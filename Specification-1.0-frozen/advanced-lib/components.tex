%%%%%%%%%%%%%%%%%%%%%%%%%%%%%%%%%%%%%%%%%%%%%%%%%%%%%%%%%%%%%%%%%%%%%%%%%%%%%%%%
%   Copyright 2009, Oracle and/or its affiliates.
%   All rights reserved.
%
%
%   Use is subject to license terms.
%
%   This distribution may include materials developed by third parties.
%
%%%%%%%%%%%%%%%%%%%%%%%%%%%%%%%%%%%%%%%%%%%%%%%%%%%%%%%%%%%%%%%%%%%%%%%%%%%%%%%%

\chapter{Components and APIs}
\chaplabel{lib:api-components}

We define a special \EXP{\TYP{Fortress}.\TYP{Components}} \apiN\ that provides
handles on components and \apisN, 
and operations on them, for use by components themselves (e.g., development
environments), allowing components to build and maintain other components,
manipulate projects and components as objects,
compile projects into components, link components together,
deploy components on specific sites over the internet, etc.
This \apiN\ is also used by the \upgapi\ and \instapi\ \apisN.
A component implementing this \apiN\ is installed along with 
\library\ on every fortress.

Note that \TYP{Component}s and \TYP{Api}s 
can be constructed only from the factory functions provided in the \apiN.
The components and \apisN\ so constructed are also installed and 
accessible via \VAR{getComponent}, \VAR{preferences} (which returns a list of components
implementing a given \apiN, in order of preference), and \VAR{getAPI}.
Calling \VAR{preferences} on an \apiN\ in \library\
returns a non-empty list of components. In particular,
\EXP{\VAR{preferences}(\TYP{Fortress}.\TYP{Components})} returns a non-empty 
list whose first element is the very component on which the call to \VAR{preferences} was made.
Conceptually, this component serves as a handle on the enclosing fortress, which might be necessary
for the purposes of certain development tools.  

The operations on a fortress provided in this \apiN\ take components and \apisN\
as arguments directly, rather than names of components and \apisN\ as the corresponding
shell operations are described. This decision is done for the sake of convenience. Note, however,
that a component name may be rebound on a fortress, or even uninstalled, while some process $p$
keeps a reference to a corresponding \TYP{Component} object. This possibility is not
problematic because the component corresponding to this object may be simply kept by the fortress
until the object is freed in $p$. Also, note that \shellcommand{upgrade} operations
on a compound component are purely
functional: they produce new compound components as a result. Thus, the structure of a component
as viewed through a \TYP{Component} object does not became stale in the face of upgrades.

We include a method
\VAR{getSourceFile} on components that returns the source file the component was compiled from.
Source files can be included with simple components during compilation as a compiler option.
Doing so allows development tools such as graphical debuggers to easily display the 
locations of errors
without the possibility that source code would not be synchronized with compiled code, as can 
happen in conventional programming models where compiled code is stored in nonencapsulated object
files.

%api Fortress.Components
%import File from Fortress.IO
%import { List, Set, Date } from Fortress.Util
%
%trait Fortress
%  fortressName: Name
%  birthDate: Date
%  getComponent(componentName: Name): Component
%  getAPI(apiName: Name): Api
%  preferences(ofAPI: Api): \langle Component \rangle
%  compile(file: File): SimpleComponent
%  install(file: File): Component
%  install(file: File, prereqs: Set[\Api\]): Component
%  upgradeAll(componentName: Name, that: Component): ()
%
%  isValidLink(constituents: \langle Component\rangle, exports = Set[\Api\], hide = Set[\Api\]): Boolean
%
%  link(result: Name, constituents: \langle Component \rangle, exports = Set[\Api\], hide = Set[\Api\]): Component
%    requires { isValidLink(constituents, exports, hide) }
%end
%
%object EnclosingFortress extends { Fortress } end
%  
%trait FortressElement
%  elementName: Name
%  vendor: String
%  owner: Fortress
%  timeStamp: Date
%  version: Version
%  uninstall(): ()
%end
%
%trait Component extends FortressElement
%  imports: Set[\Api\]
%  exports: Set[\Api\]
%  provides: Set[\Api\]
%  visibles: Set[\Api\]
%  constituents: Set[\Component\]
%  run(args: String...): ()
%  constrain(destination: Name, apis: Set[\Api\]): Component
%  hide(destination: Name, apis: Set[\Api\]): Component
%  extract(prereqs: Set[\Api\]): File
%  isValidUpgrade(that: Component): Boolean
%  abstract upgrade(result: Name, that: Component): Component
%    requires { self.isValidUpgrade(that) }
%  sourceIsAvailable: Boolean
%  getSourceFile(): File 
%    requires { sourceIsAvailable }
%  runTests(inclusive = Boolean): ()
%end
%
%trait Api extends FortressElement
%  uses: Set[\Api\]
%  extraction: File
%end
%
%trait SimpleComponent extends Component end
%
%trait Name end
%
%trait Version
%  major: NN
%  minor: NN
%end
%
%end
\begin{Fortress}
\(\KWD{api} \TYP{Fortress}.\TYP{Components}\)\\
\(\KWD{import} \TYP{File} \KWD{from} \TYP{Fortress}.\TYP{IO}\)\\
\(\KWD{import} \{\,\TYP{List}, \TYP{Set}, \TYP{Date}\,\} \KWD{from} \TYP{Fortress}.\TYP{Util}\)\\[4pt]
\(\KWD{trait} \TYP{Fortress}\)\\
{\tt~~}\pushtabs\=\+\(  \VAR{fortressName}\COLON \TYP{Name}\)\\
\(  \VAR{birthDate}\COLON \TYP{Date}\)\\
\(  \VAR{getComponent}(\VAR{componentName}\COLON \TYP{Name})\COLON \TYP{Component}\)\\
\(  \VAR{getAPI}(\VAR{apiName}\COLON \TYP{Name})\COLON \TYP{Api}\)\\
\(  \VAR{preferences}(\VAR{ofAPI}\COLON \TYP{Api})\COLON \langle \TYP{Component} \rangle\)\\
\(  \VAR{compile}(\VAR{file}\COLON \TYP{File})\COLON \TYP{SimpleComponent}\)\\
\(  \VAR{install}(\VAR{file}\COLON \TYP{File})\COLON \TYP{Component}\)\\
\(  \VAR{install}(\VAR{file}\COLON \TYP{File}, \VAR{prereqs}\COLON \TYP{Set}\llbracket\TYP{Api}\rrbracket)\COLON \TYP{Component}\)\\
\(  \VAR{upgradeAll}(\VAR{componentName}\COLON \TYP{Name}, \VAR{that}\COLON \TYP{Component})\COLON ()\)\\[4pt]
\(  \VAR{isValidLink}(\VAR{constituents}\COLON \langle \TYP{Component}\rangle, \VAR{exports} = \TYP{Set}\llbracket\TYP{Api}\rrbracket, \VAR{hide} = \TYP{Set}\llbracket\TYP{Api}\rrbracket)\COLON \TYP{Boolean}\)\\[4pt]
\(  \VAR{link}(\VAR{result}\COLON \TYP{Name}, \VAR{constituents}\COLON \langle \TYP{Component} \rangle, \VAR{exports} = \TYP{Set}\llbracket\TYP{Api}\rrbracket, \VAR{hide} = \TYP{Set}\llbracket\TYP{Api}\rrbracket)\COLON \TYP{Component}\)\\
{\tt~~}\pushtabs\=\+\(    \KWD{requires} \{\, \VAR{isValidLink}(\VAR{constituents}, \VAR{exports}, \VAR{hide})\,\}\)\-\-\\\poptabs\poptabs
\(\KWD{end}\)\\[4pt]
\(\KWD{object} \TYP{EnclosingFortress} \KWD{extends} \{\,\TYP{Fortress}\,\} \KWD{end}\)\\[4pt]
\(\KWD{trait} \TYP{FortressElement}\)\\
{\tt~~}\pushtabs\=\+\(  \VAR{elementName}\COLON \TYP{Name}\)\\
\(  \VAR{vendor}\COLON \TYP{String}\)\\
\(  \VAR{owner}\COLON \TYP{Fortress}\)\\
\(  \VAR{timeStamp}\COLON \TYP{Date}\)\\
\(  \VAR{version}\COLON \TYP{Version}\)\\
\(  \VAR{uninstall}()\COLON ()\)\-\\\poptabs
\(\KWD{end}\)\\[4pt]
\(\KWD{trait} \TYP{Component} \KWD{extends} \TYP{FortressElement}\)\\
{\tt~~}\pushtabs\=\+\(  \VAR{imports}\COLON \TYP{Set}\llbracket\TYP{Api}\rrbracket\)\\
\(  \VAR{exports}\COLON \TYP{Set}\llbracket\TYP{Api}\rrbracket\)\\
\(  \VAR{provides}\COLON \TYP{Set}\llbracket\TYP{Api}\rrbracket\)\\
\(  \VAR{visibles}\COLON \TYP{Set}\llbracket\TYP{Api}\rrbracket\)\\
\(  \VAR{constituents}\COLON \TYP{Set}\llbracket\TYP{Component}\rrbracket\)\\
\(  \VAR{run}(\VAR{args}\COLON \TYP{String}\ldots)\COLON ()\)\\
\(  \VAR{constrain}(\VAR{destination}\COLON \TYP{Name}, \VAR{apis}\COLON \TYP{Set}\llbracket\TYP{Api}\rrbracket)\COLON \TYP{Component}\)\\
\(  \VAR{hide}(\VAR{destination}\COLON \TYP{Name}, \VAR{apis}\COLON \TYP{Set}\llbracket\TYP{Api}\rrbracket)\COLON \TYP{Component}\)\\
\(  \VAR{extract}(\VAR{prereqs}\COLON \TYP{Set}\llbracket\TYP{Api}\rrbracket)\COLON \TYP{File}\)\\
\(  \VAR{isValidUpgrade}(\VAR{that}\COLON \TYP{Component})\COLON \TYP{Boolean}\)\\
\(  \KWD{abstract} \VAR{upgrade}(\VAR{result}\COLON \TYP{Name}, \VAR{that}\COLON \TYP{Component})\COLON \TYP{Component}\)\\
{\tt~~}\pushtabs\=\+\(    \KWD{requires} \{\, \mathord{\KWD{self}}.\VAR{isValidUpgrade}(\VAR{that})\,\}\)\-\\\poptabs
\(  \VAR{sourceIsAvailable}\COLON \TYP{Boolean}\)\\
\(  \VAR{getSourceFile}()\COLON \TYP{File} \)\\
{\tt~~}\pushtabs\=\+\(    \KWD{requires} \{\, \VAR{sourceIsAvailable}\,\}\)\-\\\poptabs
\(  \VAR{runTests}(\VAR{inclusive} = \TYP{Boolean})\COLON ()\)\-\\\poptabs
\(\KWD{end}\)\\[4pt]
\(\KWD{trait} \TYP{Api} \KWD{extends} \TYP{FortressElement}\)\\
{\tt~~}\pushtabs\=\+\(  \VAR{uses}\COLON \TYP{Set}\llbracket\TYP{Api}\rrbracket\)\\
\(  \VAR{extraction}\COLON \TYP{File}\)\-\\\poptabs
\(\KWD{end}\)\\[4pt]
\(\KWD{trait} \TYP{Name} \KWD{end}\)\\[4pt]
\(\KWD{trait} \TYP{SimpleComponent} \KWD{extends} \TYP{Component} \KWD{end}\)\\[4pt]
\(\KWD{trait} \TYP{Version}\)\\
{\tt~~}\pushtabs\=\+\(  \VAR{major}\COLON \mathbb{N}\)\\
\(  \VAR{minor}\COLON \mathbb{N}\)\-\\\poptabs
\(\KWD{end}\)\\[4pt]
\(\KWD{end}\)
\end{Fortress}
