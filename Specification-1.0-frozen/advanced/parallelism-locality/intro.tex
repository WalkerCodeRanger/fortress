%%%%%%%%%%%%%%%%%%%%%%%%%%%%%%%%%%%%%%%%%%%%%%%%%%%%%%%%%%%%%%%%%%%%%%%%%%%%%%%%
%   Copyright 2009, Oracle and/or its affiliates.
%   All rights reserved.
%
%
%   Use is subject to license terms.
%
%   This distribution may include materials developed by third parties.
%
%%%%%%%%%%%%%%%%%%%%%%%%%%%%%%%%%%%%%%%%%%%%%%%%%%%%%%%%%%%%%%%%%%%%%%%%%%%%%%%%

\note{Distributions are not yet supported.}

\label{parallel-intro}
Fortress is designed to make parallel programming as simple and as
painless as possible.  This chapter describes the internals of
Fortress parallelism designed for use by authors of library code (such as
\emph{distributions}, generators, and arrays).  We adopt a multi-tiered
approach to parallelism:
\begin{itemize}

\item At the highest level, we provide libraries that allocate
locality-aware distributed
shared arrays (\secref{arrays}) and implicitly
  parallel constructs such as tuples and loops.  Synchronization is
  accomplished through the use of atomic sections (\secref{atomic}).
  More complex synchronization makes use of abortable atomicity,
  described in \secref{transactions}.

\item There is an extensive library of distributions, which
  permits the programmer to specify locality and data distribution
  explicitly (\secref{distributions}).

\item
Immediately below that, the \KWD{at} expression requests that a
computation take place in a particular region of the machine
(\secref{parallelism-fundamentals}).
  We also provide a mechanism to
  terminate a spawned thread early (\secref{early-termination}).

\item Finally, there are mechanisms for constructing new generators
  via recursive subdivision into tree structures with individual
  elements at the leaves.
  \secref{defining-generators} explains how
  iterative constructs such as \KWD{for} loops and comprehensions
  are desugared into calls to methods of trait \TYP{Generator}, and how
  new instances of this trait may be defined.

\end{itemize}

We begin by describing the abstraction of \emph{regions}, which Fortress
uses to describe the machine on which a program is run.
