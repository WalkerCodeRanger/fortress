%%%%%%%%%%%%%%%%%%%%%%%%%%%%%%%%%%%%%%%%%%%%%%%%%%%%%%%%%%%%%%%%%%%%%%%%%%%%%%%%
%   Copyright 2009, Oracle and/or its affiliates.
%   All rights reserved.
%
%
%   Use is subject to license terms.
%
%   This distribution may include materials developed by third parties.
%
%%%%%%%%%%%%%%%%%%%%%%%%%%%%%%%%%%%%%%%%%%%%%%%%%%%%%%%%%%%%%%%%%%%%%%%%%%%%%%%%

\section{Early Termination of Threads}
\seclabel{early-termination}

As noted in \secref{threads-parallelism}, an implicit thread
can be terminated if its group is going to throw an exception.
Similarly, a spawned thread \VAR{t} may be terminated by calling
\EXP{t.\VAR{stop}()}.  A successful attempt to terminate a thread causes the
thread to complete asynchronously.  There is no guarantee that
termination attempts will be prompt, or that they will occur at all;
the implementation will make its best effort.  If a thread completes
normally or exceptionally before an attempt to terminate it succeeds,
the result is retained and the termination attempt is simply dropped.

At present stopping a thread immediately causes it to cease execution;
no outstanding \KWD{finally} blocks are run and the thread is not
considered to return a result.

\note{From here till the end of this section, copied from F1.0$\beta$.}

A termination attempt acts as if a special hidden \emph{stop
  exception} is thrown in that thread.  This exception cannot be
thrown by \KWD{throw} or caught by \KWD{catch}; however, \KWD{finally}
clauses are run as with any other exception.  If the stopped thread
was in the middle of an \KWD{atomic} expression, the effects of that
expression are rolled back just as with an ordinary \KWD{throw}.  A
special wrapper around every spawned thread is provided by the
Fortress implementation; it catches the stop exception and transforms
it into a deferred \TYP{Stopped} exception.  This is visible to the
programmer and should be caught by invoking the \VAR{val} method on
the thread object.  Implicit threads are terminated only if another
thread in the group completes abruptly, and the threads that are
terminated are ignored for the purposes of the completion of the
group.

Typical code for stopping a thread looks something like the following
example:
%x : ZZ64 := 0
%t = spawn do
%      try
%        atomic if x=0 then abort() else () end
%      finally
%        x := 1
%      end
%    end
%t.stop()
%try
%  t.val()
%catch s
%  Stopped => x += 2; x
%end
\begin{Fortress}
\(x \mathrel{\mathtt{:}} \mathbb{Z}64 \ASSIGN 0\)\\
\(t = \null\)\pushtabs\=\+\(\KWD{spawn}\;\;\KWD{do}\)\\
{\tt~~~~~~}\pushtabs\=\+\(      \KWD{try}\)\\
{\tt~~}\pushtabs\=\+\(        \KWD{atomic}\;\;\KWD{if} x=0 \KWD{then} \VAR{abort}() \KWD{else} () \KWD{end}\)\-\\\poptabs
\(      \KWD{finally}\)\\
{\tt~~}\pushtabs\=\+\(        x \ASSIGN 1\)\-\\\poptabs
\(      \KWD{end}\)\-\\\poptabs
\(    \KWD{end}\)\-\\\poptabs
\(t.\VAR{stop}()\)\\
\(\KWD{try}\)\\
{\tt~~}\pushtabs\=\+\(  t.\VAR{val}()\)\-\\\poptabs
\(\KWD{catch} s\)\\
{\tt~~}\pushtabs\=\+\(  \TYP{Stopped} \Rightarrow x \mathrel{+}= 2; x\)\-\\\poptabs
\(\KWD{end}\)
\end{Fortress}
Here the spawned thread \VAR{t} blocks until it is killed by the call
to \EXP{\VAR{t}.\VAR{stop}()}; it sets \VAR{x} to 1 in the
\KWD{finally} clause before exiting.  In this case, the call to
\EXP{\VAR{t}.\VAR{val}()} will throw \TYP{Stopped}, which is caught,
causing 2 to be added to \VAR{x} and returning 3.

Note that there is a race in the above code, so the \KWD{try} block in
\VAR{t} may not have been entered when \EXP{\VAR{t}.\VAR{stop}()} is
called, causing $x$ to be 2 at the end of execution.  Note also that
the call to \EXP{\VAR{t}.\VAR{stop}()} occurs asynchronously; in the
absence of the call to \EXP{\VAR{t}.\VAR{val}()}, the spawning thread
would not have waited for \VAR{t} to complete.
