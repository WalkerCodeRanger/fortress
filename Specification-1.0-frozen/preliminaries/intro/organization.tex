%%%%%%%%%%%%%%%%%%%%%%%%%%%%%%%%%%%%%%%%%%%%%%%%%%%%%%%%%%%%%%%%%%%%%%%%%%%%%%%%
%   Copyright 2009, Oracle and/or its affiliates.
%   All rights reserved.
%
%
%   Use is subject to license terms.
%
%   This distribution may include materials developed by third parties.
%
%%%%%%%%%%%%%%%%%%%%%%%%%%%%%%%%%%%%%%%%%%%%%%%%%%%%%%%%%%%%%%%%%%%%%%%%%%%%%%%%

\section{Organization}
\seclabel{organization}

This language specification is organized as follows.
In \partref{basic}, the Fortress language features for
application programmers are explained, including objects, types, and
functions. Relevant parts of the concrete syntax are provided
with many examples.  The full concrete syntax of Fortress is described in
\appref{concrete-syntax}.
\partref{advanced} describes advanced Fortress language
features for library writers.
In \partref{library}, APIs and documentation of the Fortress interpreter Library
are presented.
In \partref{basic-lib}, APIs and documentation of some of
\library\ for application programmers are presented
and \partref{advanced-lib} presents
APIs and documentation for some of \library\ for library writers.
The APIs presented in \partref{basic-lib} and \partref{advanced-lib}
are not yet tested.
Finally, in \partref{appendices}, the Fortress calculi,
support for Unicode characters, and the Fortress grammars are described.

\emph{A note on the presented libraries in
  Parts~\ref{part:basic-lib} and \ref{part:advanced-lib}:
  \Library\ presented in this draft specification
  should not be construed as exhaustive or complete.
  Presentation of additional libraries is planned for future drafts,
  as are modifications to the libraries included here.}
