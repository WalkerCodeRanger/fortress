%%%%%%%%%%%%%%%%%%%%%%%%%%%%%%%%%%%%%%%%%%%%%%%%%%%%%%%%%%%%%%%%%%%%%%%%%%%%%%%%
%   Copyright 2009, Oracle and/or its affiliates.
%   All rights reserved.
%
%
%   Use is subject to license terms.
%
%   This distribution may include materials developed by third parties.
%
%%%%%%%%%%%%%%%%%%%%%%%%%%%%%%%%%%%%%%%%%%%%%%%%%%%%%%%%%%%%%%%%%%%%%%%%%%%%%%%%

\note{This chapter including various syntax is out of date.

Imported functionals may be overloaded with declared top-level functions.

Reject identical import statements.

At most one on-demand import statement from a single API.

The component/API names should be the same as their enclosing file names.

Only a single component/API declaration should occur in a single file.

No order between \KWD{import} and \KWD{export} statements.

Import aliasing syntax should use \EXP{\Rightarrow} instead of \KWD{as}.

Eric's email titled ``Proposal: Compound APIs'' on 06/08/09:
\EXP{\KWD{api}\:\TYP{Foo} \KWD{comprises} \{\TYP{Foo1}, \TYP{Foo2}, \TYP{Foo3}, \TYP{Foo4}\} \KWD{end}}

Declaration exported by multiple APIs:
A definition in a component must not match more than one declaration in exported APIs.

Jan's email titled ``Proposed changes: scripting interface, args in a library''
on 02/05/09
}

\note{ \begin{itemize}
\item
It is a static error if
a name specified by an import %%-unqualified
statement
is also declared by a top-level or functional method declaration
in the importing component or API,
or if it is also specified by another import %%-unqualified
statement,
unless all top-level or functional declarations for that name
in the importing component or any of APIs
imported by an import %%-unqualified
statement that specifies that name
form a valid set of overloaded declarations.
It is also a static error if any specified name
is not declared by a declaration imported from the API.
\item
In addition, it is a static error if in a set of overloaded declarations,
any of the following are true:
\begin{enumerate}
\item Any declaration imported on demand is more specific than any explicit declaration.
\item Any top-level function declaration is more specific than any functional method declaration.
(This rule actually applies to declarations in a single component as well.)
\item
Any top-level function declaration imported on demand is more specific than any other declaration not in the same API as the top-level function declaration.
\end{enumerate}
It is also a static error to explicitly import a name from an API
that does not declare it, or to have multiple import-on-demand statements
with the same API.

 \end{itemize}
}

\note{API integration
  \begin{itemize}
\item
We agreed to extend Fortress to allow APIs to extend other APIs.
If an API \VAR{A} extends another API \VAR{B},
then every declaration of \VAR{B} is also a declaration of \VAR{A}.
It is a static error if a name declared in \VAR{B} is also declared in \VAR{A}
unless both declarations are of top-level functions or functional methods,
in which case, the declarations are overloaded.

The extension relation among APIs must be acyclic.
\item Selective extension of specific types
%% api FortressLibrary
%%   extends Whatever.{Boolean}
%%   extends Whatever2.{Thread}
%%   ...
%% end
\begin{Fortress}
\(\KWD{api}\:\TYP{FortressLibrary}\)\\
{\tt~~}\pushtabs\=\+\(  \KWD{extends}\:\TYP{Whatever}.\{\TYP{Boolean}\}\)\\
\(  \KWD{extends}\:\TYP{Whatever2}.\{\TYP{Thread}\}\)\\
\(  \ldots\)\-\\\poptabs
\(\KWD{end}\)
\end{Fortress}
  \end{itemize}

}

\note{Resolve ambiguity in favor of longest API name. (07/23/09)

If there are an API \EXP{A.B} and an API \VAR{A} which declares a trait \VAR{B}
declaring a method \VAR{f}, \EXP{A.B.f} in a component importing both \EXP{A.B} and \VAR{A}
is ambiguous. We agreed with the following:
\begin{itemize}
\item Importing an API: \EXP{\KWD{import}\;\;\KWD{api}\:A.B}
\item Importing an entity: \EXP{\KWD{import}\:A.B},
 \EXP{\KWD{import}\:A.\{B\:\VAR{as} \OPR{B{\char'137}A}\}}
\item Resolve ambiguity in favor of longest API name.
\end{itemize}
}

\note{Component and Value Namespace
% component Surprise
% import api Foo
% export Executable

% object Foo
%   foo():String = "foo"
%   bar():String = "foo"
%   baz():String = "foo"
% end

% run(args:String...) = do
%   Foo.foo()
%   Foo.bar()
%   Foo.baz()
% end

% end
\begin{Fortress}
{\tt~}\pushtabs\=\+\( \KWD{component}\:\TYP{Surprise}\)\\
\( \KWD{import}\;\;\KWD{api}\:\TYP{Foo}\)\\
\( \KWD{export}\:\TYP{Executable}\)\\[4pt]
\( \KWD{object}\:\TYP{Foo}\)\\
{\tt~~}\pushtabs\=\+\(   \VAR{foo}()\COLONOP\TYP{String} =\;\;\hbox{\rm``\STR{foo}''}\)\\
\(   \VAR{bar}()\COLONOP\TYP{String} =\;\;\hbox{\rm``\STR{foo}''}\)\\
\(   \VAR{baz}()\COLONOP\TYP{String} =\;\;\hbox{\rm``\STR{foo}''}\)\-\\\poptabs
\( \KWD{end}\)\\[4pt]
\( \VAR{run}(\VAR{args}\COLONOP\TYP{String}\ldots) = \;\KWD{do}\)\\
{\tt~~}\pushtabs\=\+\(   \TYP{Foo}.\VAR{foo}()\)\\
\(   \TYP{Foo}.\VAR{bar}()\)\\
\(   \TYP{Foo}.\VAR{baz}()\)\-\\\poptabs
\( \KWD{end}\)\\[4pt]
\( \KWD{end}\)\-\\\poptabs
\end{Fortress}

Static error for the declaration of the object \TYP{Foo} in the above example.
}

\note{Importing operators:

Operators with different fixities are not overloaded but different.
When an ordinary operator (i.e., not a ``vertical-line operator'') is imported,
all the operators with different fixities are imported.
When a vertical-line operator is imported, only one fixity is imported.
}

\note{``\EXP{\KWD{import}\:A.f}'' does not imply
 ``\EXP{\KWD{import}\;\;\KWD{api}\:A}''.

``\EXP{\KWD{api}\:A; \KWD{import}\;\;\KWD{api}\:B; \KWD{end}}''
 does not export \VAR{B}.
}

Fortress programs
are developed, compiled, and deployed as
\emph{encapsulated upgradable components}
that exist not only as programming language features, but also as
self-contained run-time entities that are managed
throughout the life of the software.
The imported and exported references of a component
are described with explicit \emph{\apisN}.
With components and \apisN, Fortress
provides the stability benefits of
static linking with the sharing and upgrading benefits of
dynamic linking.
\footnote{The system described in this chapter is based on that
described in \cite{allen-05-components}.}
In addition to an informal description of the component system in this
chapter, we also formally specify key functionality of the system,
and illustrate how we can reason about the correctness of the system
in \appref{components}.
