%%%%%%%%%%%%%%%%%%%%%%%%%%%%%%%%%%%%%%%%%%%%%%%%%%%%%%%%%%%%%%%%%%%%%%%%%%%%%%%%
%   Copyright 2009, Oracle and/or its affiliates.
%   All rights reserved.
%
%
%   Use is subject to license terms.
%
%   This distribution may include materials developed by third parties.
%
%%%%%%%%%%%%%%%%%%%%%%%%%%%%%%%%%%%%%%%%%%%%%%%%%%%%%%%%%%%%%%%%%%%%%%%%%%%%%%%%

\section{Tests in Components and \Apis}
\seclabel{component-tests}

A component may include definitions of tests,
as described in \chapref{tests}.
These definitions are allowed to refer to both test and non-test code defined in the same
component or declared in \apisN\ imported by the component.

An \apiN\ may also include definitions of tests.
These definitions may refer to all declarations in the \apiN\ as well as in
any \apisN\ it imports. Tests defined in \apisN\ should be thought of as ``executable documentation''
that partially specifies the required behavior of the declared entities.

See \secref{basicops} for an explanation of how tests
defined in components and \apisN\ are executed.
