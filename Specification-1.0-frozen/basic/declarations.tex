%%%%%%%%%%%%%%%%%%%%%%%%%%%%%%%%%%%%%%%%%%%%%%%%%%%%%%%%%%%%%%%%%%%%%%%%%%%%%%%%
%   Copyright 2009, Oracle and/or its affiliates.
%   All rights reserved.
%
%
%   Use is subject to license terms.
%
%   This distribution may include materials developed by third parties.
%
%%%%%%%%%%%%%%%%%%%%%%%%%%%%%%%%%%%%%%%%%%%%%%%%%%%%%%%%%%%%%%%%%%%%%%%%%%%%%%%%

\chapter{Names and Declarations}
\chaplabel{declarations}
\chaplabel{names}

\note{This chapter should be revised according to Victor's new description.

Qualified names, dimensions and units, tests and properties,
where clauses, keyword and varargs parameters, and
type aliases are not yet supported.}

\note{
Issues -- Victor
\begin{itemize}
\item opr parameter scope, etc.
\item delayed initialization of variables
\begin{itemize}
\item Synonyms?  (especially for operators, but maybe for identifers)
\item  \EXP{T.\KWD{coerce}} is a new kind of ``name'', so we need to be careful
  how we phrase this.  (If we can't call \EXP{T.\KWD{coerce}} explicitly,
  then should coercion declarations even be considered declarations?)
\end{itemize}
\item Should distinguish simple names, API names, and qualified names.
\begin{itemize}
\item Need to discuss (perhaps in components/APIs chapter) restriction
  on API names so that ``first part'' does not conflict with any
  name declared in (or imported unqualified by) the component.
\item Should also discuss declaration by import statements.  Is that
  a declaration?  Even if we don't consider normal import statements
  to be declarations--the declarations are in the APIs--when import
  statements provide aliases, that must be considered a declaration.
\end{itemize}
\end{itemize}}

\note{Expressions that declare names:
  typecase, catch clauses, generators, labeled blocks, bindings?
  the declaration in a typecase is just a binding
  so it's not actually that special, except for the static type

 I think we can consider ``self'' and ``coerce'' to be
 ``special identifiers''.  Or else the ``names'' declared
 by declarations are more than just identifiers and operators.

 Search for comments for some recommendations for local changes.
}

Names are used to refer to certain kinds of entities
in a Fortress program.
Names may be simple or qualified.
A simple name is either an identifier or an operator.
An operator may be an operator token,
a special token corresponding to a special operator character,
or a matching pair of enclosing operator tokens.
A qualified name consists of an \apiN\ name
followed by ``\EXP{.}'',
followed by an identifier,
where an \apiN\ name consists of a sequence of identifiers
separated by ``\EXP{.}'' tokens.
Note that operators cannot be qualified.
Except in \secref{qualified-names},
we consider only simple names in this chapter.

Simple names are typically introduced by declarations,
which bind the name to an entity.
In some cases,
the declaration is implicit.
Every declaration has a scope,
in which the declared name can be used to refer to the declared entity.

\emph{Declarations} introduce \emph{named entities};
we say that a declaration \emph{declares} an entity
and a name by which that entity can be referred to,
and that the declared name \emph{refers to} the declared entity.
As discussed later, there is not a one-one correspondence
between declarations and named entities:
some declarations declare multiple named entities,
some declare multiple names,
and some named entities are declared by multiple declarations.

Some declarations contain other declarations.
For example, a trait declaration may contain method declarations,
and a function declaration may contain parameter declarations.

The positions in which a declaration may legally
appear in a component are determined by the nonterminal \emph{Decl}
in the simplified Fortress grammar in \appref{concrete-syntax}.


\section{Kinds of Declarations}
\seclabel{decl-kinds}

\note{The grammar the follows is only for top-level declarations.}

\begin{Grammar}
\emph{Decl} &::=& \emph{TraitDecl}\\
&$|$& \emph{ObjectDecl}\\
&$|$& \emph{VarDecl}\\
&$|$& \emph{FnDecl}\\
&$|$& \emph{DimUnitDecl}\\
&$|$& \emph{TypeAlias}\\
&$|$& \emph{TestDecl}\\
&$|$& \emph{PropertyDecl}\\
&$|$& \emph{ExternalSyntax}
\footnote{
Fortress provides declarations of syntax expanders for domain-specific languages
as described in \chaplabel{syntax-expanders}.
 However, these declarations do not introduce named entities.
}\\
\end{Grammar}
There are two kinds of declarations: \emph{top-level declarations} and
\emph{local declarations}.

Top-level declarations occur at the top level of a component,
not within any other declaration or expression.
A top-level declaration is one of the following:\footnote{The
Fortress component system, defined in \chapref{components},
includes declarations of \emph{components} and \emph{\apisN}.
Because component names are not used in a Fortress program and
\apiN\ names are used only in
qualified names and
\KWD{import} and \KWD{export} statements,
we do not discuss them in this chapter.
However, these declarations are not proper constituents of a valid
program.  Rather, valid programs are \emph{assembled} from them.
}
\begin{itemize}

\item
trait declarations (see \chapref{traits});

\item
object declarations (see \chapref{objects}),
which may be singleton declarations or constructor declarations;

\item
top-level variable declarations
(see \secref{top-var-decls});

\item
top-level function declarations
(see \chapref{functions}),
including top-level operator declarations
(see \chapref{operators});

\item
dimension declarations (see \chapref{dimunits});

\item
unit declarations (see \chapref{dimunits});

\item
top-level type aliases (see \secref{type-alias});

\item
test declarations
(see \chapref{tests});

\item
top-level property declarations;
(see \chapref{tests})

\end{itemize}

Local declarations occur in another declaration
or in some expression (or both).
A local declaration is one of the following:
\begin{itemize}

\item
field declarations (see \secref{fields}),
which occur in object declarations and object expressions,
and include field declarations
in the parameter list of a constructor declaration;

\item
method declarations (see \secref{methods}),
which occur in trait and object declarations
and object expressions;

\item
coercion declarations (see \chapref{conversions-coercions}),
which occur in trait and object declarations;

\item
local variable declarations (see \secref{local-var-decls}),
which occur in expression blocks;

\item
local function declarations (see \secref{local-fn-decls}),
which occur in expression blocks;

\item
local property declarations (see \chapref{tests}),
which occur in trait and object declarations and object expressions;

\item
labeled blocks (see \secref{label-expr}),
which are expressions;

\item
static-parameter declarations,
which may declare
type parameters,
\KWD{nat} parameters,
\KWD{int} parameters,
\KWD{bool} parameters,
\KWD{dim} parameters,
\KWD{unit} parameters,
or \KWD{opr} parameters
(see \chapref{trait-parameters}),
and occur in static-parameter lists
of trait and object declarations,
top-level type aliases,
top-level function declarations,
method declarations,
and local function declarations.

\item
hidden-type-variable declarations,
which occur in \KWD{where} clauses
(see \secref{where-clauses})
of trait and object declarations,
top-level function declarations,
and method declarations;

\item
type aliases in \KWD{where} clauses,
of trait and object declarations,
top-level function declarations,
and method declarations;

\item
(value) parameter declarations,
which may be keyword-parameter declarations
which occur in parameter lists
of constructor declarations,
top-level function declarations,
method declarations,
local function declarations,
and function expressions

\end{itemize}

Some declarations are syntactic sugar
for other declarations.
Throughout this chapter,
we consider declarations after they have been desugared.
Thus, apparent field declarations in trait declarations
are actually method declarations (as described in \secref{methods}),
and a dimension and unit declaration
may desugar into several separate declarations
(as described in \secref{abbrev-dimunits}).
After desugaring, the kinds of declarations listed above are disjoint.

In addition to these explicit declarations,
there are two cases in which names are declared implicitly:
\begin{itemize}

\item
the special name \KWD{self} is implicitly declared
as a parameter of dotted methods
(see \secref{methods} for details);
and

\item
the name \VAR{result} is implicitly declared
as a variable for the \KWD{ensures} clause of a contract
(see \secref{contracts} for details).

\end{itemize}

\note{This is *not* the proper reference, 
but I don't know what is.}
As discussed in \chapref{variables}, 
the initialization of a local variable 
may be separated from its declaration.
In this case 
we do not consider the initialization of an immutable variable 
to be a declaration, 
even though it is syntatically indistinguishable 
from a declaration of an immutable variable.

Trait declarations,
object declarations,
top-level type aliases,
type-parameter declarations,
and hidden-type-variable declarations
are collectively called \emph{type declarations};
they declare names that refer to \emph{types}
(see \chapref{types}).
Dimension declarations
and \KWD{dim}-parameter declarations
are \emph{dimension declarations},
and unit declarations
and \KWD{unit}-parameter declarations
are \emph{unit declarations}.
Constructor declarations,
top-level function declarations,
method declarations,
and local function declarations
are collectively called \emph{functional declarations}.
Singleton declarations,
top-level variable declarations,
field declarations,
local variable declarations
(including implicit declarations of \VAR{result})
and (value) parameter declarations
(including implicit declarations of \KWD{self})
are collectively called \emph{variable declarations}.
Static-parameter declarations
and hidden-type-variable declarations are also
collectively called \emph{static-variable declarations}.
Note that static-variable declarations are disjoint
from variable declarations.
\note{
Victor: May want to add ``module-wide'' declarations, 
which are the top-level declarations plus functional method declarations.
These are the declarations whose reach is the entire component
and that can be imported from an API.}

The groups of declarations defined in the previous paragraph
are neither disjoint nor exhaustive.
For example, labeled blocks are not included in any of these groups, and
an object declaration is both a type declaration
and either a function or variable declaration,
depending on whether it is singleton.

Most declarations declare a single name
given explicitly in the declaration
(though, as discussed in \secref{namespaces},
they may declare this name in multiple namespaces).
There is one exception:
wrapped field declarations (described in \secref{abstract-fields})
in object declarations and object expressions declare both the field name
and names for methods provided
by the declared type of the field.

Method declarations in a trait may be either
\emph{abstract} or \emph{concrete}.
Abstract declarations do not have bodies;
concrete declarations, sometimes called \emph{definitions}, do.


\section{Namespaces}
\seclabel{namespaces}

Fortress supports three namespaces,
one for types, one for values, and one for labels.
(If we consider the Fortress component system,
there is another namespace for \apisN.)
These namespaces are logically disjoint:
names in one namespace do not conflict
with names in another.

Type declarations, of course,
declare names in the type namespace.
Function and variable declarations declare names
in the value namespace.
(This implies that object names
are declared in both the type and value namespaces.)
Labeled blocks declare names in the label namespace.
Although they are not all type declarations,
all the static-variable declarations declare names
in the type namespace, as do dimension declarations.
In addition,
\KWD{nat} parameters,
\KWD{int} parameters,
\KWD{bool} parameters,
\KWD{unit} parameters, and
\KWD{opr} parameters
are also declared in the value namespace.
\secref{opr-ident} describes
\KWD{opr} parameters in more detail.

Every occurrence of a name in a program
is either a declaration or a reference.
\footnote{We are abusing terminology here
by using ``declaration'' to denote
the occurrence of a name,
whereas we have generally been using ``declaration''
to denote the syntactic construct that declares the name.
Many declarations in the latter sense
allow the declared name to occur several times:
only one such occurrence is a declaration in the former sense;
the rest are references.}
A reference to a name is resolved to
the entity that the name refers to the namespace
appropriate to the context
in which the reference occurs.
For example, a name refers to a label
if and only if it occurs immediately following
the reserved word \KWD{exit}.
It refers to a type if and only if it appears in a type context
(described in \chapref{types}).
Otherwise, it refers to a value.


\section{Reach and Scope of Declarations}
\seclabel{scope}
\seclabel{shadowing}

In this section,
we define the \emph{reach} and \emph{scope} of a declaration,
which determine where a declared name may be used to refer to the entity
declared by the declaration.  
It is a static error 
for a reference to a name to occur at any point in a program
at which the name is not \emph{in scope} in the appropriate namespace
(as defined below),
except immediately after the `.'
of a dotted field access or dotted method invocation
when the name is the name of the field or dotted method
provided by the static type of the receiver expression
(see Sections~\ref{sec:field-access}
and~\ref{sec:dotted-method-calls}).
It is also a static error 
for multiple declarations with overlapping reaches 
to declare the same name in the same namespace 
unless the declarations are \emph{overloaded} 
or one declaration \emph{shadows} the other, 
as defined below.

We first define a declaration's \emph{reach}.
The reach of a labeled block is the block itself.
The reach of a functional method declaration
is the component containing that declaration.
A dotted method declaration not in an object expression
(described in \secref{object-expr}) or declaration
must be in the declaration of some trait $T$,
and its reach is the declaration of $T$
and any trait or object declarations
or object expressions
that extend $T$;
that is,
if the declaration of trait $T$
contains a method declaration,
and trait $S$ extends trait $T$,
then the reach of that method declaration
includes the declaration of trait $S$.
The reach of any other (explicit) declaration
is the smallest block strictly containing that declaration
(i.e., not just the declaration itself).
For example,
the reach of any top-level declaration
(including any imported declaration)
is the component containing (or importing) that declaration;
the reach of a field declaration
is the enclosing object declaration or expression;
the reach of a parameter declaration
is the constructor declaration, functional declaration,
or function expression in whose parameter list it occurs;
and the reach of a local variable declaration
is the smallest block in which that declaration occurs
(recall from \chapref{programs}
that a local variable declaration always starts a new block).
The reach of an implicit declaration of \KWD{self}
by a dotted method declaration
is the method declaration
(the same as the explicit parameters of the method),
and the reach of an implicit declaration of \VAR{result}
by an \KWD{ensures} clause of a contract 
is the \KWD{ensures} clause.
We say that a declaration \emph{reaches}
any point within its reach.

One declaration \emph{shadows} another for a name in a namespace 
if they both declare the name in the namespace,
and any of the following conditions hold:
\begin{itemize}

\item
The first declaration is a field or dotted method declaration 
in a trait declaration, object declaration or object expression
contained strictly within the reach of the second declaration, 
unless the first declaration is in an object expression 
that extends a trait 
that provides the second declaration 
(in which case the declarations are overloaded).

\note{
Alternatively, split above condition into two:
\begin{itemize}
\item
The first declaration is a field or dotted method declaration 
in a trait or object declaration 
and the second declaration is a top-level or functional method declaration.

\item
The first declaration is a field or dotted method declaration 
in an object expression
contained strictly entirely within the reach of the second declaration, 
unless the object expression extends a trait 
that provides the second declaration
(in which case the declarations are overloaded).
\end{itemize}
}

\item
The first declaration is a keyword-parameter declaration 
of a function or method declaration 
contained strictly in the reach of the second declaration.

\item
the name is \KWD{self}, 
and the reach of the first declaration 
is contained in the reach of the second declaration.
\note{
I'm not sure this is necessary because self should not be in scope
in an enclosed object expression in any case.
However, we still have two declarations of self with overlapping reaches, 
but which are not overloaded,
so it may be safer to say there is shadowing in any case.}

\item
the name is \VAR{result}
and the first declaration is an implicit declaration by an \KWD{ensures} clause 
contained in the reach of the second declaration.

\end{itemize}
We omit the name and the namespace when they are clear from context.
We say that the second declaration \emph{is shadowed by} the first 
at any point in the program that their reaches overlap.
No other shadowing is permitted in a Fortress program.


The \emph{scope} of a declaration for a name in a namespace 
(assuming the declaration declares the name in the namespace) 
consists of all points in the reach of the declaration except the following:
\begin{itemize}

\item
any point where the declaration is shadowed by another 
for that name and namespace;

\item
the declaration is a type alias (top-level or not),
a dimension declaration or a unit declaration,
and the program point is in the declaration;

\item
if the declaration is a field, local variable or parameter declaration, 
any point in the declaration or that textually precedes the declaration;

\item
if the name is \KWD{self}, 
any point in an object expression 
contained in the reach of the declaration;

\item
if the declaration is a labeled block,
any point in a \KWD{spawn} expression
in the labeled block.

\end{itemize}
We say that a name is \emph{in scope} in a namespace 
at any point in the program 
that is in the scope of some declaration for that name and namespace.
Again, we omit the name and/or namespace when they are clear from context.

As mentioned above,
it is a static error for a reference to a name to occur 
where it is not in scope in the appropriate namespace.
In addition, 
local variables must be properly initialized before they are used, 
as discussed in \chapref{variables}.

If the scope of multiple declarations for the same name and namespace overlap, 
then we say that the name is \emph{overloaded} in that namespace 
in the overlap, 
and we say that the declarations are \emph{overloaded} for that name.

Declarations may be overloaded only if one of the following conditions hold:
\begin{itemize}

\item
They are all dotted method declarations.

\item
They are all top-level function declarations and functional method declarations.

\item
One is a field declaration, 
and the rest are getter and setter declarations.

\end{itemize}
In the first two cases,
the declarations define an \emph{overloaded functional} 
(i.e., function or method), 
and they must satisfy certain rules 
that ensure that the functional does not admit ambiguous calls.
See \chapref{overloaded-declarations} for further discussion on this topic.
In the last case, 
the type of the field must be a subtype of the return type of any getter 
and a supertype of the parameter type of any setter.
\note{
Can we put the rules for fields overloading into the overloading chapter 
and over here simply say, "in either case, see Chapter XX"?
We might also join the first two cases, and leave the discussion of 
which kinds of functional declarations can be overloaded 
to the overloading chapter.}

\section{Imported Declarations and Qualified Names}
\seclabel{qualified-names}
\begin{Grammar}
\emph{QualifiedName} &::=& \emph{Id}(\EXP{.}\emph{Id})$^*$\\
\end{Grammar}

Fortress provides a component system
in which the entities declared in a component
are described by an \apiN.
A component may \emph{import} \apisN,
allowing it to refer to these entities
declared by the imported \apisN.
In some cases,
references to these entities must be \emph{qualified}
by the \apiN\ name.
These qualified names can be used
in any place that a simple name would be used
had the entity been declared directly in the component
rather than being imported.
Note that qualified names are distinguished from simple names
by the inclusion of a ``\EXP{.}'' token,
so they never shadow, nor are they shadowed by, simple names.
For further discussion on \apisN\ and the component system,
see \chapref{components}.
