%%%%%%%%%%%%%%%%%%%%%%%%%%%%%%%%%%%%%%%%%%%%%%%%%%%%%%%%%%%%%%%%%%%%%%%%%%%%%%%%
%   Copyright 2009, Oracle and/or its affiliates.
%   All rights reserved.
%
%
%   Use is subject to license terms.
%
%   This distribution may include materials developed by third parties.
%
%%%%%%%%%%%%%%%%%%%%%%%%%%%%%%%%%%%%%%%%%%%%%%%%%%%%%%%%%%%%%%%%%%%%%%%%%%%%%%%%

\chapter{Overloading and Multiple Dispatch}
\chaplabel{multiple-dispatch}

\note{Keyword and varargs parameters are not yet supported.}

Fortress allows functions and methods (collectively called \emph{functionals})
to be \emph{overloaded}.  That is,
there may be multiple declarations for the same functional name
visible in a single scope (which may include inherited method declarations),
and several of them may be applicable to any particular functional call.

However, with these benefits comes the potential for ambiguous calls at run time.
For an overloaded functional call,
the most specific applicable declaration is chosen, if any.
Otherwise, any of the applicable declarations such that
no other applicable declaration is more specific than them is chosen.


In this chapter, we describe how to determine which declarations
are \emph{applicable} to a particular functional call,
and when several are applicable, how to select among them.
We also provide overloading rules for the declarations of functionals
to eliminate \emph{some possibility} of ambiguous calls at run time,
whether or not these calls actually appear in the program.
\secref{overloading-terms} introduces some terminology and notation.
In \secref{functional-applicability},
we show how to determine which declarations
are applicable to a \emph{named functional call}
(a function call described in \secref{function-calls}
or a naked method invocation described in \secref{naked-method-calls})
when all declarations have only ordinary parameters
(without varargs or keyword
parameters).
We discuss how to handle dotted method calls
(described in \secref{dotted-method-calls})
in \secref{dotted-method-applicability},
and declarations with varargs and keyword
parameters in \secref{applicability-with-special-params}.
Determining which declaration is applied,
if several are applicable,
is discussed in \secref{resolving-overloading}.


\section{Principles of Overloading}
\seclabel{overloading-terms}

Fortress allows multiple functional declarations of the same name to
be declared in a single scope.  However, recall from \chapref{names}
the following shadowing rules:
\begin{itemize}
\item
dotted method declarations shadow top-level function declarations with
the same name, and
\item
dotted method declarations provided by a trait or object declaration
or object expression
shadow functional method declarations with the same name that are
provided by a different trait or object declaration
or object expression.
\end{itemize}
Also, note that a trait or object declaration
or object expression must not have a
functional method declaration and a dotted method declaration with the
same name, either directly or by inheritance.
Therefore, top-level functions can overload
with other top-level functions and functional methods,
dotted methods with other dotted methods,
and functional methods with other functional methods and top-level functions.
It is a static error if any top-level function declaration is
more specific than any functional method declaration.
If a top-level function declaration is overloaded with a functional method
declaration, the top-level function declaration must not be more specific than
the functional method declaration.


Operator declarations with the same name but different fixity
are not a valid overloading; they are unambiguous declarations.
An operator method declaration whose name is one of the operator parameters
(described in \secref{opr-ident})
of its enclosing trait or object may be overloaded with other operator
declarations in the same component.
Therefore, such an operator method declaration
must satisfy the overloading rules
(described in \chapref{overloaded-declarations})
with every operator declaration in the same component.

\note{This restriction will be relaxed.}

Although there may be multiple declarations
with the same functional name,
it is an error for their static parameters to differ
(up to $\alpha$-equivalence),
or for one declaration to have static parameters
and another to not have them.
Hence, static parameters do not enter into the determination of which
declarations are applicable,
so we ignore them for most of this chapter.



Recall from \chapref{types} that we write
$T \Ovrsubtype U$ when $T$ is a subtype of $U$,
and $T \strictsubtype U$ when $T \Ovrsubtype U$ and $T \neq U$.


\section{Applicability to Named Functional Calls}
\seclabel{functional-applicability}
In this section, we show how to determine which declarations are
applicable to a named functional call when all declarations have only
ordinary parameters (i.e., neither varargs nor keyword parameters).

For the purpose of defining applicability, a named functional call can
be characterized by the name of the functional and its argument type.
Recall that a functional has a single parameter, which may be a tuple
(a dotted method has a receiver as well).
 We abuse notation by using
\emph{static call $\f(\As)$} to refer to a named functional call with
name $\f$ and whose argument has static type $\As$, and
\emph{dynamic call $\f(\Xs)$} to refer to a named functional call with
We use \emph{call $\f(\Cs)$} to refer to a named functional call with
name $\f$ and whose argument, when evaluated, has dynamic type $\Cs$.
(Note that if the type system is sound---and we certainly hope that
it is!---then $\Xs \Ovrsubtype \As$ for all well-typed calls to $\f$.)
We use the term \emph{call $\f(\Cs)$} to refer to static and dynamic calls
collectively.
We assume throughout this chapter that all static variables in functional calls
have been instantiated or inferred.


We also use \emph{function declaration $\f(\Ps):\Us$} to refer to a
function declaration with function name $\f$, parameter type $\Ps$,
and return type $\Us$.


For method declarations, we must take into account the self parameter,
as follows:

A \emph{dotted method declaration $\Ps_0.\f(\Ps):\Us$} is a dotted
method declaration with name $f$, where $\Ps_0$ is the trait or object
type in which the declaration appears, $\Ps$ is the parameter type,
and $\Us$ is the return type.  (Note that despite the suggestive
notation, a dotted method declaration does not explicitly list its
self parameter.)

A \emph{functional method declaration $\f(\Ps):\Us$ with self
parameter at $i$} is a functional method declaration with name $f$,
with the parameter \EXP{\mathord{\KWD{self}}} in the $i$th position
of the parameter type $\Ps$, and return type $\Us$.  Note that the static
type of the self parameter is the trait or object trait type in which
the declaration $\f(\Ps):\Us$ occurs.  In the following, we will use
$\Ps_i$ to refer to the $i$th element of $\Ps$.

We elide the return type of a declaration, writing $\f(\Ps)$ and
$\Ps_0.\f(\Ps)$, when the return type is not relevant to the
discussion.
Note that static parameters may appear in the types $\Ps_0$, $\Ps$, and $\Us$.


A declaration $\f(\Ps)$ is \emph{applicable} to a call $\f(\Cs)$
if the call is in the scope of the declaration and $\Cs \Ovrsubtype \Ps$.
(See \chapref{names} for the definition of scope.)
If the parameter type $\Ps$ includes static parameters,
they are inferred as described in \chapref{type-inference}
before checking the applicability of the declaration to the call.



Note that a named functional call $\f(\Cs)$ may invoke a dotted method
declaration if the declaration is provided by the trait or object
enclosing the call.  To account for this, let $\Cs_0$ be the trait or
object declaration immediately enclosing the call.
%
Then we consider a named functional call $\f(\Cs)$ as $\Cs_0.\f(\Cs)$
if $\Cs_0$ provides dotted method declarations applicable to $\f(\Cs)$, and
use the rule for applicability to dotted method calls (described in
\secref{dotted-method-applicability}) to determine which declarations
are applicable to $\Cs_0.\f(\Cs)$.



\section{Applicability to Dotted Method Calls}
\seclabel{dotted-method-applicability}

Dotted method applications can be characterized similarly to named
functional applications, except that, analogously to dotted method declarations,
we use $\Cs_0$ to denote the dynamic type of the receiver object,
and, as for named functional calls, $\Cs$ to denote the dynamic
type of the argument of a  dotted method call.
We write $\Cs_0.\f(\Cs)$ to refer to the  call.

A dotted method declaration $\Ps_0.\f(\Ps)$ is \emph{applicable}
to a dotted method call $\Cs_0.\f(\Cs)$ if $\Cs_0 \Ovrsubtype
\Ps_0$ and $\Cs \Ovrsubtype \Ps$.
If the types $\Ps_0$ and $\Ps$ include static parameters,
they are inferred as described in \chapref{type-inference}
before checking the applicability of the declaration to the call.



\section{Applicability for Functionals with Varargs and Keyword Parameters}
\seclabel{applicability-with-special-params}

The basic idea for handling varargs and keyword
parameters is that we
can think of a functional declaration that has such parameters as
though it were (possibly infinitely) many declarations, one for each
set of arguments it may be called with.  In other words, we expand
these declarations so that there exists a declaration for each number
of arguments that can be passed to it.


A declaration with a varargs parameter corresponds to an infinite
number of declarations, one for every number of arguments that may be
passed to the varargs parameter.  In practice, we can bound that
number by the maximum number of arguments that the functional is
called with anywhere in the program (in other words, a given program
will contain only a finite number of calls with different numbers of
arguments).  The expansion described here is a conceptual one to
simplify the description of the semantics; we do not expect a real
implementation to actually expand these declarations at compile time.
%
For example, the following declaration:
%f(x : ZZ, y : ZZ, z : ZZ...):ZZ
\begin{Fortress}
\(f(x \mathrel{\mathtt{:}} \mathbb{Z}, y \mathrel{\mathtt{:}} \mathbb{Z}, z \mathrel{\mathtt{:}} \mathbb{Z}\ldots)\COLONOP\mathbb{Z}\)
\end{Fortress}
would be expanded into:
%f(x : ZZ, y : ZZ):ZZ
%f(x : ZZ, y : ZZ, z1 : ZZ):ZZ
%f(x : ZZ, y : ZZ, z1 : ZZ, z2 : ZZ):ZZ
%f(x : ZZ, y : ZZ, z1 : ZZ, z2 : ZZ, z3 : ZZ):ZZ
%...
\begin{Fortress}
\(f(x \mathrel{\mathtt{:}} \mathbb{Z}, y \mathrel{\mathtt{:}} \mathbb{Z})\COLONOP\mathbb{Z}\)\\
\(f(x \mathrel{\mathtt{:}} \mathbb{Z}, y \mathrel{\mathtt{:}} \mathbb{Z}, z_{1} \mathrel{\mathtt{:}} \mathbb{Z})\COLONOP\mathbb{Z}\)\\
\(f(x \mathrel{\mathtt{:}} \mathbb{Z}, y \mathrel{\mathtt{:}} \mathbb{Z}, z_{1} \mathrel{\mathtt{:}} \mathbb{Z}, z_{2} \mathrel{\mathtt{:}} \mathbb{Z})\COLONOP\mathbb{Z}\)\\
\(f(x \mathrel{\mathtt{:}} \mathbb{Z}, y \mathrel{\mathtt{:}} \mathbb{Z}, z_{1} \mathrel{\mathtt{:}} \mathbb{Z}, z_{2} \mathrel{\mathtt{:}} \mathbb{Z}, z_{3} \mathrel{\mathtt{:}} \mathbb{Z})\COLONOP\mathbb{Z}\)\\
\(\ldots\)
\end{Fortress}
A declaration with a varargs parameter is applicable to a call if
any one of the expanded declarations is applicable.



\section{Overloading Resolution}
\seclabel{resolving-overloading}

\note{
Victor: Does this paragraph, other than the last sentence,
really belong in this section?}

To evaluate a given functional call,
it is necessary to determine which functional declaration
to dispatch to.
To do so,
we consider the declarations that are applicable to that call at run time.
If there is exactly one such declaration,
then the call dispatches to that declaration.
If there is no such declaration,
then the call is \emph{undefined}, which is a static error.
(However, see \secref{applicability-with-coercion}
for how coercion may add to the set of applicable declarations.)
If multiple declarations are applicable to the call at run time,
then we choose an arbitrary declaration among the declarations such that
no other applicable declaration is more specific than them.



We use the subtype relation to compare parameter types to determine
a more specific declaration.
Formally, a declaration $\f(\Ps)$ is \emph{more specific} than a declaration
$\f(\Qs)$ if $\Ps \strictsubtype \Qs$.  Similarly, a declaration
$\Ps_0.\f(\Ps)$ is more specific than a declaration $\Qs_0.\f(\Qs)$
if $\Ps_0 \strictsubtype \Qs_0$ and $\Ps \strictsubtype \Qs$.
(See \secref{resolving-coercion} for how coercion
changes the definition of ``more specific''.)
Restrictions on the definition of overloaded functionals
(see \chapref{overloaded-declarations})
guarantee that among all applicable declarations,
one is more specific than all the others.
If the declarations include static parameters,
they are inferred as described in \chapref{type-inference}
before comparing their parameter types
to determine which declaration is more specific.
