%%%%%%%%%%%%%%%%%%%%%%%%%%%%%%%%%%%%%%%%%%%%%%%%%%%%%%%%%%%%%%%%%%%%%%%%%%%%%%%%
%   Copyright 2009, Oracle and/or its affiliates.
%   All rights reserved.
%
%
%   Use is subject to license terms.
%
%   This distribution may include materials developed by third parties.
%
%%%%%%%%%%%%%%%%%%%%%%%%%%%%%%%%%%%%%%%%%%%%%%%%%%%%%%%%%%%%%%%%%%%%%%%%%%%%%%%%

\section{Identifier References}
\seclabel{idn-ref}

\begin{Grammar}
\emph{Primary}
&::=& \emph{VarOrFnRef}\\
&$|$& \KWD{self} \\

\emph{VarOrFnRef} &::=& \emph{Id} \option{\emph{StaticArgs}}\\

\end{Grammar}

A name that is not an operator name appearing in an expression context is called
  an \emph{identifier reference}.
It evaluates to the value of the name in the enclosing scope in the value
  namespace.  The type of an identifier reference is the declared type of
  the name.
See \chapref{names} for a discussion of names.
An identifier reference performs a memory read operation.
\marginnote{Is this true even for immutable variables?}
If a name is not in scope,
it is a static error (as described in \secref{shadowing}).

An identifier reference which denotes a polymorphic function
may include explicit static arguments
(described in \chapref{functions})
but most identifier references do not include them.; the static arguments
are statically inferred from the context of the function call
(as described in \chapref{type-inference}).
For example,
%double[\String\]
\EXP{\VAR{double}\llbracket\TYP{String}\rrbracket} is an identifier
reference with an explicit static argument where the function
\VAR{double} is defined as follows:
\input{\home/basic/examples/Expr.VarRef.tex}

The special name \KWD{self} is declared as a parameter of a method.
When a dotted method is invoked, its receiver is bound to
the \KWD{self} parameter; the value of \KWD{self} is the receiver.
When a functional method is invoked, the corresponding argument is bound to
the \KWD{self} parameter; the value of \KWD{self} is the argument passed to it.
The type of \KWD{self} is the type of the trait or object being declared by
the innermost enclosing trait or object declaration or object expression.
\marginnote{I thought that for object expressions, the type of \EXP{\mathord{\KWD{self}}}
was the intersection of the traits it lists in its \KWD{extends} clause.
In particular, if it includes a ``method declaration'' with a new name \VAR{m}
(not one inherited from one of its supertypes), we cannot invoke \EXP{\mathord{\KWD{self}}.m},
because \VAR{m} isn't really a new method--just a local function declaration.
(There are other distinctions between the type of an object expression and
the intersection of the traits it extends, so we should think about this issue more carefully.)
}
% The new self-type idiom
If the innermost enclosing such construct is a trait declaration (for a trait called,
say, `\EXP{T}'), and that trait declaration has a \KWD{comprises} clause,
and `\EXP{U_1}', `\EXP{U_2}', ..., `\EXP{U_n}' are the traits mentioned in
the \KWD{comprises} clause, then the type of \KWD{self} is instead
``\EXP{T \cap (U_{\mathrm{1}} \cup U_{\mathrm{2}}
\cup \ldots \cup U_{\mathrm{n}})}''.
%
See \secref{methods} for details about \KWD{self} parameters.
