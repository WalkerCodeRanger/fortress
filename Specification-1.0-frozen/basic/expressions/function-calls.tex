%%%%%%%%%%%%%%%%%%%%%%%%%%%%%%%%%%%%%%%%%%%%%%%%%%%%%%%%%%%%%%%%%%%%%%%%%%%%%%%%
%   Copyright 2009, Oracle and/or its affiliates.
%   All rights reserved.
%
%
%   Use is subject to license terms.
%
%   This distribution may include materials developed by third parties.
%
%%%%%%%%%%%%%%%%%%%%%%%%%%%%%%%%%%%%%%%%%%%%%%%%%%%%%%%%%%%%%%%%%%%%%%%%%%%%%%%%

\section{Naked Method Invocations}
\seclabel{naked-method-calls}

\note{Should be a static check that there is an applicable declaration.}

\begin{Grammar}
\emph{Primary} &::=& \emph{Id} \emph{Primary}\\

\end{Grammar}

Method invocations that are not prefixed by receivers are \emph{naked
  method invocations}.  A naked method invocation is either a
functional method call (see \secref{methods} for a discussion of
functional methods) or a method invocation within a trait or object
that provides the method declaration.  Syntactically, a naked method
invocation is same as a function call except that the method name is
used instead of an arbitrary expression denoting the applied method.
Like function calls, an argument expression need not be parenthesized
unless it is a tuple.  After the evaluation of the argument expression
completes normally, the body of the method is evaluated with the parameter
of the method bound to the value of the argument expression.  The value
and the type of a naked method invocation are the value and the type
of the method body.


\section{Function Calls}
\seclabel{function-calls}

\note{Should be a static check that there is an applicable declaration.}

\begin{Grammar}
\emph{Primary} &::=& \emph{Primary} \emph{Primary}\\
\end{Grammar}


A \emph{function call} consists of two subexpressions: an expression
denoting the applied function and an argument expression.  The
argument expression and the expression denoting the applied function
are evaluated \emph{in parallel} in separate implicit threads (see
\secref{threads-parallelism}).  As with languages such as Scheme and
the Java Programming Language, function calls in Fortress are
call-by-value.  After the evaluation of the function and its arguments
completes normally,
the body of the function is evaluated with the parameter of the
function bound to the value of the argument expression.
The value and the type of a function call are the value and the type of
the function body.
See \secref{function-app} for a detailed description of function calls.


Here are some examples:
\input{\home/basic/examples/Expr.FnCall.a.tex}
If the function's argument is not a tuple, then the argument need not be
parenthesized:
\input{\home/basic/examples/Expr.FnCall.b.tex}
