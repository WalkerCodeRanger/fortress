%%%%%%%%%%%%%%%%%%%%%%%%%%%%%%%%%%%%%%%%%%%%%%%%%%%%%%%%%%%%%%%%%%%%%%%%%%%%%%%%
%   Copyright 2009, Oracle and/or its affiliates.
%   All rights reserved.
%
%
%   Use is subject to license terms.
%
%   This distribution may include materials developed by third parties.
%
%%%%%%%%%%%%%%%%%%%%%%%%%%%%%%%%%%%%%%%%%%%%%%%%%%%%%%%%%%%%%%%%%%%%%%%%%%%%%%%%

Fortress is an expression-oriented language.
The positions in which an expression may legally appear
(\emph{value context})
are determined by the nonterminal \emph{Expr}
in the Fortress grammar defined in \appref{concrete-syntax}.
We say that an expression is a \emph{subexpression}
of any expression (or any other program construct)
that (syntactically) contains it.
When evaluation of one subexpression
must complete before another subexpression is evaluated,
those subexpressions are ordered by dynamic program order
(see \chapref{evaluation}).
This constrains the memory behavior of program constructs,
as described in \chapref{memory-model}.
Unless otherwise specified,
abrupt completion of the evaluation of a subexpression
causes the evaluation of the expression as a whole
to complete abruptly in the same way.
Also, if one expression precedes another by dynamic program order,
and the evaluation of the first expression completes abruptly,
the second is not evaluated at all.
