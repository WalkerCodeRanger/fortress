%%%%%%%%%%%%%%%%%%%%%%%%%%%%%%%%%%%%%%%%%%%%%%%%%%%%%%%%%%%%%%%%%%%%%%%%%%%%%%%%
%   Copyright 2009,2010, Oracle and/or its affiliates.
%   All rights reserved.
%
%
%   Use is subject to license terms.
%
%   This distribution may include materials developed by third parties.
%
%%%%%%%%%%%%%%%%%%%%%%%%%%%%%%%%%%%%%%%%%%%%%%%%%%%%%%%%%%%%%%%%%%%%%%%%%%%%%%%%

\section{Aggregate Expressions}
\seclabel{aggregate-expr}

\note{Some operators and matrix unpasting are not yet supported.}

\note{\EXP{\mapsto} is a special syntax only in two contexts:
map aggregates and map comprehensions. (02/12/09)
}

\begin{Grammar}
\emph{Primary}
&::=& \emph{LeftEncloser} \option{\emph{StaticArgs}} \option{\emph{ExprList}} \emph{RightEncloser} \\

\emph{ExprList} &::=& \emph{Expr}(\EXP{,} \emph{Expr})$^*$ \\

\emph{ArrayExpr} &::=& \texttt{[} \option{\emph{StaticArgs}} \emph{RectElements} \texttt{]}\\

\emph{RectElements} &::=& \emph{Expr} \emph{MultiDimCons}$^*$\\

\emph{MultiDimCons} &::=& \emph{RectSeparator} \emph{Expr}\\

\emph{RectSeparator} &::=& \EXP{;}+\\
&$|$& \emph{Whitespace}\\

\emph{MapExpr} &::=& \{ \option{\emph{StaticArgs}} \option{\emph{EntryList}} \}\\
\end{Grammar}

\emph{Aggregate expressions} evaluate to values that are themselves
homogeneous collections of values.  Each subexpression of an aggregate
expression is evaluated in parallel in a separate implicit thread (see
\secref{threads-parallelism}).
With the exception of array expressions, the resulting
values are passed to the appropriate bracketing operator, which is a
function with a varargs parameter constructing the aggregate.
The array aggregate operations
construct the aggregate directly, after all
subexpressions have completed normally.  Any aggregate expression may
reserve storage for its result before its elements complete
evaluation.

Functions defining aggregate expressions are provided in \library\
for sets, maps, lists, matrices, vectors, and arrays.

\subsection{Set Expressions}
\seclabel{set-expr}

Set element expressions are enclosed in braces and separated by commas.
The type of a set expression is
\EXP{\TYP{Set}\llbracket T\rrbracket},
where \VAR{T} is the union type
of the types of all element expressions of the set expression.


Set containment is checked with the operator $\in$
and the subset relationship is checked with the operator \EXP{\subseteq}.
For example:
\input{\home/basic/examples/Expr.Set.tex}
evaluates to \VAR{true}.
and
\marginnote{This example should be added to the SpecData directory when it works.}
\begin{center}
\EXP{\{0, 1, 2\} \subseteq \{0, 3, 2\}}
\end{center}
evaluates to \VAR{false}.


\subsection{Map Expressions}

Map entries are enclosed in curly braces, separated by commas, and
matching pairs are separated by $\mapsto$.
The type of a map expression is
\EXP{\TYP{Map}\llbracket{}K,V\rrbracket} where \VAR{K}
is the union of the types of all left-hand-side expressions of the map
entries, and \VAR{V} is the union of the types of all right-hand-side
expressions of the map entries.

A map \VAR{m} is indexed by placing an element in the domain of \VAR{m}
enclosed in brackets immediately after an expression evaluating to \VAR{m}.
\marginnote{The new fortify fixes this.}
Thus, the index is rendered as a subscript.
For example, if:
\input{\home/basic/examples/Expr.Map.tex}
then \EXP{m_{\hbox{\rm``\STR{b}''}}} evaluates to \EXP{1}.
In contrast, \EXP{m_{\hbox{\rm``\STR{x}''}}} throws a \TYP{NotFound} exception,
as \EXP{\hbox{\rm``\STR{x}''}} is not an index of \VAR{m}.

\subsection{List Expressions}
\seclabel{list-expr}

List element expressions are enclosed in
angle brackets $\llbracket$ and $\rrbracket$ and are separated by commas.
The type of a list expression is \EXP{\TYP{List}\llbracket{}T\rrbracket}
 where \VAR{T}
is the union type of the types of all element expressions.

A list \VAR{l} is indexed by placing an index enclosed in square brackets
immediately after an expression evaluating to \VAR{l}.
\marginnote{The new fortify fixes this.}
Thus, the index is rendered as a subscript.
Lists are always indexed from \EXP{0}.
For example:
\input{\home/basic/examples/Expr.List.tex}
evaluates to \EXP{1}.


\subsection{Array Expressions}
\seclabel{array-expr}

Array element expressions are enclosed in brackets. Element expressions
along a row are separated only by whitespace.
Two dimensional array expressions are written by separating rows with
newlines or semicolons.  If a semicolon appears, whitespace before
and after the semicolon is ignored.
The parts of higher-dimensional array expressions are separated
by repeated-semicolons, where the dimensionality
of the result is equal to one plus the number
of repeated semicolons.
The type of a $k$-dimensional array expression is
\EXP{\TYP{Array}k\llbracket{}T, 0, n_{\mathrm{0}}, \cdots,
         0, n_{k-1}\rrbracket},
where \VAR{T} is the union type of the types of the element expressions and
$n_0, ..., n_{k-1}$
are the sizes of the array in each dimension\footnote{
As of March 2008, the Fortress library provides \EXP{\TYP{Array}k} types for
$1 \le k \le 3$.}. This type can be abbreviated as
\EXP{T [n_{\mathrm{0}}, \cdots, n_{k-1}]}.




A $k$-dimensional array \VAR{A} is indexed by placing a sequence of
$k$ indices enclosed in brackets, and separated by commas, after an expression
evaluating to \VAR{A}.
Thus, the index is rendered as a subscript.
By default, arrays are indexed from 0.  The
horizontal dimension of an array is the last dimension mentioned in the
array index.  For example:
\input{\home/basic/examples/Expr.Array.a.tex}
then \EXP{A_{1,0}} evaluates to \EXP{4}.

An array of two
\marginnote{two or more -- Jan: I continue to disbelieve.}
dimensions whose elements are a subtype of \TYP{Number}
is a matrix.
\marginnote{
can be coerced to a matrix.
Matrix types are written
\EXP{\TYP{Matrix}\llbracket{}T\rrbracket [n_{\mathrm{0}} \times \ldots \times n_{k-1}]},
where $k > 1$.
A type of this form can be abbreviated as
\EXP{T^{n_0 \times \ldots \times n_{k-1}}}.}
Matrices are indexed in the same manner as arrays.

A one-dimensional array whose elements are a subtype of \TYP{Number}
is a vector.
\marginnote{
can be coerced to a vector.
Vector types are written \EXP{\TYP{Vector}\llbracket{}T\rrbracket [n]}.
A type of this form can be abbreviated as
\EXP{T^{n}}, unless \VAR{T} is declared in the enclosing scope
to be a physical dimension or unit.

Matrices occurring inside a matrix literal on the same row are
pasted along dimension 0. Each row is pasted along dimension 1 and the
resulting matrix is ``flattened''.  This works because matrices never
contain other matrices as elements.
}

The element expressions in an array expression may be either scalars or
array expressions themselves.  If an element is an array expression,
it is ``flattened'' (pasted) into the enclosing expression.
This pasting works because array expressions never
contain other arrays as elements.  The elements along
a row (or column) must have the same number of columns (or rows),
though two elements in different rows (columns) need not have the
same number of columns (rows).
See \secref{unpasting} for a discussion of matrix unpasting.



The following four examples are all equivalent:
\input{\home/basic/examples/Expr.Array.b.tex}
\input{\home/basic/examples/Expr.Array.c.tex}
\input{\home/basic/examples/Expr.Array.d.tex}
\input{\home/basic/examples/Expr.Array.e.tex}
%Here is a $3\times3\times3\times2$ matrix example:
Here is a $3\times3\times3$ matrix example:
\input{\home/basic/examples/Expr.Array.f.tex}
