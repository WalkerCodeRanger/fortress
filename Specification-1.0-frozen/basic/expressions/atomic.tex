%%%%%%%%%%%%%%%%%%%%%%%%%%%%%%%%%%%%%%%%%%%%%%%%%%%%%%%%%%%%%%%%%%%%%%%%%%%%%%%%
%   Copyright 2009, Oracle and/or its affiliates.
%   All rights reserved.
%
%
%   Use is subject to license terms.
%
%   This distribution may include materials developed by third parties.
%
%%%%%%%%%%%%%%%%%%%%%%%%%%%%%%%%%%%%%%%%%%%%%%%%%%%%%%%%%%%%%%%%%%%%%%%%%%%%%%%%

\section{Atomic Expressions}
\seclabel{atomic}

\note{Modifiers on functionals such as \KWD{atomic} and \KWD{io} are not yet supported.}

\begin{Grammar}
\emph{FlowExpr} &::=&  \KWD{atomic} \emph{AtomicBack}\\
&$|$& \KWD{tryatomic} \emph{AtomicBack} \\

\emph{AtomicBack}
&::=& \emph{AssignExpr}\\
&$|$& \emph{OpExpr}\\
&$|$& \emph{DelimitedExpr}\\

\end{Grammar}

As Fortress is a parallel language, an executing Fortress program consists
of a set of threads (See \secref{threads-parallelism} for a discussion of
parallelism in Fortress).  In multithreaded programs,
it is often convenient for a thread to evaluate some expressions
\emph{atomically}.  For this purpose, Fortress provides \KWD{atomic}
expressions.


An \KWD{atomic} expression consists of \KWD{atomic}
followed by a \emph{body expression}.  Evaluating an \KWD{atomic}
expression is simply evaluating the body expression.  All reads and
all writes which occur as part of this evaluation will appear to occur
simultaneously in a single atomic step with respect to \emph{any}
action performed by any thread which is dynamically outside.  This is
specified in detail in \chapref{memory-model}.  The value and type of an
\KWD{atomic} expression are the value and type of its body expression.

A \KWD{tryatomic} expression consists of \KWD{tryatomic} followed by
an expression.  It acts exactly like \KWD{atomic} except that in
certain circumstances (see \secref{transactions}) it throws
\TYP{TryAtomicFailure} and discards the effects of its body.

A function or method with the modifier \KWD{atomic} acts as if its entire
body were surrounded in an \KWD{atomic} expression.
However, it is a static error if an API declares a functional \VAR{f}
with the modifier \KWD{atomic} but a component implementing the API
defines \VAR{f} whose body is an \KWD{atomic} expression without the
modifier.  Functionals with the modifier \KWD{io} cannot be called
within an \KWD{atomic} expression.  Thus, a
functional must not have both \KWD{atomic} and \KWD{io} modifiers.

When the body of an \KWD{atomic} expression completes abruptly, the
\KWD{atomic} expression completes abruptly in the same way.  If it
completes abruptly by exiting to an enclosing \KWD{label} expression,
writes within the block are retained and become visible to other
threads.  If it completes abruptly by throwing an uncaught exception,
all writes to objects allocated before the \KWD{atomic}
expression began evaluation are discarded.  Writes to newly
allocated objects are retained.  Any variable reverts to the value it
held before evaluation of the \KWD{atomic} expression began.  Thus, the
only values retained from the abruptly completed \KWD{atomic} expression
will be reachable from the exception object through a chain of newly
allocated objects.

Atomic expressions may be nested arbitrarily; the above semantics
imply that an inner \KWD{atomic} expression is atomic with respect to
evaluations which occur dynamically outside the inner \KWD{atomic}
expression but dynamically inside an enclosing \KWD{atomic}.

Implicit threads may be created dynamically within an \KWD{atomic}
expression.  These implicit threads will complete before the
\KWD{atomic} expression itself does so.  The implicit threads may
run in parallel, and will see one another's writes; they may
synchronize with one another using nested \KWD{atomic} expressions.

Note that \KWD{atomic} expressions may be evaluated in parallel with
other expressions.  An \KWD{atomic} expression experiences
\emph{conflict} when another thread attempts to read or write a memory
location which is accessed by the \KWD{atomic} expression.  The
evaluation of such an expression must be partially serialized with the
conflicting memory operation (which might be another \KWD{atomic}
expression).  The exact mechanism by which this occurs will vary; the
necessary serialization is provided by the implementation.  In
general, the evaluation of a conflicting \KWD{atomic} expression may
be abandoned, forcing the effects of execution to be discarded and
execution to be retried.  The longer an \KWD{atomic} expression
evaluates and the more memory it touches the greater the chance of
conflict and the larger the bottleneck a conflict may impose.

For example,
the following code uses a shared counter atomically:
\input{\home/basic/examples/Expr.Atomic.tex}
The loop body reads \EXP{a_i} and \VAR{sum}, then adds them and
writes the result back to \VAR{sum}; this will appear to occur
atomically with respect to all other threads---including both other
iterations of the loop body and other simultaneous calls to
\VAR{accumArray}.  Note in particular that the \KWD{atomic} expression
will appear atomic with respect to reads and writes of \EXP{a_i} and \VAR{sum} that do  not occur in
\KWD{atomic} expressions.
