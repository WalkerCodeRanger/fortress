%%%%%%%%%%%%%%%%%%%%%%%%%%%%%%%%%%%%%%%%%%%%%%%%%%%%%%%%%%%%%%%%%%%%%%%%%%%%%%%%
%   Copyright 2009, Oracle and/or its affiliates.
%   All rights reserved.
%
%
%   Use is subject to license terms.
%
%   This distribution may include materials developed by third parties.
%
%%%%%%%%%%%%%%%%%%%%%%%%%%%%%%%%%%%%%%%%%%%%%%%%%%%%%%%%%%%%%%%%%%%%%%%%%%%%%%%%

\section{Spawn Expressions}
\seclabel{spawn}

\note{Static checking for \KWD{io} actions are not yet supported.}

\begin{Grammar}
\emph{FlowExpr} &::=& \KWD{spawn} \emph{Expr} \\
\end{Grammar}

A spawned thread is created using a \KWD{spawn} expression.  A
\KWD{spawn} expression consists of \KWD{spawn} followed by an
expression.  A \KWD{spawn} expression spawns a thread which evaluates
its subexpression in parallel with any succeeding evaluation.  The
value of a \KWD{spawn} expression is the spawned thread
and the type of the expression is \EXP{\TYP{Thread}\llbracket{}T\rrbracket},
where \VAR{T} is the static type of the expression spawned.
A \KWD{spawn} expression constitutes an \KWD{io} action, and thus cannot be run
within the body of an \KWD{atomic} expression.
A \KWD{spawn} expression cannot be run
within the body of an \KWD{atomic} expression.
The semantics of spawned threads are discussed in \secref{threads-parallelism}.
