%%%%%%%%%%%%%%%%%%%%%%%%%%%%%%%%%%%%%%%%%%%%%%%%%%%%%%%%%%%%%%%%%%%%%%%%%%%%%%%%
%   Copyright 2009, Oracle and/or its affiliates.
%   All rights reserved.
%
%
%   Use is subject to license terms.
%
%   This distribution may include materials developed by third parties.
%
%%%%%%%%%%%%%%%%%%%%%%%%%%%%%%%%%%%%%%%%%%%%%%%%%%%%%%%%%%%%%%%%%%%%%%%%%%%%%%%%

\section{Operator Applications}
\seclabel{operator-app-expr}

\note{Examples/discussion of juxtaposition as an operator.

Should be a static check that an applicable declaration exists.
}

\begin{Grammar}
\emph{OpExpr}
&::=& \emph{EncloserOp} \option{\emph{OpExpr}} \option{\emph{EncloserOp}} \\
&$|$& \emph{OpExpr} \emph{EncloserOp} \option{\emph{OpExpr}} \\
&$|$& \emph{Primary}\\

\emph{EncloserOp} &::=& \emph{Encloser}\\
&$|$& \emph{Op}\\

\emph{Primary}
&::=& \emph{LeftEncloser} \option{\emph{StaticArgs}} \option{\emph{ExprList}} \emph{RightEncloser} \\
&$|$& \emph{Primary} \verb+^+ \emph{Exponent}\\
&$|$& \emph{Primary} \emph{ExponentOp}\\

\emph{Exponent}
&::=& \emph{Id}\\
&$|$& \emph{ParenthesisDelimited}\\
&$|$& \emph{LiteralExpr}\\
&$|$& \KWD{self}\\

\emph{ExponentOp} &::=& \verb+^T+ $|$ \verb+^+(\emph{Encloser} $|$ \emph{Op})\\

\end{Grammar}

To support a rich mathematical notation, Fortress allows most Unicode
characters that are specified to be mathematical operators to be used as
operators in Fortress expressions, as well as various tokens described in
\chapref{operators}.
Most of the operators can be used as prefix, infix, postfix, or
nofix operators as described in \secref{operator-fixity};
the fixity of an operator is determined syntactically, and
the same operator may have definitions for multiple fixities.

Syntactically, an operator application consists of an operator and its
argument expressions.  If the operator is a prefix operator, it is followed
by its argument expression.  If the operator is an infix operator, its two
argument expressions come both sides of the operator.  If the operator is a
postfix operator, it comes right after its argument expression.
Like function calls,
argument expressions are evaluated \emph{in parallel} in separate implicit threads.
After evaluation of arguments completes normally,
the body of the operator definition is evaluated
with the parameters of the operator bound to the values of the argument
expressions.  The value and the type of an operator application are the
value and the type of the operator body.
See \chapref{operators} for a detailed description of operators.


Here are some examples:
\input{\home/basic/examples/Expr.OprApp.tex}


\note{The following examples are not yet working:}
% a[k] b[n-k]
% 17.3 meter/second
% 17.3 m/s
% u DOT (v CROSS w)
% (A CUP B) INTERSECT C
% (A CUP B) CAP C
% i < j <= k AND p PREC q
% <|2, 3, 4, 5|>
\begin{Fortress}
{\tt~}\pushtabs\=\+\( a_k\:b_{n-k}\)\\
\( 17.3\,\TYP{meter}/\TYP{second}\)\\
\( 17.3 m/s\)\\
\( u \cdot (v \times w)\)\\
\( (A \cup B) \OPR{INTERSECT}\:C\)\\
\( (A \cup B) \cap C\)\\
\( i < j \leq k \wedge p \prec q\)\\
\( \langle{}2, 3, 4, 5\rangle\)\-\\\poptabs
\end{Fortress}
