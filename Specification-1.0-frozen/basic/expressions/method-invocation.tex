%%%%%%%%%%%%%%%%%%%%%%%%%%%%%%%%%%%%%%%%%%%%%%%%%%%%%%%%%%%%%%%%%%%%%%%%%%%%%%%%
%   Copyright 2009, Oracle and/or its affiliates.
%   All rights reserved.
%
%
%   Use is subject to license terms.
%
%   This distribution may include materials developed by third parties.
%
%%%%%%%%%%%%%%%%%%%%%%%%%%%%%%%%%%%%%%%%%%%%%%%%%%%%%%%%%%%%%%%%%%%%%%%%%%%%%%%%

\section{Dotted Method Invocations}
\seclabel{dotted-method-calls}

\note{Should be a static check that there is an applicable declaration.}

\begin{Grammar}
\emph{Primary}
&::=& \emph{Primary}\EXP{.}\emph{Id}
\option{\emph{StaticArgs}} \emph{ParenthesisDelimited}\\

\emph{ParenthesisDelimited}
&::=& \emph{Parenthesized}\\
&$|$& \emph{ArgExpr}\\
&$|$& \texttt(\texttt)\\

\emph{Parenthesized} &::=& \texttt( \emph{Expr} \texttt)\\

\emph{ArgExpr}
&::=& \emph{TupleExpr}\\
&$|$& \texttt( (\emph{Expr}\EXP{,})$^*$ \emph{Expr}\EXP{...} \texttt)\\

\emph{TupleExpr}
&::=& \texttt( (\emph{Expr}\EXP{,})$^+$ \emph{Expr} \texttt)\\

\end{Grammar}

A \emph{dotted method invocation} consists of a subexpression
(called the receiver expression), followed by `\txt{.}',
followed by an identifier, an optional list of static arguments
(described in \chapref{functions})
and a subexpression (called the \emph{argument expression}).
Unlike in function calls (described in \secref{function-calls}),
the argument expression must be parenthesized, even if it is not a tuple.
There must be no whitespace on the left-hand side of the `\txt{.}' and
the left-hand side of the left parenthesis of the argument expression.
The receiver expression evaluates to the receiver of the invocation (bound
to the \KWD{self} parameter (discussed in \secref{methods}) of the method).
A method invocation may include explicit instantiations of static parameters
but most method invocations do not include them; the static arguments
are statically inferred from the context of the method invocation (as
described in \chapref{type-inference}).

The receiver and arguments of a method invocation are each evaluated
in parallel in a separate implicit thread (see
\secref{threads-parallelism}).  After this thread group completes
normally, the body of the method is evaluated with the parameter of
the method bound to the value of the argument expression (thus
evaluation of the body occurs after evaluation of the receiver and
arguments in dynamic program order).  The value and the type of a
dotted method invocation are the value and the type of the method
body.

We say that methods or functions (collectively called as
\emph{functionals}) may be \emph{applied to}
(also ``\emph{invoked on}'' or ``\emph{called with}'') an argument.
We use ``call'', ``invocation'', and ``application'' interchangeably.

Here are some examples:
\input{\home/basic/examples/Expr.MthIvk.tex}
