%%%%%%%%%%%%%%%%%%%%%%%%%%%%%%%%%%%%%%%%%%%%%%%%%%%%%%%%%%%%%%%%%%%%%%%%%%%%%%%%
%   Copyright 2009, Oracle and/or its affiliates.
%   All rights reserved.
%
%
%   Use is subject to license terms.
%
%   This distribution may include materials developed by third parties.
%
%%%%%%%%%%%%%%%%%%%%%%%%%%%%%%%%%%%%%%%%%%%%%%%%%%%%%%%%%%%%%%%%%%%%%%%%%%%%%%%%

\section{Operator Names}
\seclabel{operator-names}

\note{Victor: This should refer to the appropriate sections of Lexical Structure
Operator names are either operator tokens or special operator characters,
or a few reserved words.}

To support a rich mathematical notation, Fortress allows most Unicode
characters
that are specified to be mathematical operators to be used as operators in
Fortress expressions, as well as these characters:
\begin{ttt}
!~~~{\char'100}~~~{\char'43}~~~{\char'44}~~~{\char'45}~~~*~~~+~~~-~~~=~~~{\char'174}~~~:~~~<~~~>~~~/~~~?~~~{\char'136}~~~{\verb+~+} \\
->~~~-->~~~=>~~~==>~~~<=~~~>=~~~=/=~~~!!~~~{\char'174}{\char'174}~~~{\char'174}{\char'174}{\char'174}\\
<<~~~<<<~~~>>~~~>>>~~~<->~~~<-/-~~~-/->~~~<=>~~~===
\end{ttt}
In addition, a token that is made up of a mixture of uppercase letters and underscores
(but no digits), does not begin or end with an underscore, and contains at least two
different letters is also considered to be an operator:
\begin{ttt}
MAX~~~MIN
\end{ttt}
The above operators are rendered as:
\EXP{\OPR{MAX}~~~\OPR{MIN}}.
(See \secref{lexical-operators} and
\appref{operator-precedence} for detailed descriptions of operator
names in Fortress.)
