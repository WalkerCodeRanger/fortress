%%%%%%%%%%%%%%%%%%%%%%%%%%%%%%%%%%%%%%%%%%%%%%%%%%%%%%%%%%%%%%%%%%%%%%%%%%%%%%%%
%   Copyright 2009, Oracle and/or its affiliates.
%   All rights reserved.
%
%
%   Use is subject to license terms.
%
%   This distribution may include materials developed by third parties.
%
%%%%%%%%%%%%%%%%%%%%%%%%%%%%%%%%%%%%%%%%%%%%%%%%%%%%%%%%%%%%%%%%%%%%%%%%%%%%%%%%

\section{Environments}
\seclabel{environments}

An \emph{environment} maps \emph{names} to values or locations.
Environments are immutable,
and two environments that map exactly the same names
to the same values or locations are identical.


A program starts executing with an empty environment.
Environments are extended with new mappings
by variable, function, and object declarations, and functional calls
(including calls to object constructors).
After initializing all top-level variables and singleton objects,
as described in \secref{initialization},
the \emph{top-level environment} for each component is constructed.

The environment of a value is determined by how it is
\emph{constructed}.
For all but object expressions (described in \secref{object-expr}),
function expressions (described in \secref{func-expr})
and local function declarations (described in \secref{local-fn-decls}),
the environment of the constructed value is the top-level environment of the component in which the expression or declaration occurs.
For object and function expressions and local function declarations,
the environment of the constructed value
is the lexical environment in which the expression or declaration was evaluated.

We carefully distinguish a spawned thread from its associated spawned
thread object.  In particular, note that the execution environment of
a spawned thread, in which the body expression is evaluated, is
distinct from the environment of the associated thread object, in
which calls to the thread methods are evaluated.
