%%%%%%%%%%%%%%%%%%%%%%%%%%%%%%%%%%%%%%%%%%%%%%%%%%%%%%%%%%%%%%%%%%%%%%%%%%%%%%%%
%   Copyright 2009, Oracle and/or its affiliates.
%   All rights reserved.
%
%
%   Use is subject to license terms.
%
%   This distribution may include materials developed by third parties.
%
%%%%%%%%%%%%%%%%%%%%%%%%%%%%%%%%%%%%%%%%%%%%%%%%%%%%%%%%%%%%%%%%%%%%%%%%%%%%%%%%

\section{Input and Output Actions}
\seclabel{io-actions}

\note{Check for I/O actions is not yet supported.}

Certain functionals in Fortress perform primitive input/output (I/O)
actions.  These actions have an externally visible effect.
Any functional which may perform an I/O action---either
because it is a primitive action or because it invokes other
functionals which perform I/O actions---must be declared
with the \KWD{io} modifier.

Any primitive I/O action may take many internal steps; each step may
read or write any memory locations referred to either directly or
transitively by object references passed as arguments to the action.
Each I/O action is free to complete either normally or abruptly.
I/O actions may block and be prevented from taking a step until any
necessary external conditions are fulfilled (input is available, data
has been written to disk, and so forth).

Each I/O action taken by an expression is considered part of that
expression's effects.  The steps taken by an I/O action are considered
part of the context in which an expression evaluates, in much the same
way as effects of simultaneously-executing threads must be considered
when describing the behavior of an expression.  For example, we cannot
consider two functionals to be equivalent unless the possible I/O
actions they take are the same, given identical internal steps by each
I/O action.
