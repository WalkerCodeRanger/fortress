%%%%%%%%%%%%%%%%%%%%%%%%%%%%%%%%%%%%%%%%%%%%%%%%%%%%%%%%%%%%%%%%%%%%%%%%%%%%%%%%
%   Copyright 2009, Oracle and/or its affiliates.
%   All rights reserved.
%
%
%   Use is subject to license terms.
%
%   This distribution may include materials developed by third parties.
%
%%%%%%%%%%%%%%%%%%%%%%%%%%%%%%%%%%%%%%%%%%%%%%%%%%%%%%%%%%%%%%%%%%%%%%%%%%%%%%%%

\section{Proof of Overloading Resolution for Functions}
\seclabel{proof-overloading-resolution}

This section proves that the overloading rules for
function declarations are sufficient to guarantee no undefined nor
ambiguous call at run time (the case for methods is analogous).

Consider a static function call $\f(\As)$ at some program point $Z$
and its corresponding dynamic function call $\f(\Xs)$.  Let $\Delta$
be the set of parameter types of function declarations of $\f$ that
are visible at $Z$ and applicable to the dynamic call $\f(\Xs)$.  Let
$\Sigma$ be the set of parameter types of function declarations of
$\f$ that are visible at $Z$ and applicable to the static call
$\f(\As)$.  Moreover, let $\sigma$ be the subset of $\Sigma$ for which
no type in $\Sigma$ is more specific and let $\delta$ be the
subset of $\Delta$ for which no type in $\Delta$ is more
specific:
\[
\begin{array}{lclcll}
\sigma & = & \set{S \in \Sigma & | & \lnot \exists S'\in\Sigma: S' \strictsubtype S & }
\\
\delta & = & \set{D \in \Delta & | & \lnot \exists D'\in\Delta: D' \strictsubtype D & }.
\end{array}
\]

Below we prove:
\[
\begin{array}{l}
|\Sigma| \neq 0 \ \text{implies} \ |\sigma| = 1, and
\\
|\Sigma| \neq 0 \ \text{implies} \ |\delta| = 1.
\end{array}
\]
Informally, if any declaration is applicable to a static call then there
exists a single most specific declaration that is applicable to the
static call and a single most specific declaration that is applicable
to the corresponding dynamic call.
\\
\begin{lemma}
\label{lem:subset}
$\Sigma \subseteq \Delta$.
\end{lemma}

\begin{proof}
Notice that $\Xs \Ovrsubtype \As$ by type soundness.
If $\f(\Ps)$ is applicable to
the call $\f(\As)$ then $\As \Ovrsubtype \Ps$.  Notice that $\Xs
\Ovrsubtype \As$ implies $\Xs \Ovrsubtype \Ps$.  Therefore
$\f(\Ps)$ is applicable to the call $\f(\Xs)$.
\end{proof}

\begin{lemma}
\label{lem:dgesge-second}
If $|\Delta| \ge 1$ then $|\delta| \ge 1$.
Also, if $|\Sigma| \ge 1$ then $|\sigma| \ge 1$.
\end{lemma}

\begin{proof}
Follows from Lemma \ref{lem:nonempty} where $S$ is the set of all
types, $A$ is $\Delta$ and $\Sigma$ respectively,
and the relation $\strictsubtype$ is acyclic and irreflexive.
\end{proof}

\begin{lemma}
\label{lem:ge}
If $|\Sigma| \ge 1$ then $|\delta| \ge 1$.
\end{lemma}

\begin{proof}
Follows from Lemmas \ref{lem:subset} and \ref{lem:dgesge-second}.
\end{proof}

\begin{lemma}
\label{lem:le}
$|\sigma| \le 1$.  Also, $|\delta| \le 1$.
\end{lemma}

\begin{proof}
We prove this for $\delta$, but the case for $\sigma$ is identical.
For the purpose of contradiction suppose there are two declarations
$\f(\Ps)$ and $\f(\Qs)$ in $\delta$.  Since both $\f(\Ps)$ and
$\f(\Qs)$ are applicable without coercion to the call $\f(\Xs)$ we
have $\Xs \Ovrsubtype \Ps \inter \Qs$.  Therefore it is not the case
that $\Ps \incompatible \Qs$.  By the overloading rules, $\Ps
\neq \Qs$ and either $\Ps \excludes \Qs$ or there is a declaration
$\f(\Ps \inter \Qs)$ visible at $Z$.  Since it cannot be the case that
$\Ps \excludes \Qs$ there must exist a declaration $\f(\Ps \inter
\Qs)$ visible at $Z$.  Since $\Ps \inter \Qs \Ovrsubtype \Ps$ and $\Ps
\inter \Qs \Ovrsubtype \Qs$ we know $\f(\Ps \inter \Qs)$ is applicable
without coercion to the call $\f(\Xs)$.  Since $\Ps \neq \Qs$ either
$\Ps \inter \Qs \strictsubtype \Ps$ or $\Ps \inter \Qs \strictsubtype
\Qs$.  Either case contradicts our assumption.
\end{proof}

\begin{theorem}
\label{thm:overloading-subtyping}
If $|\Sigma| \neq 0$ then $|\sigma| = 1$.  Also, if $|\Sigma| \neq 0$
then $|\delta| = 1$.
\end{theorem}

\begin{proof}
Follows from Lemmas \ref{lem:dgesge-second}, \ref{lem:ge} and \ref{lem:le}.
\end{proof}

\begin{theorem}
\label{thm:dynamic-subtype-static}
If $\sigma = \set{S}$ and $\delta = \set{D}$ then $D \Ovrsubtype S$.
\end{theorem}

\begin{proof}
If the declaration with parameter type $S$ and the declaration with
parameter type $D$ satisfy the Subtype Rule then the theorem is
proved.  Otherwise, by the definition of $\sigma$ we have $\sigma
\subseteq \Sigma$.  Therefore $S \in \Sigma$.  By Lemma
\ref{lem:subset}, $S \in \Delta$.  Notice that $S,D \in \Delta$ implies
$X \Ovrsubtype S$ and $X \Ovrsubtype D$.  Therefore, $S \not \excludes
D$.  By the More Specific Rule for Functions, there must exist a
declaration with parameter type $S \inter D$.  Because $X
\Ovrsubtype (S \inter D)$, $(S \inter D) \in \Delta$.
Notice $(S \inter D) \Ovrsubtype S$ and $(S \inter D) \Ovrsubtype D$.
By the definition of $\delta$, we have $\lnot \exists D'\in\Delta: D'
\strictsubtype D$.  In particular, $(S \inter D) \not \strictsubtype
D$.  Therefore $(S \inter D) = D$.
\end{proof}
