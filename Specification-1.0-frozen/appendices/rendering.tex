%%%%%%%%%%%%%%%%%%%%%%%%%%%%%%%%%%%%%%%%%%%%%%%%%%%%%%%%%%%%%%%%%%%%%%%%%%%%%%%%
%   Copyright 2009, Oracle and/or its affiliates.
%   All rights reserved.
%
%
%   Use is subject to license terms.
%
%   This distribution may include materials developed by third parties.
%
%%%%%%%%%%%%%%%%%%%%%%%%%%%%%%%%%%%%%%%%%%%%%%%%%%%%%%%%%%%%%%%%%%%%%%%%%%%%%%%%

\chapter{Rendering of Fortress Code}
\applabel{app-rendering}

\note{Identifiers including connecting punctuations are not yet supported.}

In order to more closely approximate mathematical notation,
Fortress mandates standard rendering for various input elements,
particularly for numerals and identifiers,
as specified in \secref{rendering}.

In this appendix, we describe the detailed rules
for rendering an identifier, as well as rules used for rendering other constructions.

\section{Rendering of Fortress Identifiers}
\seclabel{rendering-identifiers}

If an identifier consists of letters and possibly digits,
but no underscores, %or other connecting punctuation,
prime marks, or apostrophes, then the rules are fairly simple:

(a) If the identifier consists of two ASCII capital letters that are the same,
possibly followed by digits, then a single capital letter is rendered
double-struck, followed by full-sized (not subscripted) digits in roman font.

\begin{tabular}{rcl@{\qquad\qquad}rcl}
   \STR{QQ} & \emph{is rendered as} & \EXP{\mathbb{Q}} &
   \STR{RR64} & \emph{is rendered as} & \EXP{\mathbb{R}64} \\
   \STR{ZZ} & \emph{is rendered as} & \EXP{\mathbb{Z}} &
   \STR{ZZ512} & \emph{is rendered as} & \EXP{\mathbb{Z}512}
\end{tabular}

(b) Otherwise, if the identifier has more than two characters and
begins with a capital letter, then it is rendered in roman font.
(Such names are typically used as names of types in Fortress.
Note that an identifier cannot consist entirely of capital letters,
because such a token is considered to be an operator.)

\begin{tabular}{rcl@{\qquad\qquad}rcl}
   \STR{Integer} & \emph{is rendered as} & \TYP{Integer} &
   \STR{Matrix} & \emph{is rendered as} & \TYP{Matrix} \\
   \STR{TotalOrder} & \emph{is rendered as} & \TYP{TotalOrder} &
   \STR{BooleanAlgebra} & \emph{is rendered as} & \TYP{BooleanAlgebra} \\
   \STR{Fred17} & \emph{is rendered as} & \TYP{Fred17} &
   \STR{R2D2} & \emph{is rendered as} & \TYP{R2D2}
\end{tabular}

(c) Otherwise, if the identifier consists of one or more letters
followed by one or more digits, then the letters are rendered in italic
and the digits are rendered as roman subscripts.

\begin{tabular}{rcl@{\qquad\qquad}rcl}
   \STR{a3} & \emph{is rendered as} & \EXP{a_{3}} &
   \STR{foo7} & \emph{is rendered as} & \EXP{{foo}_{7}} \\
   \STR{M1} & \emph{is rendered as} & \EXP{M_{1}} &
   \STR{z128} & \emph{is rendered as} & \EXP{z_{128}} \\
   \txt{OMEGA13} & \emph{is rendered as} & \EXP{\Omega{}_{13}} &
   \STR{myFavoriteThings1625} & \emph{is rendered as} & \EXP{{myFavoriteThings}_{1625}}

\end{tabular}

(d) The following names are always rendered in roman type out of respect for tradition:

\begin{tabular}{llllll}
   \EXP{\mathrm{sin}} & \EXP{\mathrm{cos}} & \EXP{\mathrm{tan}} & \EXP{\mathrm{cot}} & \EXP{\mathrm{sec}} & \EXP{\mathrm{csc}} \\
   \EXP{\mathrm{sinh}} & \EXP{\mathrm{cosh}} & \EXP{\mathrm{tanh}} & \EXP{\mathrm{coth}} & \EXP{\mathrm{sech}} & \EXP{\mathrm{csch}} \\
   \EXP{\mathrm{arcsin}} & \EXP{\mathrm{arccos}} & \EXP{\mathrm{arctan}} & \EXP{\mathrm{arccot}} & \EXP{\mathrm{arcsec}} & \EXP{\mathrm{arccsc}} \\
   \EXP{\mathrm{arsinh}} & \EXP{\mathrm{arcosh}} & \EXP{\mathrm{artanh}} & \EXP{\mathrm{arcoth}} & \EXP{\mathrm{arsech}} & \EXP{\mathrm{arcsch}} \\
   \EXP{\mathrm{arg}} & \EXP{\mathrm{deg}} & \EXP{\mathrm{det}} & \EXP{\mathrm{exp}} & \EXP{\mathrm{inf}} & \EXP{\mathrm{sup}} \\
   \EXP{\mathrm{lg}} & \EXP{\mathrm{ln}} & \EXP{\mathrm{log}} & \EXP{\mathrm{gcd}} & \EXP{\mathrm{max}} & \EXP{\mathrm{min}}
\end{tabular}

(e) Otherwise the identifier is simply rendered entirely in italic type.

\begin{tabular}{rcl@{\qquad\qquad}rcl}
   \STR{a} & \emph{is rendered as} & \VAR{a} &
   \STR{foobar} & \emph{is rendered as} & \VAR{foobar} \\
   \STR{length} & \emph{is rendered as} & \VAR{length} &
   \STR{isInstanceOf} & \emph{is rendered as} & \VAR{isInstanceOf} \\
   \STR{foo7a} & \emph{is rendered as} & \VAR{foo7a} &
   \STR{l33tsp33k} & \emph{is rendered as} & \VAR{l33tsp33k}
\end{tabular}

If the identifier begins or ends with an underscore, or both,
but has no other underscores,
%% or other connecting punctuation,
or prime marks, or apostrophes:

(f) If the identifier, ignoring its underscores, consists of two ASCII capital
letters that are the same, possibly followed by one or more digits,
then a single capital letter is rendered in sans-serif (for a leading
underscore), script (for a trailing underscore), or italic san-serif
(for both a leading and a trailing underscore), and any digits are
rendered as roman subscripts.

(g) Otherwise, the identifier without its underscores is rendered
in boldface (for a leading underscore), roman (for a trailing underscore),
or bold italic (for both a leading and a trailing underscore);
except that if the identifier, ignoring its underscores, consists of
one or more letters followed by one or more digits, then the
digits are rendered as roman subscripts regardless of the underscores.

\begin{tabular}{rcl@{\qquad\qquad}rcl}
      \txt{m\_} & \emph{is rendered as} & \EXP{\mathrm{m}} &
      \txt{s\_} & \emph{is rendered as} & \EXP{\mathrm{s}} \\
      \txt{km\_} & \emph{is rendered as} & \EXP{\mathrm{km}} &
      \txt{kg\_} & \emph{is rendered as} & \EXP{\mathrm{kg}} \\
      \txt{V\_} & \emph{is rendered as} & \EXP{\mathrm{V}} &
      \txt{kW\_} & \emph{is rendered as} & \EXP{\mathrm{kW}} \\
      \txt{\_v} & \emph{is rendered as} & \EXP{\mathbf{v}} &
      \txt{\_foo13} & \emph{is rendered as} & \EXP{\mathbf{foo}_{13}}
\end{tabular}

(Roman identifiers are typically used for names of SI dimensional units.
See sections 6.1.1 and 6.2.1 of~\cite{nistSP811}
for style questions with respect to dimensions and units.)

These last two rules are actually special cases of the following general
rules that apply whenever an identifier contains at least one underscore,
%% other connecting punctuation,
prime mark, or apostrophe:

An identifier containing underscores is divided into portions by its underscores;
in addition, any apostrophe, prime, or double prime character
separates portions and is also itself a portion.

(h) If any portion is empty other than the first or last, then the entire identifier
is rendered in italics, underscores and all.

Otherwise, the
portions are rendered as follows.  The idea is that there is
a \emph{principal portion} that may be preceded and/or followed
by modifiers, and there may also be a \emph{face portion}:
\begin{itemize}
\item If the first portion is not empty, \txt{script}, \txt{fraktur}, \txt{sansserif}, or \txt{monospace},
then the principal portion is the first portion and there is no face portion.
\item If the first portion is \txt{script}, \txt{fraktur}, \txt{sansserif}, or \txt{monospace},
then the principal portion is the second portion and the face portion is the first portion.
\item If the first portion is empty and the second portion is not
\txt{script}, \txt{fraktur}, \txt{sansserif}, or \txt{monospace},
then the principal portion is the second portion and there is no face portion.
\item Otherwise the principal portion is the third portion and the face portion is the second portion.
\end{itemize}
If there is no face portion, the principal portion will be rendered in ordinary italics.
However, if the first portion is empty
(that is, the identifier begins with a leading underscore), then the
principal portion is to be rendered in roman boldface.
If the last portion is empty (that is, the identifier ends
with a trailing underscore), then the principal portion
will be roman rather than italic, or bold italic rather than bold.

If there is a face portion, then that describes an alternate typeface to be used
in rendering the principal portion.  If there is no face portion,
but the principal portion consists of two copies of the same letter,
then it is rendered as a single letter in a double-struck face
(also known as ``blackboard bold''), sans-serif, script, or italic sans-serif
font according to whether the first and last portions are (not empty, not
empty), (empty, not empty), (not empty, empty), or (empty, empty),
respectively.  Otherwise, if the first portion is empty
(that is, the identifier begins with a leading underscore), then the
principal portion is to be rendered in a bold version of the selected face,
and if the last portion is empty (that is, the identifier ends
with a trailing underscore), then the principal portion
to be rendered in an italic (or bold italic) version of the selected face.
The bold and italic modifiers may be used only in combination with certain faces; the
following are the allowed combinations:

\begin{tabular}{l}
        \txt{script}\\
        \txt{bold script}\\
        \txt{fraktur}\\
        \txt{bold fraktur}\\
        \txt{double-struck}\\
        \txt{sans-serif}\\
        \txt{bold sans-serif}\\
        \txt{italic sans-serif}\\
        \txt{bold italic sans-serif}\\
        \txt{monospace}
\end{tabular}

If a combination can be properly rendered, then the principal
portion is rendered but not any preceding portions or underscores.
If a combination cannot be properly rendered, then the principal portion
and all portions and underscores preceding it are rendered all in italics
if possible, and otherwise all in some other default face.

If the principal portion consists of a sequence of letters followed by a sequence
of digits, then the letters are rendered in the chosen face and
the digits are rendered as roman subscripts.  Otherwise the
entire principal portion is rendered in the chosen face.
The remaining portions (excepting the last, if it is empty)
are then processed according to the following rules:
\begin{itemize}
\item If a portion is \txt{bar}, then a bar is rendered
above what has already been rendered, excluding superscripts and subscripts.
For example, \STR{x{\char'137}bar} is rendered as \EXP{\bar{\VAR{x}}},
\STR{x17{\char'137}bar} is rendered as \EXP{\bar{\VAR{x}}_{17}},
\STR{x{\char'137}bar{\char'137}bar} is rendered as \EXP{\bar{\bar{\VAR{x}}}},
and \STR{foo{\char'137}bar} is rendered as \EXP{\overline{\VAR{foo}}}.
(Contrast this last with \STR{foo{\char'137}baz}, which is rendered as \EXP{\VAR{foo{\tt\_}baz}}.)

\item If a portion is \txt{vec}, then a right-pointing arrow is rendered
above what has already been rendered, excluding superscripts and subscripts.
For example, \STR{v{\char'137}vec} is rendered as \EXP{\vec{\VAR{v}}},
\STR{v17{\char'137}vec} is rendered as \EXP{\vec{\VAR{v}}_{17}}, and
\STR{zoom{\char'137}vec} is rendered as \EXP{\overrightarrow{\VAR{zoom}}}.

\item If a portion is \txt{hat}, then a hat is rendered above what has already
been rendered, excluding superscripts and subscripts.  For example,
\STR{x17{\char'137}hat} is rendered as \EXP{\hat{\VAR{x}}_{17}}.

\item If a portion is \txt{dot}, then a dot is rendered above what has already
been rendered, excluding superscripts and subscripts; but if the preceding portion was also \txt{dot},
then the new dot is rendered appropriately relative to the previous dot(s).
Up to four dots will be rendered side-by-side rather than vertically.
For example,
\STR{a{\char'137}dot} is rendered as \EXP{\dot{\VAR{a}}},
\STR{a{\char'137}dot{\char'137}dot} is rendered as \EXP{\ddot{\VAR{a}}},
\STR{a{\char'137}dot{\char'137}dot{\char'137}dot} is rendered as \EXP{\dddot{\VAR{a}}},
\STR{a{\char'137}dot{\char'137}dot{\char'137}dot{\char'137}dot} is rendered as \EXP{\ddddot{\VAR{a}}}.
Also, \STR{a{\char'137}vec{\char'137}dot} is rendered as \EXP{\dot{\vec{\VAR{v}}}}.

\item If a portion is \txt{star}, then an asterisk \STR{*} is rendered as a superscript.
For example,
\STR{a{\char'137}star} is rendered as \EXP{a^*},
\STR{a{\char'137}star{\char'137}star} is rendered as\EXP{a^{**}},
\STR{ZZ{\char'137}star} is rendered as \EXP{\mathbb{Z}^*}.

\item If a portion is \txt{splat}, then a number sign \STR{{\char'43}} is rendered as a superscript.
For example, \STR{QQ{\char'137}splat} is rendered as \EXP{\mathbb{Q}^{\#}}.

\item If a portion is \txt{prime}, then a prime mark is rendered as a superscript.

\item A prime character is treated the same as \txt{prime}, and a double prime character
is treated the same as two consecutive \txt{prime} portions.  An apostrophe is treated
the same as a prime character, but only if all characters following it in the
identifier, if any, are also apostrophes.  For example,
\STR{a'} is rendered as \EXP{a'},
\STR{a13'} is rendered as \EXP{a_{13}'}, and
\STR{a''} is rendered as \EXP{a''},
but \STR{don't} is rendered as \EXP{\hbox{\emph{don't}}}.

\item If a portion is \txt{super} and another portion follows, then that other
portion is rendered as a superscript in roman type, and enclosed in parentheses
if it is all digits.

\item If a portion is \txt{sub} and another portion follows, then that other
portion is rendered as a subscript in roman type, and enclosed in parentheses
if it is all digits, and preceded by a subscript-separating comma if this
portion was immediately preceded by another portion that was rendered
as a subscript.

\item If a portion consists entirely of capital letters and would,
if considered by itself as an identifier, be the name of a non-letter Unicode character
that would be subject to replacement by preprocessing, then that Unicode character
is rendered as a subscript.  For example,
\STR{id{\char'137}OPLUS} is rendered as \EXP{\VAR{id}_{\oplus}},
\STR{ZZ{\char'137}GT} is rendered as \EXP{\mathbb{Z}_{>}}, and
\STR{QQ{\char'137}star{\char'137}LE} is rendered as \EXP{\mathbb{Q}^*_{\leq}}.

\item If the portion is the last portion, and the principal
portion was a single letter (or two letters indicating a double-struck letter),
and none of the preceding rules in this list applies,
it is rendered as a subscript in roman type.
For example,
\STR{T{\char'137}min} is rendered as \EXP{T_{\mathrm{min}}}.
Note that \STR{T{\char'137}MAX} is rendered simply as \OPR{T{\char'137}MAX}---because
all its letters are capital letters, it is considered to be an operator---but
\STR{T{\char'137}sub{\char'137}MAX} is rendered as \EXP{T_{\mathrm{MAX}}}.

\item Otherwise, this portion and all succeeding portions are rendered
in italics, along with any underscores that appear adjacent to any of them.
\end{itemize}

Examples:

\begin{tabular}{rclrcl}
        \txt{M} & \emph{is rendered as} & $M$ &
        \txt{\_M} & \emph{is rendered as} & $\mathbf{M}$ \\
        \txt{v\_vec} & \emph{is rendered as} & $\vec{v}$ &
        \txt{\_v\_vec} & \emph{is rendered as} & $\vec{\mathbf{v}}$ \\
        \txt{v1} & \emph{is rendered as} & $v_1$ &
        \txt{v\_x} & \emph{is rendered as} & $v_{\mathrm{x}}$ \\
        \txt{\_v1} & \emph{is rendered as} & $\mathbf{v}_1$ &
        \txt{\_v\_x} & \emph{is rendered as} & $\mathbf{v}_{\mathrm{x}}$ \\
        \txt{a\_dot} & \emph{is rendered as} & $\dot{a}$ &
        \txt{a\_dot\_dot} & \emph{is rendered as} & $\ddot{a}$ \\
        \txt{a\_dot\_dot\_dot} & \emph{is rendered as} & $\dddot{a}$ &
        \txt{a\_dot\_dot\_dot\_dot} & \emph{is rendered as} & $\ddddot{a}$ \\
        \txt{a\_dot\_dot\_dot\_dot\_dot} & \emph{is rendered as} & $\dot{\ddddot{a}}$ &
        \txt{p13'} & \emph{is rendered as} & $p_{13}'$ \\
        \txt{p'} & \emph{is rendered as} & $p'$ &
        \txt{p\_prime} & \emph{is rendered as} & $p'$ \\
        \txt{T\_min} & \emph{is rendered as} & $T_{\mathrm{min}}$ &
        \txt{T\_max} & \emph{is rendered as} & $T_{\mathrm{max}}$ \\
        \txt{foo\_bar} & \emph{is rendered as} & $\overline{\mathit{foo}}$ &
        \txt{foo\_baz} & \emph{is rendered as} & $\mathit{foo\_baz}$
\end{tabular}

In this way, through the use of underscore characters and annotation portions
delimited by underscores, the programmer can exercise considerable typographical control
over the rendering of variable names; but if no underscores are used, the rendering
rules are quite simple.

\section{Rendering of Other Fortress Constructs}
\seclabel{rendering-other}

A parameterized type of the form \verb+T[\S\]+ is rendered by replacing
the brackets \verb+[\+ and \verb+\]+ with
\EXP{\llbracket} (U+27E6, \txt{MATHEMATICAL LEFT WHITE SQUARE BRACKET}) and
\EXP{\rrbracket} (U+27E7, \txt{MATHEMATICAL RIGHT WHITE} \txt{SQUARE BRACKET}),
respectively.


An expression of the form \STR{a{\char'136}b} or \STR{a{\char'136}(b)} is rendered by making the expression
\VAR{b} a superscript, and possibly removing an outer set of parentheses, if it is reasonably simple;
otherwise it is rendered as if \STR{{\char'136}} were an ordinary binary operator.

\begin{center}
\begin{tabular}{rcl@{\qquad\qquad}rcl}
\STR{x{\char'136}y} & \emph{is rendered as} & \EXP{x^{y}} \\
\STR{x{\char'136}43} & \emph{is rendered as} & \EXP{x^{43}} \\
\STR{x{\char'136}(n+1)} & \emph{is rendered as} & \EXP{x^{n+1}} \\
\STR{x{\char'136}(|s.substring(a,b)|)} & \emph{is rendered as} &
   \EXP{x\STR{{\char'136}}(\left|s.\VAR{substring}(a,b)\right|)}
\end{tabular}
\end{center}

An expression of the form \STR{a[b]} is rendered by making the expression
\VAR{b} a subscript, if it is reasonably simple.

\begin{center}
\begin{tabular}{rcl@{\qquad\qquad}rcl}
\STR{a[43]} & \emph{is rendered as} & \EXP{a_{43}} \\
\STR{a[k]} & \emph{is rendered as} & \EXP{a_k} \\
\STR{a[n{\char'137}max]} & \emph{is rendered as} & \EXP{a_{n_{\mathrm{\max}}}} \\
\STR{a[k+k']} & \emph{is rendered as} & \EXP{a_{k+k'}} \\
\STR{a[b-c]} & \emph{is rendered as} & \EXP{a_{b-c}} \\
\STR{a[3~k~+~1]} & \emph{is rendered as} & \EXP{a_{3 k + 1}} \\
\STR{a[b[c[d]]]} & \emph{is rendered as} & \EXP{a[b[c_d]]} \\
\STR{a[s.substring(p,q)]} & \emph{is rendered as} & \EXP{a[s.\VAR{substring}(p,q)]}
\end{tabular}
\end{center}

If there is more than one subscript expression, then they are rendered as subscripts
only if each subscript is a single letter or digit, or a single letter followed
by digits and/or prime marks:

\begin{center}
\begin{tabular}{rcl@{\qquad\qquad}rcl}
\STR{a[1,1]} & \emph{is rendered as} & \EXP{a_{11}} \\
\STR{a[1,n]} & \emph{is rendered as} & \EXP{a_{1n}} \\
\STR{a[j,m,n]} & \emph{is rendered as} & \EXP{a_{jmn}} \\
\STR{a[n1,n2,n3]} & \emph{is rendered as} & \EXP{a_{n_{1}n_{2}n_{3}}} \\
\STR{a[k,k',k'']} & \emph{is rendered as} & \EXP{a_{kk'k''}} \\
\STR{a[1,n-1]} & \emph{is rendered as} & \EXP{a[1,n-1]}
\end{tabular}
\end{center}

\noindent (There is, of course, some opportunity here to make the rules for
subscripted rendering more elaborate.)

If there is whitespace
adjacent to the subscripting brackets or any separating comma, then actual subscripts are not used
and the brackets remain (so putting spaces after the commas of a multidimensional subscript is a simple
way to guarantee that bracket syntax will be used in the formatted version):

\begin{center}
\begin{tabular}{rcl@{\qquad\qquad}rcl}
\STR{a[i,j]} & \emph{is rendered as} & \EXP{a_{ij}} \\
\STR{a[i,~j]} & \emph{is rendered as} & \EXP{a[i, j]} \\
\STR{a[b[c[ d ]]]} & \emph{is rendered as} & \EXP{a[b[c[\,d\,]]]}
\end{tabular}
\end{center}

The presence or absence of whitespace inside and adjacent to enclosing operators is
reflected in the rendered form by the presence or absence of a thin space:

\begin{center}
\begin{tabular}{rcl@{\qquad\qquad}rcl}
    \STR{a[s.substring(p,q)]} & \emph{is rendered as} & \EXP{a[s.\VAR{substring}(p,q)]} \\
    \STR{a[~s.substring(p,q)~]} & \emph{is rendered as} & \EXP{a[\,s.\VAR{substring}(p,q)\,]} \\
    \STR{{\char'173}{}1,2{\char'175}} & \emph{is rendered as} & \EXP{\{1,2\}} \\
    \STR{{\char'173}~1,~2~{\char'175}} & \emph{is rendered as} & \EXP{\{\,1, 2\,\}} \\
    \STR{<|1,2,3|>} & \emph{is rendered as} & \EXP{\langle{}1,2,3\rangle} \\
    \STR{<|~x+y~|~x<-a,~y<-b~|>} & \emph{is rendered as} & \EXP{\langle\,x+y \mid x\leftarrow{}a, y\leftarrow{}b\,\rangle}
\end{tabular}
\end{center}

\noindent
Rendered comprehensions look best if there is whitespace inside the enclosers
and on both sides of the separating vertical bar.

The presence or absence
of whitespace on either side of a colon is also handled especially carefully.
In general, one should put a space on both sides, or neither side, of a colon
that is used as an operator to construct a range; on the other hand,
one should put a space after,
but \emph{not} before, a colon that separates a variable or function header from a type.
This will produce the best rendered spacing.  (The assignment operator \EXP{\ASSIGN}
is recognized separately and whitespace doesn't matter.)

\begin{center}
\begin{tabular}{rcl@{\qquad\qquad}rcl}
    \STR{for~i~<-~1:10~do~print~i~end}
    & \emph{is rendered as} &
    \EXP{\KWD{for}\:i \leftarrow 1\COLONOP{}10 \:\KWD{do}\:\VAR{print}\:i\:\KWD{end}} \\
    \STR{for~i~<-~0~:~n+1~do~print~i~end}
    & \emph{is rendered as} &
    \EXP{\KWD{for}\: i \leftarrow 0 {}\mathrel{\mathtt{:}} n+1 \:\KWD{do}\: \VAR{print}\:i\:\KWD{end}} \\
    \STR{k:~ZZ32~:=~5}
    & \emph{is rendered as} &
    \EXP{k\COLON \mathbb{Z}32 \ASSIGN 5} \\
    \STR{factorial(n:~NN):~NN}
    & \emph{is rendered as} &
    \EXP{\VAR{factorial}(n\COLON \mathbb{N})\COLON \mathbb{N}} \\
    \STR{a:=b}
    & \emph{is rendered as} &
    \EXP{a\ASSIGN{}b}
\end{tabular}
\end{center}


Generators and filters in brackets after a reduction operator are rendered by stacking them
beneath the operator:

\begin{center}
\begin{tabular}{rcl@{\qquad\qquad}rcl}
    \STR{PROD[k<-1{\char'43}n]~n}
    & \emph{is rendered as} &
    \EXP{\prod\limits_{k\leftarrow{}1\mathinner{\hbox{\tt\char'43}}n} n} \\
\\
    \STR{SUM[i<-1:n,~j<-1:p,~prime~j]~a[i]~x{\char'136}j}
    & \emph{is rendered as} &
    \EXP{\sum\limits_{\genfrac{}{}{0pt}{1}{\genfrac{}{}{0pt}{1}{i\leftarrow{}1\COLONOP{}n}{j\leftarrow{}1\COLONOP{}p}}{\VAR{prime}\:j}}\:a_i\:x^{j}} \\
\\
    \STR{MAX[j<-S,~k<-j:j+m]~a[k]}
    & \emph{is rendered as} &
    \EXP{\mathop{\OPR{MAX}}\limits_{\genfrac{}{}{0pt}{1}{j\leftarrow{}S}{k\leftarrow{}j\COLONOP{}j+m}}\:a_k}
\end{tabular}
\end{center}
