%%%%%%%%%%%%%%%%%%%%%%%%%%%%%%%%%%%%%%%%%%%%%%%%%%%%%%%%%%%%%%%%%%%%%%%%%%%%%%%%
%   Copyright 2009, Oracle and/or its affiliates.
%   All rights reserved.
%
%
%   Use is subject to license terms.
%
%   This distribution may include materials developed by third parties.
%
%%%%%%%%%%%%%%%%%%%%%%%%%%%%%%%%%%%%%%%%%%%%%%%%%%%%%%%%%%%%%%%%%%%%%%%%%%%%%%%%

\chapter{Operator Precedence, Associativity, Chaining, and Enclosure}
\applabel{operator-precedence}
\applabel{operators}

\newcommand{\UnicodeOp}[4]{~~\texttt{#1}\>\texttt{\footnotesize #2}\>{#3}\>{#4}}
\newcommand{\UnicodeKillLine}{~~\texttt{U+0000}~~\={\footnotesize\texttt{LESS-THAN ABOVE SLANTED EQUAL ABOVE GREATER-THAN ABOVE SLANTED EQUAL}}~~\=\quad\quad\=\kill}


This appendix contains the detailed rules
for which \unicode\ characters may be used as operators,
which operators form enclosing pairs,
which operators may be chained,
which operators are associative,
and what precedence relationships exist among the various operators.
An infix operator is left associative, nonassociative, or chainable.
If no precedence relationship is stated explicitly
for any given pair of operators,
then there is no precedence relationship between those two operators.
Remember that precedence is not transitive in Fortress.

In each of the character lists below,
each line gives the Unicode codepoint,
the full \unicode\ name,
a rendering of the character in \TeX\ (if possible),
and any ASCII shorthand or short names for the character.



\section{Bracket Pairs for Static Type Information}
\seclabel{oxford-bracket-pairs}

These brackets are the exception: they are \emph{not} operators, but are used to enclose static parameters
and static arguments.

\begin{tabbing}
\UnicodeKillLine
\UnicodeOp{U+27E6}{MATHEMATICAL LEFT WHITE SQUARE BRACKET}{$[\![$}{\texttt{[{\char'134}}} \\
\UnicodeOp{U+27E7}{MATHEMATICAL RIGHT WHITE SQUARE BRACKET}{$]\!]$}{\texttt{{\char'134}]}}
\end{tabbing}

\section{Bracket Pairs for Enclosing Operators}
\seclabel{bracket-pairs}

Here are the bracket pairs that may be used as enclosing operators.
Note that there is one group of four brackets; within that group,
either left bracket may be paired with either right bracket to form
an enclosing operator.
\begin{tabbing}
\UnicodeKillLine
\UnicodeOp{U+005B}{LEFT SQUARE BRACKET}{[}{\texttt{[}} \\
\UnicodeOp{U+005D}{RIGHT SQUARE BRACKET}{]}{\texttt{]}} \\[4pt]
\UnicodeOp{U+007B}{LEFT CURLY BRACKET}{\{}{\texttt{{\char'173}}} \\
\UnicodeOp{U+007D}{RIGHT CURLY BRACKET}{\}}{\texttt{\char'175}} \\[4pt]
\UnicodeOp{U+2045}{LEFT SQUARE BRACKET WITH QUILL}{}{\texttt{[./}} \\
\UnicodeOp{U+2046}{RIGHT SQUARE BRACKET WITH QUILL}{}{\texttt{/.]}} \\[4pt]
\UnicodeOp{U+2308}{LEFT CEILING}{$\lceil$}{\texttt{|/}} \\
\UnicodeOp{U+2309}{RIGHT CEILING}{$\rceil$}{\texttt{{\char'134}|}} \\[4pt]
\UnicodeOp{U+230A}{LEFT FLOOR}{$\lfloor$}{\texttt{|{\char'134}}} \\
\UnicodeOp{U+230B}{RIGHT FLOOR}{$\rfloor$}{\texttt{/|}} \\[4pt]
\UnicodeOp{U+27C5}{LEFT S-SHAPED BAG DELIMITER}{}{\texttt{|.{\char'134}}} \\
\UnicodeOp{U+27C6}{RIGHT S-SHAPED BAG DELIMITER}{}{\texttt{/.|}} \\[4pt]
\UnicodeOp{U+27E8}{MATHEMATICAL LEFT ANGLE BRACKET}{$\langle$}{\texttt{<|}} \\
\UnicodeOp{U+27E9}{MATHEMATICAL RIGHT ANGLE BRACKET}{$\rangle$}{\texttt{|>}} \\[4pt]
\UnicodeOp{U+27EA}{MATHEMATICAL LEFT DOUBLE ANGLE BRACKET}{$\langle\!\langle$}{\texttt{<<|}} \\
\UnicodeOp{U+27EB}{MATHEMATICAL RIGHT DOUBLE ANGLE BRACKET}{$\rangle\!\rangle$}{\texttt{|>>}} \\[4pt]
\UnicodeOp{U+2983}{LEFT WHITE CURLY BRACKET}{$\{\!|$}{\texttt{{\char'173}{\char'134}}} \\
\UnicodeOp{U+2984}{RIGHT WHITE CURLY BRACKET}{$|\!\}$}{\texttt{{\char'134}{\char'175}}} \\[4pt]
\UnicodeOp{U+2985}{LEFT WHITE PARENTHESIS}{}{\texttt{(.{\char'134}}} \\
\UnicodeOp{U+2986}{RIGHT WHITE PARENTHESIS}{}{\texttt{{\char'134}.)}} \\[4pt]
\UnicodeOp{U+2987}{Z NOTATION LEFT IMAGE BRACKET}{}{\texttt{(./}} \\
\UnicodeOp{U+2988}{Z NOTATION RIGHT IMAGE BRACKET}{}{\texttt{/.)}} \\[4pt]
\UnicodeOp{U+2989}{Z NOTATION LEFT BINDING BRACKET}{}{\texttt{<||}} \\
\UnicodeOp{U+298A}{Z NOTATION RIGHT BINDING BRACKET}{}{\texttt{||>}} \\[4pt]
\UnicodeOp{U+298B}{LEFT SQUARE BRACKET WITH UNDERBAR}{}{\texttt{[.\char'134}} \\
\UnicodeOp{U+298C}{RIGHT SQUARE BRACKET WITH UNDERBAR}{}{\texttt{\char'134.]}} \\[4pt]
\UnicodeOp{U+298D}{LEFT SQUARE BRACKET WITH TICK IN TOP CORNER}{}{\texttt{[.//}} \\
\UnicodeOp{U+298E}{RIGHT SQUARE BRACKET WITH TICK IN BOTTOM CORNER}{}{\texttt{//.]}} \\[4pt]
\UnicodeOp{U+298F}{LEFT SQUARE BRACKET WITH TICK IN BOTTOM CORNER}{}{\texttt{[.\char'134\char'134}} \\
\UnicodeOp{U+2990}{RIGHT SQUARE BRACKET WITH TICK IN TOP CORNER}{}{\texttt{\char'134\char'134.]}} \\[4pt]
\UnicodeOp{U+2991}{LEFT ANGLE BRACKET WITH DOT}{}{\texttt{<.|}} \\
\UnicodeOp{U+2992}{RIGHT ANGLE BRACKET WITH DOT}{}{\texttt{|.>}} \\[4pt]
\UnicodeOp{U+2993}{LEFT ARC LESS-THAN BRACKET}{}{\texttt{(.<}} \\
\UnicodeOp{U+2994}{RIGHT ARC GREATER-THAN BRACKET}{}{\texttt{>.)}} \\[4pt]
\UnicodeOp{U+2995}{DOUBLE LEFT ARC GREATER-THAN BRACKET}{}{\texttt{((.>}} \\
\UnicodeOp{U+2996}{DOUBLE RIGHT ARC LESS-THAN BRACKET}{}{\texttt{<.))}} \\[4pt]
\UnicodeOp{U+2997}{LEFT BLACK TORTOISE SHELL BRACKET}{}{\texttt{[*/}} \\
\UnicodeOp{U+2998}{RIGHT BLACK TORTOISE SHELL BRACKET}{}{\texttt{/*]}} \\[4pt]
\UnicodeOp{U+29D8}{LEFT WIGGLY FENCE}{}{\texttt{[/\char'134/}} \\
\UnicodeOp{U+29D9}{RIGHT WIGGLY FENCE}{}{\texttt{/\char'134/]}} \\[4pt]
\UnicodeOp{U+29DA}{LEFT DOUBLE WIGGLY FENCE}{}{\texttt{[/\char'134/\char'134/}} \\
\UnicodeOp{U+29DB}{RIGHT DOUBLE WIGGLY FENCE}{}{\texttt{/\char'134/\char'134/]}} \\[4pt]
\UnicodeOp{U+29FC}{LEFT-POINTING CURVED ANGLE BRACKET}{}{\texttt{<|||}} \\
\UnicodeOp{U+29FD}{RIGHT-POINTING CURVED ANGLE BRACKET}{}{\texttt{|||>}} \\[4pt]
\UnicodeOp{U+300C}{LEFT CORNER BRACKET}{$\ulcorner$}{\texttt{</}} \\
\UnicodeOp{U+300D}{RIGHT CORNER BRACKET}{$\urcorner$}{\texttt{{\char'134}>}} \\[4pt]
\UnicodeOp{U+300E}{LEFT WHITE CORNER BRACKET}{}{\texttt{<</}} \\
\UnicodeOp{U+300F}{RIGHT WHITE CORNER BRACKET}{}{\texttt{{\char'134}>>}} \\[4pt]
\UnicodeOp{U+3010}{LEFT BLACK LENTICULAR BRACKET}{}{\texttt{{\char'173}*/}} \\
\UnicodeOp{U+3011}{RIGHT BLACK LENTICULAR BRACKET}{}{\texttt{/*{\char'175}}} \\[4pt]
\UnicodeOp{U+3018}{LEFT WHITE TORTOISE SHELL BRACKET}{}{\texttt{[//}} \\
\UnicodeOp{U+3014}{LEFT TORTOISE SHELL BRACKET}{}{\texttt{[/}} \\
\UnicodeOp{U+3015}{RIGHT TORTOISE SHELL BRACKET}{}{\texttt{/]}} \\
\UnicodeOp{U+3019}{RIGHT WHITE TORTOISE SHELL BRACKET}{}{\texttt{//]}} \\[4pt]
\UnicodeOp{U+3016}{LEFT WHITE LENTICULAR BRACKET}{}{\texttt{{\char'173}/}} \\
\UnicodeOp{U+3017}{RIGHT WHITE LENTICULAR BRACKET}{}{\texttt{/{\char'175}}}
\end{tabbing}



\section{Vertical-Line Operators}
\seclabel{vertical-line-ops}

The following are vertical-line operators.  They are left associative
when they are used as infix operators.

\begin{tabbing}
\UnicodeKillLine
\UnicodeOp{U+007C}{VERTICAL LINE}{$|$}{\texttt{|}} \\
\UnicodeOp{U+2016}{DOUBLE VERTICAL LINE}{$\|$}{\texttt{||}} \\
\UnicodeOp{U+2AF4}{TRIPLE VERTICAL BAR BINARY RELATION}{}{\texttt{|||}} \\
\end{tabbing}



\section{Arithmetic Operators}
\seclabel{precedence:arithops}

\subsection{Multiplication and Division}
\seclabel{precedence:multdivops}

The following are multiplication operators.  They are left associative.
\begin{tabbing}
\UnicodeKillLine
\UnicodeOp{U+002A}{ASTERISK}{*}{\texttt{*}} \\
\UnicodeOp{U+00B7}{MIDDLE DOT}{$\cdot$}{\texttt{DOT}} \\
\UnicodeOp{U+00D7}{MULTIPLICATION SIGN}{$\times$}{\texttt{TIMES}~~\texttt{BY}} \\
\UnicodeOp{U+2217}{ASTERISK OPERATOR}{$\ast$}{\texttt{*}} \\
\UnicodeOp{U+228D}{MULTISET MULTIPLICATION}{}{} \\
\UnicodeOp{U+2297}{CIRCLED TIMES}{$\otimes$}{\texttt{OTIMES}} \\
\UnicodeOp{U+2299}{CIRCLED DOT OPERATOR}{$\odot$}{\texttt{ODOT}} \\
\UnicodeOp{U+229B}{CIRCLED ASTERISK OPERATOR}{$\circledast$}{\texttt{CIRCLEDAST}} \\
\UnicodeOp{U+22A0}{SQUARED TIMES}{$\boxtimes$}{\texttt{BOXTIMES}} \\
\UnicodeOp{U+22A1}{SQUARED DOT OPERATOR}{$\boxdot$}{\texttt{BOXDOT}} \\
\UnicodeOp{U+22C5}{DOT OPERATOR}{$\cdot$}{} \\
\UnicodeOp{U+29C6}{SQUARED ASTERISK}{}{\texttt{BOXAST}} \\
\UnicodeOp{U+29D4}{TIMES WITH LEFT HALF BLACK}{}{} \\
\UnicodeOp{U+29D5}{TIMES WITH RIGHT HALF BLACK}{}{} \\
\UnicodeOp{U+2A2F}{VECTOR OR CROSS PRODUCT}{$\times$}{\texttt{CROSS}} \\
\UnicodeOp{U+2A30}{MULTIPLICATION SIGN WITH DOT ABOVE}{}{\texttt{DOTTIMES}} \\
\UnicodeOp{U+2A31}{MULTIPLICATION SIGN WITH UNDERBAR}{}{} \\
\UnicodeOp{U+2A34}{MULTIPLICATION SIGN IN LEFT HALF CIRCLE}{}{} \\
\UnicodeOp{U+2A35}{MULTIPLICATION SIGN IN RIGHT HALF CIRCLE}{}{} \\
\UnicodeOp{U+2A36}{CIRCLED MULTIPLICATION SIGN WITH CIRCUMFLEX ACCENT}{}{} \\
\UnicodeOp{U+2A37}{MULTIPLICATION SIGN IN DOUBLE CIRCLE}{}{} \\
\UnicodeOp{U+2A3B}{MULTIPLICATION SIGN IN TRIANGLE}{}{\texttt{TRITIMES}}
\end{tabbing}

The following are division operators.  They are nonassociative.
\begin{tabbing}
\UnicodeKillLine
\UnicodeOp{U+002F}{SOLIDUS}{/}{\texttt{/}} \\
\UnicodeOp{U+00F7}{DIVISION SIGN}{$\div$}{\texttt{DIV}} \\
\UnicodeOp{U+2215}{DIVISION SLASH}{$/$}{\texttt{/}} \\
\UnicodeOp{U+2298}{CIRCLED DIVISION SLASH}{$\oslash$}{\texttt{OSLASH}} \\
\UnicodeOp{U+29B8}{CIRCLED REVERSE SOLIDUS}{}{} \\
\UnicodeOp{U+29BC}{CIRCLED ANTICLOCKWISE-ROTATED DIVISION SIGN}{}{} \\
\UnicodeOp{U+29C4}{SQUARED RISING DIAGONAL SLASH}{}{\texttt{BOXSLASH}} \\
\UnicodeOp{U+29F5}{REVERSE SOLIDUS OPERATOR}{$\backslash$}{\texttt{\char'134}} \\
\UnicodeOp{U+29F8}{BIG SOLIDUS}{$\bigl/$}{} \\
\UnicodeOp{U+29F9}{BIG REVERSE SOLIDUS}{$\bigl\backslash$}{} \\
\UnicodeOp{U+2A38}{CIRCLED DIVISION SIGN}{}{\texttt{ODIV}} \\
\UnicodeOp{U+2AFD}{DOUBLE SOLIDUS OPERATOR}{$//$}{\texttt{//}} \\
\UnicodeOp{U+2AFB}{TRIPLE SOLIDUS BINARY RELATION}{}{\texttt{///}}
\end{tabbing}
Note also that \txt{per} is treated as a division operator.

\subsection{Addition and Subtraction}
\seclabel{precedence:addsubops}

Addition and subtraction operators are left associative.

The following three operators have the same precedence and may be mixed.
\begin{tabbing}
\UnicodeKillLine
\UnicodeOp{U+002B}{PLUS SIGN}{$+$}{\texttt{+}} \\
\UnicodeOp{U+002D}{HYPHEN-MINUS}{$-$}{\texttt{-}} \\
\UnicodeOp{U+2212}{MINUS SIGN}{$-$}{\texttt{-}}
\end{tabbing}
They each have lower precedence than any of the following multiplication and division operators:
\begin{tabbing}
\UnicodeKillLine
\UnicodeOp{U+002A}{ASTERISK}{*}{\texttt{*}} \\
\UnicodeOp{U+002F}{SOLIDUS}{/}{\texttt{/}} \\
\UnicodeOp{U+00B7}{MIDDLE DOT}{$\cdot$}{\texttt{DOT}} \\
\UnicodeOp{U+00D7}{MULTIPLICATION SIGN}{$\times$}{\texttt{TIMES}~~\texttt{BY}} \\
\UnicodeOp{U+00F7}{DIVISION SIGN}{$\div$}{\texttt{DIV}} \\
\UnicodeOp{U+2215}{DIVISION SLASH}{$/$}{\texttt{/}} \\
\UnicodeOp{U+2217}{ASTERISK OPERATOR}{$\ast$}{\texttt{*}} \\
\UnicodeOp{U+22C5}{DOT OPERATOR}{$\cdot$}{} \\
\UnicodeOp{U+2A2F}{VECTOR OR CROSS PRODUCT}{$\times$}{\texttt{CROSS}}
\end{tabbing}

The following two operators have the same precedence and may be mixed.
\begin{tabbing}
\UnicodeKillLine
\UnicodeOp{U+2214}{DOT PLUS}{$\dotplus$}{\texttt{DOTPLUS}} \\
\UnicodeOp{U+2238}{DOT MINUS}{$\dot{-}$}{\texttt{DOTMINUS}}
\end{tabbing}
They each have lower precedence than this multiplication operator:
\begin{tabbing}
\UnicodeKillLine
\UnicodeOp{U+2A30}{MULTIPLICATION SIGN WITH DOT ABOVE}{}{\texttt{DOTTIMES}}
\end{tabbing}

The following two operators have the same precedence and may be mixed.
\begin{tabbing}
\UnicodeKillLine
\UnicodeOp{U+2A25}{PLUS SIGN WITH DOT BELOW}{}{} \\
\UnicodeOp{U+2A2A}{MINUS SIGN WITH DOT BELOW}{}{}
\end{tabbing}

The following two operators have the same precedence and may be mixed.
\begin{tabbing}
\UnicodeKillLine
\UnicodeOp{U+2A39}{PLUS SIGN IN TRIANGLE}{}{\texttt{TRIPLUS}} \\
\UnicodeOp{U+2A3A}{MINUS SIGN IN TRIANGLE}{}{\texttt{TRIMINUS}}
\end{tabbing}
They each have lower precedence than this multiplication operator:
\begin{tabbing}
\UnicodeKillLine
\UnicodeOp{U+2A3B}{MULTIPLICATION SIGN IN TRIANGLE}{}{\texttt{TRITIMES}}
\end{tabbing}

The following two operators have the same precedence and may be mixed.
\begin{tabbing}
\UnicodeKillLine
\UnicodeOp{U+2295}{CIRCLED PLUS}{$\oplus$}{\texttt{OPLUS}} \\
\UnicodeOp{U+2296}{CIRCLED MINUS}{$\ominus$}{\texttt{OMINUS}}
\end{tabbing}
They each have lower precedence than any of the following multiplication and division operators:
\begin{tabbing}
\UnicodeKillLine
\UnicodeOp{U+2297}{CIRCLED TIMES}{$\otimes$}{\texttt{OTIMES}} \\
\UnicodeOp{U+2298}{CIRCLED DIVISION SLASH}{$\oslash$}{\texttt{OSLASH}} \\
\UnicodeOp{U+2299}{CIRCLED DOT OPERATOR}{$\odot$}{\texttt{ODOT}} \\
\UnicodeOp{U+229B}{CIRCLED ASTERISK OPERATOR}{$\circledast$}{\texttt{CIRCLEDAST}} \\
\UnicodeOp{U+2A38}{CIRCLED DIVISION SIGN}{}{\texttt{ODIV}}
\end{tabbing}

The following two operators have the same precedence and may be mixed.
\begin{tabbing}
\UnicodeKillLine
\UnicodeOp{U+229E}{SQUARED PLUS}{$\boxplus$}{\texttt{BOXPLUS}} \\
\UnicodeOp{U+229F}{SQUARED MINUS}{$\boxminus$}{\texttt{BOXMINUS}}
\end{tabbing}
They each have lower precedence than any of these multiplication or division operators:
\begin{tabbing}
\UnicodeKillLine
\UnicodeOp{U+22A0}{SQUARED TIMES}{$\boxtimes$}{\texttt{BOXTIMES}} \\
\UnicodeOp{U+22A1}{SQUARED DOT OPERATOR}{$\boxdot$}{\texttt{BOXDOT}} \\
\UnicodeOp{U+29C4}{SQUARED RISING DIAGONAL SLASH}{}{\texttt{BOXSLASH}} \\
\UnicodeOp{U+29C6}{SQUARED ASTERISK}{}{\texttt{BOXAST}} \\
\end{tabbing}

These are other miscellaneous addition and subtraction operators:
\begin{tabbing}
\UnicodeKillLine
\UnicodeOp{U+00B1}{PLUS-MINUS SIGN}{$\pm$}{} \\
\UnicodeOp{U+2213}{MINUS-OR-PLUS SIGN}{$\mp$}{} \\
\UnicodeOp{U+2242}{MINUS TILDE}{}{} \\
\UnicodeOp{U+2A22}{PLUS SIGN WITH SMALL CIRCLE ABOVE}{$\overset{\circ}{+}$}{} \\
\UnicodeOp{U+2A23}{PLUS SIGN WITH CIRCUMFLEX ACCENT ABOVE}{$\Hat{+}$}{} \\
\UnicodeOp{U+2A24}{PLUS SIGN WITH TILDE ABOVE}{$\overset{\sim}{+}$}{} \\
\UnicodeOp{U+2A26}{PLUS SIGN WITH TILDE BELOW}{$\underset{\sim}{+}$}{} \\
\UnicodeOp{U+2A27}{PLUS SIGN WITH SUBSCRIPT TWO}{$\hbox{$+$}_2$}{} \\
\UnicodeOp{U+2A28}{PLUS SIGN WITH BLACK TRIANGLE}{}{} \\
\UnicodeOp{U+2A29}{MINUS SIGN WITH COMMA ABOVE}{$\overset{,}{-}$}{} \\
\UnicodeOp{U+2A2B}{MINUS SIGN WITH FALLING DOTS}{}{} \\
\UnicodeOp{U+2A2C}{MINUS SIGN WITH RISING DOTS}{}{} \\
\UnicodeOp{U+2A2D}{PLUS SIGN IN LEFT HALF CIRCLE}{}{} \\
\UnicodeOp{U+2A2E}{PLUS SIGN IN RIGHT HALF CIRCLE}{}{}
\end{tabbing}


\subsection{Miscellaneous Arithmetic Operators}
\seclabel{precedence:miscarithops}

The operators \texttt{MAX}, \texttt{MIN}, \texttt{REM}, \texttt{MOD}, \texttt{GCD},
\texttt{LCM}, \texttt{CHOOSE}, and \texttt{per}, none of which corresponds to a single
Unicode character, are considered to be arithmetic operators, having
higher precedence than certain relational operators, as described in
a later section.
Among these operators, \texttt{MAX}, \texttt{MIN}, \texttt{GCD}, and \texttt{LCM}
are left associative and \texttt{REM}, \texttt{MOD}, \texttt{CHOOSE}, and
\texttt{per} are nonassociative.

\subsection{Set Intersection, Union, and Difference}
\seclabel{precedence:setops}

The following are the set intersection operators.  They are left associative.
\begin{tabbing}
\UnicodeKillLine
\UnicodeOp{U+2229}{INTERSECTION}{$\cap$}{\texttt{CAP}~~\texttt{INTERSECT}} \\
\UnicodeOp{U+22D2}{DOUBLE INTERSECTION}{$\Cap$}{\texttt{CAPCAP}} \\
\UnicodeOp{U+2A40}{INTERSECTION WITH DOT}{}{} \\
\UnicodeOp{U+2A43}{INTERSECTION WITH OVERBAR}{$\overline{\cap}$}{} \\
\UnicodeOp{U+2A44}{INTERSECTION WITH LOGICAL AND}{}{} \\
\UnicodeOp{U+2A4B}{INTERSECTION BESIDE AND JOINED WITH INTERSECTION}{}{} \\
\UnicodeOp{U+2A4D}{CLOSED INTERSECTION WITH SERIFS}{}{} \\
\UnicodeOp{U+2ADB}{TRANSVERSAL INTERSECTION}{}{}
\end{tabbing}

The following are the set union operators.  They are left associative.
\begin{tabbing}
\UnicodeKillLine
\UnicodeOp{U+222A}{UNION}{$\cup$}{\texttt{CUP}~~\texttt{UNION}} \\
\UnicodeOp{U+228E}{MULTISET UNION}{$\uplus$}{\texttt{UPLUS}} \\
\UnicodeOp{U+22D3}{DOUBLE UNION}{$\Cup$}{\texttt{CUPCUP}} \\
\UnicodeOp{U+2A41}{UNION WITH MINUS SIGN}{}{} \\
\UnicodeOp{U+2A42}{UNION WITH OVERBAR}{$\overline{\cup}$}{} \\
\UnicodeOp{U+2A45}{UNION WITH LOGICAL OR}{}{} \\
\UnicodeOp{U+2A4A}{UNION BESIDE AND JOINED WITH UNION}{}{} \\
\UnicodeOp{U+2A4C}{CLOSED UNION WITH SERIFS}{}{} \\
\UnicodeOp{U+2A50}{CLOSED UNION WITH SERIFS AND SMASH PRODUCT}{}{}
\end{tabbing}
They each have lower precedence than any of the set intersection operators.

This is a miscellaneous set operator.  It is nonassociative.
\begin{tabbing}
\UnicodeKillLine
\UnicodeOp{U+2216}{SET MINUS}{$\setminus$}{\texttt{SETMINUS}}
\end{tabbing}


\subsection{Square Arithmetic Operators}
\seclabel{precedence:squareops}

Square arithmetic operators are left associative.

The following are the square intersection operators.
\begin{tabbing}
\UnicodeKillLine
\UnicodeOp{U+2293}{SQUARE CAP}{$\sqcap$}{\texttt{SQCAP}} \\
\UnicodeOp{U+2A4E}{DOUBLE SQUARE INTERSECTION}{}{\texttt{SQCAPCAP}}
\end{tabbing}

The following are the square union operators:
\begin{tabbing}
\UnicodeKillLine
\UnicodeOp{U+2294}{SQUARE CUP}{$\sqcup$}{\texttt{SQCUP}} \\
\UnicodeOp{U+2A4F}{DOUBLE SQUARE UNION}{}{\texttt{SQCUPCUP}}
\end{tabbing}
They each have lower precedence than either of the square intersection operators.


\subsection{Curly Arithmetic Operators}
\seclabel{precedence:curlyops}

Curly arithmetic operators are left associative.

The following is the curly intersection operator:
\begin{tabbing}
\UnicodeKillLine
\UnicodeOp{U+22CF}{CURLY LOGICAL AND}{$\curlywedge$}{\texttt{CURLYAND}}
\end{tabbing}

The following is the curly union operator:
\begin{tabbing}
\UnicodeKillLine
\UnicodeOp{U+22CE}{CURLY LOGICAL OR}{$\curlyvee$}{\texttt{CURLYOR}}
\end{tabbing}
It has lower precedence than the curly intersection operator.

\section{Relational Operators}
\seclabel{precedence:relops}

\subsection{Equivalence and Inequivalence Operators}
\seclabel{precedence:equivops}

Every operator listed in this section
has lower precedence than any operator listed in
\secref{precedence:arithops}.

The following are equivalence operators.  They may be chained.
Moreover, they may be chained with any other single group of
chainable relational operators, as described in later sections.
\begin{tabbing}
\UnicodeKillLine
\UnicodeOp{U+003D}{EQUALS SIGN}{$=$}{\texttt{EQ}} \\
\UnicodeOp{U+2243}{ASYMPTOTICALLY EQUAL TO}{$\simeq$}{\texttt{SIMEQ}} \\
\UnicodeOp{U+2245}{APPROXIMATELY EQUAL TO}{$\cong$}{} \\
\UnicodeOp{U+2246}{APPROXIMATELY BUT NOT ACTUALLY EQUAL TO}{}{} \\
\UnicodeOp{U+2248}{ALMOST EQUAL TO}{$\approx$}{\texttt{APPROX}} \\
\UnicodeOp{U+224A}{ALMOST EQUAL OR EQUAL TO}{$\approxeq$}{\texttt{APPROXEQ}} \\
\UnicodeOp{U+224C}{ALL EQUAL TO}{}{} \\
\UnicodeOp{U+224D}{EQUIVALENT TO}{$\asymp$}{} \\
\UnicodeOp{U+224E}{GEOMETRICALLY EQUIVALENT TO}{$\Bumpeq$}{\texttt{BUMPEQV}} \\
\UnicodeOp{U+2251}{GEOMETRICALLY EQUAL TO}{$\doteqdot$}{\texttt{DOTEQDOT}} \\
\UnicodeOp{U+2252}{APPROXIMATELY EQUAL TO OR THE IMAGE OF}{$\fallingdotseq$}{} \\
\UnicodeOp{U+2253}{IMAGE OF OR APPROXIMATELY EQUAL TO}{$\risingdotseq$}{} \\
\UnicodeOp{U+2256}{RING IN EQUAL TO}{$\eqcirc$}{\texttt{EQRING}} \\
\UnicodeOp{U+2257}{RING EQUAL TO}{$\circeq$}{\texttt{RINGEQ}} \\
\UnicodeOp{U+225B}{STAR EQUALS}{}{} \\
\UnicodeOp{U+225C}{DELTA EQUAL TO}{$\triangleq$}{\texttt{EQDEL}} \\
\UnicodeOp{U+225D}{EQUAL TO BY DEFINITION}{}{\texttt{EQDEF}} \\
\UnicodeOp{U+225F}{QUESTIONED EQUAL TO}{}{} \\
\UnicodeOp{U+2261}{IDENTICAL TO}{$\equiv$}{\texttt{EQV}~~\texttt{EQUIV}} \\
\UnicodeOp{U+2263}{STRICTLY EQUIVALENT TO}{}{\texttt{===}~~\texttt{SEQV}} \\
\UnicodeOp{U+229C}{CIRCLED EQUALS}{}{} \\
\UnicodeOp{U+22CD}{REVERSED TILDE EQUALS}{$\backsimeq$}{} \\
\UnicodeOp{U+22D5}{EQUAL AND PARALLEL TO}{}{} \\
\UnicodeOp{U+29E3}{EQUALS SIGN AND SLANTED PARALLEL}{}{} \\
\UnicodeOp{U+29E4}{EQUALS SIGN AND SLANTED PARALLEL WITH TILDE ABOVE}{}{} \\
\UnicodeOp{U+29E5}{IDENTICAL TO AND SLANTED PARALLEL}{}{} \\
\UnicodeOp{U+2A66}{EQUALS SIGN WITH DOT BELOW}{}{} \\
\UnicodeOp{U+2A67}{IDENTICAL WITH DOT ABOVE}{}{} \\
\UnicodeOp{U+2A6C}{SIMILAR MINUS SIMILAR}{}{} \\
\UnicodeOp{U+2A6E}{EQUALS WITH ASTERISK}{}{} \\
\UnicodeOp{U+2A6F}{ALMOST EQUAL TO WITH CIRCUMFLEX ACCENT}{}{} \\
\UnicodeOp{U+2A70}{APPROXIMATELY EQUAL OR EQUAL TO}{}{} \\
\UnicodeOp{U+2A71}{EQUALS SIGN ABOVE PLUS SIGN}{}{} \\
\UnicodeOp{U+2A72}{PLUS SIGN ABOVE EQUALS SIGN}{}{} \\
\UnicodeOp{U+2A73}{EQUALS SIGN ABOVE TILDE OPERATOR}{}{} \\
\UnicodeOp{U+2A75}{TWO CONSECUTIVE EQUALS SIGNS}{}{} \\
\UnicodeOp{U+2A76}{THREE CONSECUTIVE EQUALS SIGNS}{}{} \\
\UnicodeOp{U+2A77}{EQUALS SIGN WITH TWO DOTS ABOVE AND TWO DOTS BELOW}{}{} \\
\UnicodeOp{U+2A78}{EQUIVALENT WITH FOUR DOTS ABOVE}{}{} \\
\UnicodeOp{U+2AAE}{EQUALS SIGN WITH BUMPY ABOVE}{}{} \\
\UnicodeOp{U+FE66}{SMALL EQUALS SIGN}{}{} \\
\UnicodeOp{U+FF1D}{FULLWIDTH EQUALS SIGN}{}{}
\end{tabbing}

The following are inequivalence operators.  They might not be chained
and they are nonassociative.
\begin{tabbing}
\UnicodeKillLine
\UnicodeOp{U+2244}{NOT ASYMPTOTICALLY EQUAL TO}{$\not\simeq$}{\texttt{NSIMEQ}} \\
\UnicodeOp{U+2247}{NEITHER APPROXIMATELY NOR ACTUALLY EQUAL TO}{$\ncong$}{} \\
\UnicodeOp{U+2249}{NOT ALMOST EQUAL TO}{$\not\approx$}{\texttt{NAPPROX}} \\
\UnicodeOp{U+2260}{NOT EQUAL TO}{$\neq$}{\texttt{=/=}~~\texttt{NE}} \\
\UnicodeOp{U+2262}{NOT IDENTICAL TO}{$\not\equiv$}{\texttt{NEQV}} \\
\UnicodeOp{U+226D}{NOT EQUIVALENT TO}{$\not\asymp$}{}
\end{tabbing}


\subsection{Plain Comparison Operators}

Every operator listed in this section
has lower precedence than any operator listed in
Sections~\ref{sec:precedence:multdivops}, \ref{sec:precedence:addsubops},
and \ref{sec:precedence:miscarithops}.

The following are less-than operators.  They may be mixed and chained
with each other and with equivalence operators
(\see{precedence:equivops}).
\begin{tabbing}
\UnicodeKillLine
\UnicodeOp{U+003C}{LESS-THAN SIGN}{$<$}{\texttt{<}~~\texttt{LT}} \\
\UnicodeOp{U+2264}{LESS-THAN OR EQUAL TO}{$\leq$}{\texttt{<=}~~\texttt{LE}} \\
\UnicodeOp{U+2266}{LESS-THAN OVER EQUAL TO}{$\leqq$}{} \\
\UnicodeOp{U+2268}{LESS-THAN BUT NOT EQUAL TO}{$\lneqq$}{} \\
\UnicodeOp{U+226A}{MUCH LESS-THAN}{$\ll$}{\texttt{<<}} \\
\UnicodeOp{U+2272}{LESS-THAN OR EQUIVALENT TO}{$\lesssim$}{} \\
\UnicodeOp{U+22D6}{LESS-THAN WITH DOT}{$\lessdot$}{\texttt{DOTLT}} \\
\UnicodeOp{U+22D8}{VERY MUCH LESS-THAN}{$\lll$}{\texttt{<<<}} \\
\UnicodeOp{U+22DC}{EQUAL TO OR LESS-THAN}{}{} \\
\UnicodeOp{U+22E6}{LESS-THAN BUT NOT EQUIVALENT TO}{$\lnsim$}{} \\
\UnicodeOp{U+29C0}{CIRCLED LESS-THAN}{}{} \\
\UnicodeOp{U+2A79}{LESS-THAN WITH CIRCLE INSIDE}{}{} \\
\UnicodeOp{U+2A7B}{LESS-THAN WITH QUESTION MARK ABOVE}{}{} \\
\UnicodeOp{U+2A7D}{LESS-THAN OR SLANTED EQUAL TO}{}{} \\
\UnicodeOp{U+2A7F}{LESS-THAN OR SLANTED EQUAL TO WITH DOT INSIDE}{}{} \\
\UnicodeOp{U+2A81}{LESS-THAN OR SLANTED EQUAL TO WITH DOT ABOVE}{}{} \\
\UnicodeOp{U+2A83}{LESS-THAN OR SLANTED EQUAL TO WITH DOT ABOVE RIGHT}{}{} \\
\UnicodeOp{U+2A85}{LESS-THAN OR APPROXIMATE}{}{} \\
\UnicodeOp{U+2A87}{LESS-THAN AND SINGLE-LINE NOT EQUAL TO}{}{} \\
\UnicodeOp{U+2A89}{LESS-THAN AND NOT APPROXIMATE}{}{} \\
\UnicodeOp{U+2A8D}{LESS-THAN ABOVE SIMILAR OR EQUAL}{}{} \\
\UnicodeOp{U+2A95}{SLANTED EQUAL TO OR LESS-THAN}{}{} \\
\UnicodeOp{U+2A97}{SLANTED EQUAL TO OR LESS-THAN WITH DOT INSIDE}{}{} \\
\UnicodeOp{U+2A99}{DOUBLE-LINE EQUAL TO OR LESS-THAN}{}{} \\
\UnicodeOp{U+2A9B}{DOUBLE-LINE SLANTED EQUAL TO OR LESS-THAN}{}{} \\
\UnicodeOp{U+2A9D}{SIMILAR OR LESS-THAN}{}{} \\
\UnicodeOp{U+2A9F}{SIMILAR ABOVE LESS-THAN ABOVE EQUALS SIGN}{}{} \\
\UnicodeOp{U+2AA1}{DOUBLE NESTED LESS-THAN}{}{} \\
\UnicodeOp{U+2AA3}{DOUBLE NESTED LESS-THAN WITH UNDERBAR}{}{} \\
\UnicodeOp{U+2AA6}{LESS-THAN CLOSED BY CURVE}{}{} \\
\UnicodeOp{U+2AA8}{LESS-THAN CLOSED BY CURVE ABOVE SLANTED EQUAL}{}{} \\
\UnicodeOp{U+2AF7}{TRIPLE NESTED LESS-THAN}{}{} \\
\UnicodeOp{U+2AF9}{DOUBLE-LINE SLANTED LESS-THAN OR EQUAL TO}{}{} \\
\UnicodeOp{U+FE64}{SMALL LESS-THAN SIGN}{}{} \\
\UnicodeOp{U+FF1C}{FULLWIDTH LESS-THAN SIGN}{}{}
\end{tabbing}

The following are greater-than operators.  They may be mixed and chained
with each other and with equivalence operators
(\see{precedence:equivops}).
\begin{tabbing}
\UnicodeKillLine
\UnicodeOp{U+003E}{GREATER-THAN SIGN}{$>$}{\texttt{>}~~\texttt{GT}} \\
\UnicodeOp{U+2265}{GREATER-THAN OR EQUAL TO}{$\geq$}{\texttt{>=}~~\texttt{GE}} \\
\UnicodeOp{U+2267}{GREATER-THAN OVER EQUAL TO}{$\geqq$}{} \\
\UnicodeOp{U+2269}{GREATER-THAN BUT NOT EQUAL TO}{$\gneqq$}{} \\
\UnicodeOp{U+226B}{MUCH GREATER-THAN}{$\gg$}{\texttt{>>}} \\
\UnicodeOp{U+2273}{GREATER-THAN OR EQUIVALENT TO}{$\gtrsim$}{} \\
\UnicodeOp{U+22D7}{GREATER-THAN WITH DOT}{$\gtrdot$}{\texttt{DOTGT}} \\
\UnicodeOp{U+22D9}{VERY MUCH GREATER-THAN}{$\ggg$}{\texttt{>>>}} \\
\UnicodeOp{U+22DD}{EQUAL TO OR GREATER-THAN}{}{} \\
\UnicodeOp{U+22E7}{GREATER-THAN BUT NOT EQUIVALENT TO}{$\gnsim$}{} \\
\UnicodeOp{U+29C1}{CIRCLED GREATER-THAN}{}{} \\
\UnicodeOp{U+2A7A}{GREATER-THAN WITH CIRCLE INSIDE}{}{} \\
\UnicodeOp{U+2A7C}{GREATER-THAN WITH QUESTION MARK ABOVE}{}{} \\
\UnicodeOp{U+2A7E}{GREATER-THAN OR SLANTED EQUAL TO}{}{} \\
\UnicodeOp{U+2A80}{GREATER-THAN OR SLANTED EQUAL TO WITH DOT INSIDE}{}{} \\
\UnicodeOp{U+2A82}{GREATER-THAN OR SLANTED EQUAL TO WITH DOT ABOVE}{}{} \\
\UnicodeOp{U+2A84}{GREATER-THAN OR SLANTED EQUAL TO WITH DOT ABOVE LEFT}{}{} \\
\UnicodeOp{U+2A86}{GREATER-THAN OR APPROXIMATE}{}{} \\
\UnicodeOp{U+2A88}{GREATER-THAN AND SINGLE-LINE NOT EQUAL TO}{}{} \\
\UnicodeOp{U+2A8A}{GREATER-THAN AND NOT APPROXIMATE}{}{} \\
\UnicodeOp{U+2A8E}{GREATER-THAN ABOVE SIMILAR OR EQUAL}{}{} \\
\UnicodeOp{U+2A96}{SLANTED EQUAL TO OR GREATER-THAN}{}{} \\
\UnicodeOp{U+2A98}{SLANTED EQUAL TO OR GREATER-THAN WITH DOT INSIDE}{}{} \\
\UnicodeOp{U+2A9A}{DOUBLE-LINE EQUAL TO OR GREATER-THAN}{}{} \\
\UnicodeOp{U+2A9C}{DOUBLE-LINE SLANTED EQUAL TO OR GREATER-THAN}{}{} \\
\UnicodeOp{U+2A9E}{SIMILAR OR GREATER-THAN}{}{} \\
\UnicodeOp{U+2AA0}{SIMILAR ABOVE GREATER-THAN ABOVE EQUALS SIGN}{}{} \\
\UnicodeOp{U+2AA2}{DOUBLE NESTED GREATER-THAN}{}{} \\
\UnicodeOp{U+2AA7}{GREATER-THAN CLOSED BY CURVE}{}{} \\
\UnicodeOp{U+2AA9}{GREATER-THAN CLOSED BY CURVE ABOVE SLANTED EQUAL}{}{} \\
\UnicodeOp{U+2AF8}{TRIPLE NESTED GREATER-THAN}{}{} \\
\UnicodeOp{U+2AFA}{DOUBLE-LINE SLANTED GREATER-THAN OR EQUAL TO}{}{} \\
\UnicodeOp{U+FE65}{SMALL GREATER-THAN SIGN}{}{} \\
\UnicodeOp{U+FF1E}{FULLWIDTH GREATER-THAN SIGN}{}{}
\end{tabbing}

The following are miscellaneous plain comparison operators.
They might not be mixed or chained and they are nonassociative.
\begin{tabbing}
\UnicodeKillLine
\UnicodeOp{U+226E}{NOT LESS-THAN}{$\nless$}{\texttt{NLT}} \\
\UnicodeOp{U+226F}{NOT GREATER-THAN}{$\ngtr$}{\texttt{NGT}} \\
\UnicodeOp{U+2270}{NEITHER LESS-THAN NOR EQUAL TO}{$\nleq$}{\texttt{NLE}} \\
\UnicodeOp{U+2271}{NEITHER GREATER-THAN NOR EQUAL TO}{$\ngeq$}{\texttt{NGE}} \\
\UnicodeOp{U+2274}{NEITHER LESS-THAN NOR EQUIVALENT TO}{$\not\lesssim$}{} \\
\UnicodeOp{U+2275}{NEITHER GREATER-THAN NOR EQUIVALENT TO}{$\not\gtrsim$}{} \\
\UnicodeOp{U+2276}{LESS-THAN OR GREATER-THAN}{$\lessgtr$}{} \\
\UnicodeOp{U+2277}{GREATER-THAN OR LESS-THAN}{$\gtrless$}{} \\
\UnicodeOp{U+2278}{NEITHER LESS-THAN NOR GREATER-THAN}{}{} \\
\UnicodeOp{U+2279}{NEITHER GREATER-THAN NOR LESS-THAN}{}{} \\
\UnicodeOp{U+22DA}{LESS-THAN EQUAL TO OR GREATER-THAN}{$\lesseqgtr$}{} \\
\UnicodeOp{U+22DB}{GREATER-THAN EQUAL TO OR LESS-THAN}{$\gtreqless$}{} \\
\UnicodeOp{U+2A8B}{LESS-THAN ABOVE DOUBLE-LINE EQUAL ABOVE GREATER-THAN}{}{} \\
\UnicodeOp{U+2A8C}{GREATER-THAN ABOVE DOUBLE-LINE EQUAL ABOVE LESS-THAN}{}{} \\
\UnicodeOp{U+2A8F}{LESS-THAN ABOVE SIMILAR ABOVE GREATER-THAN}{}{} \\
\UnicodeOp{U+2A90}{GREATER-THAN ABOVE SIMILAR ABOVE LESS-THAN}{}{} \\
\UnicodeOp{U+2A91}{LESS-THAN ABOVE GREATER-THAN ABOVE DOUBLE-LINE EQUAL}{}{} \\
\UnicodeOp{U+2A92}{GREATER-THAN ABOVE LESS-THAN ABOVE DOUBLE-LINE EQUAL}{}{} \\
\UnicodeOp{U+2A93}{LESS-THAN ABOVE SLANTED EQUAL ABOVE GREATER-THAN ABOVE SLANTED EQUAL}{}{} \\
\UnicodeOp{U+2A94}{GREATER-THAN ABOVE SLANTED EQUAL ABOVE LESS-THAN ABOVE SLANTED EQUAL}{}{} \\
\UnicodeOp{U+2AA4}{GREATER-THAN OVERLAPPING LESS-THAN}{}{} \\
\UnicodeOp{U+2AA5}{GREATER-THAN BESIDE LESS-THAN}{}{}
\end{tabbing}

The following are not really comparison operators, but it is convenient to list them here
because they also have lower precedence than any operator listed in
Sections~\ref{sec:precedence:multdivops}, \ref{sec:precedence:addsubops},
and \ref{sec:precedence:miscarithops}.
They are left associative and they have the same precedence.
\begin{tabbing}
\UnicodeKillLine
\UnicodeOp{U+0023}{NUMBER SIGN}{\#}{\texttt{\#}} \\
\UnicodeOp{U+003A}{COLON}{:}{\texttt{:}} \\
\UnicodeOp{U+2237}{PROPORTION}{}{\texttt{::}}
\end{tabbing}

\subsection{Set Comparison Operators}

Every operator listed in this section
has lower precedence than any operator listed in \secref{precedence:setops}.

The following are subset comparison operators.  They may be mixed and chained
with each other and with equivalence operators
(\see{precedence:equivops}).
\begin{tabbing}
\UnicodeKillLine
\UnicodeOp{U+2282}{SUBSET OF}{$\subset$}{\texttt{SUBSET}} \\
\UnicodeOp{U+2286}{SUBSET OF OR EQUAL TO}{$\subseteq$}{\texttt{SUBSETEQ}} \\
\UnicodeOp{U+228A}{SUBSET OF WITH NOT EQUAL TO}{$\subsetneq$}{\texttt{SUBSETNEQ}} \\
\UnicodeOp{U+22D0}{DOUBLE SUBSET}{$\Subset$}{\texttt{SUBSUB}} \\
\UnicodeOp{U+27C3}{OPEN SUBSET}{}{} \\
\UnicodeOp{U+2ABD}{SUBSET WITH DOT}{}{} \\
\UnicodeOp{U+2ABF}{SUBSET WITH PLUS SIGN BELOW}{}{} \\
\UnicodeOp{U+2AC1}{SUBSET WITH MULTIPLICATION SIGN BELOW}{}{} \\
\UnicodeOp{U+2AC3}{SUBSET OF OR EQUAL TO WITH DOT ABOVE}{}{} \\
\UnicodeOp{U+2AC5}{SUBSET OF ABOVE EQUALS SIGN}{}{} \\
\UnicodeOp{U+2AC7}{SUBSET OF ABOVE TILDE OPERATOR}{}{} \\
\UnicodeOp{U+2AC9}{SUBSET OF ABOVE ALMOST EQUAL TO}{}{} \\
\UnicodeOp{U+2ACB}{SUBSET OF ABOVE NOT EQUAL TO}{}{} \\
\UnicodeOp{U+2ACF}{CLOSED SUBSET}{}{} \\
\UnicodeOp{U+2AD1}{CLOSED SUBSET OR EQUAL TO}{}{} \\
\UnicodeOp{U+2AD5}{SUBSET ABOVE SUBSET}{}{}
\end{tabbing}

The following are superset comparison operators.  They may be mixed and chained
with each other and with equivalence operators
(\see{precedence:equivops}).
\begin{tabbing}
\UnicodeKillLine
\UnicodeOp{U+2283}{SUPERSET OF}{$\supset$}{\texttt{SUPSET}} \\
\UnicodeOp{U+2287}{SUPERSET OF OR EQUAL TO}{$\supseteq$}{\texttt{SUPSETEQ}} \\
\UnicodeOp{U+228B}{SUPERSET OF WITH NOT EQUAL TO}{$\supsetneq$}{\texttt{SUPSETNEQ}} \\
\UnicodeOp{U+22D1}{DOUBLE SUPERSET}{$\Supset$}{\texttt{SUPSUP}} \\
\UnicodeOp{U+27C4}{OPEN SUPERSET}{}{} \\
\UnicodeOp{U+2ABE}{SUPERSET WITH DOT}{}{} \\
\UnicodeOp{U+2AC0}{SUPERSET WITH PLUS SIGN BELOW}{}{} \\
\UnicodeOp{U+2AC2}{SUPERSET WITH MULTIPLICATION SIGN BELOW}{}{} \\
\UnicodeOp{U+2AC4}{SUPERSET OF OR EQUAL TO WITH DOT ABOVE}{}{} \\
\UnicodeOp{U+2AC6}{SUPERSET OF ABOVE EQUALS SIGN}{}{} \\
\UnicodeOp{U+2AC8}{SUPERSET OF ABOVE TILDE OPERATOR}{}{} \\
\UnicodeOp{U+2ACA}{SUPERSET OF ABOVE ALMOST EQUAL TO}{}{} \\
\UnicodeOp{U+2ACC}{SUPERSET OF ABOVE NOT EQUAL TO}{}{} \\
\UnicodeOp{U+2AD0}{CLOSED SUPERSET}{}{} \\
\UnicodeOp{U+2AD2}{CLOSED SUPERSET OR EQUAL TO}{}{} \\
\UnicodeOp{U+2AD6}{SUPERSET ABOVE SUPERSET}{}{}
\end{tabbing}

The following are miscellaneous set comparison operators.
They might not be mixed or chained and they are nonassociative.

\begin{tabbing}
\UnicodeKillLine
\UnicodeOp{U+2284}{NOT A SUBSET OF}{$\not\subset$}{\texttt{NSUBSET}} \\
\UnicodeOp{U+2285}{NOT A SUPERSET OF}{$\not\supset$}{\texttt{NSUPSET}} \\
\UnicodeOp{U+2288}{NEITHER A SUBSET OF NOR EQUAL TO}{$\nsubseteq$}{\texttt{NSUBSETEQ}} \\
\UnicodeOp{U+2289}{NEITHER A SUPERSET OF NOR EQUAL TO}{$\nsupseteq$}{\texttt{NSUPSETEQ}} \\
\UnicodeOp{U+2AD3}{SUBSET ABOVE SUPERSET}{}{} \\
\UnicodeOp{U+2AD4}{SUPERSET ABOVE SUBSET}{}{} \\
\UnicodeOp{U+2AD7}{SUPERSET BESIDE SUBSET}{}{} \\
\UnicodeOp{U+2AD8}{SUPERSET BESIDE AND JOINED BY DASH WITH SUBSET}{}{}
\end{tabbing}

\subsection{Square Comparison Operators}

Every operator listed in this section
has lower precedence than any operator listed in \secref{precedence:squareops}.

The following are square ``image of'' comparison operators.  They may be mixed and chained
with each other and with equivalence operators
(\see{precedence:equivops}).
\begin{tabbing}
\UnicodeKillLine
\UnicodeOp{U+228F}{SQUARE IMAGE OF}{$\sqsubset$}{\texttt{SQSUBSET}} \\
\UnicodeOp{U+2291}{SQUARE IMAGE OF OR EQUAL TO}{$\sqsubseteq$}{\texttt{SQSUBSETEQ}} \\
\UnicodeOp{U+22E4}{SQUARE IMAGE OF OR NOT EQUAL TO}{}{}
\end{tabbing}

The following are square ``original of'' comparison operators.  They may be mixed and chained
with each other and with equivalence operators (\see{precedence:equivops}).
\begin{tabbing}
\UnicodeKillLine
\UnicodeOp{U+2290}{SQUARE ORIGINAL OF}{$\sqsupset$}{\texttt{SQSUPSET}} \\
\UnicodeOp{U+2292}{SQUARE ORIGINAL OF OR EQUAL TO}{$\sqsupseteq$}{\texttt{SQSUPSETEQ}} \\
\UnicodeOp{U+22E5}{SQUARE ORIGINAL OF OR NOT EQUAL TO}{}{}
\end{tabbing}

The following are miscellaneous square comparison operators.
They might not be mixed or chained and they are nonassociative.
\begin{tabbing}
\UnicodeKillLine
\UnicodeOp{U+22E2}{NOT SQUARE IMAGE OF OR EQUAL TO}{$\not\sqsubseteq$}{} \\
\UnicodeOp{U+22E3}{NOT SQUARE ORIGINAL OF OR EQUAL TO}{$\not\sqsupseteq$}{}
\end{tabbing}

\subsection{Curly Comparison Operators}

Every operator listed in this section
has lower precedence than any operator listed in \secref{precedence:curlyops}.

The following are curly ``precedes'' comparison operators.  They may be mixed and chained
with each other and with equivalence operators (\see{precedence:equivops}).
\begin{tabbing}
\UnicodeKillLine
\UnicodeOp{U+227A}{PRECEDES}{$\prec$}{\texttt{PREC}} \\
\UnicodeOp{U+227C}{PRECEDES OR EQUAL TO}{$\preccurlyeq$}{\texttt{PRECEQ}} \\
\UnicodeOp{U+227E}{PRECEDES OR EQUIVALENT TO}{$\precsim$}{\texttt{PRECSIM}} \\
\UnicodeOp{U+22B0}{PRECEDES UNDER RELATION}{}{} \\
\UnicodeOp{U+22DE}{EQUAL TO OR PRECEDES}{$\curlyeqprec$}{\texttt{EQPREC}} \\
\UnicodeOp{U+22E8}{PRECEDES BUT NOT EQUIVALENT TO}{$\precnsim$}{\texttt{PRECNSIM}} \\
\UnicodeOp{U+2AAF}{PRECEDES ABOVE SINGLE-LINE EQUALS SIGN}{}{} \\
\UnicodeOp{U+2AB1}{PRECEDES ABOVE SINGLE-LINE NOT EQUAL TO}{}{} \\
\UnicodeOp{U+2AB3}{PRECEDES ABOVE EQUALS SIGN}{}{} \\
\UnicodeOp{U+2AB5}{PRECEDES ABOVE NOT EQUAL TO}{}{} \\
\UnicodeOp{U+2AB7}{PRECEDES ABOVE ALMOST EQUAL TO}{}{} \\
\UnicodeOp{U+2AB9}{PRECEDES ABOVE NOT ALMOST EQUAL TO}{}{} \\
\UnicodeOp{U+2ABB}{DOUBLE PRECEDES}{}{}
\end{tabbing}

The following are curly ``succeeds'' comparison operators.  They may be mixed and chained
with each other and with equivalence operators (\see{precedence:equivops}).
\begin{tabbing}
\UnicodeKillLine
\UnicodeOp{U+227B}{SUCCEEDS}{$\succ$}{\texttt{SUCC}} \\
\UnicodeOp{U+227D}{SUCCEEDS OR EQUAL TO}{$\succcurlyeq$}{\texttt{SUCCEQ}} \\
\UnicodeOp{U+227F}{SUCCEEDS OR EQUIVALENT TO}{$\succsim$}{\texttt{SUCCSIM}} \\
\UnicodeOp{U+22B1}{SUCCEEDS UNDER RELATION}{}{} \\
\UnicodeOp{U+22DF}{EQUAL TO OR SUCCEEDS}{$\curlyeqsucc$}{\texttt{EQSUCC}} \\
\UnicodeOp{U+22E9}{SUCCEEDS BUT NOT EQUIVALENT TO}{$\succnsim$}{\texttt{SUCCNSIM}} \\
\UnicodeOp{U+2AB0}{SUCCEEDS ABOVE SINGLE-LINE EQUALS SIGN}{}{} \\
\UnicodeOp{U+2AB2}{SUCCEEDS ABOVE SINGLE-LINE NOT EQUAL TO}{}{} \\
\UnicodeOp{U+2AB4}{SUCCEEDS ABOVE EQUALS SIGN}{}{} \\
\UnicodeOp{U+2AB6}{SUCCEEDS ABOVE NOT EQUAL TO}{}{} \\
\UnicodeOp{U+2AB8}{SUCCEEDS ABOVE ALMOST EQUAL TO}{}{} \\
\UnicodeOp{U+2ABA}{SUCCEEDS ABOVE NOT ALMOST EQUAL TO}{}{} \\
\UnicodeOp{U+2ABC}{DOUBLE SUCCEEDS}{}{}
\end{tabbing}

The following are miscellaneous curly comparison operators.
They might not be mixed or chained and they are nonassociative.
\begin{tabbing}
\UnicodeKillLine
\UnicodeOp{U+2280}{DOES NOT PRECEDE}{$\nprec$}{\texttt{NPREC}} \\
\UnicodeOp{U+2281}{DOES NOT SUCCEED}{$\nsucc$}{\texttt{NSUCC}} \\
\UnicodeOp{U+22E0}{DOES NOT PRECEDE OR EQUAL}{$\not\preccurlyeq$}{} \\
\UnicodeOp{U+22E1}{DOES NOT SUCCEED OR EQUAL}{$\not\succcurlyeq$}{}
\end{tabbing}

\subsection{Triangular Comparison Operators}

The following are triangular ``subgroup'' comparison operators.  They may be mixed and chained
with each other and with equivalence operators
(\see{precedence:equivops}).
\begin{tabbing}
\UnicodeKillLine
\UnicodeOp{U+22B2}{NORMAL SUBGROUP OF}{$\lhd$}{} \\
\UnicodeOp{U+22B4}{NORMAL SUBGROUP OF OR EQUAL TO}{$\unlhd$}{}
\end{tabbing}

The following are triangular ``contains as subgroup'' comparison operators.  They may be mixed and chained
with each other and with equivalence operators
(\see{precedence:equivops}).
\begin{tabbing}
\UnicodeKillLine
\UnicodeOp{U+22B3}{CONTAINS AS NORMAL SUBGROUP}{$\rhd$}{} \\
\UnicodeOp{U+22B5}{CONTAINS AS NORMAL SUBGROUP OR EQUAL TO}{$\unrhd$}{}
\end{tabbing}

The following are miscellaneous triangular comparison operators.
They might not be mixed or chained and they are nonassociative.
\begin{tabbing}
\UnicodeKillLine
\UnicodeOp{U+22EA}{NOT NORMAL SUBGROUP OF}{$\ntriangleleft$}{} \\
\UnicodeOp{U+22EB}{DOES NOT CONTAIN AS NORMAL SUBGROUP}{$\ntriangleright$}{} \\
\UnicodeOp{U+22EC}{NOT NORMAL SUBGROUP OF OR EQUAL TO}{$\ntrianglelefteq$}{} \\
\UnicodeOp{U+22ED}{DOES NOT CONTAIN AS NORMAL SUBGROUP OR EQUAL}{$\ntrianglerighteq$}{}
\end{tabbing}

\subsection{Chickenfoot Comparison Operators}

The following are chickenfoot ``smaller than'' comparison operators.  They may be mixed and chained
with each other and with equivalence operators (\see{precedence:equivops}).
\begin{tabbing}
\UnicodeKillLine
\UnicodeOp{U+2AAA}{SMALLER THAN}{$<\!\llap{$-$}$}{\texttt{SMALLER}} \\
\UnicodeOp{U+2AAC}{SMALLER THAN OR EQUAL TO}{$\leq\!\llap{\raisebox{.15ex}[0cm][0cm]{$-$}}$}{\texttt{SMALLEREQ}}
\end{tabbing}

The following are chickenfoot ``larger than'' comparison operators.  They may be mixed and chained
with each other and with equivalence operators (\see{precedence:equivops}).
\begin{tabbing}
\UnicodeKillLine
\UnicodeOp{U+2AAB}{LARGER THAN}{$\rlap{$-$}\!>$}{\texttt{LARGER}} \\
\UnicodeOp{U+2AAD}{LARGER THAN OR EQUAL TO}{$\rlap{\raisebox{.15ex}[0cm][0cm]{$-$}}\!\geq$}{\texttt{LARGEREQ}}
\end{tabbing}


\subsection{Miscellaneous Relational Operators}

The following operators are considered to be relational operators, having
higher precedence than certain boolean operators,
as described in a later section.  They are nonassociative.
\begin{tabbing}
\UnicodeKillLine
\UnicodeOp{U+2208}{ELEMENT OF}{$\in$}{\texttt{IN}} \\
\UnicodeOp{U+2209}{NOT AN ELEMENT OF}{$\notin$}{\texttt{NOTIN}} \\
\UnicodeOp{U+220A}{SMALL ELEMENT OF}{$\hbox{\footnotesize$\in$}$}{} \\
\UnicodeOp{U+220B}{CONTAINS AS MEMBER}{$\ni$}{\texttt{CONTAINS}} \\
\UnicodeOp{U+220C}{DOES NOT CONTAIN AS MEMBER}{$\not\ni$}{} \\
\UnicodeOp{U+220D}{SMALL CONTAINS AS MEMBER}{$\hbox{\footnotesize$\ni$}$}{} \\
\UnicodeOp{U+22F2}{ELEMENT OF WITH LONG HORIZONTAL STROKE}{}{} \\
\UnicodeOp{U+22F3}{ELEMENT OF WITH VERTICAL BAR AT END OF HORIZONTAL STROKE}{}{} \\
\UnicodeOp{U+22F4}{SMALL ELEMENT OF WITH VERTICAL BAR AT END OF HORIZONTAL STROKE}{}{} \\
\UnicodeOp{U+22F5}{ELEMENT OF WITH DOT ABOVE}{$\dot{\in}$}{} \\
\UnicodeOp{U+22F6}{ELEMENT OF WITH OVERBAR}{$\overline{\in}$}{} \\
\UnicodeOp{U+22F7}{SMALL ELEMENT OF WITH OVERBAR}{$\hbox{\footnotesize$\overline{\in}$}$}{} \\
\UnicodeOp{U+22F8}{ELEMENT OF WITH UNDERBAR}{$\underline{\in}$}{} \\
\UnicodeOp{U+22F9}{ELEMENT OF WITH TWO HORIZONTAL STROKES}{}{} \\
\UnicodeOp{U+22FA}{CONTAINS WITH LONG HORIZONTAL STROKE}{}{} \\
\UnicodeOp{U+22FB}{CONTAINS WITH VERTICAL BAR AT END OF HORIZONTAL STROKE}{}{} \\
\UnicodeOp{U+22FC}{SMALL CONTAINS WITH VERTICAL BAR AT END OF HORIZONTAL STROKE}{}{} \\
\UnicodeOp{U+22FD}{CONTAINS WITH OVERBAR}{$\overline{\ni}$}{} \\
\UnicodeOp{U+22FE}{SMALL CONTAINS WITH OVERBAR}{$\hbox{\footnotesize$\overline{\ni}$}$}{} \\
\UnicodeOp{U+22FF}{Z NOTATION BAG MEMBERSHIP}{}{}
\end{tabbing}


\section{Boolean Operators}

Every operator listed in this section
has lower precedence than any operator listed in
\secref{precedence:relops}.

The following are the boolean conjunction operators.
They are left associative.
\begin{tabbing}
\UnicodeKillLine
\UnicodeOp{U+2227}{LOGICAL AND}{$\wedge$}{\texttt{AND}} \\
\UnicodeOp{U+27D1}{AND WITH DOT}{}{} \\
\UnicodeOp{U+2A51}{LOGICAL AND WITH DOT ABOVE}{$\dot{\wedge}$}{} \\
\UnicodeOp{U+2A53}{DOUBLE LOGICAL AND}{}{} \\
\UnicodeOp{U+2A55}{TWO INTERSECTING LOGICAL AND}{$\twointersectand$}{} \\
\UnicodeOp{U+2A5A}{LOGICAL AND WITH MIDDLE STEM}{}{} \\
\UnicodeOp{U+2A5C}{LOGICAL AND WITH HORIZONTAL DASH}{}{} \\
\UnicodeOp{U+2A5E}{LOGICAL AND WITH DOUBLE OVERBAR}{}{} \\
\UnicodeOp{U+2A60}{LOGICAL AND WITH DOUBLE UNDERBAR}{}{}
\end{tabbing}

The following are the boolean disjunction operators.
They are left associative.
\begin{tabbing}
\UnicodeKillLine
\UnicodeOp{U+2228}{LOGICAL OR}{$\vee$}{\texttt{OR}} \\
\UnicodeOp{U+2A52}{LOGICAL OR WITH DOT ABOVE}{$\dot{\vee}$}{} \\
\UnicodeOp{U+2A54}{DOUBLE LOGICAL OR}{}{} \\
\UnicodeOp{U+2A56}{TWO INTERSECTING LOGICAL OR}{$\twointersector$}{} \\
\UnicodeOp{U+2A5B}{LOGICAL OR WITH MIDDLE STEM}{}{} \\
\UnicodeOp{U+2A5D}{LOGICAL OR WITH HORIZONTAL DASH}{}{} \\
\UnicodeOp{U+2A62}{LOGICAL OR WITH DOUBLE OVERBAR}{}{} \\
\UnicodeOp{U+2A63}{LOGICAL OR WITH DOUBLE UNDERBAR}{}{}
\end{tabbing}
They each have lower precedence than any of the boolean conjunction operators.

The following are miscellaneous boolean operators that are nonassociative.
\begin{tabbing}
\UnicodeKillLine
\UnicodeOp{U+2192}{RIGHTWARDS ARROW}{$\rightarrow$}{\texttt{->}~~\texttt{IMPLIES}} \\
\UnicodeOp{U+2194}{LEFT RIGHT ARROW}{$\leftrightarrow$}{\texttt{<->}~~\texttt{IFF}} \\
\UnicodeOp{U+22BC}{NAND}{$\barwedge$}{} \\
\UnicodeOp{U+22BD}{NOR}{$\overline{\vee}$}{}
\end{tabbing}


The following is a miscellaneous boolean operator that is left associative.
\begin{tabbing}
\UnicodeKillLine
\UnicodeOp{U+22BB}{XOR}{$\veebar$}{} \\
\end{tabbing}

\section{Other Operators}
All the operators listed in this section are nonassociative.

As specified in \secref{operator-precedence},
the following operator has higher precedence than any other operator.

\begin{tabbing}
\UnicodeKillLine
\UnicodeOp{U+005E}{CIRCUMFLEX ACCENT}{$\texttt{{\char'136}}$}{\texttt{\char'136}} \\
\end{tabbing}

Each of the following operators has no defined precedence relationships
to any of the other operators listed in this appendix.
\begin{tabbing}
\UnicodeKillLine
\UnicodeOp{U+0021}{EXCLAMATION MARK}{!}{\texttt{!}} \\
\UnicodeOp{U+0024}{DOLLAR SIGN}{\$}{\texttt{\$}} \\
\UnicodeOp{U+0025}{PERCENT SIGN}{\%}{\texttt{\%}} \\
\UnicodeOp{U+003F}{QUESTION MARK}{?}{\texttt{?}} \\
\UnicodeOp{U+0040}{COMMERCIAL AT}{@}{\texttt{@}} \\
\UnicodeOp{U+007E}{TILDE}{{\char'176}}{\texttt{\char'176}} \\
\UnicodeOp{U+00A1}{INVERTED EXCLAMATION MARK}{\char'74}{} \\
\UnicodeOp{U+00A2}{CENT SIGN}{}{\texttt{CENTS}} \\
\UnicodeOp{U+00A3}{POUND SIGN}{}{} \\
\UnicodeOp{U+00A4}{CURRENCY SIGN}{}{} \\
\UnicodeOp{U+00A5}{YEN SIGN}{}{} \\
\UnicodeOp{U+00A6}{BROKEN BAR}{}{} \\
\UnicodeOp{U+00AC}{NOT SIGN}{$\neg$}{\texttt{NOT}} \\
\UnicodeOp{U+00B0}{DEGREE SIGN}{${}^\circ$}{\texttt{DEGREES}} \\
\UnicodeOp{U+00BF}{INVERTED QUESTION MARK}{\char'76}{} \\
\UnicodeOp{U+2020}{DAGGER}{$\dagger$}{} \\
\UnicodeOp{U+2021}{DOUBLE DAGGER}{$\ddagger$}{} \\
\UnicodeOp{U+203C}{DOUBLE EXCLAMATION MARK}{$!!$}{\texttt{!!}} \\
\UnicodeOp{U+2190}{LEFTWARDS ARROW}{$\leftarrow$}{\texttt{<-}} \\
\UnicodeOp{U+2191}{UPWARDS ARROW}{$\uparrow$}{\texttt{UPARROW}} \\
\UnicodeOp{U+2193}{DOWNWARDS ARROW}{$\downarrow$}{\texttt{DOWNARROW}} \\
\UnicodeOp{U+2195}{UP DOWN ARROW}{$\updownarrow$}{\texttt{UPDOWNARROW}} \\
\UnicodeOp{U+2196}{NORTH WEST ARROW}{$\nwarrow$}{\texttt{NWARROW}} \\
\UnicodeOp{U+2197}{NORTH EAST ARROW}{$\nearrow$}{\texttt{NEARROW}} \\
\UnicodeOp{U+2198}{SOUTH EAST ARROW}{$\searrow$}{\texttt{SEARROW}} \\
\UnicodeOp{U+2199}{SOUTH WEST ARROW}{$\swarrow$}{\texttt{SWARROW}} \\
\UnicodeOp{U+219A}{LEFTWARDS ARROW WITH STROKE}{$\nleftarrow$}{\texttt{<-/-}} \\
\UnicodeOp{U+219B}{RIGHTWARDS ARROW WITH STROKE}{$\nrightarrow$}{\texttt{-/->}} \\
\UnicodeOp{U+219C}{LEFTWARDS WAVE ARROW}{}{} \\
\UnicodeOp{U+219D}{RIGHTWARDS WAVE ARROW}{$\leadsto$}{\texttt{LEADSTO}} \\
\UnicodeOp{U+219E}{LEFTWARDS TWO HEADED ARROW}{}{} \\
\UnicodeOp{U+219F}{UPWARDS TWO HEADED ARROW}{}{} \\
\UnicodeOp{U+21A0}{RIGHTWARDS TWO HEADED ARROW}{}{} \\
\UnicodeOp{U+21A1}{DOWNWARDS TWO HEADED ARROW}{}{} \\
\UnicodeOp{U+21A2}{LEFTWARDS ARROW WITH TAIL}{}{} \\
\UnicodeOp{U+21A3}{RIGHTWARDS ARROW WITH TAIL}{}{} \\
\UnicodeOp{U+21A4}{LEFTWARDS ARROW FROM BAR}{}{} \\
\UnicodeOp{U+21A5}{UPWARDS ARROW FROM BAR}{}{} \\
\UnicodeOp{U+21A6}{RIGHTWARDS ARROW FROM BAR}{$\mapsto$}{\texttt{MAPSTO}~~\texttt{|->}} \\
\UnicodeOp{U+21A7}{DOWNWARDS ARROW FROM BAR}{}{} \\
\UnicodeOp{U+21A8}{UP DOWN ARROW WITH BASE}{}{} \\
\UnicodeOp{U+21A9}{LEFTWARDS ARROW WITH HOOK}{}{} \\
\UnicodeOp{U+21AA}{RIGHTWARDS ARROW WITH HOOK}{}{} \\
\UnicodeOp{U+21AB}{LEFTWARDS ARROW WITH LOOP}{}{} \\
\UnicodeOp{U+21AC}{RIGHTWARDS ARROW WITH LOOP}{}{} \\
\UnicodeOp{U+21AD}{LEFT RIGHT WAVE ARROW}{}{} \\
\UnicodeOp{U+21AE}{LEFT RIGHT ARROW WITH STROKE}{}{} \\
\UnicodeOp{U+21AF}{DOWNWARDS ZIGZAG ARROW}{}{} \\
\UnicodeOp{U+21B0}{UPWARDS ARROW WITH TIP LEFTWARDS}{}{} \\
\UnicodeOp{U+21B1}{UPWARDS ARROW WITH TIP RIGHTWARDS}{}{} \\
\UnicodeOp{U+21B2}{DOWNWARDS ARROW WITH TIP LEFTWARDS}{}{} \\
\UnicodeOp{U+21B3}{DOWNWARDS ARROW WITH TIP RIGHTWARDS}{}{} \\
\UnicodeOp{U+21B4}{RIGHTWARDS ARROW WITH CORNER DOWNWARDS}{}{} \\
\UnicodeOp{U+21B5}{DOWNWARDS ARROW WITH CORNER LEFTWARDS}{}{} \\
\UnicodeOp{U+21B6}{ANTICLOCKWISE TOP SEMICIRCLE ARROW}{}{} \\
\UnicodeOp{U+21B7}{CLOCKWISE TOP SEMICIRCLE ARROW}{}{} \\
\UnicodeOp{U+21B8}{NORTH WEST ARROW TO LONG BAR}{}{} \\
\UnicodeOp{U+21B9}{LEFTWARDS ARROW TO BAR OVER RIGHTWARDS ARROW TO BAR}{}{} \\
\UnicodeOp{U+21BA}{ANTICLOCKWISE OPEN CIRCLE ARROW}{}{} \\
\UnicodeOp{U+21BB}{CLOCKWISE OPEN CIRCLE ARROW}{}{} \\
\UnicodeOp{U+21BC}{LEFTWARDS HARPOON WITH BARB UPWARDS}{$\leftharpoonup$}{\texttt{LEFTHARPOONUP}} \\
\UnicodeOp{U+21BD}{LEFTWARDS HARPOON WITH BARB DOWNWARDS}{$\leftharpoondown$}{\texttt{LEFTHARPOONDOWN}} \\
\UnicodeOp{U+21BE}{UPWARDS HARPOON WITH BARB RIGHTWARDS}{$\upharpoonright$}{\texttt{UPHARPOONRIGHT}} \\
\UnicodeOp{U+21BF}{UPWARDS HARPOON WITH BARB LEFTWARDS}{$\upharpoonleft$}{\texttt{UPHARPOONLEFT}} \\
\UnicodeOp{U+21C0}{RIGHTWARDS HARPOON WITH BARB UPWARDS}{$\rightharpoonup$}{\texttt{RIGHTHARPOONUP}} \\
\UnicodeOp{U+21C1}{RIGHTWARDS HARPOON WITH BARB DOWNWARDS}{$\rightharpoondown$}{\texttt{RIGHTHARPOONDOWN}} \\
\UnicodeOp{U+21C2}{DOWNWARDS HARPOON WITH BARB RIGHTWARDS}{$\downharpoonright$}{\texttt{DOWNHARPOONRIGHT}} \\
\UnicodeOp{U+21C3}{DOWNWARDS HARPOON WITH BARB LEFTWARDS}{$\downharpoonleft$}{\texttt{DOWNHARPOONLEFT}} \\
\UnicodeOp{U+21C4}{RIGHTWARDS ARROW OVER LEFTWARDS ARROW}{$\rightleftarrows$}{\texttt{RIGHTLEFTARROWS}} \\
\UnicodeOp{U+21C5}{UPWARDS ARROW LEFTWARDS OF DOWNWARDS ARROW}{}{} \\
\UnicodeOp{U+21C6}{LEFTWARDS ARROW OVER RIGHTWARDS ARROW}{$\leftrightarrows$}{\texttt{LEFTRIGHTARROWS}} \\
\UnicodeOp{U+21C7}{LEFTWARDS PAIRED ARROWS}{$\leftleftarrows$}{\texttt{LEFTLEFTARROWS}} \\
\UnicodeOp{U+21C8}{UPWARDS PAIRED ARROWS}{$\upuparrows$}{\texttt{UPUPARROWS}} \\
\UnicodeOp{U+21C9}{RIGHTWARDS PAIRED ARROWS}{$\rightrightarrows$}{\texttt{RIGHTRIGHTARROWS}} \\
\UnicodeOp{U+21CA}{DOWNWARDS PAIRED ARROWS}{$\downdownarrows$}{\texttt{DOWNDOWNARROWS}} \\
\UnicodeOp{U+21CB}{LEFTWARDS HARPOON OVER RIGHTWARDS HARPOON}{}{} \\
\UnicodeOp{U+21CC}{RIGHTWARDS HARPOON OVER LEFTWARDS HARPOON}{$\rightleftharpoons$}{\texttt{RIGHTLEFTHARPOONS}} \\
\UnicodeOp{U+21CD}{LEFTWARDS DOUBLE ARROW WITH STROKE}{$\nLeftarrow$}{\texttt{}} \\
\UnicodeOp{U+21CE}{LEFT RIGHT DOUBLE ARROW WITH STROKE}{$\nLeftrightarrow$}{\texttt{}} \\
\UnicodeOp{U+21CF}{RIGHTWARDS DOUBLE ARROW WITH STROKE}{$\nRightarrow$}{\texttt{}} \\
\UnicodeOp{U+21D0}{LEFTWARDS DOUBLE ARROW}{$\Leftarrow$}{} \\
\UnicodeOp{U+21D1}{UPWARDS DOUBLE ARROW}{$\Uparrow$}{} \\
\UnicodeOp{U+21D2}{RIGHTWARDS DOUBLE ARROW}{$\Rightarrow$}{\texttt{=>}} \\
\UnicodeOp{U+21D3}{DOWNWARDS DOUBLE ARROW}{$\Downarrow$}{} \\
\UnicodeOp{U+21D4}{LEFT RIGHT DOUBLE ARROW}{$\Leftrightarrow$}{\texttt{<=>}} \\
\UnicodeOp{U+21D5}{UP DOWN DOUBLE ARROW}{$\Updownarrow$}{} \\
\UnicodeOp{U+21D6}{NORTH WEST DOUBLE ARROW}{}{} \\
\UnicodeOp{U+21D7}{NORTH EAST DOUBLE ARROW}{}{} \\
\UnicodeOp{U+21D8}{SOUTH EAST DOUBLE ARROW}{}{} \\
\UnicodeOp{U+21D9}{SOUTH WEST DOUBLE ARROW}{}{} \\
\UnicodeOp{U+21DA}{LEFTWARDS TRIPLE ARROW}{$\Lleftarrow$}{} \\
\UnicodeOp{U+21DB}{RIGHTWARDS TRIPLE ARROW}{$\Rrightarrow$}{} \\
\UnicodeOp{U+21DC}{LEFTWARDS SQUIGGLE ARROW}{}{} \\
\UnicodeOp{U+21DD}{RIGHTWARDS SQUIGGLE ARROW}{$\rightsquigarrow$}{} \\
\UnicodeOp{U+21DE}{UPWARDS ARROW WITH DOUBLE STROKE}{}{} \\
\UnicodeOp{U+21DF}{DOWNWARDS ARROW WITH DOUBLE STROKE}{}{} \\
\UnicodeOp{U+21E0}{LEFTWARDS DASHED ARROW}{$\dashleftarrow$}{} \\
\UnicodeOp{U+21E1}{UPWARDS DASHED ARROW}{}{} \\
\UnicodeOp{U+21E2}{RIGHTWARDS DASHED ARROW}{$\dashrightarrow$}{} \\
\UnicodeOp{U+21E3}{DOWNWARDS DASHED ARROW}{}{} \\
\UnicodeOp{U+21E4}{LEFTWARDS ARROW TO BAR}{}{} \\
\UnicodeOp{U+21E5}{RIGHTWARDS ARROW TO BAR}{}{} \\
\UnicodeOp{U+21E6}{LEFTWARDS WHITE ARROW}{}{} \\
\UnicodeOp{U+21E7}{UPWARDS WHITE ARROW}{}{} \\
\UnicodeOp{U+21E8}{RIGHTWARDS WHITE ARROW}{}{} \\
\UnicodeOp{U+21E9}{DOWNWARDS WHITE ARROW}{}{} \\
\UnicodeOp{U+21EA}{UPWARDS WHITE ARROW FROM BAR}{}{} \\
\UnicodeOp{U+21EB}{UPWARDS WHITE ARROW ON PEDESTAL}{}{} \\
\UnicodeOp{U+21EC}{UPWARDS WHITE ARROW ON PEDESTAL WITH HORIZONTAL BAR}{}{} \\
\UnicodeOp{U+21ED}{UPWARDS WHITE ARROW ON PEDESTAL WITH VERTICAL BAR}{}{} \\
\UnicodeOp{U+21EE}{UPWARDS WHITE DOUBLE ARROW}{}{} \\
\UnicodeOp{U+21EF}{UPWARDS WHITE DOUBLE ARROW ON PEDESTAL}{}{} \\
\UnicodeOp{U+21F0}{RIGHTWARDS WHITE ARROW FROM WALL}{}{} \\
\UnicodeOp{U+21F1}{NORTH WEST ARROW TO CORNER}{}{} \\
\UnicodeOp{U+21F2}{SOUTH EAST ARROW TO CORNER}{}{} \\
\UnicodeOp{U+21F3}{UP DOWN WHITE ARROW}{}{} \\
\UnicodeOp{U+21F4}{RIGHT ARROW WITH SMALL CIRCLE}{}{} \\
\UnicodeOp{U+21F5}{DOWNWARDS ARROW LEFTWARDS OF UPWARDS ARROW}{}{} \\
\UnicodeOp{U+21F6}{THREE RIGHTWARDS ARROWS}{}{} \\
\UnicodeOp{U+21F7}{LEFTWARDS ARROW WITH VERTICAL STROKE}{}{} \\
\UnicodeOp{U+21F8}{RIGHTWARDS ARROW WITH VERTICAL STROKE}{}{} \\
\UnicodeOp{U+21F9}{LEFT RIGHT ARROW WITH VERTICAL STROKE}{}{} \\
\UnicodeOp{U+21FA}{LEFTWARDS ARROW WITH DOUBLE VERTICAL STROKE}{}{} \\
\UnicodeOp{U+21FB}{RIGHTWARDS ARROW WITH DOUBLE VERTICAL STROKE}{}{} \\
\UnicodeOp{U+21FC}{LEFT RIGHT ARROW WITH DOUBLE VERTICAL STROKE}{}{} \\
\UnicodeOp{U+21FD}{LEFTWARDS OPEN-HEADED ARROW}{}{} \\
\UnicodeOp{U+21FE}{RIGHTWARDS OPEN-HEADED ARROW}{}{} \\
\UnicodeOp{U+21FF}{LEFT RIGHT OPEN-HEADED ARROW}{}{} \\
\UnicodeOp{U+2201}{COMPLEMENT}{$\complement$}{} \\
\UnicodeOp{U+2202}{PARTIAL DIFFERENTIAL}{$\partial$}{\texttt{DEL}} \\
\UnicodeOp{U+2204}{THERE DOES NOT EXIST}{$\not\exists$}{} \\
\UnicodeOp{U+2206}{INCREMENT}{$\Delta$}{} \\
\UnicodeOp{U+220F}{N-ARY PRODUCT}{$\prod$}{\texttt{PROD}} \\
\UnicodeOp{U+2210}{N-ARY COPRODUCT}{$\coprod$}{\texttt{COPROD}} \\
\UnicodeOp{U+2211}{N-ARY SUMMATION}{$\sum$}{\texttt{SUM}} \\
\UnicodeOp{U+2218}{RING OPERATOR}{$\circ$}{\texttt{CIRC}~~\texttt{RING}~~\texttt{COMPOSE}} \\
\UnicodeOp{U+2219}{BULLET OPERATOR}{$\bullet$}{} \\
\UnicodeOp{U+221A}{SQUARE ROOT}{$\surd$}{\texttt{SQRT}} \\
\UnicodeOp{U+221B}{CUBE ROOT}{}{\texttt{CBRT}} \\
\UnicodeOp{U+221C}{FOURTH ROOT}{}{\texttt{FOURTHROOT}} \\
\UnicodeOp{U+221D}{PROPORTIONAL TO}{$\propto$}{\texttt{PROPTO}} \\
\UnicodeOp{U+2223}{DIVIDES}{$\mid$}{\texttt{DIVIDES}} \\
\UnicodeOp{U+2224}{DOES NOT DIVIDE}{$\nmid$}{} \\
\UnicodeOp{U+2225}{PARALLEL TO}{$\parallel$}{\texttt{PARALLEL}} \\
\UnicodeOp{U+2226}{NOT PARALLEL TO}{$\nparallel$}{\texttt{NPARALLEL}} \\
\UnicodeOp{U+222B}{INTEGRAL}{$\int$}{} \\
\UnicodeOp{U+222C}{DOUBLE INTEGRAL}{}{} \\
\UnicodeOp{U+222D}{TRIPLE INTEGRAL}{}{} \\
\UnicodeOp{U+222E}{CONTOUR INTEGRAL}{$\oint$}{} \\
\UnicodeOp{U+222F}{SURFACE INTEGRAL}{}{} \\
\UnicodeOp{U+2230}{VOLUME INTEGRAL}{}{} \\
\UnicodeOp{U+2231}{CLOCKWISE INTEGRAL}{}{} \\
\UnicodeOp{U+2232}{CLOCKWISE CONTOUR INTEGRAL}{}{} \\
\UnicodeOp{U+2233}{ANTICLOCKWISE CONTOUR INTEGRAL}{}{} \\
\UnicodeOp{U+2234}{THEREFORE}{$\therefore$}{} \\
\UnicodeOp{U+2235}{BECAUSE}{$\because$}{} \\
\UnicodeOp{U+2236}{RATIO}{}{} \\
\UnicodeOp{U+2239}{EXCESS}{}{} \\
\UnicodeOp{U+223A}{GEOMETRIC PROPORTION}{}{} \\
\UnicodeOp{U+223B}{HOMOTHETIC}{}{} \\
\UnicodeOp{U+223C}{TILDE OPERATOR}{$\sim$}{} \\
\UnicodeOp{U+223D}{REVERSED TILDE}{$\backsim$}{} \\
\UnicodeOp{U+223E}{INVERTED LAZY S}{}{} \\
\UnicodeOp{U+223F}{SINE WAVE}{}{} \\
\UnicodeOp{U+2240}{WREATH PRODUCT}{$\wr$}{\texttt{WREATH}} \\
\UnicodeOp{U+2241}{NOT TILDE}{$\nsim$}{} \\
\UnicodeOp{U+224B}{TRIPLE TILDE}{}{} \\
\UnicodeOp{U+224F}{DIFFERENCE BETWEEN}{$\bumpeq$}{\texttt{BUMPEQ}} \\
\UnicodeOp{U+2250}{APPROACHES THE LIMIT}{$\doteq$}{\texttt{DOTEQ}} \\
\UnicodeOp{U+2258}{CORRESPONDS TO}{}{} \\
\UnicodeOp{U+2259}{ESTIMATES}{}{} \\
\UnicodeOp{U+225A}{EQUIANGULAR TO}{}{} \\
\UnicodeOp{U+225E}{MEASURED BY}{}{} \\
\UnicodeOp{U+226C}{BETWEEN}{$\between$}{} \\
\UnicodeOp{U+228C}{MULTISET}{}{} \\
\UnicodeOp{U+229A}{CIRCLED RING OPERATOR}{$\circledcirc$}{\texttt{CIRCLEDRING}} \\
\UnicodeOp{U+229D}{CIRCLED DASH}{$\circleddash$}{} \\
\UnicodeOp{U+22A2}{RIGHT TACK}{$\vdash$}{\texttt{VDASH}~~\texttt{TURNSTILE}} \\
\UnicodeOp{U+22A3}{LEFT TACK}{$\dashv$}{\texttt{DASHV}} \\
\UnicodeOp{U+22A6}{ASSERTION}{$\vdash$}{} \\
\UnicodeOp{U+22A7}{MODELS}{$\vDash$}{} \\
\UnicodeOp{U+22A8}{TRUE}{$\models$}{} \\
\UnicodeOp{U+22A9}{FORCES}{$\Vdash$}{} \\
\UnicodeOp{U+22AA}{TRIPLE VERTICAL BAR RIGHT TURNSTILE}{$\Vvdash$}{} \\
\UnicodeOp{U+22AB}{DOUBLE VERTICAL BAR DOUBLE RIGHT TURNSTILE}{}{} \\
\UnicodeOp{U+22AC}{DOES NOT PROVE}{$\nvdash$}{} \\
\UnicodeOp{U+22AD}{NOT TRUE}{}{} \\
\UnicodeOp{U+22AE}{DOES NOT FORCE}{$\nVdash$}{} \\
\UnicodeOp{U+22AF}{NEGATED DOUBLE VERTICAL BAR DOUBLE RIGHT TURNSTILE}{$\nVDash$}{} \\
\UnicodeOp{U+22B6}{ORIGINAL OF}{}{} \\
\UnicodeOp{U+22B7}{IMAGE OF}{}{} \\
\UnicodeOp{U+22B8}{MULTIMAP}{$\multimap$}{} \\
\UnicodeOp{U+22B9}{HERMITIAN CONJUGATE MATRIX}{}{} \\
\UnicodeOp{U+22BA}{INTERCALATE}{$\intercal$}{} \\
\UnicodeOp{U+22BE}{RIGHT ANGLE WITH ARC}{}{} \\
\UnicodeOp{U+22BF}{RIGHT TRIANGLE}{}{} \\
\UnicodeOp{U+22C0}{N-ARY LOGICAL AND}{$\bigwedge$}{\texttt{BIGAND}~~\texttt{ALL}} \\
\UnicodeOp{U+22C1}{N-ARY LOGICAL OR}{$\bigvee$}{\texttt{BIGOR}~~\texttt{ANY}} \\
\UnicodeOp{U+22C2}{N-ARY INTERSECTION}{$\bigcap$}{\texttt{BIGCAP}~~\texttt{BIGINTERSECT}} \\
\UnicodeOp{U+22C3}{N-ARY UNION}{$\bigcup$}{\texttt{BIGCUP}~~\texttt{BIGUNION}} \\
\UnicodeOp{U+22C4}{DIAMOND OPERATOR}{$\diamond$}{\texttt{DIAMOND}} \\
\UnicodeOp{U+22C6}{STAR OPERATOR}{$\star$}{\texttt{STAR}} \\
\UnicodeOp{U+22C7}{DIVISION TIMES}{$\divideontimes$}{} \\
\UnicodeOp{U+22C8}{BOWTIE}{$\bowtie$}{} \\
\UnicodeOp{U+22C9}{LEFT NORMAL FACTOR SEMIDIRECT PRODUCT}{$\ltimes$}{} \\
\UnicodeOp{U+22CA}{RIGHT NORMAL FACTOR SEMIDIRECT PRODUCT}{$\rtimes$}{} \\
\UnicodeOp{U+22CB}{LEFT SEMIDIRECT PRODUCT}{$\leftthreetimes$}{} \\
\UnicodeOp{U+22CC}{RIGHT SEMIDIRECT PRODUCT}{$\rightthreetimes$}{} \\
\UnicodeOp{U+22D4}{PITCHFORK}{$\pitchfork$}{} \\
\UnicodeOp{U+22EE}{VERTICAL ELLIPSIS}{}{} \\
\UnicodeOp{U+22EF}{MIDLINE HORIZONTAL ELLIPSIS}{}{} \\
\UnicodeOp{U+22F0}{UP RIGHT DIAGONAL ELLIPSIS}{}{} \\
\UnicodeOp{U+22F1}{DOWN RIGHT DIAGONAL ELLIPSIS}{}{} \\
\UnicodeOp{U+27C0}{THREE DIMENSIONAL ANGLE}{}{} \\
\UnicodeOp{U+27C1}{WHITE TRIANGLE CONTAINING SMALL WHITE TRIANGLE}{}{} \\
\UnicodeOp{U+27C2}{PERPENDICULAR}{}{\texttt{PERP}} \\
\UnicodeOp{U+27D0}{WHITE DIAMOND WITH CENTRED DOT}{}{} \\
\UnicodeOp{U+27D2}{ELEMENT OF OPENING UPWARDS}{}{} \\
\UnicodeOp{U+27D3}{LOWER RIGHT CORNER WITH DOT}{}{} \\
\UnicodeOp{U+27D4}{UPPER LEFT CORNER WITH DOT}{}{} \\
\UnicodeOp{U+27D5}{LEFT OUTER JOIN}{}{} \\
\UnicodeOp{U+27D6}{RIGHT OUTER JOIN}{}{} \\
\UnicodeOp{U+27D7}{FULL OUTER JOIN}{}{} \\
\UnicodeOp{U+27D8}{LARGE UP TACK}{}{} \\
\UnicodeOp{U+27D9}{LARGE DOWN TACK}{}{} \\
\UnicodeOp{U+27DA}{LEFT AND RIGHT DOUBLE TURNSTILE}{}{} \\
\UnicodeOp{U+27DB}{LEFT AND RIGHT TACK}{}{} \\
\UnicodeOp{U+27DC}{LEFT MULTIMAP}{}{} \\
\UnicodeOp{U+27DD}{LONG RIGHT TACK}{}{} \\
\UnicodeOp{U+27DE}{LONG LEFT TACK}{}{} \\
\UnicodeOp{U+27DF}{UP TACK WITH CIRCLE ABOVE}{}{} \\
\UnicodeOp{U+27E0}{LOZENGE DIVIDED BY HORIZONTAL RULE}{}{} \\
\UnicodeOp{U+27E1}{WHITE CONCAVE-SIDED DIAMOND}{}{} \\
\UnicodeOp{U+27E2}{WHITE CONCAVE-SIDED DIAMOND WITH LEFTWARDS TICK}{}{} \\
\UnicodeOp{U+27E3}{WHITE CONCAVE-SIDED DIAMOND WITH RIGHTWARDS TICK}{}{} \\
\UnicodeOp{U+27E4}{WHITE SQUARE WITH LEFTWARDS TICK}{}{} \\
\UnicodeOp{U+27E5}{WHITE SQUARE WITH RIGHTWARDS TICK}{}{} \\
\UnicodeOp{U+27F0}{UPWARDS QUADRUPLE ARROW}{}{} \\
\UnicodeOp{U+27F1}{DOWNWARDS QUADRUPLE ARROW}{}{} \\
\UnicodeOp{U+27F2}{ANTICLOCKWISE GAPPED CIRCLE ARROW}{}{} \\
\UnicodeOp{U+27F3}{CLOCKWISE GAPPED CIRCLE ARROW}{}{} \\
\UnicodeOp{U+27F4}{RIGHT ARROW WITH CIRCLED PLUS}{}{} \\
\UnicodeOp{U+27F5}{LONG LEFTWARDS ARROW}{}{} \\
\UnicodeOp{U+27F6}{LONG RIGHTWARDS ARROW}{}{} \\
\UnicodeOp{U+27F7}{LONG LEFT RIGHT ARROW}{}{} \\
\UnicodeOp{U+27F8}{LONG LEFTWARDS DOUBLE ARROW}{}{} \\
\UnicodeOp{U+27F9}{LONG RIGHTWARDS DOUBLE ARROW}{}{} \\
\UnicodeOp{U+27FA}{LONG LEFT RIGHT DOUBLE ARROW}{}{} \\
\UnicodeOp{U+27FB}{LONG LEFTWARDS ARROW FROM BAR}{}{} \\
\UnicodeOp{U+27FC}{LONG RIGHTWARDS ARROW FROM BAR}{}{} \\
\UnicodeOp{U+27FD}{LONG LEFTWARDS DOUBLE ARROW FROM BAR}{}{} \\
\UnicodeOp{U+27FE}{LONG RIGHTWARDS DOUBLE ARROW FROM BAR}{}{} \\
\UnicodeOp{U+27FF}{LONG RIGHTWARDS SQUIGGLE ARROW}{}{} \\
\UnicodeOp{U+2900}{RIGHTWARDS TWO-HEADED ARROW WITH VERTICAL STROKE}{}{} \\
\UnicodeOp{U+2901}{RIGHTWARDS TWO-HEADED ARROW WITH DOUBLE VERTICAL STROKE}{}{} \\
\UnicodeOp{U+2902}{LEFTWARDS DOUBLE ARROW WITH VERTICAL STROKE}{}{} \\
\UnicodeOp{U+2903}{RIGHTWARDS DOUBLE ARROW WITH VERTICAL STROKE}{}{} \\
\UnicodeOp{U+2904}{LEFT RIGHT DOUBLE ARROW WITH VERTICAL STROKE}{}{} \\
\UnicodeOp{U+2905}{RIGHTWARDS TWO-HEADED ARROW FROM BAR}{}{} \\
\UnicodeOp{U+2906}{LEFTWARDS DOUBLE ARROW FROM BAR}{}{} \\
\UnicodeOp{U+2907}{RIGHTWARDS DOUBLE ARROW FROM BAR}{}{} \\
\UnicodeOp{U+2908}{DOWNWARDS ARROW WITH HORIZONTAL STROKE}{}{} \\
\UnicodeOp{U+2909}{UPWARDS ARROW WITH HORIZONTAL STROKE}{}{} \\
\UnicodeOp{U+290A}{UPWARDS TRIPLE ARROW}{}{} \\
\UnicodeOp{U+290B}{DOWNWARDS TRIPLE ARROW}{}{} \\
\UnicodeOp{U+290C}{LEFTWARDS DOUBLE DASH ARROW}{}{} \\
\UnicodeOp{U+290D}{RIGHTWARDS DOUBLE DASH ARROW}{}{} \\
\UnicodeOp{U+290E}{LEFTWARDS TRIPLE DASH ARROW}{}{} \\
\UnicodeOp{U+290F}{RIGHTWARDS TRIPLE DASH ARROW}{}{} \\
\UnicodeOp{U+2910}{RIGHTWARDS TWO-HEADED TRIPLE DASH ARROW}{}{} \\
\UnicodeOp{U+2911}{RIGHTWARDS ARROW WITH DOTTED STEM}{}{} \\
\UnicodeOp{U+2912}{UPWARDS ARROW TO BAR}{}{} \\
\UnicodeOp{U+2913}{DOWNWARDS ARROW TO BAR}{}{} \\
\UnicodeOp{U+2914}{RIGHTWARDS ARROW WITH TAIL WITH VERTICAL STROKE}{}{} \\
\UnicodeOp{U+2915}{RIGHTWARDS ARROW WITH TAIL WITH DOUBLE VERTICAL STROKE}{}{} \\
\UnicodeOp{U+2916}{RIGHTWARDS TWO-HEADED ARROW WITH TAIL}{}{} \\
\UnicodeOp{U+2917}{RIGHTWARDS TWO-HEADED ARROW WITH TAIL WITH VERTICAL STROKE}{}{} \\
\UnicodeOp{U+2918}{RIGHTWARDS TWO-HEADED ARROW WITH TAIL WITH DOUBLE VERTICAL STROKE}{}{} \\
\UnicodeOp{U+2919}{LEFTWARDS ARROW-TAIL}{}{} \\
\UnicodeOp{U+291A}{RIGHTWARDS ARROW-TAIL}{}{} \\
\UnicodeOp{U+291B}{LEFTWARDS DOUBLE ARROW-TAIL}{}{} \\
\UnicodeOp{U+291C}{RIGHTWARDS DOUBLE ARROW-TAIL}{}{} \\
\UnicodeOp{U+291D}{LEFTWARDS ARROW TO BLACK DIAMOND}{}{} \\
\UnicodeOp{U+291E}{RIGHTWARDS ARROW TO BLACK DIAMOND}{}{} \\
\UnicodeOp{U+291F}{LEFTWARDS ARROW FROM BAR TO BLACK DIAMOND}{}{} \\
\UnicodeOp{U+2920}{RIGHTWARDS ARROW FROM BAR TO BLACK DIAMOND}{}{} \\
\UnicodeOp{U+2921}{NORTH WEST AND SOUTH EAST ARROW}{}{} \\
\UnicodeOp{U+2922}{NORTH EAST AND SOUTH WEST ARROW}{}{} \\
\UnicodeOp{U+2923}{NORTH WEST ARROW WITH HOOK}{}{} \\
\UnicodeOp{U+2924}{NORTH EAST ARROW WITH HOOK}{}{} \\
\UnicodeOp{U+2925}{SOUTH EAST ARROW WITH HOOK}{}{} \\
\UnicodeOp{U+2926}{SOUTH WEST ARROW WITH HOOK}{}{} \\
\UnicodeOp{U+2927}{NORTH WEST ARROW AND NORTH EAST ARROW}{}{} \\
\UnicodeOp{U+2928}{NORTH EAST ARROW AND SOUTH EAST ARROW}{}{} \\
\UnicodeOp{U+2929}{SOUTH EAST ARROW AND SOUTH WEST ARROW}{}{} \\
\UnicodeOp{U+292A}{SOUTH WEST ARROW AND NORTH WEST ARROW}{}{} \\
\UnicodeOp{U+292B}{RISING DIAGONAL CROSSING FALLING DIAGONAL}{}{} \\
\UnicodeOp{U+292C}{FALLING DIAGONAL CROSSING RISING DIAGONAL}{}{} \\
\UnicodeOp{U+292D}{SOUTH EAST ARROW CROSSING NORTH EAST ARROW}{}{} \\
\UnicodeOp{U+292E}{NORTH EAST ARROW CROSSING SOUTH EAST ARROW}{}{} \\
\UnicodeOp{U+292F}{FALLING DIAGONAL CROSSING NORTH EAST ARROW}{}{} \\
\UnicodeOp{U+2930}{RISING DIAGONAL CROSSING SOUTH EAST ARROW}{}{} \\
\UnicodeOp{U+2931}{NORTH EAST ARROW CROSSING NORTH WEST ARROW}{}{} \\
\UnicodeOp{U+2932}{NORTH WEST ARROW CROSSING NORTH EAST ARROW}{}{} \\
\UnicodeOp{U+2933}{WAVE ARROW POINTING DIRECTLY RIGHT}{}{} \\
\UnicodeOp{U+2934}{ARROW POINTING RIGHTWARDS THEN CURVING UPWARDS}{}{} \\
\UnicodeOp{U+2935}{ARROW POINTING RIGHTWARDS THEN CURVING DOWNWARDS}{}{} \\
\UnicodeOp{U+2936}{ARROW POINTING DOWNWARDS THEN CURVING LEFTWARDS}{}{} \\
\UnicodeOp{U+2937}{ARROW POINTING DOWNWARDS THEN CURVING RIGHTWARDS}{}{} \\
\UnicodeOp{U+2938}{RIGHT-SIDE ARC CLOCKWISE ARROW}{}{} \\
\UnicodeOp{U+2939}{LEFT-SIDE ARC ANTICLOCKWISE ARROW}{}{} \\
\UnicodeOp{U+293A}{TOP ARC ANTICLOCKWISE ARROW}{}{} \\
\UnicodeOp{U+293B}{BOTTOM ARC ANTICLOCKWISE ARROW}{}{} \\
\UnicodeOp{U+293C}{TOP ARC CLOCKWISE ARROW WITH MINUS}{}{} \\
\UnicodeOp{U+293D}{TOP ARC ANTICLOCKWISE ARROW WITH PLUS}{}{} \\
\UnicodeOp{U+293E}{LOWER RIGHT SEMICIRCULAR CLOCKWISE ARROW}{}{} \\
\UnicodeOp{U+293F}{LOWER LEFT SEMICIRCULAR ANTICLOCKWISE ARROW}{}{} \\
\UnicodeOp{U+2940}{ANTICLOCKWISE CLOSED CIRCLE ARROW}{}{} \\
\UnicodeOp{U+2941}{CLOCKWISE CLOSED CIRCLE ARROW}{}{} \\
\UnicodeOp{U+2942}{RIGHTWARDS ARROW ABOVE SHORT LEFTWARDS ARROW}{}{} \\
\UnicodeOp{U+2943}{LEFTWARDS ARROW ABOVE SHORT RIGHTWARDS ARROW}{}{} \\
\UnicodeOp{U+2944}{SHORT RIGHTWARDS ARROW ABOVE LEFTWARDS ARROW}{}{} \\
\UnicodeOp{U+2945}{RIGHTWARDS ARROW WITH PLUS BELOW}{}{} \\
\UnicodeOp{U+2946}{LEFTWARDS ARROW WITH PLUS BELOW}{}{} \\
\UnicodeOp{U+2947}{RIGHTWARDS ARROW THROUGH X}{}{} \\
\UnicodeOp{U+2948}{LEFT RIGHT ARROW THROUGH SMALL CIRCLE}{}{} \\
\UnicodeOp{U+2949}{UPWARDS TWO-HEADED ARROW FROM SMALL CIRCLE}{}{} \\
\UnicodeOp{U+294A}{LEFT BARB UP RIGHT BARB DOWN HARPOON}{}{} \\
\UnicodeOp{U+294B}{LEFT BARB DOWN RIGHT BARB UP HARPOON}{}{} \\
\UnicodeOp{U+294C}{UP BARB RIGHT DOWN BARB LEFT HARPOON}{}{} \\
\UnicodeOp{U+294D}{UP BARB LEFT DOWN BARB RIGHT HARPOON}{}{} \\
\UnicodeOp{U+294E}{LEFT BARB UP RIGHT BARB UP HARPOON}{}{} \\
\UnicodeOp{U+294F}{UP BARB RIGHT DOWN BARB RIGHT HARPOON}{}{} \\
\UnicodeOp{U+2950}{LEFT BARB DOWN RIGHT BARB DOWN HARPOON}{}{} \\
\UnicodeOp{U+2951}{UP BARB LEFT DOWN BARB LEFT HARPOON}{}{} \\
\UnicodeOp{U+2952}{LEFTWARDS HARPOON WITH BARB UP TO BAR}{}{} \\
\UnicodeOp{U+2953}{RIGHTWARDS HARPOON WITH BARB UP TO BAR}{}{} \\
\UnicodeOp{U+2954}{UPWARDS HARPOON WITH BARB RIGHT TO BAR}{}{} \\
\UnicodeOp{U+2955}{DOWNWARDS HARPOON WITH BARB RIGHT TO BAR}{}{} \\
\UnicodeOp{U+2956}{LEFTWARDS HARPOON WITH BARB DOWN TO BAR}{}{} \\
\UnicodeOp{U+2957}{RIGHTWARDS HARPOON WITH BARB DOWN TO BAR}{}{} \\
\UnicodeOp{U+2958}{UPWARDS HARPOON WITH BARB LEFT TO BAR}{}{} \\
\UnicodeOp{U+2959}{DOWNWARDS HARPOON WITH BARB LEFT TO BAR}{}{} \\
\UnicodeOp{U+295A}{LEFTWARDS HARPOON WITH BARB UP FROM BAR}{}{} \\
\UnicodeOp{U+295B}{RIGHTWARDS HARPOON WITH BARB UP FROM BAR}{}{} \\
\UnicodeOp{U+295C}{UPWARDS HARPOON WITH BARB RIGHT FROM BAR}{}{} \\
\UnicodeOp{U+295D}{DOWNWARDS HARPOON WITH BARB RIGHT FROM BAR}{}{} \\
\UnicodeOp{U+295E}{LEFTWARDS HARPOON WITH BARB DOWN FROM BAR}{}{} \\
\UnicodeOp{U+295F}{RIGHTWARDS HARPOON WITH BARB DOWN FROM BAR}{}{} \\
\UnicodeOp{U+2960}{UPWARDS HARPOON WITH BARB LEFT FROM BAR}{}{} \\
\UnicodeOp{U+2961}{DOWNWARDS HARPOON WITH BARB LEFT FROM BAR}{}{} \\
\UnicodeOp{U+2962}{LEFTWARDS HARPOON WITH BARB UP ABOVE LEFTWARDS HARPOON WITH BARB DOWN}{}{} \\
\UnicodeOp{U+2963}{UPWARDS HARPOON WITH BARB LEFT BESIDE UPWARDS HARPOON WITH BARB RIGHT}{}{} \\
\UnicodeOp{U+2964}{RIGHTWARDS HARPOON WITH BARB UP ABOVE RIGHTWARDS HARPOON WITH BARB DOWN}{}{} \\
\UnicodeOp{U+2965}{DOWNWARDS HARPOON WITH BARB LEFT BESIDE DOWNWARDS HARPOON WITH BARB RIGHT}{}{} \\
\UnicodeOp{U+2966}{LEFTWARDS HARPOON WITH BARB UP ABOVE RIGHTWARDS HARPOON WITH BARB UP}{}{} \\
\UnicodeOp{U+2967}{LEFTWARDS HARPOON WITH BARB DOWN ABOVE RIGHTWARDS HARPOON WITH BARB DOWN}{}{} \\
\UnicodeOp{U+2968}{RIGHTWARDS HARPOON WITH BARB UP ABOVE LEFTWARDS HARPOON WITH BARB UP}{}{} \\
\UnicodeOp{U+2969}{RIGHTWARDS HARPOON WITH BARB DOWN ABOVE LEFTWARDS HARPOON WITH BARB DOWN}{}{} \\
\UnicodeOp{U+296A}{LEFTWARDS HARPOON WITH BARB UP ABOVE LONG DASH}{}{} \\
\UnicodeOp{U+296B}{LEFTWARDS HARPOON WITH BARB DOWN BELOW LONG DASH}{}{} \\
\UnicodeOp{U+296C}{RIGHTWARDS HARPOON WITH BARB UP ABOVE LONG DASH}{}{} \\
\UnicodeOp{U+296D}{RIGHTWARDS HARPOON WITH BARB DOWN BELOW LONG DASH}{}{} \\
\UnicodeOp{U+296E}{UPWARDS HARPOON WITH BARB LEFT BESIDE DOWNWARDS HARPOON WITH BARB RIGHT}{}{} \\
\UnicodeOp{U+296F}{DOWNWARDS HARPOON WITH BARB LEFT BESIDE UPWARDS HARPOON WITH BARB RIGHT}{}{} \\
\UnicodeOp{U+2970}{RIGHT DOUBLE ARROW WITH ROUNDED HEAD}{}{} \\
\UnicodeOp{U+2971}{EQUALS SIGN ABOVE RIGHTWARDS ARROW}{}{} \\
\UnicodeOp{U+2972}{TILDE OPERATOR ABOVE RIGHTWARDS ARROW}{}{} \\
\UnicodeOp{U+2973}{LEFTWARDS ARROW ABOVE TILDE OPERATOR}{}{} \\
\UnicodeOp{U+2974}{RIGHTWARDS ARROW ABOVE TILDE OPERATOR}{}{} \\
\UnicodeOp{U+2975}{RIGHTWARDS ARROW ABOVE ALMOST EQUAL TO}{}{} \\
\UnicodeOp{U+2976}{LESS-THAN ABOVE LEFTWARDS ARROW}{}{} \\
\UnicodeOp{U+2977}{LEFTWARDS ARROW THROUGH LESS-THAN}{}{} \\
\UnicodeOp{U+2978}{GREATER-THAN ABOVE RIGHTWARDS ARROW}{}{} \\
\UnicodeOp{U+2979}{SUBSET ABOVE RIGHTWARDS ARROW}{}{} \\
\UnicodeOp{U+297A}{LEFTWARDS ARROW THROUGH SUBSET}{}{} \\
\UnicodeOp{U+297B}{SUPERSET ABOVE LEFTWARDS ARROW}{}{} \\
\UnicodeOp{U+297C}{LEFT FISH TAIL}{}{} \\
\UnicodeOp{U+297D}{RIGHT FISH TAIL}{}{} \\
\UnicodeOp{U+297E}{UP FISH TAIL}{}{} \\
\UnicodeOp{U+297F}{DOWN FISH TAIL}{}{} \\
\UnicodeOp{U+2980}{TRIPLE VERTICAL BAR DELIMITER}{}{} \\
\UnicodeOp{U+2981}{Z NOTATION SPOT}{}{} \\
\UnicodeOp{U+2982}{Z NOTATION TYPE COLON}{}{} \\
\UnicodeOp{U+2999}{DOTTED FENCE}{}{} \\
\UnicodeOp{U+299A}{VERTICAL ZIGZAG LINE}{}{} \\
\UnicodeOp{U+299B}{MEASURED ANGLE OPENING LEFT}{}{} \\
\UnicodeOp{U+299C}{RIGHT ANGLE VARIANT WITH SQUARE}{}{} \\
\UnicodeOp{U+299D}{MEASURED RIGHT ANGLE WITH DOT}{}{} \\
\UnicodeOp{U+299E}{ANGLE WITH S INSIDE}{}{} \\
\UnicodeOp{U+299F}{ACUTE ANGLE}{}{} \\
\UnicodeOp{U+29A0}{SPHERICAL ANGLE OPENING LEFT}{}{} \\
\UnicodeOp{U+29A1}{SPHERICAL ANGLE OPENING UP}{}{} \\
\UnicodeOp{U+29A2}{TURNED ANGLE}{}{} \\
\UnicodeOp{U+29A3}{REVERSED ANGLE}{}{} \\
\UnicodeOp{U+29A4}{ANGLE WITH UNDERBAR}{}{} \\
\UnicodeOp{U+29A5}{REVERSED ANGLE WITH UNDERBAR}{}{} \\
\UnicodeOp{U+29A6}{OBLIQUE ANGLE OPENING UP}{}{} \\
\UnicodeOp{U+29A7}{OBLIQUE ANGLE OPENING DOWN}{}{} \\
\UnicodeOp{U+29A8}{MEASURED ANGLE WITH OPEN ARM ENDING IN ARROW POINTING UP AND RIGHT}{}{} \\
\UnicodeOp{U+29A9}{MEASURED ANGLE WITH OPEN ARM ENDING IN ARROW POINTING UP AND LEFT}{}{} \\
\UnicodeOp{U+29AA}{MEASURED ANGLE WITH OPEN ARM ENDING IN ARROW POINTING DOWN AND RIGHT}{}{} \\
\UnicodeOp{U+29AB}{MEASURED ANGLE WITH OPEN ARM ENDING IN ARROW POINTING DOWN AND LEFT}{}{} \\
\UnicodeOp{U+29AC}{MEASURED ANGLE WITH OPEN ARM ENDING IN ARROW POINTING RIGHT AND UP}{}{} \\
\UnicodeOp{U+29AD}{MEASURED ANGLE WITH OPEN ARM ENDING IN ARROW POINTING LEFT AND UP}{}{} \\
\UnicodeOp{U+29AE}{MEASURED ANGLE WITH OPEN ARM ENDING IN ARROW POINTING RIGHT AND DOWN}{}{} \\
\UnicodeOp{U+29AF}{MEASURED ANGLE WITH OPEN ARM ENDING IN ARROW POINTING LEFT AND DOWN}{}{} \\
\UnicodeOp{U+29B0}{REVERSED EMPTY SET}{}{} \\
\UnicodeOp{U+29B1}{EMPTY SET WITH OVERBAR}{}{} \\
\UnicodeOp{U+29B2}{EMPTY SET WITH SMALL CIRCLE ABOVE}{}{} \\
\UnicodeOp{U+29B3}{EMPTY SET WITH RIGHT ARROW ABOVE}{}{} \\
\UnicodeOp{U+29B4}{EMPTY SET WITH LEFT ARROW ABOVE}{}{} \\
\UnicodeOp{U+29B5}{CIRCLE WITH HORIZONTAL BAR}{}{} \\
\UnicodeOp{U+29B6}{CIRCLED VERTICAL BAR}{}{} \\
\UnicodeOp{U+29B7}{CIRCLED PARALLEL}{}{} \\
\UnicodeOp{U+29B9}{CIRCLED PERPENDICULAR}{}{} \\
\UnicodeOp{U+29BA}{CIRCLE DIVIDED BY HORIZONTAL BAR AND TOP HALF DIVIDED BY VERTICAL BAR}{}{} \\
\UnicodeOp{U+29BB}{CIRCLE WITH SUPERIMPOSED X}{}{} \\
\UnicodeOp{U+29BD}{UP ARROW THROUGH CIRCLE}{}{} \\
\UnicodeOp{U+29BE}{CIRCLED WHITE BULLET}{}{} \\
\UnicodeOp{U+29BF}{CIRCLED BULLET}{}{} \\
\UnicodeOp{U+29C2}{CIRCLE WITH SMALL CIRCLE TO THE RIGHT}{}{} \\
\UnicodeOp{U+29C3}{CIRCLE WITH TWO HORIZONTAL STROKES TO THE RIGHT}{}{} \\
\UnicodeOp{U+29C5}{SQUARED FALLING DIAGONAL SLASH}{}{} \\
\UnicodeOp{U+29C7}{SQUARED SMALL CIRCLE}{}{} \\
\UnicodeOp{U+29C8}{SQUARED SQUARE}{}{} \\
\UnicodeOp{U+29C9}{TWO JOINED SQUARES}{}{} \\
\UnicodeOp{U+29CA}{TRIANGLE WITH DOT ABOVE}{}{} \\
\UnicodeOp{U+29CB}{TRIANGLE WITH UNDERBAR}{}{} \\
\UnicodeOp{U+29CC}{S IN TRIANGLE}{}{} \\
\UnicodeOp{U+29CD}{TRIANGLE WITH SERIFS AT BOTTOM}{}{} \\
\UnicodeOp{U+29CE}{RIGHT TRIANGLE ABOVE LEFT TRIANGLE}{}{} \\
\UnicodeOp{U+29CF}{LEFT TRIANGLE BESIDE VERTICAL BAR}{}{} \\
\UnicodeOp{U+29D0}{VERTICAL BAR BESIDE RIGHT TRIANGLE}{}{} \\
\UnicodeOp{U+29D1}{BOWTIE WITH LEFT HALF BLACK}{}{} \\
\UnicodeOp{U+29D2}{BOWTIE WITH RIGHT HALF BLACK}{}{} \\
\UnicodeOp{U+29D3}{BLACK BOWTIE}{}{} \\
\UnicodeOp{U+29D6}{WHITE HOURGLASS}{}{} \\
\UnicodeOp{U+29D7}{BLACK HOURGLASS}{}{} \\
\UnicodeOp{U+29DC}{INCOMPLETE INFINITY}{}{} \\
\UnicodeOp{U+29DD}{TIE OVER INFINITY}{}{} \\
\UnicodeOp{U+29DE}{INFINITY NEGATED WITH VERTICAL BAR}{}{} \\
\UnicodeOp{U+29DF}{DOUBLE-ENDED MULTIMAP}{}{} \\
\UnicodeOp{U+29E0}{SQUARE WITH CONTOURED OUTLINE}{}{} \\
\UnicodeOp{U+29E1}{INCREASES AS}{}{} \\
\UnicodeOp{U+29E2}{SHUFFLE PRODUCT}{}{} \\
\UnicodeOp{U+29E6}{GLEICH STARK}{}{} \\
\UnicodeOp{U+29E7}{THERMODYNAMIC}{}{} \\
\UnicodeOp{U+29E8}{DOWN-POINTING TRIANGLE WITH LEFT HALF BLACK}{}{} \\
\UnicodeOp{U+29E9}{DOWN-POINTING TRIANGLE WITH RIGHT HALF BLACK}{}{} \\
\UnicodeOp{U+29EA}{BLACK DIAMOND WITH DOWN ARROW}{}{} \\
\UnicodeOp{U+29EB}{BLACK LOZENGE}{}{} \\
\UnicodeOp{U+29EC}{WHITE CIRCLE WITH DOWN ARROW}{}{} \\
\UnicodeOp{U+29ED}{BLACK CIRCLE WITH DOWN ARROW}{}{} \\
\UnicodeOp{U+29EE}{ERROR-BARRED WHITE SQUARE}{}{} \\
\UnicodeOp{U+29EF}{ERROR-BARRED BLACK SQUARE}{}{} \\
\UnicodeOp{U+29F0}{ERROR-BARRED WHITE DIAMOND}{}{} \\
\UnicodeOp{U+29F1}{ERROR-BARRED BLACK DIAMOND}{}{} \\
\UnicodeOp{U+29F2}{ERROR-BARRED WHITE CIRCLE}{}{} \\
\UnicodeOp{U+29F3}{ERROR-BARRED BLACK CIRCLE}{}{} \\
\UnicodeOp{U+29F4}{RULE-DELAYED}{}{} \\
\UnicodeOp{U+29F6}{SOLIDUS WITH OVERBAR}{}{} \\
\UnicodeOp{U+29F7}{REVERSE SOLIDUS WITH HORIZONTAL STROKE}{}{} \\
\UnicodeOp{U+29FA}{DOUBLE PLUS}{}{\texttt{++}} \\
\UnicodeOp{U+29FB}{TRIPLE PLUS}{}{\texttt{+++}} \\
\UnicodeOp{U+29FE}{TINY}{}{} \\
\UnicodeOp{U+29FF}{MINY}{}{} \\
\UnicodeOp{U+2A00}{N-ARY CIRCLED DOT OPERATOR}{$\bigodot$}{\texttt{BIGODOT}} \\
\UnicodeOp{U+2A01}{N-ARY CIRCLED PLUS OPERATOR}{$\bigoplus$}{\texttt{BIGOPLUS}} \\
\UnicodeOp{U+2A02}{N-ARY CIRCLED TIMES OPERATOR}{$\bigotimes$}{\texttt{BIGOTIMES}} \\
\UnicodeOp{U+2A03}{N-ARY UNION OPERATOR WITH DOT}{}{\texttt{BIGUDOT}} \\
\UnicodeOp{U+2A04}{N-ARY UNION OPERATOR WITH PLUS}{}{\texttt{BIGUPLUS}} \\
\UnicodeOp{U+2A05}{N-ARY SQUARE INTERSECTION OPERATOR}{}{\texttt{BIGSQCAP}} \\
\UnicodeOp{U+2A06}{N-ARY SQUARE UNION OPERATOR}{}{\texttt{BIGSQCUP}} \\
\UnicodeOp{U+2A07}{TWO LOGICAL AND OPERATOR}{}{} \\
\UnicodeOp{U+2A08}{TWO LOGICAL OR OPERATOR}{}{} \\
\UnicodeOp{U+2A09}{N-ARY TIMES OPERATOR}{}{\texttt{BIGTIMES}} \\
\UnicodeOp{U+2A0A}{MODULO TWO SUM}{}{} \\
\UnicodeOp{U+2A10}{CIRCULATION FUNCTION}{}{} \\
\UnicodeOp{U+2A11}{ANTICLOCKWISE INTEGRATION}{}{} \\
\UnicodeOp{U+2A12}{LINE INTEGRATION WITH RECTANGULAR PATH AROUND POLE}{}{} \\
\UnicodeOp{U+2A13}{LINE INTEGRATION WITH SEMICIRCULAR PATH AROUND POLE}{}{} \\
\UnicodeOp{U+2A14}{LINE INTEGRATION NOT INCLUDING THE POLE}{}{} \\
\UnicodeOp{U+2A1D}{JOIN}{$\Join$}{\texttt{JOIN}} \\
\UnicodeOp{U+2A1E}{LARGE LEFT TRIANGLE OPERATOR}{}{} \\
\UnicodeOp{U+2A1F}{Z NOTATION SCHEMA COMPOSITION}{}{} \\
\UnicodeOp{U+2A20}{Z NOTATION SCHEMA PIPING}{}{} \\
\UnicodeOp{U+2A21}{Z NOTATION SCHEMA PROJECTION}{}{} \\
\UnicodeOp{U+2A32}{SEMIDIRECT PRODUCT WITH BOTTOM CLOSED}{}{} \\
\UnicodeOp{U+2A33}{SMASH PRODUCT}{}{} \\
\UnicodeOp{U+2A3C}{INTERIOR PRODUCT}{}{} \\
\UnicodeOp{U+2A3D}{RIGHTHAND INTERIOR PRODUCT}{}{} \\
\UnicodeOp{U+2A3E}{Z NOTATION RELATIONAL COMPOSITION}{}{} \\
\UnicodeOp{U+2A3F}{AMALGAMATION OR COPRODUCT}{}{} \\
\UnicodeOp{U+2A57}{SLOPING LARGE OR}{}{} \\
\UnicodeOp{U+2A58}{SLOPING LARGE AND}{}{} \\
\UnicodeOp{U+2A61}{SMALL VEE WITH UNDERBAR}{}{} \\
\UnicodeOp{U+2A64}{Z NOTATION DOMAIN ANTIRESTRICTION}{}{} \\
\UnicodeOp{U+2A65}{Z NOTATION RANGE ANTIRESTRICTION}{}{} \\
\UnicodeOp{U+2A68}{TRIPLE HORIZONTAL BAR WITH DOUBLE VERTICAL STROKE}{}{} \\
\UnicodeOp{U+2A69}{TRIPLE HORIZONTAL BAR WITH TRIPLE VERTICAL STROKE}{}{} \\
\UnicodeOp{U+2A6A}{TILDE OPERATOR WITH DOT ABOVE}{}{} \\
\UnicodeOp{U+2A6B}{TILDE OPERATOR WITH RISING DOTS}{}{} \\
\UnicodeOp{U+2A6D}{CONGRUENT WITH DOT ABOVE}{}{} \\
\UnicodeOp{U+2ACD}{SQUARE LEFT OPEN BOX OPERATOR}{}{} \\
\UnicodeOp{U+2ACE}{SQUARE RIGHT OPEN BOX OPERATOR}{}{} \\
\UnicodeOp{U+2AD9}{ELEMENT OF OPENING DOWNWARDS}{}{} \\
\UnicodeOp{U+2ADA}{PITCHFORK WITH TEE TOP}{}{} \\
\UnicodeOp{U+2ADC}{FORKING}{}{} \\
\UnicodeOp{U+2ADD}{NONFORKING}{}{} \\
\UnicodeOp{U+2ADE}{SHORT LEFT TACK}{}{} \\
\UnicodeOp{U+2ADF}{SHORT DOWN TACK}{}{} \\
\UnicodeOp{U+2AE0}{SHORT UP TACK}{}{} \\
\UnicodeOp{U+2AE1}{PERPENDICULAR WITH S}{}{} \\
\UnicodeOp{U+2AE2}{VERTICAL BAR TRIPLE RIGHT TURNSTILE}{}{} \\
\UnicodeOp{U+2AE3}{DOUBLE VERTICAL BAR LEFT TURNSTILE}{}{} \\
\UnicodeOp{U+2AE4}{VERTICAL BAR DOUBLE LEFT TURNSTILE}{}{} \\
\UnicodeOp{U+2AE5}{DOUBLE VERTICAL BAR DOUBLE LEFT TURNSTILE}{}{} \\
\UnicodeOp{U+2AE6}{LONG DASH FROM LEFT MEMBER OF DOUBLE VERTICAL}{}{} \\
\UnicodeOp{U+2AE7}{SHORT DOWN TACK WITH OVERBAR}{}{} \\
\UnicodeOp{U+2AE8}{SHORT UP TACK WITH UNDERBAR}{}{} \\
\UnicodeOp{U+2AE9}{SHORT UP TACK ABOVE SHORT DOWN TACK}{}{} \\
\UnicodeOp{U+2AEA}{DOUBLE DOWN TACK}{}{} \\
\UnicodeOp{U+2AEB}{DOUBLE UP TACK}{}{} \\
\UnicodeOp{U+2AEC}{DOUBLE STROKE NOT SIGN}{}{} \\
\UnicodeOp{U+2AED}{REVERSED DOUBLE STROKE NOT SIGN}{}{} \\
\UnicodeOp{U+2AEE}{DOES NOT DIVIDE WITH REVERSED NEGATION SLASH}{}{} \\
\UnicodeOp{U+2AEF}{VERTICAL LINE WITH CIRCLE ABOVE}{}{} \\
\UnicodeOp{U+2AF0}{VERTICAL LINE WITH CIRCLE BELOW}{}{} \\
\UnicodeOp{U+2AF1}{DOWN TACK WITH CIRCLE BELOW}{}{} \\
\UnicodeOp{U+2AF2}{PARALLEL WITH HORIZONTAL STROKE}{}{} \\
\UnicodeOp{U+2AF3}{PARALLEL WITH TILDE OPERATOR}{}{} \\
\UnicodeOp{U+2AF5}{TRIPLE VERTICAL BAR WITH HORIZONTAL STROKE}{}{} \\
\UnicodeOp{U+2AF6}{TRIPLE COLON OPERATOR}{}{} \\
\UnicodeOp{U+2AFC}{LARGE TRIPLE VERTICAL BAR OPERATOR}{}{} \\
\UnicodeOp{U+2AFE}{WHITE VERTICAL BAR}{}{} \\
\UnicodeOp{U+2AFF}{N-ARY WHITE VERTICAL BAR}{}{}
\end{tabbing}
