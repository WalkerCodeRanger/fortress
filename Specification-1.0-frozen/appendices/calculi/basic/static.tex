%%%%%%%%%%%%%%%%%%%%%%%%%%%%%%%%%%%%%%%%%%%%%%%%%%%%%%%%%%%%%%%%%%%%%%%%%%%%%%%%
%   Copyright 2009, Oracle and/or its affiliates.
%   All rights reserved.
%
%
%   Use is subject to license terms.
%
%   This distribution may include materials developed by third parties.
%
%%%%%%%%%%%%%%%%%%%%%%%%%%%%%%%%%%%%%%%%%%%%%%%%%%%%%%%%%%%%%%%%%%%%%%%%%%%%%%%%

\subsection{Static Semantics}\label{basic-static}
A static semantics for \basiccore\ is provided in
Figures~\ref{fig:basic-static1}, \ref{fig:basic-static2}, and
\ref{fig:basic-static3}.
The \basiccore\ static semantics is based on the static
semantics of Featherweight Generic Java (FGJ)~\cite{fgj}. The major
difference is the division of classes into traits and objects. 
Both trait and object definitions
include method definitions but only object definitions include field
definitions. With traits, \basiccore\ supports multiple inheritance.
However, due to the similarity of traits and objects, many of the rules
in the \basiccore\ dynamic and static semantics combine
the two cases.  Note that \basiccore\ allows parametric 
polymorphism, subtype polymorphism, and overriding in much the
same way that FGJ does.

\begin{figure}[htbp!]
Environments\\

$
\begin{array}{llll}
\tvenv &::=& \seq{\tvone \subtype \tappone}\rulesep\rulesep
 & \mbox{bound environment}\\
\tyenv &::=& \seq{\vname:\ty} & \mbox{type environment}\\ \\
\end{array}
$

Program typing: \fbox{\provesP{\pgm}{\ty}} \\ \\
\begin{tabular}{lc}
\newinfrule{
\begin{array}{c}
\pgm = \seq{\d}~\exp
\rulesep
\provesD{\seq{\d}}
\rulesep
\provesE{\emptyset}{\emptyset}{\exp}{\ty}
\end{array}
}
{\provesP{\pgm}{\ty}}
{\tProgramRule} \\ \\
\end{tabular}

Definition typing: \fbox{\provesD{\d}} \\ \\

\begin{tabular}{lc}
\newinfrule{
\begin{array}{c}
\tvenv = \seq{\tvone \subtype \tappone}
\rulesep
\provesW{\seq{\tappone}}\rulesep
\provesW{\seq{\tapptwo}}\rulesep
\provesM{\self:\tname\bsTP{\seq{\tvone}}}{T}{\seq{\fd}}\\
\oneOwner(\tname)
\end{array}
}
{\provesD{\tdsyntaxTD}}
{\tTraitDefRule} \\ \\

\newinfrule{
\begin{array}{c}
\tvenv = \seq{\tvone \subtype \tappone}\rulesep
\provesW{\seq{\tappone}}\rulesep
\provesW{\tys}\rulesep
\provesW{\seq{\tapptwo}}\\
\provesM{\self:\oname\bsTP{\seq{\tvone}}~\seq{\vname:\ty}}{O}{\seq{\fd}}
\rulesep
\oneOwner(\oname)
\end{array}
}
{\provesD{\odsyntaxOD}}
{\tObjectDefRule} \\ \\

\end{tabular}

Method typing: \fbox{\provesM{\tyenv}{C}{\fd}} \\ \\

\begin{tabular}{lc}
\newinfrule{
\begin{array}{c}
\dsyntaxMT \inp
\rulesep
\override(\fname, \set{\seq{\tapptwo}}, \ftnty)
\\
\prm\tvenv = \tvenv~\seq{\tvone \subtype \tappone}
\rulesep
\provesWD{\prm\tvenv}{\seq{\tappone}}\rulesep
\provesWD{\prm\tvenv}{\tys}\rulesep
\provesWD{\prm\tvenv}{\retty}\rulesep
\\
{\provesE {\prm\tvenv}
          {\tyenv~\seq{\vname:\ty}}
          {\exp}{\tyP}}\rulesep
\provesSD{\prm\tvenv}{\tyP}{\retty}
\end{array}
}
{\provesM{\tyenv}{C}{\fdsyntaxMD}}
{\tMethodDefRule} \\ \\
\end{tabular}

Method overriding: \fbox{\overrideTemplate} \\ \\
\begin{tabular}{lc}
\newinfrule{
\begin{array}{c}
{
\bigcup_{\sub\tappthree i\in\set{\seq{\tappthree}}} 
\mtypeF{\sub\tappthree i} = \set{\ftntytwo}
}
\\
\begin{array}{c}
\seq{\tappone} = \substseq{\tvone}{\tvtwo}\seq{\tapptwo}\rulesep 
\tys = \substseq{\tvone}{\tvtwo}\seq{\tyP}\rulesep 
\provesSD{\tvenvone}{\retty}{\substseq{\tvone}{\tvtwo}\rettytwo}
\end{array}
\end{array}
}
{\overrideD}
{\overrideRule}\\ \\
\end{tabular}

Method type lookup: \fbox{\mtypeF{\ty} = \set{\ftntyTemplate}}\\

\begin{tabular}{lc}
\newinfrule{
\begin{array}{c}
\dsyntaxMB\inp\rulesep
\fdsyntaxMTB\in\set{\seq\fd}
\end{array}
}
{\mtypeF{\capp} = \set{\substseq{\ty}{\tvone}\ftntytwo}}
{\mtBothRule} \\ \\

\newinfrule{
\begin{array}{c}
\dsyntaxMS\inp\rulesep
\fname\not\in\set{\seq{\Fname(\fd)}}
\end{array}
}
{\mtypeF{\capp} = 
\bigcup_{\sub\tappone i\in{\set{\seq\tappone}} }
 \mtypeF{\substseq{\ty}{\tvone} \sub\tappone i}}
{\mtSuperRule} \\ \\

\newrule{\mtypeF\obj = \emptyset}{\mtObjRule}\\ \\

\end{tabular}

\caption{Static Semantics of \basiccore\ (I)}\label{fig:basic-static1}
\end{figure}

\begin{figure}[htbp]
Expression typing: \fbox{\provesEd{\exp}{\ty}} \\ \\

\begin{tabular}{lc}
\newrule{\provesEd{\vname}{\tyenv(\vname)}}{\tVarRule} \\ \\

\newrule{\provesEd{\self}{\tyenv(\self)}}{\tSelfRule} \\ \\

\newinfrule{
\begin{array}{c}
\odsyntaxObj\inp\rulesep
\provesW{\oapp}\\
\provesEd{\seq{\exp}}{\seq{\tyPP}}\rulesep
\provesS{\seq{\tyPP}}{\substseq{\ty}{\tvone}\seq{\tyP}}
\end{array}
}
{\provesEd{\oapp\eargs}{\oapp}}
{\tObjectRule} \\ \\

\newinfrule{
\begin{array}{c}
\provesEd{\sub\exp 0}{\retty}\rulesep
\boundF(\retty) = \oappP\rulesep
\odsyntaxFld\inp
\end{array}
}
{\provesEd{\sub\exp 0\mt{.}\sub\vname i}{\substseq{\tyP}{\tvone}{\tyn i}}}
{\tFieldRule} \\ \\

\newinfrule{
\begin{array}{c}
\provesEd{\sub\exp 0}{\retty}\rulesep
\mtypeF{\boundF(\retty)} = \set{\ftntyone}\\
\provesW{\tys}\rulesep
\provesS{\tys}{\substseq{\ty}{\tvone}\seq\tappone}\\
\provesEd{\seq\exp}{\seq{\tyPP}}\rulesep
\provesS{\seq{\tyPP}}{\substseq{\ty}{\tvone}\tysP}\\
\end{array}
}
{\provesEd{\invoke{\exp_0}{\ftapp\eargs}}{\substseq{\ty}{\tvone}\rettytwo}}
{\tInvkRule} \\ \\
\end{tabular}

Subtyping: \fbox{\provesS{\ty}{\ty}}\\\\

\begin{tabular}{lc}
\newrule{{\provesS{\ty}{\obj}}}{\sObjRule} \\ \\

\newrule{{\provesS{\ty}{\ty}}}{\sReflRule} \\ \\

\newinfrule{
\begin{array}{c}
\provesS{\tyn 1}{\tyn 2}\rulesep
\provesS{\tyn 2}{\tyn 3}
\end{array}
}
{\provesS{\tyn 1}{\tyn 3}}{\sTransRule} \\ \\

\newrule{\provesS{\tvone}{\tvenv(\tvone)}}{\sVarRule} \\ \\

\newinfrule{
\begin{array}{c}
\dsyntaxSB\inp\rulesep
\end{array}
}
{\provesS{\capp}{\substseq{\ty}{\tvone}\sub\tappone i}}{\sBothRule} \\ \\
\end{tabular}

Well-formed types: \fbox{\provesW\ty} \\ \\
\begin{tabular}{lc}
\newrule{\provesW{\obj}}{\wObjRule} \\ \\

\newinfrule{
\begin{array}{c}
\tvone\in\me{dom}(\tvenv)
\end{array}
}
{\provesW{\tvone}}{\wVarRule} \\ \\

\newinfrule{
\begin{array}{c}
\dsyntaxWB\inp\rulesep
\provesW{\tys}\rulesep
\provesS{\tys}{\substseq{\ty}{\tvone}\seq\tappone}
\end{array}
}
{\provesW{\capp}}{\wBothRule} \\ \\
\end{tabular}

\caption{Static Semantics of \basiccore\ (II)}\label{fig:basic-static2}
\end{figure}

\begin{figure}[htbp]


Bound of type: \fbox{\boundF(\ty) = \tynontv} \\

$\begin{array}{lcl}
\boundF(\tvone) &=& \tvenv(\tvone)\\
\boundF(\tynontv) &=& \tynontv \\
\end{array}$\\ \\

One owner for all the visible methods: \fbox{\oneOwner(\cname)} \\ \\
\begin{tabular}{lc}
\newinfrule{
\begin{array}{c}
\forall\fname\in\visible(\cname)~.~
\mbox{$f$ \emph{only occurs once in }}\visible(\cname)
\end{array}
}
{\oneOwner(\cname)}
{\oneOwnerRule}\\ \\
\end{tabular}

Auxiliary functions for methods:
\fbox{\defined~/~\inherited~/~\visible(\cname) = \set{\seq\fname}}\\

$
\begin{array}{rlll}
\defined(\cname) = & 
\set{\seq{\Fname(\fd)}}
& \mbox{where } \ignore\ \cname\ignore\fds\ignore\ \inp
\\ \\

\inherited(\cname) = & 
\biguplus_{\tappone_i\in\set{\seq\tappone}}\
\set{\fname_i~|~
\fname_i\in\visible(\tappone_i),
\fname_i\not\in\defined(\cname)}
& \mbox{where } \dsyntaxOVER \inp 
\\ \\

\visible(\cname) = & \defined(\cname) \uplus\ \inherited(\cname)
\\ \\
\end{array}
$

\caption{Static Semantics of \basiccore\ (III)}\label{fig:basic-static3}
\end{figure}

We proved the type soundness of \basiccore\ using the standard
technique of proving a progress theorem and a subject reduction
theorem.
