%%%%%%%%%%%%%%%%%%%%%%%%%%%%%%%%%%%%%%%%%%%%%%%%%%%%%%%%%%%%%%%%%%%%%%%%%%%%%%%%
%   Copyright 2012, Oracle and/or its affiliates.
%   All rights reserved.
%
%
%   Use is subject to license terms.
%
%   This distribution may include materials developed by third parties.
%
%%%%%%%%%%%%%%%%%%%%%%%%%%%%%%%%%%%%%%%%%%%%%%%%%%%%%%%%%%%%%%%%%%%%%%%%%%%%%%%%

%%%%%%%%%%%%%%%%%%%%%%%%%%%%%%%%%%%%%%%%%%%%%%%%%%%%%%%%%%%%%%%%%%%%%%%%%%%%%%%%
%%    Copyright 2010 Sun Microsystems, Inc.,
%%    4150 Network Circle, Santa Clara, California 95054, U.S.A.
%%    All rights reserved.
%%
%%    U.S. Government Rights - Commercial software.
%%    Government users are subject to the Sun Microsystems, Inc. standard
%%    license agreement and applicable provisions of the FAR and its supplements.
%%
%%    Use is subject to license terms.
%%
%%    This distribution may include materials developed by third parties.
%%
%%    Sun, Sun Microsystems, the Sun logo and Java are trademarks or registered
%%    trademarks of Sun Microsystems, Inc. in the U.S. and other countries.
%%
%%
%% Stuff for the EMACS "fortify" command (M-&) %%%%%%%%%%%%%%%%%
%%%%%%%%%%%%%%%%%%%%%%%%%%%%%%%%%%%%%%%%%%%%%%%%%%%%%%%%%%%%%%%%

\usepackage{color}
\usepackage{amssymb} %maths
\usepackage{amsmath} %maths
\usepackage{stmaryrd}
\usepackage{graphicx}
\usepackage[utf8]{inputenc} %useful to type directly accentuated characters


\newcommand\twointersectand{\mathbin{\wedge\mskip-8mu\wedge}}
\newcommand\twointersector{\mathbin{\vee\mskip-8mu\vee}}
\newcommand\twointersectxor{\mathbin{\underline{\vee\mskip-8mu\vee}}}



\newlength{\FortressCodeIndent}
\setlength{\FortressCodeIndent}{1em}
\newlength{\FortressParSkip}
\setlength{\FortressParSkip}{\parskip}
\def\Fortress{\list{}{\leftmargin \FortressCodeIndent
                      \itemindent\listparindent
                      \parsep 0pt plus 1pt}\item
                      \begingroup\FortressInnerMacros\tabbing}
\makeatletter
% Need to be slightly different from \endtabbing
\def\endFortress{\unskip\@stopfield\@addfield\ifdim\wd\@curline>0pt\@startfield\@stopline\fi
  \ifnum\@tabpush >\z@ \@badpoptabs \fi\endtrivlist\endgroup\endlist}
\makeatother
\newdimen\FortressMathsurround   \FortressMathsurround=0.166666em
\newcommand\FortressMathsurroundFixup[1]{\ifdim \lastskip = 0pt \hskip-#1\FortressMathsurround \fi}

\newcommand\FortressOuterKWD[1]{{\FortressMathsurroundFixup{1}\ensuremath{\mathtt{#1}\mathsurround=\FortressMathsurround}}}
\newcommand\FortressOuterKWDVAR[1]{{\FortressMathsurroundFixup{1}\ensuremath{\mathtt{#1}\mathsurround=\FortressMathsurround}}}
\newcommand\FortressOuterOPR[1]{{\FortressMathsurroundFixup{1}\ensuremath{\mathtt{#1}\mathsurround=\FortressMathsurround}}}
\newcommand\FortressOuterTYP[1]{{\FortressMathsurroundFixup{0.5}\ensuremath{\mathrm{#1}\mathsurround=0.5\FortressMathsurround}}}
\newcommand\FortressOuterVAR[1]{{\FortressMathsurroundFixup{0.5}\ensuremath{\mathit{#1}\mathsurround=0.5\FortressMathsurround}}}
\newcommand\FortressOuterSTR[1]{{\FortressMathsurroundFixup{1}\ensuremath{\hbox{\tt\usefont{T1}{pcr}{m}{n}\selectfont#1}\mathsurround=\FortressMathsurround}}}
\newcommand\EXP[1]{{\FortressMathsurroundFixup{1}\ensuremath{{\protect\FortressInnerMacros\let\*=\fortresslinecomment#1}\mathsurround=\FortressMathsurround}}}
\newcommand\FortressOuterMacros{\let\KWD\FortressOuterKWD \let\KWDVAR\FortressOuterKWDVAR
  \let\OPR\FortressOuterOPR \let\TYP\FortressOuterTYP \let\VAR\FortressOuterVAR \let\STR\FortressOuterSTR}

\FortressOuterMacros

\newcommand\FortressInnerKWD[1]{\mathrel{\mathtt{#1}}}
\newcommand\FortressInnerKWDVAR[1]{\mathord{\mathtt{#1}}}
\newcommand\FortressInnerOPR[1]{\mathbin{\mathtt{#1}}}
\newcommand\FortressInnerBIGOPR[1]{\mathop{\lower0.4ex\hbox{\vrule height 1.2em depth 0.1em width 0pt\Large$\mathtt{#1}$}}}
\newcommand\FortressInnerBIGOP[1]{\mathop{\lower0.2ex\hbox{\Large$#1$}}}
\newcommand\FortressInnerTYP[1]{\mathord{\mathrm{#1}}}
\newcommand\FortressInnerVAR[1]{\mathord{\mathit{#1}}}
\newcommand\FortressInnerSTR[1]{\hbox{\tt\usefont{T1}{pcr}{m}{n}\selectfont#1}}
\newcommand\FortressInnerMacros{\let\KWD\FortressInnerKWD \let\KWDVAR\FortressInnerKWDVAR
  \let\OPR\FortressInnerOPR \let\BIGOPR\FortressInnerBIGOPR \let\BIGOP\FortressInnerBIGOP
  \let\TYP\FortressInnerTYP \let\VAR\FortressInnerVAR \let\STR\FortressInnerSTR}

\newcommand\CONDEQ{\mathrel{\mathtt{:}}=\mathrel{\mathtt{:}}}
\newcommand\ASSIGN{\mathrel{\mathtt{:}}=}
\newcommand\COLON{\mathpunct{\mathtt{:}}}
\newcommand\COLONOP{\mathinner{\mathtt{:}}}
\newcommand\SHORTCUT[1]{\mathrel{\mathord{#1}\mathord{:}}}
\newcommand\verythin{{\ensuremath{\mskip 1.5mu}}}
\newcommand\ultrathin{{\ensuremath{\mskip 0.75mu}}}
\def\Vvert{\mathrel{\vert\mskip-1.5mu\vert\mskip-1.5mu\vert}}
\def\VVert{\mathrel{\vert\mskip-1.5mu\vert\mskip-1.5mu\vert\mskip-1.5mu\vert}}
\def\sequiv{\mathrel{\hbox{\raise0.215ex\hbox to 0pt{$\equiv$\hss}\lower0.215ex\hbox{$\equiv$}}}}

\newcommand\bigllbracket{\bigl[\mkern-5mu\bigl[}
\newcommand\bigrrbracket{\bigr]\mkern-5mu\bigr]}
\newcommand\Bigllbracket{\Bigl[\mkern-5.5mu\Bigl[}
\newcommand\Bigrrbracket{\Bigr]\mkern-5.5mu\Bigr]}
\newcommand\biggllbracket{\biggl[\mkern-6mu\biggl[}
\newcommand\biggrrbracket{\biggr]\mkern-6mu\biggr]}
\newcommand\Biggllbracket{\Biggl[\mkern-6.5mu\Biggl[}
\newcommand\Biggrrbracket{\Biggr]\mkern-6.5mu\Biggr]}
\newcommand\leftllbracket{\left[\mkern-5.5mu\left[}
\newcommand\rightrrbracket{\right]\mkern-5.5mu\right]}


\newcommand\FortressUnknownCharacter[1]{{\fboxsep=0.166666em\fbox{\small\tt#1}}}

%% The old FortressDoc (introduced in r1437) is not working any more.
%% It is going to be replaced by the new fortify tool soon.
%% We'll not adjust the indentation of the Fortress documentation
%% embedded in the Fortress code for now.
%%
\newenvironment{FortressDoc}[1]{\tt {#1}}{}
% \newenvironment{FortressDoc}[1]{\vspace{0.5ex}%
% \begin{list}{}{%
%   \setlength{\parskip}{\FortressParSkip}
%   \settowidth{\leftmargin}{{\tt {#1}}}%
%   \addtolength{\leftmargin}{\FortressCodeIndent}}%
%      \item[]\raggedright%
% }{\unskip\end{list}\vspace{-3ex}}


\newcommand\FortexSettings{%
\setlength{\FortressCodeIndent}{0em}%
}

\medmuskip=4mu plus 2mu minus 1mu\relax      %Normal LaTeX has 4mu plus 2mu minus 4mu


\newdimen\fortressblanklineskip  \fortressblanklineskip=4pt
\newdimen\fortresscommentparskip  \fortresscommentparskip=4pt
\newdimen\fortresscommentrulethickness  \fortresscommentrulethickness=0.5pt
\newdimen\fortressleftindentincrement
\newdimen\fortressrightindentincrement  \fortressrightindentincrement=1em
\newdimen\fortresscommentruleraise
\newdimen\fortresscommentrulemargin
\newdimen\fortresscommentrulecorner
\newdimen\fortresscommentparensoffset

\newbox\fortresscodeline
\newbox\fortresstempbox
\newbox\fortressresultbox
\newbox\fortresscommentbox
\newbox\fortresstabstack
\newbox\fortresssavedtabstack
\newdimen\fortresstempdimen
\newdimen\fortresscurindent
\newdimen\fortressnewindent
\newif\iffortresstopcommentline
\newif\iffortressbottomcommentline
\newif\iffortresscommentalignblock
\newif\iffortresspng  \fortresspngfalse

\definecolor{almostwhite}{rgb}{0.99,0.99,0.99}

\newcommand\FortressCode{\lineskip=1.5pt\lineskiplimit=1.5pt
  \setbox\fortresssavedtabstack=\box\fortresstabstack
  \global\setbox\fortresstabstack=\hbox{\hbox{}}\relax
  \FortressInnerMacros
  \let\+=\fortresspushtab
  \let\-=\fortresspoptab
  \let\1=\fortresscodeindentone
  \let\2=\fortresscodeindenttwo
  \let\3=\fortresscodeindentthree
  \let\*=\fortresslinecomment
  \def~{{\tt\ }}\let\\=\fortressline
  \let\[=\fortressbegincommentalignblock
  \let\]=\fortressendcommentalignblock
  \def\digitblank{{\setbox0=\hbox{0}\hbox to \wd0{\hss}}}%
  \fortresscommentalignblockfalse
  \topsep=4pt plus 2pt   \parsep=\parskip  \partopsep=1pt plus 1pt
  \trivlist
  \item[]%
  % The \mathopen in the next line prevents a leading \mathrel (as for a keyword) from inserting glue.
  \fortressbeginline\iffortresspng\mathopen{\hbox to 0pt{\hskip\hsize\color{almostwhite}\vrule height 0.001pt width 0.001pt\hss}}\fi\ignorespaces}
\def\endFortressCode{\fortressendline\endtrivlist
  \global\setbox\fortresstabstack=\box\fortresssavedtabstack}

\newcommand\fortressbeginline{\fortressretrievenewindent
  \fortresscurindent=\fortressnewindent
  \setbox\fortresscodeline=\hbox{\iffortresspng\hbox to 0pt{\color{almostwhite}\vrule height 0.001pt width 0.001pt\hss}\fi
       \hskip\fortresscurindent\strut}%
  \fortressbeginfield}
\newcommand\fortressendline{\unskip\fortressendfield
  \ifdim \wd\fortresscodeline > \fortresscurindent
    \noindent\box\fortresscodeline\par
  \else
    \iffortresscommentalignblock
      \noalign{\vskip\fortressblanklineskip}\relax
    \else
      \vskip\fortressblanklineskip
    \fi
  \fi}
\newcommand\fortresscancelline{\fortressendfield}

\newcommand\fortressbeginfield{\setbox\fortresscodeline=\hbox\bgroup
  \box\fortresscodeline\(\displaystyle\ignorespaces}
  
\newcommand\fortressendfield{\null\)\egroup}

\newcommand\fortresspushtab{\fortressendfield
  \fortressnewindent=\wd\fortresscodeline
  \global\setbox\fortresstabstack=\hbox{\unhbox\fortresstabstack
                                        \hbox to \fortressnewindent{\hss}}\relax
  \fortressbeginfield}
  
\newcommand\fortresspoptab{\fortressendfield
  \global\setbox\fortresstabstack=\hbox{\unhbox\fortresstabstack
                                        \setbox\fortresstempbox=\lastbox}\relax
  \fortressretrievenewindent
  \fortressbeginfield}

\newcommand\fortressretrievenewindent{\relax
  \global\setbox\fortresstabstack=\hbox{\unhbox\fortresstabstack
                                        \global\setbox\fortresstempbox=\lastbox}\relax
  \fortressnewindent=\wd\fortresstempbox
  \global\setbox\fortresstabstack=\hbox{\unhbox\fortresstabstack
                                        \box\fortresstempbox}}

\newcommand\fortresscodeindentone{\mskip 8mu }
\newcommand\fortresscodeindenttwo{\1\1}
\newcommand\fortresscodeindentthree{\1\1\1}

\newcommand\fortresscommentsep{\hskip 1.5em }

%\newcommand\fortresslinecommentsymbol{\hbox{\raise 2pt\hbox{$\scriptscriptstyle\circledast$}}}
%\newcommand\fortresslinecommentsymbol{\hbox{\raise 0.37ex\hbox{\hbox to 0pt{$\scriptstyle\otimes$}\hbox{$\scriptstyle\oplus$}}}}
%\newcommand\fortresslinecommentsymbol{\hbox{\setbox0=\hbox{$\scriptstyle\odot$}\relax
%  \hbox{\raise 0.2ex\hbox to \wd0{\hss$\ast$\hss}\raise 0.37ex\hbox to 0pt{\hss\box0}}}}
\newcommand\fortresslinecommentsymbol{\hbox{\setbox0=\hbox{$\scriptstyle\odot$\hss}\relax
  \hbox{\raise 0.07ex\hbox to \wd0{\hss\tt*\hss}\raise 0.37ex\hbox to 0pt{\hss\box0}}}}
\newcommand\fortresslinecomment[1]{\fortresslinecommentsymbol\hbox{\FortressOuterMacros\def~{\penalty10000\ }\rm#1}}
\newcommand\fortressalignedlinecomment[1]{\fortressendfield
  \leavevmode\box\fortresscodeline&\setbox\fortresscodeline=\hbox{}\fortressbeginfield
  \fortresslinecommentsymbol\hbox{\FortressOuterMacros\def~{\penalty10000\ }\rm#1}}  

\newcommand\fortressonelinecomment[3]{\hbox{{\tt#1}\FortressOuterMacros\def~{\penalty10000\ }\rm#2{\tt#3}}}

\newcommand\fortressline{\fortressendline\fortressbeginline}
\newcommand\fortresscommentalignblockline{\fortressendfield
  \leavevmode\box\fortresscodeline\cr}

\newcommand\fortressbegincommentalignblock{\fortresscancelline\begingroup
  \let\\=\fortresscommentalignblockline
  \let\*=\fortressalignedlinecomment
  \tabskip=0pt\halign\bgroup
  \fortressbeginline##\hfil&##\hfil\cr}
\newcommand\fortressendcommentalignblock{\egroup\endgroup\fortressbeginline}

\newcommand\FortressBeginComment{\fortressblockcomment{}{}}
\newcommand\FortressBeginFlushComment{\fortressblockcomment{\leftskip=0pt\rightskip=0pt}{}}
\newcommand\FortressBeginFlushLeftComment{\fortressblockcomment{\leftskip=0pt}{}}
\newcommand\FortressBeginFlushRightComment{\fortressblockcomment{\rightskip=0pt}{}}
\newcommand\FortressTaggedBeginComment{\fortressblockcomment{}}
\newcommand\FortressTaggedBeginFlushComment{\fortressblockcomment{\leftskip=0pt\rightskip=0pt}}
\newcommand\FortressTaggedBeginFlushLeftComment{\fortressblockcomment{\leftskip=0pt}}
\newcommand\FortressTaggedBeginFlushRightComment{\fortressblockcomment{\rightskip=0pt}}

%%% Important invariant in block comments: whenever you are in horizontal mode, you are
%%% within a \begingroup, and returning to vertical mode must always end that group.

\newcommand\fortressblockcomment[2]{\fortresscancelline
  \begingroup\lineskip=1pt\lineskiplimit=1pt
  \setbox\fortresscommentbox=\vbox\bgroup
  \advance\hsize by -\fortresscurindent
  \setbox\fortresstempbox=\hbox{\tt (*~}\leftskip=\wd\fortresstempbox
  \setbox\fortresstempbox=\hbox{\tt ~*)}\rightskip=\wd\fortresstempbox
  \sloppy
  #1\relax
  \FortressOuterMacros
  \let\item=\fortressitem
  \let\enum=\fortressenum
  \let\desc=\fortressdesc
  \let\lind=\fortressleftindent
  \def~{\penalty10000\ }
  \let\\=\fortressendcommentparagraph
  \let\fortressdocommentrule=\relax
  \def\leftrule{\def\fortressdocommentrule{\fortresscommentrule{\fortresspositionleftrule}}}\relax
  \def\rightrule{\def\fortressdocommentrule{\fortresscommentrule{\fortresspositionrightrule}}}\relax
  \def\leftrightrule{\def\fortressdocommentrule{\fortresscommentrule{\fortresspositionleftrightrule}}}\relax
  \fortresstopcommentlinefalse
  \fortressbottomcommentlinefalse
  \def\topcommentline{\fortresscommentline{\rightskip}{0pt}{0pt}\fortresstopcommentlinetrue\par\endgroup}\relax
  \def\bottomcommentline{\ifhmode\par\endgroup\fi\begingroup\leavevmode
    \fortresscommentline{\leftskip}{0pt}{0pt}\fortressbottomcommentlinetrue}\relax
  \def\toprightcommentline{\fortresscommentline{\rightskip}{0pt}{\fortresscommentrulemargin}\fortresstopcommentlinetrue\par\endgroup}\relax
  \def\bottomleftcommentline{\ifhmode\par\endgroup\fi\begingroup\leavevmode
    \fortresscommentline{\leftskip}{\fortresscommentrulemargin}{0pt}\fortressbottomcommentlinetrue}\relax
  \setbox\fortresstempbox=\hbox{99.$\mskip 7mu$}\relax
  \fortressleftindentincrement=\wd\fortresstempbox
  \parindent=0pt
  \parskip=\fortresscommentparskip
  \tt \fortresscommentruleraise=0.6ex
      \fortresscommentrulecorner=0.5ex
      \fortresscommentparensoffset=0.3ex
      \setbox\fortresstempbox=\hbox{\tt ~}\relax
      \fortresscommentrulemargin=0.8\wd\fortresstempbox
  \rm
  \leavevmode\begingroup
  \hbox{\hskip -\leftskip \hskip-\fortresscommentparensoffset{\tt (*#2~}\hskip\fortresscommentparensoffset}}

\newcommand\FortressEndCommentLeft{\ifhmode\par\fortressendcommentendgroup\fi\leavevmode
  \hbox{\hskip -\leftskip \hskip-\fortresscommentparensoffset{\tt ~*)}\hskip\fortresscommentparensoffset}\relax
  \fortressendblockcomment}

\newcommand\FortressEndComment{\ifhmode\fortressendcommentendgroup\else\leavevmode\fi
  \penalty10000\hfill
  \hbox{\hskip\fortresscommentparensoffset
        {\tt ~*)}\hskip-\fortresscommentparensoffset\hskip -\rightskip}\relax
        \fortressendblockcomment}

\newcommand\FortressTaggedEndCommentLeft[1]{\ifhmode\par\fortressendcommentendgroup\fi\leavevmode
%\newcommand\FortressTaggedEndCommentLeft[1]{\ifhmode\par\endgroup\fi\fortressendcommentendgroup\leavevmode
  \hbox{\hskip -\leftskip \hskip-\fortresscommentparensoffset{\tt ~#1*)}\hskip\fortresscommentparensoffset}\relax
  \fortressendblockcomment}

\newcommand\FortressTaggedEndComment[1]{\ifhmode\fortressendcommentendgroup\else\leavevmode\fi
  \penalty10000\hfill
  \hbox{\hskip\fortresscommentparensoffset{\tt ~#1*)}\hskip -\fortresscommentparensoffset
        \hskip -\rightskip}\fortressendblockcomment}

\newcommand\fortressendcommentendgroup{\global\let\fortresstemp=\fortressdocommentrule
  \iffortressbottomcommentline \global\fortresstempdimen=1pt \else \global\fortresstempdimen=0pt \fi
  \endgroup
  \let\fortressdocommentrule=\fortresstemp
  \ifdim \fortresstempdimen > 0pt \fortressbottomcommentlinetrue \else \fortressbottomcommentlinefalse \fi}

\newcommand\fortressendblockcomment{\par
  \fortressdocommentrule\egroup
  \unvbox\fortresscommentbox\endgroup
  \fortressbeginline}
  
\newcommand\fortresscommentline[3]{{\fortresstempdimen=\hsize
  \advance\fortresstempdimen by -#1\relax
  \setbox\fortresstempbox=\hbox{\tt (*~}\relax
  \advance\fortresstempdimen by -\wd\fortresstempbox
  \leavevmode
  \raise \fortresscommentruleraise\hbox{\hbox to 0pt{\hss
        \vrule height \fortresscommentrulethickness
               width \ifdim \leftskip=0pt #2 \else\fortresscommentrulemargin\fi}\relax
        \vrule height \fortresscommentrulethickness
               width \fortresstempdimen
        \hbox to 0pt{\relax
          \vrule height \fortresscommentrulethickness
               width\ifdim \rightskip=0pt #3 \else\fortresscommentrulemargin\fi\hss}\relax}}}

\newcommand\fortressrightindent[1]{\advance\rightskip by #1\fortressrightindentincrement}

\newcommand\fortressleftindent[1]{\ifhmode\par\endgroup\fi\begingroup
  \advance\leftskip by #1\fortressleftindentincrement
  \leavevmode\ignorespaces}

\newcommand\fortressitem[2]{\fortressleftindent{#1}%
  \hbox to 0pt{\hss\csname fortressitem\romannumeral#2\endcsname$\mskip 7mu$}\ignorespaces}
  
\newcommand\fortressitemi{\raise 0.2ex\hbox{$\scriptstyle\bullet$}}
\newcommand\fortressitemii{\raise 0.2ex\hbox{$\scriptstyle\circ$}}
\newcommand\fortressitemiii{\raise 0.2ex\hbox{$\scriptscriptstyle\blacktriangleright$}}
\newcommand\fortressitemiv{\raise 0.2ex\hbox{$\scriptstyle\triangleright$}}
\newcommand\fortressitemv{$\star$}
\newcommand\fortressitemvi{$\diamond$}
\newcommand\fortressitemvii{$\sim$}
\newcommand\fortressitemviii{$\cdot$}

\newcommand\fortressenum[3]{\fortressleftindent{#1}%
  \hbox to 0pt{\hss\csname fortressenum\romannumeral#2\endcsname{#3}$\mskip 7mu$}\ignorespaces}

\newcommand\fortressenumi[1]{(\number #1)}
\newcommand\fortressenumii[1]{(\romanletter{#1})}
\newcommand\fortressenumiii[1]{(\romannumeral #1)}
\newcommand\fortressenumiv[1]{(\Romanletter{#1})}
\newcommand\fortressenumv[1]{\number #1.}
\newcommand\fortressenumvi[1]{\Romanletter{#1}.}
\newcommand\fortressenumvii[1]{\romannumeral #1.}
\newcommand\fortressenumviii[1]{\romanletter{#1}.}

\makeatletter
\newcommand\romanletter[1]{\@alph{#1}}   % Makes use of LaTeX internals
\newcommand\Romanletter[1]{\@Alph{#1}}   % Makes use of LaTeX internals
\makeatother

% \FortressDescription{level}{labeltext}{leftindent}
\newcommand\fortressdesc[3]{\ifhmode\par\endgroup\fi\begingroup
  \advance\leftskip by #1\fortressleftindentincrement
  \advance\leftskip by #3\fortressleftindentincrement
  \advance\leftskip by -\fortressleftindentincrement
 \setbox\fortresstempbox=\hbox{\hskip-#3\fortressleftindentincrement
                               \hbox{#2}\relax   % This inner \hbox makes \FortressMathsurroundFixup work properly
                               \quad}\relax
  \leavevmode\copy\fortresstempbox
  \ifdim \wd\fortresstempbox < 0pt \hskip-\wd\fortresstempbox \fi
  \ignorespaces}

\newcommand\fortressendcommentparagraph{\ifhmode\par\endgroup\else\vskip\parskip\fi}

\newskip\fortressskipA
\newskip\fortressskipB
\newskip\fortressskipC
\newskip\fortressskipD
\newskip\fortressskipE
\newskip\fortressskipF
\newcount\fortresspenaltyA
\newcount\fortresspenaltyB
\newcount\fortresspenaltyC
\newcount\fortresspenaltyD
\newcount\fortresspenaltyE
\newcount\fortresspenaltyF
\newbox\fortressboxA
\newbox\fortressboxB
\newif\iffortress

\newcommand\fortresscommentrule[1]{\begingroup
    \fortressskipF=\lastskip \unskip
    \fortresspenaltyF=\lastpenalty \unpenalty
    \fortressskipE=\lastskip \unskip
    \fortresspenaltyE=\lastpenalty \unpenalty
    \fortressskipD=\lastskip \unskip
    \fortresspenaltyD=\lastpenalty \unpenalty
    \setbox\fortressboxB\lastbox
    \setbox\fortressresultbox=\vbox{
      \hbox{#1{\fortressbottomleftrule}{\fortressbottomrightrule}\relax
            \box\fortressboxB}
      \penalty\fortresspenaltyD\fortressrulevskip{#1}{\fortressskipD}
      \penalty\fortresspenaltyE\fortressrulevskip{#1}{\fortressskipE}
      \penalty\fortresspenaltyF\fortressrulevskip{#1}{\fortressskipF}}
    \fortressskipC=\lastskip \unskip
    \fortresspenaltyC=\lastpenalty \unpenalty
    \fortressskipB=\lastskip \unskip
    \fortresspenaltyB=\lastpenalty \unpenalty
    \fortressskipA=\lastskip \unskip
    \fortresspenaltyA=\lastpenalty \unpenalty
    \setbox\fortressboxA\lastbox
    \iffortressbottomcommentline
      \fortressskipC=0pt
      \fortressskipA=0pt
    \fi
  \loop
    \fortressskipF=\fortressskipC
    \fortresspenaltyF=\fortresspenaltyC
    \fortressskipE=\fortressskipB
    \fortresspenaltyE=\fortresspenaltyB
    \fortressskipD=\fortressskipA
    \fortresspenaltyD=\fortresspenaltyA
    \setbox\fortressboxB=\box\fortressboxA
    \fortressskipC=\lastskip \unskip
    \fortresspenaltyC=\lastpenalty \unpenalty
    \fortressskipB=\lastskip \unskip
    \fortresspenaltyB=\lastpenalty \unpenalty
    \fortressskipA=\lastskip \unskip
    \fortresspenaltyA=\lastpenalty \unpenalty
    \setbox\fortressboxA=\lastbox
    \ifvoid\fortressboxA \fortressfalse \else \fortresstrue \fi
  \iffortress
    \setbox\fortressresultbox=\vbox{
      \hbox{#1{\fortressmaincommentrule}{\fortressmaincommentrule}\relax
            \box\fortressboxB}
      \penalty\fortresspenaltyD\fortressrulevskip{#1}{\fortressskipD}
      \penalty\fortresspenaltyE\fortressrulevskip{#1}{\fortressskipE}
      \penalty\fortresspenaltyF\fortressrulevskip{#1}{\fortressskipF}
      \unvbox\fortressresultbox}
  \repeat
  \penalty\fortresspenaltyA\vskip\fortressskipA
  \penalty\fortresspenaltyB\vskip\fortressskipB
  \penalty\fortresspenaltyC\vskip\fortressskipC
  \hbox{#1{\fortresstopleftrule}{\fortresstoprightrule}\relax
        \box\fortressboxB}
  \iffortresstopcommentline\else
    \penalty\fortresspenaltyD\fortressrulevskip{#1}{\fortressskipD}
  \fi
  \penalty\fortresspenaltyE\fortressrulevskip{#1}{\fortressskipE}
  \iffortresstopcommentline\else
    \penalty\fortresspenaltyF\fortressrulevskip{#1}{\fortressskipF}
  \fi
  \unvbox\fortressresultbox
  \endgroup}

\newcommand\fortressmaincommentrule{\vrule width \fortresscommentrulethickness
                                           height \ht\fortressboxB
                                           depth \dp\fortressboxB}
\newcommand\fortresstopleftrule{\begingroup
             \fortresstempdimen=\fortresscommentrulethickness
             \advance\fortresstempdimen by \fortresscommentruleraise
             \vrule width \fortresscommentrulethickness
                    height \fortresstempdimen
                    depth \dp\fortressboxB
             \iffortresstopcommentline
               \raise \fortresscommentruleraise\hbox to 0pt{\relax
                 \vrule width \fortresscommentrulecorner
                        height \fortresscommentrulethickness
                        depth 0pt\hss}\fi
           \endgroup}
\newcommand\fortresstoprightrule{\begingroup
             \fortresstempdimen=\fortresscommentrulethickness
             \advance\fortresstempdimen by \fortresscommentruleraise
             \vrule width \fortresscommentrulethickness
                    height \fortresstempdimen
                    depth \dp\fortressboxB
           \endgroup}
\newcommand\fortressbottomleftrule{\begingroup
                 \fortresstempdimen=\ht\fortressboxB
                 \advance\fortresstempdimen by -\fortresscommentruleraise
                 \raise \fortresscommentruleraise\hbox{\relax
                   \vrule width \fortresscommentrulethickness
                          height \fortresstempdimen
                          depth 0pt}\relax
               \endgroup}
\newcommand\fortressbottomrightrule{\begingroup
                 \fortresstempdimen=\ht\fortressboxB
                 \advance\fortresstempdimen by -\fortresscommentruleraise
                 \raise \fortresscommentruleraise\hbox{\relax
                   \iffortressbottomcommentline
                      \hbox to 0pt{\hss
                         \vrule width \fortresscommentrulecorner
                                height \fortresscommentrulethickness
                                depth 0pt}\fi
                   \vrule width \fortresscommentrulethickness
                          height \fortresstempdimen
                          depth 0pt}\relax
               \endgroup}

\newcommand\fortressrulevskip[2]{\cleaders\vbox{\hbox to 0pt{\relax
  \vrule width 0pt height #2\relax
  \vbox to 0pt{\vss
               #1{\fortressgluerule{#2}}{\fortressgluerule{#2}}\relax
               \vskip -1pt}\hss}}\vskip#2}

\newcommand\fortressgluerule[1]{\begingroup
                    \fortresstempdimen=#1\relax
                    \advance\fortresstempdimen by 1pt
                    \vrule width \fortresscommentrulethickness
                           height \fortresstempdimen
                           depth 1pt
                  \endgroup}

\newcommand\fortresspositionleftrule[2]{\hbox to 0pt{\hskip\leftskip
  \hbox to 0pt{\hss#1\hskip\fortresscommentrulemargin}\hss}}

\newcommand\fortresspositionrightrule[2]{\hbox to 0pt{\hskip\hsize \hskip-\rightskip
  \hskip\fortresscommentrulemargin{#2}\hss}}
 
\newcommand\fortresspositionleftrightrule[2]{\hbox to 0pt{\hskip\leftskip
  \hbox to 0pt{\hss#1\hskip\fortresscommentrulemargin}\hskip-\leftskip
  \hskip\hsize \hskip-\rightskip \hskip\fortresscommentrulemargin{#2}\hss}}

\newcommand\FortressTable[1]{\ifhmode\par\endgroup\fi
  \leavevmode\begingroup
  \advance\leftskip by #1\fortressleftindentincrement
  \begin{tabular}}
\newcommand\FortressEndTable{\end{tabular}\par\endgroup}

\newcommand\FortressParagraphImage[1]{%
  \fortresstempdimen=\hsize
  \advance\fortresstempdimen by -\leftskip
  \advance\fortresstempdimen by -\rightskip
  \edef\fortresstemp{[width=\the\fortresstempdimen]}
  \expandafter\includegraphics\fortresstemp{#1}}

\newcommand\FortressInlineImage[1]{\includegraphics[width=1in]{#1}}


% Notes

\newdimen\notewidth
\newcommand{\note}[1]{{\ifhmode\unskip\hfill\break\else\noindent\fi
    \notewidth=\textwidth  \advance\notewidth by -2em
                \framebox[\textwidth][t]{\parbox[t]{\notewidth}{#1}}\break}}

\newcommand{\redundant}[1][other parts of the specification]{\note{This is redundant with {#1}.}}
% Typical usage: \redundant[\secref{label}]

% Sectioning
\newcommand{\newpart}[2]{\part{#1}\partlabel{#2}}
\newcommand{\newchap}[2]{\chapter{#1}\chaplabel{#2}}
\newcommand{\newapp}[2]{\chapter{#1}\applabel{#2}}
\newcommand{\newsec}[2]{\section{#1}\seclabel{#2}}
\newcommand{\newsubsec}[2]{\subsection{#1}\seclabel{#2}}
\newcommand{\newsubsubsec}[2]{\subsubsection{#1}\seclabel{#2}}

% Cross-references
\newcommand{\partref}[1]{Part~\ref{part:#1}}
\newcommand{\partlabel}[1]{\label{part:#1}}
\newcommand{\chapref}[1]{Chapter~\ref{chap:#1}}
\newcommand{\chaprefs}[2]{Chapters~\ref{chap:#1} and~\ref{chap:#2}}
% Victor: To handle general case, could split into two commands, 
%   one to produce Chapters and first ref, the other to produce other refs.
\newcommand{\chaplabel}[1]{\label{chap:#1}}
\newcommand{\appref}[1]{Appendix~\ref{app:#1}}
\newcommand{\apprefs}[2]{Appendices~\ref{app:#1} and~\ref{app:#2}}
\newcommand{\applabel}[1]{\label{app:#1}}
\newcommand{\secref}[1]{Section~\ref{sec:#1}}
\newcommand{\secrefs}[2]{Sections~\ref{sec:#1} and~\ref{sec:#2}}
\newcommand{\seclabel}[1]{\label{sec:#1}}
\newcommand{\figref}[1]{Figure~\ref{fig:#1}}
\newcommand{\figrefs}[2]{Figures~\ref{fig:#1} and~\ref{fig:#2}}
\newcommand{\figlabel}[1]{\label{fig:#1}}
% Victor: we should eliminate the following commands
\newcommand{\see}[1]{see \secref{#1}}
\newcommand{\See}[1]{See \secref{#1}}

% Literal text
\newcommand{\txt}[1]{\mbox{\ttfamily\small{#1}}}
\newcommand{\A}[1]{\underline{\texttt{#1}}}
\newcommand{\txtr}[1]{\mbox{\ttfamily{#1}}}
\newcommand{\csdot}{\thinspace\cdot\thinspace}
\newcommand{\scap}[1]{\textsc{#1}}

% Tabbing in teletype font, so we can embed formulas, italics, roman, etc...
\newenvironment{ttt}{%
\begin{tt}
\begin{tabbing}
}{%
\end{tabbing}
\end{tt}
}

%%%%%%%%%%%%%%%%%%%%%%%%%%%%%%%%%%%%%%%%%%%%%%%%%%%%%%%%%%%%%%
%% brackets, braces, and parentheses %%%%%%%%%%%%%%%%%%%%%%%%%
%%%%%%%%%%%%%%%%%%%%%%%%%%%%%%%%%%%%%%%%%%%%%%%%%%%%%%%%%%%%%%
\newcommand{\set}[1]{\ensuremath{\{#1\}}}
\newcommand{\braces}[1]{\ensuremath{\code\{#1\code\}}}
\newcommand{\parens}[1]{\ensuremath{\code(#1\code)}}
\newcommand{\brackets}[1]{\ensuremath{\code[#1\code]}}
%% brackets for type parameters
\newcommand{\bTPl}{\ensuremath{\llbracket}}
\newcommand{\bTPr}{\ensuremath{\rrbracket}}
\newcommand{\bsTP}[1]{\bTPl#1\bTPr}
\newcommand{\bINTl}{\ensuremath{\langle\!|}}
\newcommand{\bINTr}{\ensuremath{|\!\rangle}}
\newcommand{\bsINT}[1]{\bINTl#1\bINTr}
%% brackets for lists
\newcommand{\bLISTl}{\ensuremath{\langle}}
\newcommand{\bLISTr}{\ensuremath{\rangle}}
\newcommand{\bsLIST}[1]{\bLISTl#1\bLISTr}
%% brackets for where clauses
\newcommand{\bWHl}{\ensuremath{\{}}
\newcommand{\bWHr}{\ensuremath{\}}}
\newcommand{\bsWH}[1]{\bWHl#1\bWHr}

%% Numeric type names
\newcommand{\ZZ}{\ensuremath{\mathbb{Z}}}
\newcommand{\RR}{\ensuremath{\mathbb{R}}}

\newcommand{\AmBYn}[3]{$\txtr{#1}^{\txtr{#2}\BY\txtr{#3}}$}
\newcommand{\aSUBmBYn}[3]{$\txtr{#1}_{\txtr{#2}\BY\txtr{#3}}$}

%%%%%%%%%%%%%%%%%%%%%%%%%%%%%%%%%%%%%%%%%%%%%%%%%%%%%%%%%%%%%%
\newcommand{\unicode}{Unicode 5.0}
% Victor: Do we want to specify the version?

%% shell command rendering used in the Overview chapter
\newcommand{\shellcommand}[1]{\txt{#1}}
\newcommand{\library}{the Fortress standard libraries}
\newcommand{\Library}{The Fortress standard libraries}

%%%%%%%%%%%%%%%%%%%%%%%%%%%%%%%%%%%%%%%%%%%%%%%%%%%%%%%%%%%%%%%%
%% Stuff for the grammar excerpt %%%%%%%%%%%%%%%%%%%%%%%%%%%%%%%
%%%%%%%%%%%%%%%%%%%%%%%%%%%%%%%%%%%%%%%%%%%%%%%%%%%%%%%%%%%%%%%%

\newcommand{\option}[1]{#1?}
\newcommand{\options}[1]{(#1)?}


\def\Grammar{\list{}{\leftmargin 2em
                      \itemindent\listparindent
                      \parsep 0pt plus 1pt}
\item[{\sf Syntax:}]
\item
                      \begingroup\tabbing\begin{tabular}{llll}}
\makeatletter
\def\endGrammar{\end{tabular}\unskip\@stopfield\@addfield\ifdim\wd\@curline>0pt\@startfield\@stopline\fi
  \ifnum\@tabpush >\z@ \@badpoptabs \fi\endtrivlist\endgroup\endlist}
\makeatother

\def\GrammarTwo{\list{}{\leftmargin 2em
                      \itemindent\listparindent
                      \parsep 0pt plus 1pt}
\item
                      \begingroup\tabbing\begin{tabular}{llll}}
\makeatletter
\def\endGrammarTwo{\end{tabular}\unskip\@stopfield\@addfield\ifdim\wd\@curline>0pt\@startfield\@stopline\fi
  \ifnum\@tabpush >\z@ \@badpoptabs \fi\endtrivlist\endgroup\endlist}
\makeatother

%%%%%%%%%%%%%%%%%%%%%%%%%%%%%%%%%%%%%%%%%%%%
%% Sectioning commands for libraries %%%%%%%
%%%%%%%%%%%%%%%%%%%%%%%%%%%%%%%%%%%%%%%%%%%%

\makeatletter
\newcommand\librarysection{\@ifstar{\vskip-\lastskip  \vskip-\parskip
   \@startsection{section}{1}{\z@}{\z@}%
                 {2.3ex \@plus .2ex}{\normalfont\Large\bfseries}}{%
   \vskip 0pt plus .00001fil\nobreak\null
   \@startsection{section}{1}{\z@}{-3.5ex\@plus -1ex \@minus -.2ex}%
                 {2.3ex \@plus .2ex}{\normalfont\Large\bfseries}}}
\newcommand\librarysubsection{\@ifstar{\vskip-\lastskip  \vskip-\parskip
   \@startsection{subsection}{2}{\z@}{\z@}%
                 {0.5ex \@plus .2ex}{\normalfont\bfseries}}{%
   \vskip 0pt plus .00001fil\nobreak\null
   \@startsection{subsection}{2}{\z@}{-3.25ex\@plus -1ex \@minus -.2ex}%
                 {0.5ex \@plus .2ex}{\normalfont\bfseries}}}
\makeatother
\newcommand\traitdecl[1]{\librarysection{The Trait #1}}
\newcommand\objectdecl[1]{\librarysection{The Object #1}}
\newcommand\methoddecl{\librarysubsection}
\newcommand\functiondecl{\librarysubsection}
\newcommand\variabledecl{\librarysubsection}

%%%%%%%%%%%%%%%%%%%%%
%% API vs api %%%%%%%
%%%%%%%%%%%%%%%%%%%%%
\newcommand{\apiN}{\mbox{API}}
\newcommand{\apisN}{\mbox{APIs}}
\newcommand{\Api}{API}
\newcommand{\Apis}{APIs}

%%%%%%%%%%%%%%%%%%%%%%%%%%%%%%%%%%%%%%%%%%%%%%%%%%%%%%%%%%%%%%%%
%% mathematical characters not already covered by LaTeX %%%%%%%
%%%%%%%%%%%%%%%%%%%%%%%%%%%%%%%%%%%%%%%%%%%%%%%%%%%%%%%%%%%%%%%%
\newcommand{\OverStrike}[2]{{\setbox1\hbox{${#1}$}\setbox2\hbox{${#2}$}\relax
   \setbox3\vbox{\box1\box2}\hbox{\hbox to \wd3{\hfil${#1}$\hfil}\hskip-\wd3
                                  \hbox to \wd3{\hfil${#2}$\hfil}}}}
\newcommand{\UPTRI}{\vbox to 0pt{\vss\hbox{\raise 0.1em\hbox{\large$\bigtriangleup$}}}}
\newcommand{\DOWNTRI}{\vbox to 0pt{\vss\hbox{\lower 0.2em\hbox{\large$\bigtriangledown$}}}}

\newcommand{\upplus}{\mathbin{\OverStrike{\UPTRI}{\scriptstyle +}}}
\newcommand{\upminus}{\mathbin{\OverStrike{\UPTRI}{\scriptstyle -}}}
\newcommand{\uptimes}{\mathbin{\OverStrike{\UPTRI}{\scriptstyle\times}}}
\newcommand{\updot}{\mathbin{\OverStrike{\UPTRI}{\cdot}}}
\newcommand{\upslash}{\mathbin{\OverStrike{\UPTRI}{\scriptstyle\diagup}}}
\newcommand{\downplus}{\mathbin{\OverStrike{\DOWNTRI}{\raise0.1ex\hbox{$\scriptstyle +$}}}}
\newcommand{\downminus}{\mathbin{\OverStrike{\DOWNTRI}{\raise0.1ex\hbox{$\scriptstyle -$}}}}
\newcommand{\downtimes}{\mathbin{\OverStrike{\DOWNTRI}{\raise0.1ex\hbox{$\scriptstyle\times$}}}}
\newcommand{\downdot}{\mathbin{\OverStrike{\DOWNTRI}{\raise0.1ex\hbox{$\cdot$}}}}
\newcommand{\downslash}{\mathbin{\OverStrike{\DOWNTRI}{\raise0.1ex\hbox{$\scriptstyle\diagup$}}}}
\newcommand{\chopplus}{\mathbin{\OverStrike{\sqcup}{\raise0.1ex\hbox{$\scriptstyle +$}}}}
\newcommand{\chopminus}{\mathbin{\OverStrike{\sqcup}{\raise0.1ex\hbox{$\scriptstyle -$}}}}
\newcommand{\choptimes}{\mathbin{\OverStrike{\sqcup}{\raise0.1ex\hbox{$\scriptstyle\times$}}}}
\newcommand{\chopdot}{\mathbin{\OverStrike{\sqcup}{\raise0.1ex\hbox{$\cdot$}}}}
\newcommand{\chopslash}{\mathbin{\OverStrike{\sqcup}{\raise0.1ex\hbox{$\scriptstyle\diagup$}}}}
\newcommand{\exactplus}{\mathbin{\OverStrike{\diamond}{\scriptstyle +}}}
\newcommand{\exactminus}{\mathbin{\OverStrike{\diamond}{\scriptstyle -}}}
\newcommand{\exacttimes}{\mathbin{\OverStrike{\diamond}{\scriptstyle\times}}}
\newcommand{\exactdot}{\mathbin{\OverStrike{\diamond}{\cdot}}}
\newcommand{\exactslash}{\mathbin{\OverStrike{\diamond}{\scriptstyle\diagup}}}
\newcommand{\totallss}{\mathbin{\OverStrike{\bigcirc}{\scriptstyle>}}}
\newcommand{\totalleq}{\mathbin{\OverStrike{\bigcirc}{\scriptstyle\leq}}}
\newcommand{\totalgeq}{\mathbin{\OverStrike{\bigcirc}{\scriptstyle\geq}}}
\newcommand{\totalgtr}{\mathbin{\OverStrike{\bigcirc}{\scriptstyle<}}}
\newcommand{\xor}{\mathbin{\underline{\vee}}}
\newcommand{\nand}{\mathbin{\overline{\wedge}}}
\newcommand{\nor}{\mathbin{\overline{\vee}}}
\newcommand{\twointersectnot}{\mathbin{\OverStrike{\raise0.3ex\hbox{$\neg$}}{\lower0.2ex\hbox{$\neg$}}}}
\newcommand{\lhfloor}{\mathopen{\lfloor\mskip-1.8\thinmuskip\lfloor}}
\newcommand{\rhfloor}{\mathclose{\rfloor\mskip-1.8\thinmuskip\rfloor}}
\newcommand{\lhceil}{\mathopen{\lceil\mskip-1.8\thinmuskip\lceil}}
\newcommand{\rhceil}{\mathclose{\rceil\mskip-1.8\thinmuskip\rceil}}
\newcommand{\lhhfloor}{\mathopen{\lfloor\mskip-1.8\thinmuskip\lfloor\mskip-1.8\thinmuskip\lfloor}}
\newcommand{\rhhfloor}{\mathclose{\rfloor\mskip-1.8\thinmuskip\rfloor\mskip-1.8\thinmuskip\rfloor}}
\newcommand{\lhhceil}{\mathopen{\lceil\mskip-1.8\thinmuskip\lceil\mskip-1.8\thinmuskip\lceil}}
\newcommand{\rhhceil}{\mathclose{\rceil\mskip-1.8\thinmuskip\rceil\mskip-1.8\thinmuskip\rceil}}
\newcommand{\dotminus}{\mathbin{\dot{-}}}
\newcommand{\dottimes}{\mathbin{\dot{\times}}}
%\newcommand{\sequiv}{\mathrel{\OverStrike{\raise0.45ex\hbox{$=$}}{\lower0.4ex\hbox{$=$}}}}

%%%%%%%%%%%%%%%%%%%%%%%%%%%%%%%%%%%%%%%%%%
%% object trait vs object trait type %%%%%
%%%%%%%%%%%%%%%%%%%%%%%%%%%%%%%%%%%%%%%%%%
\newcommand{\objecttype}{object trait type}
\newcommand{\objecttypes}{object trait types}

%%%%%%%%%%%%%%%%%%%%%%%%%%%%%
%% identifier parameter %%%%%
%%%%%%%%%%%%%%%%%%%%%%%%%%%%%
\newcommand{\nameparam}{\KWD{ident}}

%Token
\newcommand{\T}[1]{\textsc{#1}}
%Nonterminal
\newcommand{\N}[1]{\textit{#1}}

%subscripted nonterminals
\newcommand{\NS}[1]{$\textit{#1}_\sigma$}
\newcommand{\NWO}[1]{$\textit{#1}_{\not\hookleftarrow}$}
\newcommand{\NWN}[1]{$\textit{#1}_{\emptyset}$}

% Whitespace glyphs
\newcommand{\WW}{$\Diamond$}%Any white, including empty
\newcommand{\WP}{$\Box$}%Non-empty white
\newcommand{\WO}{$\not\hookleftarrow$}%Any white except newline
\newcommand{\WN}{$\emptyset$}%White containing at least one newline
\newcommand{\NL}{$\hookleftarrow$}%A new line

%Troublesome characters
\newcommand{\OR}{$|$}
\newcommand{\GT}{>}
\newcommand{\LT}{<}
\newcommand{\HAT}{\boldmath$\hat{\ }$}
\def\BY{\ensuremath{\times}}

% Types

\newcommand{\Any}{\TYP{Any}}
\newcommand{\Bottom}{\TYP{Bottom}}
\newcommand{\Void}{\TYP{()}}
\newcommand{\Object}{\TYP{Object}}

\newcommand{\arrow}[2]{\ensuremath{{#1} \rightarrow {#2}}}
\newcommand{\ioarrow}[2]{\ensuremath{\texttt{io} {#1} \rightarrow {#2}}}
\newcommand{\arrowthrows}[3]{\ensuremath{{#1} \rightarrow {#2} \texttt{throws} {#3}}}
\newcommand{\ioarrowthrows}[3]{\ensuremath{\texttt{io} {#1} \rightarrow {#2} \texttt{throws} {#3}}}

\newcommand{\uniontype}[2]{\ensuremath{{#1} \cup {#2}}}
\newcommand{\intertype}[2]{\ensuremath{{#1} \cap {#2}}}

\newcommand{\univtype}[2]{\ensuremath{\forall [{#1}] {#2}}}
\newcommand{\existype}[2]{\ensuremath{\exists [{#1}] {#2}}}

