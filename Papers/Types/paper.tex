\documentclass[10pt,preprint]{sigplanconf}
\usepackage{amsmath,graphicx,url,color,alltt,fortify,verbatim,bcprules,tabularx,theorem}
\advance \textheight by 4pt

% make a big red TODO label
\newcommand{\TODO}[1]{\text{\textbf{\emph{\textcolor{red}{TODO}}}: \textsf{\footnotesize #1}}}

%newcommand
\newcommand{\ms}{\preceq}
\renewcommand{\bar}{\overline}
\newcommand{\meet}{\wedge}
\newcommand{\C}{\mathcal{C}}
\newcommand{\quoted}[1]{\begin{quote}#1\end{quote}}
\newcommand{\exc}{\mathrel{\lozenge}}
\newcommand{\nexc}{\mathrel{\hbox to 0pt{$\mskip -1.4mu\not$\hss}\lozenge}}
\newcommand{\smalllozenge}{\vcenter{\hbox{\scalebox{.5}{$\lozenge$}}}}
\newcommand{\normallozenge}{\vcenter{\hbox{$\lozenge$}}}
% \newcommand{\altlozenge}{\ooalign{\hfil$\normallozenge$\hfil\cr\hfil$\smalllozenge$\hfil}}
\newcommand{\altlozenge}{\ooalign{\hfil$\vcenter{\hbox{$\lozenge$}}$\hfil\cr\hfil$\cdot$\hfil}}
\newcommand{\bexc}{\mathrel{\altlozenge}}
\newcommand{\bexcp}{\mathrel{\altlozenge}_\textrm{p}}
\newcommand{\bnexc}{\mathrel{\hbox to 0pt{$\mskip -1.4mu\not$\hss}\altlozenge}}

\newcommand{\fresh}[1]{\textit{fresh}({#1})}

\newcommand{\excr}{\triangleright}
\newcommand{\excl}{\triangleleft}
\newcommand{\excre}{\excr_\textrm{e}}
\newcommand{\excle}{\excl_\textrm{e}}
\newcommand{\excrc}{\excr_\textrm{c}}
\newcommand{\exclc}{\excl_\textrm{c}}
\newcommand{\excro}{\excr_\textrm{o}}
\newcommand{\exclo}{\excl_\textrm{o}}
\newcommand{\excrp}{\excr_\textrm{p}}
\newcommand{\exclp}{\excl_\textrm{p}}
\newcommand{\excrx}{\excr_x}
\newcommand{\exclx}{\excl_x}

\newcommand{\exce}{\exc_\textrm{e}}
\newcommand{\excc}{\exc_\textrm{c}}
\newcommand{\exco}{\exc_\textrm{o}}
\newcommand{\excp}{\exc_\textrm{p}}

\newcommand{\ancestors}{\textit{ancestors}}
\newcommand{\ancexcludes}{\textit{excludes}^*}
\newcommand{\myexcludes}[1]{{#1}.\textit{excludes}}
\newcommand{\mycomprises}[1]{{#1}.\textit{comprises}}
\newcommand{\myextends}[1]{{#1}.\textit{extends}}

\newcommand{\extends}{\ensuremath{<:}}
\newcommand{\subtypeof}{\ensuremath{<:}}

\newcommand{\arrowtype}[2]{\mbox{\ensuremath{{#1} \rightarrow {#2}}}}
\newcommand{\tuple}[1]{\ensuremath{#1}}

% indented code block
\newenvironment{ttquote}%
{\begin{quote}\begin{alltt}}
{\end{alltt}\end{quote}}

% put in oxford brackets
\newcommand{\ob}[1]{\ensuremath{\llbracket {#1} \rrbracket}}
% put in oxford brackets and an overbar
\newcommand{\obb}[1]{\ensuremath{\llbracket \bar{#1} \rrbracket}}
% make a type param bound with the given name
\newcommand{\bd}[1]{\ensuremath{\{\bar{#1}\}}}
% syntactic definition
\newcommand{\syndef}{\ensuremath{\overset{\mathrm{def}}{=}}}
% make a substitution
\newcommand{\subst}[2]{\ensuremath{[#1/#2]}}
% make a substitution with bars
\newcommand{\substb}[2]{\ensuremath{[\bar{#1}/\bar{#2}]}}
% list of bounds/type environment
\newcommand{\bds}[2]{\ensuremath{\bar{{#1} \extends \bd{#2}}}}
% type parameter list with bounds and oxford brackets
\newcommand{\tplist}[2]{\ensuremath{\ob{\bds{#1}{#2}}}}
% monomorphic fn decl
\newcommand{\decl}[3]{\mbox{\ensuremath{{#1}\,{#2}\!:\!{#3}}}}
% a generic function declaration 
\newcommand{\declg}[5]{\mbox{\ensuremath{#1 \tplist{#2}{#3}\, #4\!:\!#5}}}
\newcommand{\hdeclg}[4]{\mbox{\ensuremath{#1 \ob{#2}\, #3\!:\!#4}}}
% a class table T
\newcommand{\T}{\ensuremath{\mathcal{T}}}
% class table extension
\newcommand{\ctext}{\ensuremath{\supseteq}}
% a declaration set D
\newcommand{\D}{\ensuremath{\mathcal{D}}}
% existential type
\newcommand{\exttype}[2][\Delta]{\ensuremath{\exists\ob{#1}{#2}}}
% universal type
\newcommand{\unitype}[2][\Delta]{\ensuremath{\forall\ob{#1}{#2}}}

%%%%% Any and Bottom %%%%

\newcommand{\Any}{\TYP{Any}}
\newcommand{\BottomType}{\TYP{BottomType}}

\newcommand{\FALSE}{\textrm{false}}
\newcommand{\TRUE}{\textrm{true}}

\newcommand{\NONE}{\bullet}

\newcommand{\eqred}{\overset{\equiv}{\longrightarrow}}

%%%%%%% JUDGMENTS %%%%%%%%

%%% NEW SYNTACTIC JUDGMENT
\newcommand{\newjudge}[2]{\fbox{\textbf{#1:} \quad \ensuremath{#2}}}

% generic
\newcommand{\jgtemplate}[4][\Delta]{\ensuremath{{#1}\,\vdash\,{#2}\;{#3}\;{#4}}}
% constrained judgments
\newcommand{\jgconstrtemplate}[5][\Delta]{\ensuremath{{#1}\,\vdash\,{#2}\;{#3}\;{#4}}\;|\,{#5}}
% ground subtyping
\newcommand{\jgsub}[3][\Delta]{\jgtemplate[#1]{#2}{\subtypeof}{#3}}
% ground subtyping with constraints
\newcommand{\jcsub}[4][\Delta]{\jgconstrtemplate[#1]{#2}{\subtypeof}{#3}{#4}}
% subtyping on quantified types
\newcommand{\jqsub}[3][\Delta]{\jgtemplate[#1]{#2}{\le}{#3}}
% type exclusion
\newcommand{\jexc}[4][\Delta]{\jgconstrtemplate[#1]{#2}{\exc}{#3}{#4}}
% type non-exclusion
\newcommand{\jnexc}[4][\Delta]{\jgconstrtemplate[#1]{#2}{\nexc}{#3}{#4}}
% applicability of a domain or fndecl to a type
\newcommand{\japp}[3][\Delta]{\jgtemplate[#1]{#2}{\ni}{#3}}
% specificity between fndecls
\newcommand{\jms}[3][\Delta]{\jgtemplate[#1]{#2}{\ms}{#3}}
% nonequivalence
\newcommand{\jcnonequiv}[4][\Delta]{\jgconstrtemplate[#1]{#2}{\not\equiv}{#3}{#4}}
% equivalence
\newcommand{\jcequiv}[4][\Delta]{\jgconstrtemplate[#1]{#2}{\equiv}{#3}{#4}}
% constraint solving
\newcommand{\jcsolve}[3][\Delta]{\ensuremath{{#1}\,\vdash\,\textit{solve}({#2})\,=\,{#3}}}
% type reduction
\newcommand{\jtred}[2]{\ensuremath{\vdash\,{#1} \eqred {#2}}}
\newcommand{\jtreds}[3]{\ensuremath{\vdash\,{#1} \eqred {#2}\,,\;{#3}}}


% for tabularx environments to have a right-aligned, stretched col
\newcolumntype{R}{>{\raggedleft\arraybackslash}X}%

\theorembodyfont{\rm}
\newtheorem{lemma}{Lemma}
\newtheorem{theorem}{Theorem}
% Our proofs are more like proof sketches!! EricAllen 7/15/2011
\newenvironment{proof}{\noindent \textbf{Proof Sketch:} }{\hfill $\Box$}

\begin{document}

\conferenceinfo{OOPSLA '11}{October 22--27, 2011, Portland, Oregon, USA.}
\CopyrightYear{2011}
\copyrightdata{978-1-4503-0940-0/11/10}

\titlebanner{draft}        % These are ignored unless
\preprintfooter{draft}     % 'preprint' option specified.

\title{Type-checking Modular Multiple Dispatch with Parametric Polymorphism and Multiple Inheritance}
\subtitle{}
\authorinfo{Eric Allen}{Oracle Labs}{eric.allen@oracle.com}
\authorinfo{Justin Hilburn}{Oracle Labs}{justin.hilburn@oracle.com}
\authorinfo{Scott Kilpatrick}{University of Texas \\ at Austin}{scottk@cs.utexas.edu}
\authorinfo{Victor Luchangco}{Oracle Labs}{victor.luchangco@oracle.com}
\authorinfo{Sukyoung Ryu}{KAIST}{sryu@cs.kaist.ac.kr}
\authorinfo{David Chase}{Oracle Labs}{david.r.chase@oracle.com}
\authorinfo{Guy L. Steele Jr.}{Oracle Labs}{guy.steele@oracle.com}

\makeatletter
\def \@maketitle {%
 \begin{center}
 \@settitlebanner
 \let \thanks = \titlenote
 \noindent \LARGE \bfseries \@titletext \par
 %\vskip 6pt
 %\noindent \Large \@subtitletext \par
 \vskip 6pt
   \noindent \@setauthor{9pc}{i}{\@false}\hspace{1.5pc}%
             \@setauthor{9pc}{ii}{\@false}\hspace{1.5pc}%
             \@setauthor{9pc}{iii}{\@false}\hspace{1.5pc}%
             \@setauthor{9pc}{iv}{\@true}\par
\vspace{12pt plus 2pt}
 \noindent \@setauthor{9pc}{v}{\@false}\hspace{1.5pc}%
           \@setauthor{9pc}{vi}{\@false}\hspace{1.5pc}%
           \@setauthor{9pc}{vii}{\@false}\par
\vspace{10pt plus 2pt}
 \end{center}}
\makeatother
\maketitle


%% llncs
%% \title{Type-checking Modular Multiple Dispatch with Parametric Polymorphism and
%% \mbox{Multiple Inheritance}}
%% \titlerunning{Modular Multiple Dispatch with Polymorphism and Multiple Inheritance}

%% \author{Eric Allen\inst{1} \and
%% Justin Hilburn\inst{2} \and
%% Scott Kilpatrick\inst{3} \and
%% Sukyoung Ryu\inst{4} \and\\
%% David Chase\inst{1} \and
%% Victor Luchangco\inst{1} \and
%% Guy L. Steele Jr.\inst{1}
%% }
%% \authorrunning{Allen, Hilburn, Kilpatrick, Ryu, Chase, Luchangco, and Steele Jr.}
%% %
%% %%%% list of authors for the TOC (use if author list has to be modified)
%% %% \tocauthor{Ivar Ekeland, Roger Temam, Jeffrey Dean, David Grove,
%% %% Craig Chambers, Kim B. Bruce, and Elisa Bertino}
%% %
%% \institute{Sun Labs / Oracle, USA
%% %\email{\{firstname.lastname, david.r.chase\}@oracle.com}
%% \and
%% University of Oregon, USA
%% %\email{jrhil47@gmail.com}
%% \and
%% MPI-SWS, Germany
%% %\email{skilpat@mpi-sws.org}
%% \and
%% KAIST, Korea
%% %\email{sryu@cs.kaist.ac.kr}
%% }

%% \maketitle

\begin{abstract}

\begin{abstract}

The Fortress programming language integrates traditional mathematical
notation into an object-oriented framework based on traits with
multiple inheritance, overloading (of both methods and functions)
resolved by symmetric dynamic dispatch, static types, and separately
compiled modules.  One innovation is
\emph{functional methods}, which (like conventional ``dotted methods'')
are declared within traits and may be inherited, but are invoked by
ordinary function calls (or mathematical operator syntax) rather
than conventional ``dotted method calls,'' and therefore compete
in overloading resolution with ordinary function declarations.
A component/API system governs visibility of traits, objects, and
functions, and allows separate compilation of components.

A longstanding problem with multiple inheritance is what to do when
methods inherited from several parents conflict.  Many approaches have
been explored in the literature; most fail to obey the
intuitively desirable requirement that the function or method invoked
be the uniquely most specific one that is both accessible and
applicable.  Fortress requires that the signatures in every overload
set form a meet-bounded lattice; therefore it is impossible for any
function or method call to be ambiguous.  This idea goes back nearly
two decades, but Fortress appears to be the first programming language
to adopt and statically enforce it.  Because this rule guarantees
confluence, it enables a distributed implementation of dispatching
that allows selective export and selective optimization.

We exhibit a source-to-source rewrite from a source language
(a stripped-down version of Fortress) to a related target language
that is simpler than the Java\texttrademark\ programming language and is readily
supported by the Java Virtual Machine.  The demonstrated rewriting is
a practical basis for separate compilation and is easily extended to
explicitly type-parameterized methods and functions.

\end{abstract}





\end{abstract}

\category{D.3.3}{Programming Languages}{Language Constructs and Features---classes and objects, inheritance, modules, packages, polymorphism}

\terms{Languages}

\keywords{object-oriented programming, multiple dispatch, 
symmetric dispatch, multiple inheritance, overloading, modularity, methods, 
multimethods, static types, run-time types, ilks, 
components, separate compilation, Fortress, meet rule}

\section{Introduction}\label{sec:intro}
A key feature of object-oriented languages is \emph{dynamic dispatch}: 
there may be multiple definitions of a function (or method) with the same name---%
we say the function is \emph{overloaded}---%
and a call to a function of that name is resolved
based on the ``run-time types''---we use the term \emph{ilks}---of the arguments. 
With \emph{single dispatch}, 
a particular argument is designated as the \emph{receiver}, 
and the call is resolved only with respect to that argument.
With \emph{multiple dispatch}, 
the run-time types of  all arguments to a call are used to resolve the call.
\emph{Symmetric multiple dispatch} is a special case of multiple dispatch 
in which all arguments are considered equally when resolving a call.

Multiple dispatch provides great expressivity.
In particular, 
mathematical operators such as $+$ and $\leq$ and $\cup$
and especially $\cdot$ and $\times$
have different definitions depending on the types of the arguments
to an application of the operator
(even the number of arguments may vary between calls); 
in a language with multiple dispatch, 
it is natural to define these operators as overloaded functions. 
Similarly, 
many binary operations on collections such as \VAR{append} and \VAR{zip} 
have different definitions 
depending on the types of both arguments. 
\TODO{Add (reference to) argument for symmetric multiple dispatch?}

% \TODO{Do we want something like the following paragraph here?}
% In this paper,
% we present rules for defining overloaded functions 
% to ensure type soundness and non-ambiguity of function calls
% under symmetric multiple dispatch
% in an object-oriented language 
% that supports parametric polymorphism and multiple inheritance.
% \TODO{If we do, trim text on next page.}

% \TODO{Alternative is to have a shorter intro, 
% which mostly mimics the abstract, but with a bit more elaboration,
% and I would probably leave the discussion of our prior work till later.
% We may want to mention Fortress early as a context for this work.
% Then have a long ``background'' section 
% containing the discussion starting from Castagna to Bourdoncle and Merz,
% and including the discussion of our prior work and former thoughts.}

% To preserve type soundness and avoid ambiguous function calls 
% while incorporating multiple dispatch 
% into an object-oriented language with a static semantics, 
% the sets of valid overloaded definitions must be restricted.
% For example, to avoid ambiguous function calls,
% we must ensure that for every call site 
% (knowing only the static types of the arguments),
% there exists a unique ``best'' function to dispatch to at run time.\footnote{
% In languages with static overloading, 
% such as Scala, C\#, and the Java\texttrademark\ programming language 
% \cite{scala,CSharpSpec,JavaSpec}, 
% it is possible to simply reject ambiguous call sites of overloaded functions.
% However, as Millstein and Chambers have observed, 
% it is impossible to statically forbid ambiguity 
% in the presence of multiple dynamic dispatch 
% without imposing constraints at the definition sites of overloaded functions
% \cite{millstein02,millstein03}.
% \TODO{Is this true for asymmetric multiple dispatch?}}

In an object-oriented language with symmetric multiple dispatch,
some restrictions must be placed on overloaded function definitions
to guarantee type soundness 
and avoid ambiguous function calls.
% \cite{castagna95,millstein02,millstein03}.
For example, 
consider the following overloaded function definitions:
\small
\begin{FortressCode}
{\tt ~~}\+f(b\COLONOP{}B, a\COLONOP{}A)\COLON \mathbb{Z} = 1 \\
  f(a\COLONOP{}A, b\COLONOP{}B)\COLON \mathbb{Z} = 2\-
\end{FortressCode}
\normalsize
If $A$ is a subtype of $B$ (we write this as \EXP{A \SHORTCUT{<} B}),
to which of these definitions do we dispatch 
when $f$ is called with two arguments of type $A$? 
% Note that the ambiguity is inherent in these definitions:
% there is a real question as to what behavior the programmer intended
% in this case.  

Castagna \textit{et al.} \cite{castagna95} address this problem 
in the context of a type system 
without parametric polymorphism or multiple inheritance
by requiring every pair of overloaded function definitions 
to satisfy the following properties:
(\emph{i}) whenever the parameter type of one 
is a subtype of the parameter type of the other, 
the return type of the first
must also be a subtype of the return type of the second; 
and 
(\emph{ii}) whenever the parameter types of the two definitions 
have a common lower bound (i.e., nontrivial subtype), 
there is a unique definition for the same function 
whose parameter type is the greatest lower bound 
of the parameter types of the two definitions.
Thus, for the example above, 
% the solution of Castagna {\it et~al}.\ is to require the programmer to
the programmer must provide a third definition
to satisfy the latter property:
 \small
\begin{FortressCode}
{\tt ~~}\+f(a\COLONOP{}A,a'\COLONOP{}A)\COLON \mathbb{Z} = \ldots\-
\end{FortressCode}
 \normalsize

This latter property is equivalent to requiring
that the definitions for each overloaded function form a meet semilattice 
partially ordered by the subtype relation on their parameter types.
We call this partial order the \emph{more specific than} relation,\!\footnote{%
Despite its name,
this relation, like the subtype relation, is reflexive: 
two function definitions with the same parameter type 
are each more specific than the other.
In that case, we say the definitions are equally specific.}
and we call the property the \emph{meet rule}.
We call the first property above the \emph{return type rule} or \emph{subtype rule}.

In this paper, 
we give variants of these rules for ensuring safe overloaded functions 
in a language that supports symmetric multiple dispatch, 
multiple inheritance, and parametric polymorphism 
(that is, generic types \emph{and} generic functions).
We prove that these rules guarantee type soundness
and that there are no ambiguous calls at run time 
(see Section~\ref{sec:safety}).
We do this by extending our earlier rules 
for a core of the Fortress programming language 
that did not support generics \cite{allen07,Fortress},
for which we proved the analogous theorems.



The type system considered by Castagna \emph{et al.} 
assumed knowledge of the entire type hierarchy 
(to determine whether two types have a common subtype), 
and the type hierarchy was assumed to be a meet semilattice 
(to ensure that any two types have a greatest lower bound).
In previous work~\cite{allen07},
we applied variants of the meet and return type rules
to a simplified version of the Fortress programming language~\cite{Fortress}, 
which supports multiple inheritance 
and does not require that types have expressible meets 
(i.e., the types that can be expressed in the language 
need not form a meet semilattice).
We showed that we could check these rules in a modular way, 
so that the type hierarchy could be extended safely by new modules
without rechecking old modules.

Because the type hierarchy defined by a module may be extended,
and because Fortress supports multiple inheritance,
two types may have a common nontrivial subtype 
even if no declared type extends them both.
Thus,
for any pair of overloaded function definitions with incomparable parameter types
(i.e., neither definition is more specific than the other),
the meet rule requires some other definition to resolve the potential ambiguity.
Because explicit intersection types cannot be expressed in Fortress, 
it is not always possible to provide such a function definition.
However, 
Fortress defines an \emph{exclusion relation} on types, 
such that types related by exclusion must have no common nontrivial subtypes,
and thus definitions with such types as parameter types 
need not be disambiguated.

In this paper, we extend our prior rules 
to handle parametric polymorphism, 
where both types and functions may be parameterized by type variables, 
which Fortress also supports.
To do so, 
it is helpful to have an interpretation for parametric types 
and type-parametric functions.

One way to think about a parametric type such as \EXP{\TYP{List}\llbracket{}T\rrbracket}
(a list with elements of type \VAR{T}---type parameter lists 
in Fortress are delimited by white square brackets) 
is that it represents an infinite set of ground types 
\EXP{\TYP{List}\llbracket\TYP{Object}\rrbracket} (lists of objects),
\EXP{\TYP{List}\llbracket\TYP{String}\rrbracket} (lists of strings), 
\EXP{\TYP{List}\llbracket\mathbb{Z}\rrbracket} (lists of integers), 
and so on.
An actual type checker must have rules 
for working with uninstantiated (non-ground) parametric types, 
but for many purposes this model of ``an infinite set of ground types'' 
is adequate for explanatory purposes.
Not so, however, for type-parametric functions.  

For some time during the development of Fortress, 
one of us (Steele) pushed for an interpretation of type-parametric functions
analogous to the one above for parametric types;
that is, 
that the type-parametric function definition\footnote{%
The first pair of white square brackets delimits the declaration of a type parameter \VAR{T},
but the other pairs of white brackets indicate 
that this type variable \VAR{T} is the static argument to the parametric type \TYP{List}.}
\small
\begin{FortressCode}
{\tt ~~}\+\VAR{append}\llbracket{}T\rrbracket\bigl(x\COLON \TYP{List}\llbracket{}T\rrbracket, y\COLON \TYP{List}\llbracket{}T\rrbracket\bigr)\COLON \TYP{List}\llbracket{}T\rrbracket = e\-
\end{FortressCode}
\normalsize
should be understood as if it stood for an infinite set of monomorphic definitions:
\small
\begin{FortressCode}
{\tt ~~}\+\VAR{append}\bigl(x\COLON \TYP{List}\llbracket\TYP{Object}\rrbracket, y\COLON \TYP{List}\llbracket\TYP{Object}\rrbracket\bigr)\COLON \TYP{List}\llbracket\TYP{Object}\rrbracket = e \\
  \VAR{append}\bigl(x\COLON \TYP{List}\llbracket\TYP{String}\rrbracket, y\COLON \TYP{List}\llbracket\TYP{String}\rrbracket\bigr)\COLON \TYP{List}\llbracket\TYP{String}\rrbracket = e \\
  \VAR{append}\bigl(x\COLON \TYP{List}\llbracket\mathbb{Z}\rrbracket, y\COLON \TYP{List}\llbracket\mathbb{Z}\rrbracket\bigr)\COLON \TYP{List}\llbracket\mathbb{Z}\rrbracket = e \\
  \ldots\-
\end{FortressCode}
\normalsize
The intuition was that for any specific function call,
the usual rule for dispatch would then choose 
the appropriate most specific definition 
for this (infinitely) overloaded function.

Although that intuition worked well enough 
for a single polymorphic function definition,
it failed utterly when we considered multiple function definitions.
For example, 
a programmer might want to provide definitions 
for specific monomorphic special cases, as in:
\small
\begin{FortressCode}
{\tt ~~}\+\VAR{append}\llbracket{}T\rrbracket\bigl(x\COLON \TYP{List}\llbracket{}T\rrbracket, y\COLON \TYP{List}\llbracket{}T\rrbracket\bigr)\COLON \TYP{List}\llbracket{}T\rrbracket = e_1 \\
  \VAR{append}\bigl(x\COLON \TYP{List}\llbracket\mathbb{Z}\rrbracket, y\COLON \TYP{List}\llbracket\mathbb{Z}\rrbracket\bigr)\COLON \TYP{List}\llbracket\mathbb{Z}\rrbracket = e_2\-
\end{FortressCode}
\normalsize
But if the interpretation above is taken seriously, 
this would be equivalent to:
\small
\begin{FortressCode}
{\tt ~~}\+\VAR{append}\bigl(x\COLON \TYP{List}\llbracket\TYP{Object}\rrbracket, y\COLON \TYP{List}\llbracket\TYP{Object}\rrbracket\bigr)\COLON \TYP{List}\llbracket\TYP{Object}\rrbracket = e_1 \\
  \VAR{append}\bigl(x\COLON \TYP{List}\llbracket\TYP{String}\rrbracket, y\COLON \TYP{List}\llbracket\TYP{String}\rrbracket\bigr)\COLON \TYP{List}\llbracket\TYP{String}\rrbracket = e_1 \\
  \VAR{append}\bigl(x\COLON \TYP{List}\llbracket\mathbb{Z}\rrbracket, y\COLON \TYP{List}\llbracket\mathbb{Z}\rrbracket\bigr)\COLON \TYP{List}\llbracket\mathbb{Z}\rrbracket = e_1 \\
  \ldots \\
  \VAR{append}\bigl(x\COLON \TYP{List}\llbracket\mathbb{Z}\rrbracket, y\COLON \TYP{List}\llbracket\mathbb{Z}\rrbracket\bigr)\COLON \TYP{List}\llbracket\mathbb{Z}\rrbracket = e_2\-
\end{FortressCode}
\normalsize
and we can see that there is an ambiguity 
when the arguments are both of type \EXP{\TYP{List}\llbracket\mathbb{Z}\rrbracket}.

It gets worse if the programmer wishes to handle an infinite set of cases specially.  
It would seem natural to write
\small
\begin{FortressCode}
{\tt ~~}\+\VAR{append}\llbracket{}T\rrbracket\bigl(x\COLON \TYP{List}\llbracket{}T\rrbracket, y\COLON \TYP{List}\llbracket{}T\rrbracket\bigr)\COLON \TYP{List}\llbracket{}T\rrbracket = e_1 \\
  \VAR{append}\llbracket{}T \SHORTCUT{<} \TYP{Number}\rrbracket\bigl(x\COLON \TYP{List}\llbracket{}T\rrbracket, y\COLON \TYP{List}\llbracket{}T\rrbracket\bigr)\COLON \TYP{List}\llbracket{}T\rrbracket = e_2\-
\end{FortressCode}
\normalsize
to handle specially all cases where \VAR{T} is a subtype of \TYP{Number}.
But the model would regard this as an overloading 
with an infinite number of ambiguities.


To resolve this problem, 
we had to develop an alternate model 
and an associated type system 
that could handle overloaded type-parametric functions 
in a manner that would accord with programmer intuition 
and support the plausible examples shown above.

Two other authors of this paper 


The key insight 


Credit for championing key insights---regarding each polymorphic definition
as a single definition (rather than an infinite set of definitions)
competing in the overload set, and using universal and existential types
to describe them in the type system (an idea reported by
Bourdoncle and Merz~\cite{bourdoncle97})---%
belongs to two other authors of this paper (Hilburn and Kilpatrick).
Adopting this new approach has made
overloaded polymorphic functions both tractable and effective
for writing Fortress code.

In this paper, 
we give rules for ensuring safe overloaded functions 
in a language that supports symmetric multiple dispatch, 
multiple inheritance, and parametric polymorphism 
(that is, generic types \emph{and} generic functions),
and we prove that these rules guarantee 
that there are no ambiguous calls at run time 
(see Section~\ref{sec:safety}).
We do this by extending our earlier rules 
for a core of the Fortress programming language 
that did not support generics \cite{allen07,Fortress},
for which we proved the analogous theorem.
To minimize syntactic overhead 
and avoid having to translate 
between a concrete language syntax 
and a formal semantics, 
we present these rules (see Section~\ref{sec:rules}) 
in the context of a straightforward formalization 
of a type system supporting multiple inheritance 
and parametric polymorphism, 
which we define in Section~\ref{sec:pre}.

The problem of dynamic dispatch 
in the presence of overloaded \emph{generic} functions 
is challenging 
because the overloaded definitions
might have not only distinct argument types, 
but also distinct type parameters 
(even different numbers of type parameters), 
so the type values of these parameters 
make sense only in distinct type environments. 
For example, consider the following overloaded function definitions in Fortress:
\small
\begin{FortressCode}
{\tt ~~}\+\VAR{combine}\llbracket{}T\rrbracket\bigl(\VAR{xs}\COLON \TYP{List}\llbracket{}T\rrbracket, \VAR{ys}\COLON \TYP{List}\llbracket{}T\rrbracket\bigr)\COLON \TYP{List}\llbracket{}T\rrbracket \\
  \VAR{combine}\llbracket{}S,T\rrbracket\bigl(s\COLON \TYP{Table}\llbracket{}S,T\rrbracket, t\COLON \TYP{Table}\llbracket{}S,T\rrbracket\bigr)\COLON \TYP{Table}\llbracket{}S,T\rrbracket\-
\end{FortressCode}
\normalsize
The first definition declares a single
type parameter denoting the types of the elements of the two
list arguments $xs$ and $ys$. The second definition declares two 
type parameters corresponding to the domains and ranges of the two
table arguments $s$ and $t$. But the type parameter of the first
definition bears no relation to the type parameters of the second.
How should we compare such function definitions 
to determine which is the best to dispatch to?
How can we ensure that there even is a best one in all cases?
Furthermore, the rules must be compatible with type inference, 
since instantiation of type parameters at a call site 
is typically done automatically.
So even determining which definitions are applicable 
to a particular call is not always obvious.

In providing rules to ensure 
that any valid set of overloaded function definitions 
guarantees that there is always a unique function to call at run time, 
we strive to be maximally permissive: 
A set of overloaded definitions should be disallowed 
only if it permits ambiguity
that cannot be resolved at run time.  
Nonetheless, 
we show in Section~\ref{sec:problems} 
that some seemingly valid sets of overloaded functions are rejected by our rules, 
and rightly so: 
although intuitively appealing, 
these overloaded functions admit ambiguous calls.

Many of these overloaded functions can, 
and we believe should, 
be allowed 
if the type system supports an \emph{exclusion relation},
which asserts that two types have no common instances.
If the domains of two function definitions exclude each other, 
then these definitions can never be applicable to the same call,
and so no ambiguity can arise between them.
Many languages provide a way of declaring some exclusion relations
implicitly. For example, single inheritance ensures that, for any 
two types, if one is not a subtype of the other, then the two types exclude each other.
Fortress enables programmers to declare ``nominal exclusion''
in addition to determining many exclusions implicitly, 
and in Section~\ref{sec:exclusion}, 
we formalize how Fortress does this, 
and show how this exclusion relation is used 
to improve expressivity 
by accommodating overloadings that would otherwise be rejected.
The proof of safety in Section~\ref{sec:safety} 
covers the rules under this extended type system.

% The remainder of this paper is organized thus:
% In Section~\ref{sec:pre}, we define the concepts and notation necessary
% to explain our formal rules for checking overloaded function definitions,
% which we present using universal and existential types in Section~\ref{sec:rules}.
% In Section~\ref{sec:problems}, 
% we explain why some apparently valid overloadings
% are (correctly) rejected by our rules 
% and why a multiple-inheritance language
% should include features for ``nominal exclusion'' (as Fortress does)
% to improve expressiveness and accommodate such overloadings.
% In Section~\ref{sec:exclusion}, we formalize the exclusion
% relation and use it to extend the overloading rules of Section~\ref{sec:rules}.
% %use it to augment the subtyping relation for universal and existential types.
% Section~\ref{sec:safety} explains that the overloading rules are
% sufficient to guarantee no undefined or ambiguous calls at run time.
In Section~\ref{sec:discussion}, we discuss type inference and modularity.
We discuss related work in Section~\ref{sec:related} and
conclude in Section~\ref{sec:conclusion}.


% INTRO
% Symmetric Multiple Dispatch
% Ambiguity
% State the old rules
% Parametric polymorphism
% Dispatch

\section{Preliminaries}\label{sec:pre}
\subsection{Types}

Following Kennedy and Pierce \cite{kennedy07},
we define a world of types ranged over by metavariables $S$, $T$, $U$, $V$, and $W$. 
Types are of four forms: 
\emph{type variables} 
(ranged over by metavariables $X$, $Y$, and $Z$);
\emph{constructed types} 
(ranged over by metavariables $K$, $L$, $M$ and $N$), 
written \EXP{C\llbracket\bar{T}\rrbracket} 
where \VAR{C} is a type constructor 
and \EXP{\bar{T}} is a list of types; 
\emph{structural types},
consisting of arrow and tuple types;
and \emph{compound types},
consisting of intersection and union types. 
In addition, 
there are two special constructed types, \Any\ and \BottomType, explained below.
\TODO{Make \BottomType\ special, not a constructed type.}
% A type is of one of five forms:
% a \emph{type variable} 
% (represented by metavariable $X$, $Y$, or $Z$);
% \emph{constructed types} 
% (represented by metavariable $K$, $L$, $M$ or $N$), 
% which is either a type constructor application \EXP{C\llbracket\bar{T}\rrbracket},  
% where \VAR{C} is a type constructor 
% and \EXP{\bar{T}} is a list of types, 
% or the special constructed type \Any;
% a \emph{structural type},
% which is either an arrow type or a tuple type;
% a \emph{compound type},
% which is either an intersection or union type; 
% or the special type \BottomType, 
% which represents the uninhabited type 
% (i.e., no value belongs to \BottomType).
The abstract syntax of types is defined in BNF as follows
(where $\bar{\emph{A}}$ indicates 
a possibly empty comma-separated sequence of syntactic elements $\emph{A}$):
%and we often abbreviate sequences
%of larger syntactic constructs such as $T \extends U$ 
%by writing bars over each variable:
%$\bar{T \extends U}$.)
\[
\begin{array}{@{}l@{\;}l@{\;}l@{\;\;\;\;\;\;}l@{}}
\emph{T} &::=& \emph{X} & \hbox{\rm type variable}\\
&\mid& \emph{C}\llbracket\bar{\emph{T}}\rrbracket & \hbox{\rm type constructor application}\\
&\mid& \emph{T} \rightarrow \emph{T} & \hbox{\rm arrow type}\\
&\mid& ( \bar{\emph{T}} ) & \hbox{\rm tuple type}\\
&\mid& \emph{T} \cap \emph{T} & \hbox{\rm intersection type}\\
&\mid& \emph{T} \cup \emph{T} & \hbox{\rm union type}\\
&\mid& \Any \\
&\mid& \BottomType \\
\end{array}
\]
% \[
% \begin{array}{@{}l@{\;}l@{\;}l@{\;\;\;\;\;\;}l@{}}
% \emph{Type} &::=& \emph{Id} & \hbox{\rm type variable}\\
% &\mid& \emph{Id}\llbracket\bar{\emph{Type}}\rrbracket & \hbox{\rm constructed type}\\
% &\mid& \emph{Type} \rightarrow \emph{Type} & \hbox{\rm arrow type}\\
% &\mid& ( \bar{\emph{Type}} ) & \hbox{\rm tuple type}\\
% &\mid& \emph{Type} \cap \emph{Type} & \hbox{\rm intersection type}\\
% &\mid& \emph{Type} \cup \emph{Type} & \hbox{\rm union type}\\
% %&\mid& \Any \\
% %&\mid& \BottomType \\
% \end{array}
% \]

A tuple type of length one is synonymous with its element type. 
A tuple type with any \BottomType\ element 
is synonymous with \BottomType.
As in Fortress, 
compound types---intersection and union types---and \BottomType\ 
are \emph{not} first-class:
they cannot be written in a program; 
rather, they are used by the type analyzer during type checking.
For example, type variables may have multiple bounds, 
so that any valid instantiation of such a variable
must be a subtype of the intersection of its bounds.

% All type checking \TODO{Do we prefer ``type-checking'',
% ``type checking'', or ``typechecking''?)
% is done within the context of a \emph{class table} $\T$, 
% which is a set of type constructor declarations 
% (at most one declaration for each type constructor) 
% of the following form:
% \[
% C\tplist{X}{M} \extends \{\bar{N}\}
% \]

\TODO{Only considering ground types at first.
(Define ground types.)}

Constructed types (other than \Any\ and \BottomType)
are applications of \emph{type constructors}.
A \emph{class table} $\T$
is a set of type constructor declarations 
(at most one declaration for each type constructor) 
of the following form:
\[
C\tplist{X}{M} \extends \{\bar{N}\}
\]
This declaration indicates that an application $C\obb{U}$ 
(\emph{i}) is \emph{well-formed} (with respect to $\T$)
if and only if $|\bar{U}| = |\bar{X}|$ and
for all $i$ and $j$, 
$U_i \subtypeof \substb{U}{X}M_{ij}$ for each bound $M_{ij}$
(where $\subtypeof$ is the subtyping relation defined below, 
and $\substb{U}{X}M_{ij}$ is $M_{ij}$ 
with $U_k$ substituted for each occurrence of $X_k$ in $M_{ij}$ 
for $1 \leq k \leq |\bar{U}|$);
and 
(\emph{ii}) is a subtype of $\substb{U}{X}N_l$ for $1 \leq l \leq |\bar{N}|$.
Thus, a class table induces 
a (nominal) \emph{subtyping relation} over the constructed types 
by taking the reflexive and transitive closure 
of the subtyping relation derived from the declarations in the class table.
In addition, 
every type is a subtype of \Any\ and a supertype of \BottomType.
For any type $T$ (of any form),
we write $T \in \T$ to mean that 
any constructed type occurring in type $T$
is well-formed with respect to $\T$.

A class table $\T$ is \emph{well-formed} if the resulting subtyping relation 
on its constructed types is a partial order.
As usual for languages with nominal subtyping, 
we allow recursive and mutually recursive references in $\T$.
\TODO{What does ``recursive and mutually recursive references'' mean?}
A class table $\T'$ is an \emph{extension} of $\T$ (written $\T' \ctext \T$)
if every constructor declaration in $\T$ is also in $\T'$ 
and the subtype relation on $\T'$ agrees with that of $\T$. 
Consequently, if $\T' \ctext \T$ then, for all types $T$, $T \in \T$ implies $T \in \T'$.
We typically omit explicit reference to the class table when it is understood, 
and we assume that the class table is well-formed.

Given a well-formed class table 
containing the type constructor declaration for $C$ above 
and a well-formed application $C\obb{T}$,
we denote the set of its explicitly declared supertypes by
\[
\myextends{C\obb{T}} = \{ \bar{\substb{T}{X}N} \}
\]
and the set of ancestors of $C\obb{T}$ 
(defined recursively) by
\[
\ancestors(C\obb{T}) 
   = \{C\obb{T}\} \cup 
     \hspace*{-4ex} \bigcup_{M \in \myextends{C\obb{T}}} \hspace{-4ex} \ancestors(M).
\]
(To reduce clutter, 
nullary applications are written without brackets; 
for example, \EXP{C\llbracket\,\rrbracket} is written \VAR{C}.) 
\TODO{Other for \Any\ and \BottomType, 
do we actually do this?}

Structural and compound types are \emph{well-formed} 
(with respect to a class table) 
if their constituent types are well-formed. 
We extend the subtyping relation to
structural and compound types in the usual way:
Arrow types are contravariant in their domain types 
and covariant in their range types 
(i.e., $\arrowtype{S}{T} \subtypeof \arrowtype{U}{V}$
if and only if $U \subtypeof S$ and $T \subtypeof V$).
One tuple type is a subtype of another 
if and only if they have the same number of elements, 
and each element of the first is a subtype of the corresponding element of the other 
(i.e., $( \tuple{\bar{S}} ) \subtypeof ( \tuple{\bar{T}} )$
if and only if $|\bar{S}| = |\bar{T}|$
and $S_i \subtypeof T_i$ for all $1 \leq i \leq |\bar{S}|$).
An intersection type is by definition the most general type that is a subtype
of each of its element types: $(A \cap B) <: A$, $(A \cap B) <: B$, and for all types $T$,
if $T <: A$ and $T <: B$ then $T <: (A \cap B)$.
Similarly, a union type is by definition the most specific type that is a supertype
of each of its element types: $A <: (A \cup B)$, $B <: (A \cup B)$, and for all types $T$,
if $A <: T$ and $B <: T$ then $(A \cup B) <: T$.

To extend the subtyping relation to type variables,
we require  a \emph{type environment}, 
which maps type variables to bounds:
\[
\Delta = \bds{X}{M}
\]
In the context of $\Delta$, 
each type variable $X_i$ is a subtype of each of its bounds $M_{ij}$.
Note that the type variables $X_i$ may appear within the bounds $M_{ij}$.
We write $\jgsub{S}{T}$ 
to indicate the judgment that $S$ is a subtype of $T$ 
in the context of $\Delta$. When $\Delta$ is understood to be empty, 
we write this judgment as simply $S \subtypeof T$. The subtype judgment
can only be made on types $S, T \in \T$, so the judgment takes a class table $\T$ as an implicit parameter.
The types $S$ and $T$ are said to be \emph{equivalent}, written $S \equiv T$, when $S \subtypeof T$ and $T \subtypeof S$.

To allow separate compilation of program components, 
we do not assume that the class table is complete;
there might be declarations yet unknown.
Specifically, 
we cannot infer that two constructed types 
have no common constructed subtype (other than \BottomType) 
from the lack of any such type in the class table.
However, we do assume that each declaration is complete, 
and furthermore, 
that any type constructor used in the class table 
(e.g., in a bound or a supertype of another declaration)
is declared in the table, 
so that all the supertypes of a constructed type 
are known.


\subsection{Values and Ilks}

Types are intended to describe the values that might be produced by
an expression or passed into a function.
In Fortress, for example, there are three kinds of values: 
objects, functions, and tuples;
every object belongs to at least one constructed type,
every function belongs to at least one arrow type,
and every tuple belongs to at least one tuple type.
When we say that two types \VAR{T} and \VAR{U} have \emph{the same extent},
we mean that for every value \VAR{v}, 
\VAR{v} belongs to \VAR{T} if and only if \VAR{v} belongs to \VAR{U}.

We place a requirement on values and on the type system that describes them: 
Although a value may belong to more than one type, 
for every value \VAR{v} there is a unique type \EXP{\VAR{ilk}(v)} 
(the \emph{ilk} of the value) 
that is \emph{representable in the type system}\footnote{The
type system presented here satisfies this requirement 
simply by providing intersection types.  
The Fortress type system happens to satisfy it in another way as well, 
which is typical of object-oriented language designs: 
every object is created as an instance of a single nominal constructed type, 
and this type is its ilk.} 
and has the property that for every type \VAR{T}, 
if \VAR{v} belongs to \VAR{T} then \EXP{\VAR{ilk}(v) \SHORTCUT{<} T};
moreover, \EXP{\VAR{ilk}(v) \neq \TYP{BottomType}}.  
(This notion of \VAR{ilk} corresponds to what is sometimes called the
``class'' or ``run-time type'' of the value.\footnote{%
We prefer the term ``ilk'' to ``run-time type'' 
because the notion---and usefulness---of 
the most specific type to which a value belongs 
is not confined to run time.
We prefer it to the term ``class,'' 
which is used in {\it The Java Language Specification}~\cite{JavaSpec}, 
because not every language uses the term ``class'' 
or requires that every value belong to a class.  
For those who like acronyms, 
we offer the mnemonic retronyms 
``implementation-level kind'' 
and ``intrinsically least kind.''})

The implementation significance of ilks is that it is possible to
select the dynamically most specific applicable function
from an overload set using only the ilks of the argument values; no
other information about the arguments is needed.

In a sound type system,
if an expression is determined by the type system to have type \VAR{T}, 
then every value computed by the expression at run time
will belong to type \VAR{T}; 
moreover, 
whenever a function whose ilk is \EXP{U\rightarrow{}V} is applied to an argument value,
then the argument value will belong to type \VAR{U}.


\subsection{Overloaded Functions}
\label{terms}

A function declaration consists of 
a name, 
a sequence of type parameter declarations 
(enclosed in white square brackets), 
a type indicating the domain of the function, 
and a type indicating the range of the function.  
A type parameter declaration consists of
a type parameter name and its bounds.
We omit the white square brackets of a declaration 
when the sequence of type parameter declarations is empty.
The abstract syntax of function declarations is as follows:
\[
\begin{array}{lll}
\emph{Decl} &::=& 
%\emph{Id}\llbracket\bar{\emph{Id}}\SHORTCUT{<}\bar{\emph{Type}}\rrbracket \emph{Type} \COLON \emph{Type}\\
\declg{\textit{Id}}{\textit{Id}}{\textit{Type}}{\textit{Type}}{\textit{Type}}  \\
&\mid& 
%\emph{Id}\ \emph{Type} \COLON \emph{Type}\\
\decl{\textit{Id}}{\textit{Type}}{\textit{Type}}
\end{array}
\]

For example, in the following function declaration:

\begin{FortressCode}
{\tt ~~}\+f\llbracket{}X \SHORTCUT{<} M, Y \SHORTCUT{<} N\rrbracket\bigl(\TYP{List}\llbracket{}X\rrbracket, \TYP{Tree}\llbracket{}Y\rrbracket\bigr)\COLON \TYP{Map}\llbracket{}X, Y\rrbracket\-
\end{FortressCode}
the name of the function is \VAR{f}, 
the type parameter declarations are \EXP{X \SHORTCUT{<} M} and \EXP{Y \SHORTCUT{<} N}, 
the domain type is \EXP{\bigl(\TYP{List}\llbracket{}X\rrbracket, \TYP{Tree}\llbracket{}Y\rrbracket\bigr)}, which is a tuple type,
and the range type is \EXP{\TYP{Map}\llbracket{}X, Y\rrbracket}. We will often abbreviate a function
as \hdeclg{f}{\Delta}{S}{T} when we do not want to emphasize the bounds (we are abusing notation by letting
$\Delta$ range over both type environments and bounds definitions).

A function declaration \declg{f}{X}{N}{S}{T}
may be \emph{instantiated} with type arguments $\bar{W}$ 
if $|\bar{W}| = |\bar{X}|$ and $W_i \subtypeof \substb{W}{X} N_{ij}$ for all $i$ and $j$;
we call $\substb{W}{X} \decl{f}{S}{T}$
the \emph{instantiation} of $f$ with $\bar{W}$. 
When we do not care about $\bar{W}$, 
we just say that $\decl{f}{U}{V}$
is an \emph{instance} of $f$ (and it is understood that $U=\substb{W}{X}S$
and $V=\substb{W}{X}T$ for some (unstated) $\bar{W}$).
%
We use the metavariable $\D$ for a finite collection of sets of
function declarations and $\D(f)$ for the set in
$\D$ that contains all declarations named $f$.
%
An instance \decl{f}{U}{V} of a declaration $f$ 
is \emph{applicable} to a type $T$ 
if and only if $T \subtypeof U$.
A function declaration is \emph{applicable} to a type 
if and only if at least one of its instances is.
%
For any two function declarations $f_1, f_2 \in \D(f)$, 
$f_1$ is \emph{more specific} than $f_2$ 
(written $f_1 \ms f_2$)
if and only if for every type $T$ 
such that $f_1$ is applicable to $T$, 
$f_2$ is also applicable to $T$.


%% \begin{figure}
%%   \begin{minipage}{.462\textwidth}
    
%%   \fbox{\textbf{Type equivalence reduction:} \quad \jtred{\Delta}{T}{T}}
%%   \TODO{rules for flattening/distributing $\cap, \cup$}
%%   \TODO{is subtype/exclusion judgment in premises ok?}
  
%%   % INTERSECTION
%%   \infrule
%%     {\jgsub{\Delta}{T}{U}}
%%     {\jtred{\Delta}{T \cap U}{T}}
%%   \infrule
%%     {\jgsub{\Delta}{U}{T}}
%%     {\jtred{\Delta}{T \cap U}{U}}
%%   \infrule
%%     {\jexc{\Delta}{T}{U}}
%%     {\jtred{\Delta}{T \cap U}{\TYP{BottomType}}}
  
%%   % UNION
%%   \infrule
%%     {\jgsub{\Delta}{T}{U}}
%%     {\jtred{\Delta}{T \cup U}{U}}
%%   \infrule
%%     {\jgsub{\Delta}{U}{T}}
%%     {\jtred{\Delta}{T \cup U}{T}}
  
%%   % ARROW
%%   % \infrule
%%   %   {\jtred{\Delta}{T}{T'}}
%%   %   {\jtred{\Delta}{T \rightarrow U}{T' \rightarrow U}}
%%   % \infrule
%%   %   {\jtred{\Delta}{U}{U'}}
%%   %   {\jtred{\Delta}{T \rightarrow U}{T \rightarrow U'}}
  
%%   % TUPLE
%%   % \infrule
%%   %   {\jtred{\Delta}{T_i}{T_i'}}
%%   %   {\jtred{\Delta}{(T_1, \ldots, T_i, \ldots, T_n)}{(T_1, \ldots, T_i', \ldots, T_n)}}
%%   \infrule
%%     {T_i = \TYP{BottomType}}
%%     {\jtred{\Delta}{(T_1, \ldots, T_i, \ldots, T_n)}{\TYP{BottomType}}}
  
%%   % CONSTRUCTED
%%   % \infrule
%%   %   {\jtred{\Delta}{T_i}{T_i'}}
%%   %   {\jtred{\Delta}{C\ob{T_1, \ldots, T_i, \ldots, T_n}}{C\ob{T_1, \ldots, T_i', \ldots, T_n}}}
  
%%   % VARIABLE
%%   \TODO{should be a subtype judgment instead?}
%%   \infrule
%%     {\EXP{X \SHORTCUT{<} \TYP{BottomType} \in \Delta}}
%%     {\jtred{\Delta}{X}{\TYP{BottomType}}}
  
%%   % REDUCTION CONTEXT GRAMMAR
%%   \newcommand{\OR}{\;|\;}
%%   \[ E \; ::= \; [] \OR E \rightarrow U \OR T \rightarrow E \]
%%   \[ \OR (T_1, \ldots, E, \ldots, T_n) \]
%%   \[ \OR C\ob{T_1, \ldots, E, \ldots, T_n} \]
  
%%   % CONGRUENCE
%%   \infrule
%%     {\jtred{\Delta}{T}{T'}}
%%     {\jtred{\Delta}{E[T]}{E[T']}}
  
%%   \end{minipage}
%%   \caption{Type equivalence reduction}
%%   \label{fig:tred}
%% \end{figure}


% TYPES
% Classes
% Class Table
% Subtyping
% Tuples + Arrows

% OVERLOADED FUNCTIONS
% Dispatch: Static vs Dynamics
% Modularity: Class table never complete
% Old Overloading Rules
% Dispatch: Parametric functions

\section{Parametrically Polymorphic Overloading}\label{sec:rules}
% Stuff needed in set up
% intersection and union types
%   can be structural types
%   distribute over structural types (tuples, arrows, intersection, union)
% Extensions of the class table
%   what does it mean for an extension to be well formed
%   what does this mean for condition 1)
% Language is call by value

Guaranteeing valid overloading requires constraints on the sets of overloaded function declarations
that are allowed to appear in a legal program \cite{millstein02,millstein03}.
Some languages require program constructs that encapsulate all overloaded definitions;
such constructs essentially guarantee valid overloading ``by construction''.
Examples in other languages include type classes \cite{wadler89,dreyer07,siek05}
and multimethods \cite{millstein02,millstein03,bourdoncle97}.
We take a different approach: Rather than introducing additional language facilities,
we impose rules on the function declarations themselves.
% These aren't necessarily strictly "minimal" though, so:
We intend these rules to be ``minimal'' in that they should be as unrestrictive as possible
while preserving the ability to guarantee valid overloading and be checked in a modular way.
We took a similar approach to guarantee safety for overloaded monomorphic functions \cite{allen07}.
However, handling parametric polymorphism and implicit instantiation (i.e., type inference)
requires more sophisticated type analysis.
We achieve this analysis by introducing universal and existential quantification
over the ground types defined by class tables.

Specifically, in this section, we define three rules---the \emph{No Duplicates Rule},
\emph{Meet Rule}, and \emph{Return Type Rule}---for sets of overloaded function definitions,
and say that such a set is \emph{well-formed with respect to a class table}
if it satisfies all these rules using the subtyping relation induced by the class table.
We describe how to mechanically verify these rules in a modular way
in terms of subtyping relations on universal and existential types in Section~\ref{sec:checking},
and we show that any valid set of overloaded function declarations is safe in Section~\ref{sec:safety}.

\subsection{Overloading Rules}\label{sec:threerules}
In this section, we describe the rules for valid overloading.
For each function name $f$,
we determine whether a set of overloaded function declarations $\D(f)$ is valid
by independently considering every pair of declarations in the set.
A pair of declarations is a valid overloading if it satisfies one of three rules described below.
%the more-specific-than relation on these declarations by the subtyping relation induced by a class table $\T$.


To avoid the obvious ambiguity, $\D(f)$ should not contain equally specific declarations:
for each pair of overloaded declarations, either one declaration is strictly more specific
than the other or they are incomparable.
\begin{description}
\item[No Duplicates Rule]
$\D(f)$ does not contain any two declarations that are equally specific.
%(i.e., each declaration is more specific than the other).
In other words, there are no (distinct) declarations $f_1, f_2 \in \D(f)$ such that $f_1 \ms f_2$ and $f_2 \ms f_1$.
\end{description}


A pair of declarations is a valid overloading if for any call to which both declarations are applicable,
there is a {\em disambiguating declaration} (possibly one of the pair)
that is also applicable to the call and is at least as specific as both declarations.
Thus, at run time, the disambiguating declaration is preferred.
\begin{description}
\item[Meet Rule]
For each pair of declarations $f_1, f_2 \in \D(f)$,
and for every type $T \in \T$, there should exist a third declaration $f_0 \in \D(f)$
(possibly one of the pair) such that $f_0$ is applicable to $T$ if and only if
both $f_1$ and $f_2$ are applicable to $T$.
\end{description}



If one monomorphic declaration is more specific than another monomorphic one
then there is no ambiguity between these two declarations:
for any call to which both are applicable, the first is more specific.
In the parametrically polymorphic setting,
if one declaration is to be regarded as more specific than another,
we require that for every instance of the second that is applicable to a call,
there exist an instance of the first that is also applicable to the call.
As in other object-oriented languages, to ensure type safety in the face of dynamic dispatch,
we also require that the return type of the latter declaration be a subtype of the return type of the former.
% We can ensure that the second well-formedness condition for overloading
% holds by checking the following rule for each function name $f$:
\begin{description}
\item[Return Type Rule]
For every $f_1, f_2 \in \D(f)$ with $f_1 \ms f_2$, for every type $W$ to which $f_1$ is applicable
(and therefore $f_2$ is also applicable)
and every instance $\decl{f_2'}{S_2'}{T_2'}$ of $f_2$ that is applicable to $W$,
there must exist an instance \decl{f_1'}{S_1'}{T_1'} of $f_1$
that is applicable to $W$ and satisfies \mbox{$T_1' <: T_2'$}.
\end{description}

% GENERALIZED RULES
% subtyping on domain = more specific for arrows
% subtyping for existential types = more specific for universal arrows
% using Derek's extension of System F_sub
% no duplicates rule
% meet rule
% exclusion rule
% subtype rule
% Nothing satisfies them

\section{Overloading Resolution Safety}\label{sec:safety}
\begin{lemma}[Progress]
\label{lem:progress}
If some declaration in $\D(f)$ is applicable to $W$
then there is a unique most specific declaration $f_W \in \D(f)$ that is applicable to $W$.
\end{lemma}
\begin{proof}
Our proof strategy for satisfying the Progress condition
makes use of the old idea from Castagna \emph{et al.} \cite{castagna95} that
for each name $f$ the set of function declarations $\D(f)$ should form a meet
semilattice under the \emph{specificity} order defined in
Section~\ref{terms}.
If any declaration of name $f$ is applicable to a type $W$,
then the set $\D_W(f) \subseteq \D(f)$
of all declarations named $f$ and applicable to $W$
also forms a nonempty meet semilattice under specificity. Therefore
$\D_W(f)$ must have a least element.
\end{proof}

\begin{lemma}[Preservation]
\label{lem:preservation}
If $\decl{f}{S}{T}$ is an instance of some declaration 
and $\decl{f}{S}{T}$ is applicable to $W$, 
then there exists some instance $\decl{f_W}{U}{V}$ of $f_W$ such that
$\decl{f_W}{U}{V}$ is applicable to $W$ and  $V \extends T$.
\end{lemma}
\begin{proof}
Note that subtyping is preserved under class table extension,
so if $\D(f)$ satisfies the No Duplicates Rule and the Meet Rule
with respect to the class table $\T$ then it satisfies them
with respect to $\T'$ for any $\T' \ctext \T$.
Therefore, we can be certain that adding more types will not
invalidate the Progress guarantee.
Just as in our discussion of the Meet Rule, the fact that subtyping is preserved
under class table extension makes sure that the property that $\D(f)$ satisfies
the Return Type Rule is preserved under class table extension. Therefore, the
Return Type Rule and the rules from the last section are sufficient to ensure safety.
\end{proof}

\begin{theorem}[Overloading Resolution Safety]
\label{thm:safety}
There exists always a unique function to call at run time among a collection of overloaded function declarations.
%There are no ambiguous calls at run time due to a collection of overloaded function declarations.
\end{theorem}
\begin{proof}
Lemma~\ref{lem:progress} ensures that we never get stuck resolving an application;
every call to a function declaration is unambiguous.
Lemma~\ref{lem:preservation} ensures that the static type of an application
is a supertype of the ilk of each value produced by the application at run time,
provided that the ilks for the argument values in function applications 
are always subtypes of the static types of the argument expressions
(as they are in any language with a sound type system).
\end{proof}
Note that as a monomorphic function declaration is a special case of a generic function declaration,
where the sequence of the type parameters is empty,
these conditions apply also for monomorphic function declarations.



\section{Disallowed Valid Overloading}\label{sec:problems}
% Outline for this section

% Many programs that users would like to write are unsafe
% Undefinable meets
%  ex1: String, ZZ example
%  ex2: [\X <: Any\]ArrayList[\X\], [\Y<: Foo\]List[\Y\] ([\X <: Foo\]ArrayList[\X\] doesn't work)
% Don't meet the subtype rule
%  ex3: ArrayList[\T\], List[\T\]

% Solution
% 1) Exclusion Relation
%  symmetric, irreflexive
%  change subtyping to ensure that the intersection of two excluding types is bottom
%  to lift to existential type schema: all instantiations exclude
%  we want same behavior when type checking programs with type variables
% 2) Outlaw multiple instance inheritance
%  lets us add a new kind of exclusion
%  can use that to reason that ex2 is safe
%  need to build this reasoning into subtyping for existential types
%  same machinery will help us fix ex3 in the next section

% What we will do in this section 
%  need to know constraints under which a type is not bottom => need to know when two types don't exclude
%  can get regular exclusion from not exclude

% Helpful notation
% Not Bottom \[\Delta \vdash T \not\equiv \TYP{BottomType}|\C\]
% Don't exclude under constraints \[\Delta \vdash T  \nexc S | \C\]

% How to get all instantiations exclude (written \exc)
% \infrule
%  {\Delta \vdash S \nexc T|False}
%  {\Delta \vdash S \exc T}

While the rules presented in Section~\ref{sec:rules} allow programmers to
write valid sets of overloaded generic function declarations,
they sometimes reject ``seemingly'' valid overloadings.
In fact, these are not false negatives;
many declarations that programmers would like to write are actually unsafe due to multiple inheritance.

For example, even though the following function declarations look like
a valid overloading, they are not because the Meet Rule is not satisfied:
\begin{FortressCode}
{\tt ~~}\+\VAR{simple}\, \TYP{String}\COLON\TYP{String} \\
  \VAR{simple}\, \mathbb{Z}\COLON\mathbb{Z}\-
\end{FortressCode}
Moreover, in Fortress it is impossible to disambiguate the declarations by providing the meet
because intersection types are not allowed in the Fortress syntax.
In a language with single inheritance, we might infer that these
overloadings were safe because a class can only have a single superclass.
However, due to multiple inheritance, we cannot be sure that these types
do not have a common subtype, no matter what the programmer intends.

Now consider this less trivial set of overloaded functions:
\begin{FortressCode}
{\tt ~~}\+\VAR{foo}\llbracket{}X \SHORTCUT{<} \TYP{Any}\rrbracket\TYP{ArrayList}\llbracket{}X\rrbracket\COLON\mathbb{Z} \\
  \VAR{foo}\llbracket{}Y\mathord{\SHORTCUT{<}}\: \mathbb{Z}\rrbracket\TYP{List}\llbracket{}Y\rrbracket\COLON\mathbb{Z} \\
  \VAR{foo}\llbracket{}W\mathord{\SHORTCUT{<}}\: \mathbb{Z}\rrbracket\TYP{ArrayList}\llbracket{}W\rrbracket\COLON\mathbb{Z}\-
\end{FortressCode}
where \EXP{\TYP{ArrayList}\llbracket{}T\rrbracket \SHORTCUT{<} \TYP{List}\llbracket{}T\rrbracket} for all types $T$.\footnote{We use this standard declaration for \TYP{ArrayList} throughout.}
The first two declarations are incomparable under specificity---%
the first declaration applies to all instantiations of type constructor \TYP{ArrayList},
whereas the second declaration applies only to instantiations of
type constructor \TYP{List} with subtypes of type \EXP{\mathbb{Z}}. The third definition,
which is the ``obvious" candidate to disambiguate the two, is not actually the meet;
the domain of this meet candidate is the existential type:
\begin{FortressCode}
{\tt ~~}\+\exists\llbracket{}W \SHORTCUT{<} \mathbb{Z}\rrbracket\TYP{ArrayList}\llbracket{}W\rrbracket\-
\end{FortressCode}
and needs to be proven equivalent to the domain of the computed meet:
\begin{FortressCode}
{\tt ~~}\+\exists\llbracket{}X \SHORTCUT{<} \TYP{Any}, Y \SHORTCUT{<} \mathbb{Z}\rrbracket\bigl(\TYP{ArrayList}\llbracket{}X\rrbracket \cap \TYP{List}\llbracket{}Y\rrbracket\bigr)\-
\end{FortressCode}
which requires that the latter be a subtype of the former.
However, there is no type \EXP{W \SHORTCUT{<} \mathbb{Z}} such that:
\\[.5em]
\hspace*{.5em}
%\jgsub[\EXP{X \SHORTCUT{<} \TYP{Any}, Y \SHORTCUT{<} \mathbb{Z}}]{\EXP{\TYP{ArrayList}\llbracket{}X\rrbracket \cap \TYP{List}\llbracket{}Y\rrbracket}}{\EXP{\TYP{ArrayList}\llbracket{}W\rrbracket}}
\ensuremath{
\begin{array}{l}
{\EXP{X \SHORTCUT{<} \TYP{Any}, Y \SHORTCUT{<} \mathbb{Z}}}\,\vdash
\\
\quad\quad\quad\quad
{\EXP{\TYP{ArrayList}\llbracket{}X\rrbracket \cap \TYP{List}\llbracket{}Y\rrbracket}}\;{\subtypeof}\;{\EXP{\TYP{ArrayList}\llbracket{}W\rrbracket}}
\end{array}
}
\\[.5em]
Our definition of the meet is not faulty: these declarations actually are unsafe.
Consider the (user-defined) constructed type:
\begin{FortressCode}
{\tt ~~}\+\TYP{BadList} \SHORTCUT{<} \bigl\lbrace\TYP{ArrayList}\llbracket\TYP{String}\rrbracket, \TYP{List}\llbracket\mathbb{Z}\rrbracket\bigr\rbrace\-
\end{FortressCode}
Our meet candidate is not applicable to \TYP{BadList}, but the two other definitions of \VAR{foo} are.
Since neither of those is more specific than the other,
this set of overloaded declarations must be rejected.



The following overloading example,
in which the second declaration is more specific than the first,
is also ill-formed:
\begin{FortressCode}
{\tt ~~}\+\VAR{tail}\llbracket{}X \SHORTCUT{<} \TYP{Any}\rrbracket\TYP{List}\llbracket{}X\rrbracket\COLON\TYP{List}\llbracket{}X\rrbracket \\
  \VAR{tail}\llbracket{}Y \SHORTCUT{<} \TYP{Any}\rrbracket\TYP{ArrayList}\llbracket{}Y\rrbracket\COLON\TYP{ArrayList}\llbracket{}Y\rrbracket\-
\end{FortressCode}
The declarations do not satisfy the Return Type Rule
because we cannot find a specific type \EXP{V \SHORTCUT{<} \TYP{Any}} such that:
\\[.8em]
\begin{tabular}{lr}
%% \jgsub[\EXP{X \SHORTCUT{<} \TYP{Any}, Y \SHORTCUT{<} \TYP{Any}}]{\EXP{\TYP{ArrayList}\llbracket{}V\rrbracket \rightarrow \TYP{ArrayList}\llbracket{}V\rrbracket}}{\EXP{\TYP{ArrayList}\llbracket{}Y\rrbracket \cap \TYP{List}\llbracket{}X\rrbracket \rightarrow \TYP{List}\llbracket{}X\rrbracket}}
\multicolumn{2}{l}{\EXP{X \SHORTCUT{<} \TYP{Any}, Y \SHORTCUT{<} \TYP{Any}} $\vdash$} \\
\quad\quad\quad\quad\quad\quad
&\EXP{\TYP{ArrayList}\llbracket{}V\rrbracket \rightarrow \TYP{ArrayList}\llbracket{}V\rrbracket} \\
&$\subtypeof$ \EXP{\TYP{ArrayList}\llbracket{}Y\rrbracket \cap \TYP{List}\llbracket{}X\rrbracket \rightarrow \TYP{List}\llbracket{}X\rrbracket}
\end{tabular}
\\[.8em]
%% \EXP{X \SHORTCUT{<} \TYP{Any}, Y \SHORTCUT{<} \TYP{Any}} \; &\vdash \EXP{\TYP{ArrayList}\llbracket{}V\rrbracket \rightarrow \TYP{ArrayList}\llbracket{}V\rrbracket} \\
%% \;&\subtypeof\; \EXP{\TYP{ArrayList}\llbracket{}Y\rrbracket \cap \TYP{List}\llbracket{}X\rrbracket \rightarrow \TYP{List}\llbracket{}X\rrbracket}
Once again, the type \TYP{BadList} proves that this set of overloaded declarations must be rejected. Consider the instance:
\begin{FortressCode}
{\tt ~~}\+\VAR{tail}\, \TYP{List}\llbracket\mathbb{Z}\rrbracket\COLON\TYP{List}\llbracket\mathbb{Z}\rrbracket\-
\end{FortressCode}
of the first declaration. We need to find an instance of the second declaration that is applicable to \TYP{BadList}
and has a return type that is a subtype of \EXP{\TYP{List}\llbracket\mathbb{Z}\rrbracket},
but the only instance of the second declaration applicable to \TYP{BadList} is:
\begin{FortressCode}
{\tt ~~}\+\VAR{tail}\, \TYP{ArrayList}\llbracket\TYP{String}\rrbracket\COLON\TYP{ArrayList}\llbracket\TYP{String}\rrbracket\-
\end{FortressCode}
whose return type is not a subtype of \EXP{\TYP{List}\llbracket\mathbb{Z}\rrbracket}.
Therefore, this set of overloaded declarations must be rejected.


\section{More Permissive Overloading}\label{sec:exclusion}
% EXCLUSION
% Exclusion on base types
% Exclusion on parametric types
%   restriction in kennedy-pierce paper
% Powerful for computing meets
% \input{exc-intro}
% \subsection{Constraints and Inference}
% \input{constraints}
% \subsection{Exclusion under Constraints}
% \input{doesnotexclude}
% %% \subsection{Not BottomType}
% %% \input{notbottom}
% \subsection{Subtyping}
% \input{reduce}

To allow more overloaded functions such as \VAR{simple}, \VAR{tail}, and \VAR{foo}
described in Section~\ref{sec:problems},
we extend the overloading rules using \emph{exclusion} relations on types.

\subsection{Type Exclusion}\label{sec:excdec}
To provide more expressive power to describe richer type relationships,
we augment our formalism with an \emph{exclusion} relation $\exc$ on types:
$S \exc T$ asserts that types $S$ and $T$ have no common subtypes other than \BottomType.
This allows us to describe explicitly what is typically only implicit in single-inheritance
class hierarchies.

We define the exclusion relation by extending type constructor declarations with
two new optional clauses, \KWD{excludes} and \KWD{comprises}:
\[
C\tplist{X}{L} \extends \EXP{\lbrace\bar{N}\rbrace\, \bigl[\KWD{excludes} \lbrace\bar{M}\rbrace\bigr]\, \bigl[\KWD{comprises} \lbrace\bar{K}\rbrace\bigr]}
\]
We also allow another form of declaration that is frequently convenient for defining ``leaf types'':
\[
\KWD{object} \; C\tplist{X}{L} \extends \EXP{\lbrace\bar{N}\rbrace}
\]

% The \KWD{excludes} clause explicitly states that
% the intersection of any application \EXP{C\llbracket\bar{T}\rrbracket} of type constructor \VAR{C}
% and any of \EXP{\bar{M}} is \TYP{BottomType} 
% (i.e., 
% \EXP{C\llbracket\bar{T}\rrbracket \cap [\bar{T} / \bar{X}]M_{i} \equiv \TYP{BottomType}} 
% for any \EXP{\bar{T}} and for each \EXP{M_{i}} in \EXP{\bar{M}}).
% This implies, of course, 
% that any subtype of \EXP{C\llbracket\bar{T}\rrbracket} also excludes 
% \EXP{[\bar{T} / \bar{X}]\, M_{i}}.
% The \KWD{comprises} clause stipulates that
% an application of type constructor \VAR{C} with \EXP{\bar{T}}
% consists of exactly the types in \EXP{\bar{K}} that are instantiated with \EXP{\bar{T}}
% (i.e., 
% \EXP{C\llbracket\bar{T}\rrbracket \equiv [\bar{T} / \bar{X}]K_{1}} $\cup \ldots \cup$ \EXP{[\bar{T} / \bar{X}]\, K_{n}}).
% Thus, if a type excludes \EXP{[\bar{T} / \bar{X}]K_{i}} for each $i$, then it
% also necessarily excludes \EXP{C\llbracket\bar{T}\rrbracket}.
%
The exclusion relation on constructed types can then be described in terms of
more precise sub-relations on those types, each of which corresponds to a certain
reason for (or proof of) exclusion:
\begin{enumerate}
  \item The \KWD{excludes} clause explicitly states that
the constructed type \EXP{C\llbracket\bar{T}\rrbracket}
excludes $\substb{T}{X}M_i$ for each $M_i$ in $\bar{M}$, which
implies that any subtype of \EXP{C\llbracket\bar{T}\rrbracket} also excludes
each $\substb{T}{X}M_i$. We write this exclusion sub-relation as
$C\obb{T} \,\excre\, \substb{T}{X}M_i$.

  \item The \KWD{comprises} clause stipulates that any subtype of \EXP{C\llbracket\bar{T}\rrbracket} \emph{must} be a subtype of \EXP{[\bar{T} / \bar{X}]K_{i}} for some \EXP{K_{i}} in \EXP{\bar{K}}. Then if every \EXP{[\bar{T} / \bar{X}]K_{i}} in \EXP{\bar{K}} excludes some type $U$, \EXP{C\llbracket\bar{T}\rrbracket} must also exclude $U$. We write this exclusion sub-relation as $C\obb{T} \,\excrc\, U$.
  
  \item The \KWD{object} keyword denotes a type constructor whose applications have no non-trivial subtypes; an \KWD{object} type constructor is a leaf of the class hierarchy. Since such a constructed type \EXP{C\llbracket\bar{T}\rrbracket} has no subtypes other than itself and \BottomType, we know that it excludes any type \VAR{U} of which it is not a subtype. We write this exclusion sub-relation as $C\obb{T} \,\excro\, U$.
\end{enumerate}
We take the symmetric closure of each of these relations 
to get the relations $\exce$, $\excc$ and $\exco$.
Exclusion between constructed types is informally defined as the union
of these symmetric relations.
(We introduce another sub-relation $\excp$ in Section~\ref{sec:exc-polyrules}.)

We can extend the exclusion relation 
to structural and compound types as follows:
Every arrow type excludes every non-arrow type.
Every singleton tuple type excludes exactly those types excluded by its element type.
Every non-singleton tuple type excludes every non-tuple type.
Tuple type $(\bar{V})$ excludes $(\bar{W})$ if either $|\bar{V}| \neq |\bar{W}|$
or $V_i$ excludes $W_i$ for some $i$. An intersection type excludes any type excluded by \emph{any} of its constituent types, 
while a union type excludes any type excluded by \emph{all} of its constituent types.
$\BottomType$ excludes every type (including itself---it is the only type 
that excludes itself), 
and $\Any$ does not exclude any type other than $\BottomType$.
(We define the exclusion relation formally in Figure~\ref{fig:jexc}
in Section~\ref{sec:constraints}.)

We augment our notion of a well-formed class table 
to require that the subtyping and exclusion relations it induces 
``respect'' each other.
That is, for all constructed types $M$ and $N$ other than \BottomType, 
\begin{enumerate}
\item  If $M$ excludes $N$ then $M$ must not be a subtype of $N$.

% FUTURE FIX REQUIRED:
% Technically the following two statements should be about instantiations
% of what is in a comprises clause.
% Leaving this for now, but we should fix it.

\item 
If $N \subtypeof M$ and $M$ has a \KWD{comprises} clause,
then there is some constructed type $K$ in the \KWD{comprises} clause of $M$
such that $N \subtypeof K$.

\item
If the type constructor $C$ is declared as an \KWD{object}, then its \KWD{comprises}
clause must be empty, and there is no other constructed type $N$ such that
$N \subtypeof C\obb{T}$ for any types $\bar{T}$.

\end{enumerate}
As with the subtyping relation, 
a valid extension to a class table $\T$ 
must preserve these well-formedness properties.

For convenience, 
we allow the \KWD{excludes} and \KWD{comprises} clauses to be omitted.
Omitting an \KWD{excludes} clause is equivalent to having \EXP{\KWD{excludes} \lbrace\ultrathin\rbrace}, and
omitting a \KWD{comprises} clause is equivalent to having \EXP{\KWD{comprises} \lbrace\,\TYP{Any}\,\rbrace}.\footnote{For
the sake of catching likely programming errors, the Fortress language requires that every $K_i$
in a \KWD{comprises} clause for $C\obb{T}$ be a subtype of $C\obb{T}$, but allowing
\TYP{Any} to appear in a \KWD{comprises} clause simplifies our presentation here.}
For an application $C\obb{T}$ 
of a declaration with the \KWD{excludes} and/or \KWD{comprises} clause above, 
we define the sets of instantiations of the types 
in these clauses analogously to $\myextends{C\obb{T}}$.
That is,\\[-.2em]
\begin{align*}
\myexcludes{C\obb{T}}  &= \{ \bar{\substb{T}{X}M} \} \\
\mycomprises{C\obb{T}} &= \{ \bar{\substb{T}{X}K} \}
\end{align*}


\subsection{Overloading Rule with Exclusion}\label{sec:exc-rules}
If the parameter types of a set of overloaded functions are disjoint,
the set does not introduce ambiguity: they are never applicable to the same call.
Therefore, to allow the function \VAR{simple} as a valid overloading,
any ``reasonable" class table $\T$ that declares types \EXP{\mathbb{Z}} and \TYP{String} would also
declare one to exclude the other, so \hbox{\EXP{\mathbb{Z}} $\exc$ \TYP{String}}.

However, the Meet Rule still requires a third declaration in $\D(\VAR{simple})$ that is applicable
to every type $T$ if and only if $\VAR{simple}_1$ and $\VAR{simple}_2$ are applicable to $T$.
Such $T$ would necessarily be a subtype of both \EXP{\mathbb{Z}} and \TYP{String}, but since
these types exclude, no such $T$ exists (in $\T$ or in any extension thereof).
We thus augment every collection of overloaded function declarations $\D$ such
that each $\D(f)$ includes an additional, implicit declaration $\decl{f_\bot}{\BottomType}{\Any}$.
This declaration is trivially more specific than any declaration possibly written
by a programmer, but it does not conflict with overloading safety since it is
only applicable to \BottomType.

With $\VAR{simple}_\bot$ implicitly part of $\D(\VAR{simple})$, the two \VAR{simple}
declarations \emph{almost} satisfy the Meet Rule: the checker now must verify
that $\textit{dom}(\VAR{simple}_\bot)$, \BottomType, is equivalent to the computed meet,
\EXP{\TYP{String} \cap \mathbb{Z}}. Therefore we augment our judgment for the subtype relation
with the rule necessary for constructing this equivalence:
\infrule
{S \;\exc\; T}
{\Delta \vdash S \cap T <: \BottomType}
With these adjustments, $\D(\VAR{simple})$ is now a valid overloading.


% \subsection{Polymorphic Exclusion}
\subsection{Overloading Rule with Polymorphic Exclusion}\label{sec:exc-polyrules}
Checking the declarations $\D(\VAR{tail})$ and $\D(\VAR{foo})$ is trickier.
Certainly neither \TYP{List} nor \TYP{ArrayList} declares an exclusion of the other,
so seemingly the exclusion relation does not help us check these declarations.
A fundamental problem is that a type such as \TYP{BadList} might be a subtype
of multiple instantiations of the type constructor \TYP{List} (in this case,
a subtype of \EXP{\TYP{List}\llbracket\mathbb{Z}\rrbracket} directly and of \EXP{\TYP{List}\llbracket\TYP{String}\rrbracket} via \EXP{\TYP{ArrayList}\llbracket\TYP{String}\rrbracket}).
A type hierarchy in which
some type extends multiple instantiations of the same type constructor is said
to exhibit \emph{multiple instantiation inheritance} \cite{kennedy07}.

We address this problem by imposing an additional restriction on well-formed class tables,
which, in effect, adds additional power to the exclusion relation:
\begin{description}
\item[Polymorphic Exclusion Rule] 
A type (other than \BottomType) 
must not be a subtype of multiple distinct instantiations of a type constructor. 
\end{description}
Thus,
distinct well-formed applications $C\obb{T}$ and $C\obb{U}$ 
of a type constructor $C$ exclude each other; 
we use $\bexcp$ for this relation.
Furthermore, 
if $M$ is a subtype of $C\obb{T}$ 
and $N$ is a subtype of $C\obb{U}$, 
then $M$ and $N$ also exclude each other;
we use $\excp$ for this relation, 
and we add it to the other exclusion sub-relations 
whose union forms the exclusion relation over constructed types.

Polymorphic exclusion is easy to statically enforce, 
and practice suggests that it is not onerous: 
it is already in the Java\texttrademark\ programming language.
Furthermore, there is good reason to enforce it:
Kennedy and Pierce have shown that multiple instantiation inheritance
is one of three conditions that lead to the undecidability of
nominal subtyping with variance \cite{kennedy07}.

With polymorphic exclusion in place, 
the type \TYP{BadList} is no longer well-formed in its class table.
However, our subtype judgment still cannot find a type \EXP{W \SHORTCUT{<} \mathbb{Z}} to prove that:
\\[.5em]
\hspace*{.5em}
%\jgsub[\EXP{X \SHORTCUT{<} \TYP{Any}, Y \SHORTCUT{<} \mathbb{Z}}]{\EXP{\TYP{ArrayList}\llbracket{}X\rrbracket \cap \TYP{List}\llbracket{}Y\rrbracket}}{\EXP{\TYP{ArrayList}\llbracket{}W\rrbracket}}
\ensuremath{
\begin{array}{l}
{\EXP{X \SHORTCUT{<} \TYP{Any}, Y \SHORTCUT{<} \mathbb{Z}}}\,\vdash
\\
\quad\quad\quad\quad
{\EXP{\TYP{ArrayList}\llbracket{}X\rrbracket \cap \TYP{List}\llbracket{}Y\rrbracket}}\;{\subtypeof}\;{\EXP{\TYP{ArrayList}\llbracket{}W\rrbracket}}
\end{array}
}
\\[.5em]
% \begin{align*}
% \jgsub[\EXP{X \SHORTCUT{<} \TYP{Any}, Y \SHORTCUT{<} \mathbb{Z}}]{\EXP{\TYP{ArrayList}\llbracket{}X\rrbracket \cap \TYP{List}\llbracket{}Y\rrbracket}}{\EXP{\TYP{ArrayList}\llbracket{}W\rrbracket}}
% \end{align*}
which is necessary to satisfy the Meet Rule.
All that is known about $X$ and $Y$ for this subtype check is that
they are subtypes of \TYP{Any} and \EXP{\mathbb{Z}}, respectively.
If we could somehow equate $X$ and $Y$, transferring $Y$'s tighter
bound \EXP{\mathbb{Z}} to \VAR{X}, then we could use $W = X$ to prove the assertion.
Syntactically, the types $X$ and $Y$ do not provide enough information
to prove the subtype assertion, but with additional assumptions about
their structure (namely, that their extents are equal) we can indeed
prove the assertion.  Incorporating this kind of deduction into our
subtyping judgment as described in Section~\ref{sec:constraints}
allows many more sets of overloaded function declarations
such as \VAR{tail} and \VAR{foo} as valid overloadings.



\begin{comment}
%%%%%%%%%%%%%%%%%%%%%%%%%%%%%%%%%%%%%%%%%%%%%%%%%%%%%%%%%%%%%%%%
% \noindent \textbf{--- scott has not gotten to this line ---}
% 
% Polymorphic exclusion lets us determine whether one existential type
% has the same extent (has the same monomorphic subtypes) as another.
% For example, using polymorphic exclusion, we know that
% the following existential type:
% \begin{FortressCode}
\% (1) \exists\llbracket{}X \SHORTCUT{<} \TYP{Any}, Y \SHORTCUT{<} \TYP{Foo}\,\rrbracket\, \bigl(\TYP{ArrayList}\llbracket{}X\rrbracket \cap \TYP{List}\llbracket{}Y\rrbracket\bigr) \\
\% 
\end{FortressCode}
has the same extent as:
% \begin{FortressCode}
\% (2) \exists\llbracket{}Z \SHORTCUT{<} \TYP{Foo}\rrbracket\, \TYP{ArrayList}\llbracket{}Z\rrbracket \\
\% 
\end{FortressCode}
because, for all ground types \VAR{S} and \VAR{T}, the intersection type
% \EXP{\TYP{List}\llbracket{}S\rrbracket \cap \TYP{ArrayList}\llbracket{}T\rrbracket} is equivalent to \TYP{BottomType}
% unless \VAR{S} is equivalent to \VAR{T}.
% This lets us deduce that \VAR{S} (equivalent to\VAR{T})
% is an appropriate instantiation for \VAR{Z}.
% Therefore, now the following declarations:
% \begin{FortressCode}
%  \\
% \2\2\+h\llbracket{}X \SHORTCUT{<} \TYP{Any}\rrbracket\TYP{ArrayList}\llbracket{}X\rrbracket\COLON\mathbb{Z} \\
% h\llbracket{}Y\mathord{\SHORTCUT{<}}\: \TYP{Foo}\rrbracket\TYP{List}\llbracket{}Y\rrbracket\COLON\mathbb{Z}\-
% \end{FortressCode}
% satisfy the Meet Rule thanks to the following disambiguating declaration:
% \begin{FortressCode}
%  \\
% \2\2\+h\llbracket{}Z \SHORTCUT{<} \TYP{Foo}\rrbracket\TYP{ArrayList}\llbracket{}Z\rrbracket\COLON\mathbb{Z}\-
% \end{FortressCode}
% If we incorporate this form of derivation into the subtyping rules for
% existential and universal types, we can prove that many more sets of
% overloaded declarations satisfy the overloading rules.
% 
% In this section,
% we extend the exclusion relation to existential and universal types and
% refine the subtyping rules for them.
% %
% %  need to know constraints under which a type is not bottom => need to know when two types don't exclude
% %  can get regular exclusion from not exclude
% %
% We also augment our judgments to include \emph{constraints},
% which state subtyping among types that we have inferred to hold.
% For example, because of polymorphic exclusion,
% if we infer that a given type $S$ is a subtype of two instantiations \EXP{C\llbracket\bar{T}\rrbracket} and
% \EXP{C\llbracket\bar{U}\rrbracket} of the same type constructor \VAR{C}, we can infer constraints that
% \EXP{T_{i} \equiv U_{i}} for each \EXP{T_{i} \in \bar{T}} and \EXP{U_{i} \in \bar{U}}.
% We also carry constraints that hold under the supposition
% that an intersection type is not \TYP{BottomType}.
% For example, when we decide whether two existential types (1) and (2) above have the same extent,
% if we explore the ramifications of the possibility that
% \EXP{\TYP{List}\llbracket{}S\rrbracket \cap \TYP{ArrayList}\llbracket{}T\rrbracket} is not equivalent to \TYP{BottomType},
% we need to infer and carry the constraint that \EXP{S \equiv T}.
% 
% To augment the rules in this section, we introduce the following notation:
% \begin{enumerate}
% 
% \item 
% We write $\jcnonequiv{T}{\BottomType}{\C}$
% to indicate that a type $T$ is not equivalent to \BottomType\ 
% under the environment $\Delta$ and
% the assumption that constraints $\C$ hold.
% 
% \item 
% We write  $\jnexc{T}{S}{\C}$ to indicate that
% $T$ does not exclude $S$ under the environment $\Delta$ and
% the assumption that constraints $\C$ hold.
% \end{enumerate}
% 
% % How to get all instantiations exclude (written \exc)
% % \infrule
% %  {\Delta \vdash S \nexc T|False}
% %  {\Delta \vdash S \exc T}
\end{comment}




\subsection{Exclusion with Constraints}\label{sec:constraints}
\paragraph{Constraints}
We now augment our subtyping judgment to include \emph{constraints}.
With constraint-based subtyping, we perform backward reasoning from the
desired subtyping assertion to derive constraints on types under which the
assertion can be proved; 
when those constraints are satisfied, the assertion is also satisfied. 
These constraints on types are first generated and gathered, 
and solved later when more information about the environment is known.

% For example, because of polymorphic exclusion,
% if we need to prove that a given type $S$ is a subtype of two instantiations \EXP{C\llbracket\bar{T}\rrbracket} and
% \EXP{C\llbracket\bar{U}\rrbracket} of the same type constructor \VAR{C}, we can infer the constraints 
% \EXP{T_{i} \equiv U_{i}} for each \EXP{T_{i} \in \bar{T}} and \EXP{U_{i} \in \bar{U}} as necessary conditions. If these conditions cannot immediately be solved, we can propagate them so that with more information about $\bar{T}$ and $\bar{U}$ we can prove that \EXP{S \SHORTCUT{<} C\llbracket\bar{T}\rrbracket} and \EXP{S \SHORTCUT{<} C\llbracket\bar{U}\rrbracket}.
% % More concretely, when we must prove that the two existential types from $foo$ in the previous section are equivalent, we can determine that $S \equiv T$ implies the subtyping judgment we need. We propagate the constraint that $S \equiv T$ because, if it were shown to hold, we could prove the desired subtyping judgment. The equivalence of the computed meet and candidate meet for $foo$ could thus be checked.

Our grammar for type constraints is defined as follows:\\[-.2em]
\[
\begin{array}{llll}
\C &::=& X <: S~~~~ &\\
&\mid& S <: X &\\
&\mid& X \exc S &\\
&\mid& X \not<: S~~~~ &\\
&\mid& S \not<: X &\\
&\mid& X \nexc S &\\
&\mid& \C \wedge \C\\%& \mbox{conjunction constraint}\\
&\mid& \C \vee \C\\%& \mbox{disjunction constraint}\\
&\mid& \FALSE\\%& \mbox{never satisfied}\\
&\mid& \TRUE\\%& \mbox{always satisfied}\\
\end{array}
\]
\\
A primitive constraint is either {\it positive} or {\it negative}.
A positive primitive constraint of the form \EXP{X \SHORTCUT{<} S} specifies that
a type variable $X$ is a subtype of $S$, and likewise for \EXP{S \SHORTCUT{<} X}.
The positive primitive constraint $X \exc S$ specifies that a type variable $X$ must exclude a type $S$.
To generate constraints for exclusion, we also define negative primitive constraints:
$X \not<: S$, $S \not<: X$, and $X \nexc S$.
A conjunction constraint $\C_1 \wedge \C_2$ is satisfied exactly
when both $\C_1$ and $\C_2$ are satisfied,
and a disjunction constraint $\C_1 \vee \C_2$ is satisfied exactly
when one or both of $\C_1$ and $\C_2$ are satisfied.
The constraint $\FALSE$ is never satisfied, and the constraint $\TRUE$ is always satisfied.

\paragraph{Constraint Generation}
We introduce the following syntactic judgments to generate these constraints:
\[ \jcsub{S}{T}{\C} \andalso \jcequiv{S}{T}{\C} \andalso \jexc{S}{T}{\C} \]%
Each judgment indicates that, under assumptions $\Delta$,
the respective predicate can be proved if the constraint $\C$ is
satisfied. An important point about these judgments is that the
predicates $S \subtypeof T$, $S \equiv T$, and $S \exc T$ should
be considered \emph{inputs} to our algorithmic checker, while the
constraint $\C$ should be considered its \emph{output}.

We do not give 
%in Figure~\ref{fig:jexc} 
the full semantics of
constraint generation for subtyping. Instead, we refer the reader to
the recent work by Smith and
Cartwright \cite{smith08} on which our system was based. Smith and
Cartwright provide a sound and complete algorithm for generating
constraints from the subtyping relation and an algorithm for
normalizing constraints to simplify, for example, those involving
contradictions or redundancies. We assume that constraints are
implicitly simplified in this manner. The soundness of constraint
generation entails that the predicate is a logical consequence of the
constraint; the completeness of generation entails the opposite.



%%%%%%%%%%%%%%%%%%%%% BEGIN FIGURE %%%%%%%%%%%%%%%%%%%%%%
\begin{figure*}[htbp]
\newjudge{Generating constraints for exclusion}{\jexc{T}{T}{\C}}
\\\\
  \begin{tabular}{c|c}
  %% MINIPAGE FOR LEFT COLUMN
  \begin{minipage}[t]{.45\textwidth}
%\vspace*{-3em}
\textbf{Symmetry}
\vspace*{-1em}
\infrule
  {\jexc{T}{S}{\C}}
  {\jexc{S}{T}{\C}}

\vspace*{3ex}
\textbf{Structural rules}
%\vspace*{-1em}
\infrule
  {}
  {\jexc{\BottomType}{T}{\TRUE}}

\infrule
  {\jcsub{T}{\BottomType}{\C}}
  {\jexc{\Any}{T}{\C}}

\infrule
  {|\bar{S}| \neq |\bar{T}|}
  {\jexc{(\bar{S})}{(\bar{T})}{\TRUE}}

\infrule
  {|\bar{S}|=|\bar{T}|  \andalso
   %\forall  i \in \{1, \ldots, |\bar{S}| \} . \jexc{S_i}{T_i}{\C_i}}
   \bar{\jexc{S}{T}{\C}}
   }
  {\jexc{(\bar{S})}{(\bar{T})}{\bigvee \C_i}}

\infrule
  {|\bar{T}| \neq 1}
  {\jexc{\EXP{(S \rightarrow R)}}{(\bar{T})}{\TRUE}}

\infrule
  {C \neq \Any \andalso
   |\bar{T}| \neq 1}
  {\jexc{\EXP{C\llbracket\,\bar{S}\,\rrbracket}}{(\bar{T})}{\TRUE}}

\infrule
  {C \neq \Any}
  {\jexc{\EXP{C\llbracket\,\bar{S}\,\rrbracket}}{\EXP{(T \rightarrow U)}}{\TRUE}}

\infrule
  {\jexc{S}{U}{\C} \andalso 
   \jexc{T}{U}{\C'} \\
   \jexc{S}{T}{\C''}}
  {\jexc{S \cap T}{U}{\C \vee \C' \vee \C''}}

\infrule
  {}
  {\jexc{\EXP{(S \rightarrow T)}}{\EXP{(U \rightarrow V)}}{\FALSE}}

\infrule
  {\jexc{S}{U}{\C}    \andalso 
   \jexc{T}{U}{\C'}}
  {\jexc{S \cup T}{U}{\C \wedge \C'}}

\vspace*{3ex}
\textbf{Type variables}
\infrule
  {}
  {\jexc{X}{T}{\VAR{X} \exc \VAR{T}}}

  \end{minipage}
  
  %%% END LEFT COLUMN
  &
  
  %% MINIPAGE FOR RIGHT COLUMN
  \begin{minipage}[t]{.53\textwidth}
\textbf{Constructed types}
\vspace*{-1em}
\infrule
  {\jgconstrtemplate{\EXP{C\llbracket\bar{S}\rrbracket}}{\exce}{\EXP{D\llbracket\bar{T}\rrbracket}}{\C_e} \\
   \jgconstrtemplate{\EXP{C\llbracket\bar{S}\rrbracket}}{\excc}{\EXP{D\llbracket\bar{T}\rrbracket}}{\C_c} \\
   \jgconstrtemplate{\EXP{C\llbracket\bar{S}\rrbracket}}{\exco}{\EXP{D\llbracket\bar{T}\rrbracket}}{\C_o} \\
   \jgconstrtemplate{\EXP{C\llbracket\bar{S}\rrbracket}}{\excp}{\EXP{D\llbracket\bar{T}\rrbracket}}{\C_p}}
  {\jexc{\EXP{C\llbracket\bar{S}\rrbracket}}{\EXP{D\llbracket\bar{T}\rrbracket}}{\C_e \vee \C_c \vee \C_o \vee \C_p}}

\infrule
  {\jgconstrtemplate{\EXP{C\llbracket\bar{S}\rrbracket}}{\excrx}{\EXP{D\llbracket\bar{T}\rrbracket}}{\C} \\
   \jgconstrtemplate{\EXP{D\llbracket\bar{T}\rrbracket}}{\excrx}{\EXP{C\llbracket\bar{S}\rrbracket}}{\C'}}
  {\jgconstrtemplate{\EXP{C\llbracket\bar{S}\rrbracket}}{\exc_{x}}{\EXP{D\llbracket\bar{T}\rrbracket}}{\C \vee \C'}}
\vspace{-3ex}
\hfill where $x \in \{\mathrm{e,c,o}\}$
\\%[-.2em]

\vspace*{-1em}
\infrule
  %{C = D}
  {}
  {\jgconstrtemplate{\EXP{C\llbracket\bar{S}\rrbracket}}{\excrx}{\EXP{C\llbracket\bar{T}\rrbracket}}{\FALSE}}
\vspace{-3ex}
\hfill where $x \in \{\mathrm{e,c,o}\}$
\\%[-.2em]
\vspace*{-1ex}

\infrule
  {C \neq D \andalso A = \ancestors(\EXP{C\llbracket\bar{S}\rrbracket}) \\
   \forall U \in \left(\bigcup_{N \in A} \myexcludes{N}\right). \quad \jcsub{\EXP{D\llbracket\bar{T}\rrbracket}}{U}{\C_{U}}}
  {\jgconstrtemplate{\EXP{C\llbracket\bar{S}\rrbracket}}{\excre}{\EXP{D\llbracket\bar{T}\rrbracket}}{\bigvee \C_{U}}}

\infrule
  {C \neq D \\
\forall U \in \mycomprises{\EXP{C\llbracket\bar{S}\rrbracket}}.\quad \jexc{U}{\EXP{D\llbracket\bar{T}\rrbracket}}{\C_U}
% \andalso \jgconstrtemplate{\EXP{C\llbracket\bar{S}\rrbracket}}{\not \subtypeof}{\EXP{D\llbracket\bar{T}\rrbracket}}{\C}
}
  {\jgconstrtemplate{\EXP{C\llbracket\bar{S}\rrbracket}}{\excrc}{\EXP{D\llbracket\bar{T}\rrbracket}}{\C \wedge \bigwedge \C_U}}

\infrule
  {\text{$C$ does not have a \KWD{comprises} clause}}
  {\jgconstrtemplate{\EXP{C\llbracket\bar{S}\rrbracket}}{\excrc}{\EXP{D\llbracket\bar{T}\rrbracket}}{\FALSE}}

\infrule
  {\KWD{object}\ C \quad C \neq D \quad
   \jgconstrtemplate{\EXP{C\llbracket\bar{S}\rrbracket}}{\not \subtypeof}{\EXP{D\llbracket\bar{T}\rrbracket}}{\C}}
  {\jgconstrtemplate{\EXP{C\llbracket\bar{S}\rrbracket}}{\excro}{\EXP{D\llbracket\bar{T}\rrbracket}}{\C}}
%\vspace{2ex}

\infrule
  {\neg(\KWD{object}\ C)}
  {\jgconstrtemplate{\EXP{C\llbracket\bar{S}\rrbracket}}{\excro}{\EXP{D\llbracket\bar{T}\rrbracket}}{\FALSE}}
%\vspace{2ex}

\infrule
  {\forall U \in \ancestors(\EXP{C\llbracket\bar{S}\rrbracket}).\\
   \forall V \in \ancestors(\EXP{D\llbracket\bar{T}\rrbracket}).\\
     \jgconstrtemplate{U}{\bexcp}{V}{\C_{U,V}}}
  {\jgconstrtemplate{\EXP{C\llbracket\bar{S}\rrbracket}}{\excp}{\EXP{D\llbracket\bar{T}\rrbracket}}{\bigvee \C_{U,V}}}

\infrule
  %{C = D \andalso \bar{\jcnonequiv{S}{T}{\C}}}
  {\bar{\jcnonequiv{S}{T}{\C}}}
  {\jgconstrtemplate{\EXP{C\llbracket\bar{S}\rrbracket}}{\bexcp}{\EXP{C\llbracket\bar{T}\rrbracket}}{\bigvee \C_i}}

\infrule
  {C \neq D}
  {\jgconstrtemplate{\EXP{C\llbracket\bar{S}\rrbracket}}{\bexcp}{\EXP{D\llbracket\bar{T}\rrbracket}}{\FALSE}}

  \end{minipage}
  \end{tabular}

  \caption{Generating constraints for exclusion}
  \label{fig:jexc}
\end{figure*}

The constraint generation judgment for equivalence is defined entirely by the following rule:\\[-1.5em]
\infrule
  {\jcsub{S}{T}{\C} \andalso \jcsub{T}{S}{\C'}}
  {\jgconstrtemplate{S}{\equiv}{T}{\C \wedge \C'}}
Equivalence constraint generation is complete and sound if and only if
the constraint generation for $\subtypeof$ is.

With negative constraints, we define a new judgment for ``not a subtype'' by the rule:\\[-1.5em]
\infrule
  {\jcsub{S}{T}{\C}}
  {\jgconstrtemplate{S}{\not <:}{T}{\neg\C}}

\noindent where $\neg\C$ is the negated constraint formed by applying De Morgan's laws on $\C$. Similarly, we can define ``not equivalent'' by the rule:\\[-1em]
\infrule
  {\jcequiv{S}{T}{\C}}
  {\jcnonequiv{S}{T}{\neg\C}}
These judgements are sound because constraint generation for $\subtypeof$ is complete (and complete because $\subtypeof$ is sound).




% We define the following exclusion judgment:\\[-.2em]
% \[ \jexc{S}{T}{\C} \]
% \\
% which states that, under assumptions $\Delta$,
% which reads as ``the types \VAR{S} and \VAR{T} exclude each other under the assumptions $\Delta$
% if the constraints $\C$ hold,'' in Figure~\ref{fig:jexc}.
% With these
With these additional constraints and judgments, we can now define
constraint generation for exclusion.
Figure~\ref{fig:jexc} presents our algorithm, which
formalizes our original definition of exclusion in
Section~\ref{sec:excdec}.
As before, exclusion on constructed types is
satisfied when any of the sub-relations from the previous section is
satisfied. Each sub-relation checks the conditions described before
by recursively generating and propagating constraints. Each
sub-relation except $\excrc$ depends only on other constraint
generation judgments, meaning the algorithmic checking
terminates. The $C\obb{T} \excrc D\obb{U}$ predicate recursively
checks exclusion on the comprised types of $C\obb{T}$, but due to
the acyclic nature of the class table, this process will also terminate. 
Thus, the algorithm is complete and sound if $\subtypeof$ is.


For subtyping with constraints, we essentially preserve the semantics
of \cite{smith08}, but that system lacks our notion of
exclusion. Since we need our subtyping to take advantage of
exclusion, we must add an additional rule to the judgment:\\[-1em] 

\infrule
  {\jgconstrtemplate{S}{\exc}{T}{\C} \andalso
   \jcsub{S}{U}{\C'} \andalso\jcsub{T}{U}{\C''}}
  {\jcsub{S \cap T}{U}{\C \vee \C' \vee \C''}}

\noindent This new rule states three possibilities for proving that the intersection $S \cap T$ is a subtype of $U$: $S$ and $T$ exclude (which means their intersection is a subtype of \BottomType), $S$ is a subtype of $U$, or $T$ is a subtype of $U$. If any of the constraints needed for these three judgments is satisfied, then $S \cap T$ is a subtype of $U$. Adding this rule should not affect completeness or soundness of constraint generation for $\subtypeof$ (since constraint generation for $\exc$ is complete and sound if constraint generation for $\subtypeof$ is).

\paragraph{Constraint Solving}
Solving constraints
% We will also need to have the ability to solve constraints.
% $\Delta \vdash \textit{solve}(\C)$
% This
needs to be sound but not necessarily complete.\footnote{A 
failure in solving constraints may cause us to reject some overloadings that
were actually safe, 
but not to allow overloadings that are unsafe.}
Here is a simple algorithm which is sound but not complete. (We assume that all
constraints are kept in disjunctive normal form.)
\begin{enumerate}
\item Take $\C$ which is a union of intersection constraints and
try to solve each conjunct at a time (take the first one that succeeds).
\item Split the conjunct into positive and negative parts.
\item Deduce a set of equivalences from the positive part.
\item Solve the equivalences using unification (with subtyping) to get a substitution $\phi$.
\item Check whether applying $\phi$ to $\C$ reduces $\C$ to the trivial constraint 
$\TRUE$.
\item If so return $\TYP{Some}(\phi)$, otherwise return \TYP{None}.
\end{enumerate}
To get better completeness properties,
one can iterate constraint solving if $\phi(\C)$ is not false.
However, this might converge to a fixed point instead of terminating
because the constraints that are negative or that do not imply an equivalence
are never used to generate $\phi$.


%\subsection{Polymorphic Exclusion Checking}
%\subsection{Overloading with Polymorphic Exclusion, Again}

\section{Overloading Rules Checking}
\label{sec:checking}

This section describes how to mechanically check 
the three overloading rules described in Section~\ref{sec:threerules}.
Because the rules use the notions of applicability and specificity for generic functions,
we first formalize them which in turn need formalization of
the ``arrow type of $f$'' for a generic function declaration $f$.

% For a generic function declaration \declg{f}{X}{N}{S}{T},
% its arrow type (written $\textit{arrow}(f)$) is
% the \emph{universal type} $\forall\tplist{X}{N}S \rightarrow T$.
For a generic function declaration $\hdeclg{f}{\Delta}{S}{T}$,
its arrow type (written $\textit{arrow}(f)$) is
the \emph{universal type} \mbox{$\forall\ob{\Delta}S \rightarrow T$}.
A universal type binds type parameter declarations over some type and
can be instantiated by any types valid for those type parameters.
We write $\forall\tplist{X}{N}T$ to quantify each type variable $X_i$ with bounds
$\{\bar{N_i}\}$ over the type $T$, and we use the metavariable $\sigma$ to range over universal types.
The meta-level function \OPR{FV} maps a type to all free variables contained in that type. 

%applicability of a generic function
Recall that a generic function declaration $\hdeclg{f}{\Delta}{S}{T}$ 
is applicable to a type $U$
if and only if some instance of $f$ is applicable to $U$.
A monomorphic function declaration $\decl{f}{S'}{T'}$ is applicable to a type $U$
if and only if $U <: \textit{dom}(f)$,
where $\textit{dom}(f)$ is the domain $S'$ of $S' \rightarrow T'$, the arrow type of $f$.
%We must then extend our notion of the ``arrow type of $f$'' for a generic function declaration $f$,
%but first we need to characterize the type of a generic function.
% In particular, we need a higher-level notion of universally
% quantified types separate from the language of types $W$.
This existential quantification over possible
instantiations of the domain directly corresponds to an existentially
quantified type as in \cite{bourdoncle97}. An \emph{existential type}
$\exists\tplist{X}{N}T$ also binds type parameter declarations over a type, but
unlike a universal type, it cannot be instantiated; instead, it represents
some hidden type instantiation $\bar{W}$ and the corresponding instantiated
type $\substb{W}{X}T$. Therefore, we say that the domain of the universal arrow type for $f$,
$\forall\ob{\Delta}S \rightarrow T$ (again written $\textit{dom}(f)$ as an abuse of notation),
is the existential type $\exists\ob{\Delta}{S}$.
We use the metavariable $\delta$ to range over existential types.
Note that we abbreviate both the universal type $\forall\ob{}T$ and
the existential type $\exists\ob{}T$ as simply $T$ when the meaning is clear from context.
Figure~\ref{fig:sub} presents the formal definition of applicability for a generic function $f$
to a type $T$, $f \ni T$.

%%%%%%%%%%%%%%%%%%%%% BEGIN SUBTYPING %%%%%%%%%%%%%%%%%%%%%%
\begin{figure*}[ht]
\begin{center}
  \begin{tabular}{c|c}

  %% MINIPAGE FOR LEFT COLUMN
  \begin{minipage}{.55\textwidth}

    %%% UNIVERSAL SUBTYPING RULE
    \begin{center}
      \fbox{\textbf{Universal subtyping:} \jqsub{\sigma}{\sigma}}
    \end{center}

    \infrule
      {\fresh{\bar{Z}}
         \andalso \Delta' = \Delta, \bds{Z}{N} \\
       \jgsub[\Delta']{\substb{V}{X}T}{\substb{Z}{Y}U} \andalso
           \forall i\,.\;\jgsub[\Delta']{V_i}{\substb{V}{X}\bd{M_i}}
           % \bar{\jgsub[\Delta']{V}{\substb{V}{X}\bar{M}}}
      }
      {\jqsub{\forall\obb{X <: \bd{M}}T}{\forall\obb{Y <: \bd{N}}U}}

\TODO{Perhaps $\bar{Z}$ should be defined in a ``where'' side condition.}

    %%% SYNTAX DEF
    \begin{tabularx}{\textwidth}{RcX}
    $\sigma_1 \le \sigma_2$ &
      \syndef &
      $\jqsub[\emptyset]{\sigma_1}{\sigma_2}$ \\
    % $\textit{dom}(\unitype{\arrowtype{T}{U}})$ &
    %   \syndef &
    %   $\exttype{T}$
    \end{tabularx}

    %%% DOMAIN SUBTYPING RULE
    \begin{center}
      \fbox{\textbf{Existential subtyping:} \jqsub{\delta}{\delta}}
    \end{center}
    \infrule
      {\fresh{\bar{Z}}
         \andalso \Delta' = \Delta, \bds{Z}{N} \\
       \jgsub[\Delta']{\substb{Z}{X}T}{\substb{V}{Y}U} 
          \andalso \forall i\,.\;\jgsub[\Delta']{V_i}{\substb{V}{Y}\bd{N_i}}
                   % \bar{\jgsub[\Delta']{V}{\substb{V}{Y}\bar{N}}}
          }
      {\jqsub{\exists\obb{X <: \bd{M}}T}{\exists\obb{Y <: \bd{N}}U}}
    %%% SYNTAX DEF
    \begin{tabularx}{\textwidth}{RcX}
    $\delta_1 \le \delta_2$ & \syndef & $\jqsub[\emptyset]{\delta_1}{\delta_2}$
    \end{tabularx}
  \end{minipage}

  %%% RIGHT COLUMN
  &

  %% MINIPAGE FOR RIGHT COLUMN
  \begin{minipage}{.40\textwidth}
    %%% APPLICABILITY
    \begin{center}
%      \fbox{\textbf{Applicability:} \japp{\delta}{T}~~$f \ni T$}
      \fbox{
\begin{tabular}{lc}
\textbf{Applicability:}& \japp{\delta}{T}\\
&$f \ni T$
\end{tabular}
}
    \end{center}
    \infrule
      {\jqsub{\exists\ob{}T}{\delta}}
      {\japp{\delta}{T}}
    %%% SYNTAX DEF

    \begin{tabularx}{\textwidth}{RcX}
    $\delta \ni T$ & \syndef & $\japp[\emptyset]{\delta}{T}$ \\
    $f \ni T$ & \syndef & $\japp[\emptyset]{\textit{dom}(f)}{T}$\\
    \end{tabularx}

    %%% MORE SPECIFIC
    \begin{center}
      \fbox{\textbf{Specificity:} \jms{f}{f}}
    \end{center}
    \infrule
      {\jqsub{\textit{dom}(f_1)}{\textit{dom}(f_2)}}
      {\jms{f_1}{f_2}}
    %%% SYNTAX DEF
    \begin{tabularx}{\textwidth}{RcX}
    $f_1 \ms f_2$ & \syndef & $\jms[\emptyset]{f_1}{f_2}$
    \end{tabularx}
  \end{minipage} \\
  \hline
  \end{tabular}

  %%% SECOND ROW
  \begin{tabular}{c}
  \begin{minipage}{0.95\textwidth}
%  \begin{center}
\vspace*{.5em}
    \fbox{\textbf{Existential meet:} \quad $\delta_1 \;\meet\; \delta_2$}
    \[
\begin{array}{c}
      \left(\exists\tplist{X}{M}T\right) \; \meet \;
          \left(\exists\tplist{Y}{N}U\right)
      \quad \syndef \quad
      \exists\ob{\bds{X'}{M}, \bds{Y'}{N}}(\substb{X'}{X}T \cap \substb{Y'}{Y}U)
\\[.5em]
\text{where}\quad \fresh{\bar{X'}, \bar{Y'}}
\end{array}
    \] % \vspace{-1em}
    %     %%% SYNTAX DEF
    %     \[
    %       f_1 \;\meet\; f_2
    %       \quad \syndef \quad
    %       \textit{dom}(f_1) \;\meet\; \textit{dom}(f_2)
    %     \]
%  \end{center}
\end{minipage}
  \end{tabular}
\end{center}
  \caption{Subtyping on universal and existential types, applicability and specificity on generic function declarations,
and meet of existential types}
  \label{fig:sub}
\end{figure*}
%%%%%%%%%%%%%%%%%%%%% END SUBTYPING %%%%%%%%%%%%%%%%%%%%%%


%specificity for a generic function
A generic function declaration is \emph{more specific} than another generic function declaration
if and only if the domain of the former is a subtype of the latter as presented in Figure~\ref{fig:sub}.
Figure~\ref{fig:sub} also presents the subtype relation on existential types
as originally given by Mitchell \cite{mitchell88}.
Roughly, an existential type $\delta_1 = \exists \ob{\Delta_1} T_1$ is a
subtype of another existential type $\delta_2 = \exists\ob{\Delta_2}T_2$
(written $\delta_1 \le \delta_2$)
if $T_2$ can be instantiated to a supertype of $T_1$ in the environment $\Delta_1$.


Now, we can mechanically check the No Duplicates Rule
by mechanically determining if two declarations are equally specific.
To check whether one declaration is more specific than another,
we check whether the domain of the former is a subtype of the latter.
Checking whether one existential type is a subtype of another
is described in Figure~\ref{fig:sub}.


To check the Meet Rule, for every pair of declarations $f_1, f_2 \in \D(f)$,
we must find a function declaration $f_0 \in \D(f)$ that is
applicable to a type $T$ if and only if both $f_1$ and $f_2$ are applicable to $T$.
In other words, we need to find $f_0$ that is equivalent under specificity to
the meet of $f_1$ and $f_2$.
To mechanically find the meet of two generic function declarations,
we define the \emph{computed meet} of $f_1$ and $f_2$ as
a declaration $f_\meet$, not necessarily in $\D(f)$,
such that $\textit{dom}(f_\meet) \equiv \textit{dom}(f_1) \meet \textit{dom}(f_2)$\footnote{Note that the computed meet, as defined,
is not actually unique since the return type is unspecified.
By an abuse of notation, we refer to ``the'' computed meet to mean any such computed meet.}: \begin{align*}
  \textit{dom}(f_0) &\;\le\; \textit{dom}(f_1) \meet \textit{dom}(f_2) \\
  \textit{dom}(f_1) \meet \textit{dom}(f_2) &\;\le\; \textit{dom}(f_0)
\end{align*}
Figure~\ref{fig:sub} defines the meet of two existential types $\delta_1 \meet \delta_2$,
which may require alpha renaming on the existential types' parameters.
The following lemma shows that the definition of the meet of two existential types is correct:
\begin{lemma}\label{lem:meet}
$\delta_1 \meet \delta_2$ is the meet of $\delta_1$ and $\delta_2$ under $\le$.
\end{lemma}
\begin{proof}
That $\delta_1 \meet \delta_2$ is a subtype of both $\delta_1$ and $\delta_2$ is obvious.
For any $\delta_0$, if $\bar{U}$ and $\bar{V}$ are instantiations that prove
$\delta_0$ is a subtype of $\delta_1$ and $\delta_2$, respectively,
then we can use $\bar{U},\bar{V}$ to prove that $\delta_0$ is a subtype of $\delta_1 \meet \delta_2$.
\end{proof}


We can check the Return Type Rule using the subtype relation on universal types.
Figure~\ref{fig:sub} presents the subtype relation on universal types, again,
as originally given by Mitchell \cite{mitchell88}.
Roughly, a universal type $\sigma_1 = \forall\ob{\Delta_1} T_1$ is a subtype of another universal
type $\sigma_2 = \forall\ob{\Delta_2} T_2$ (written $\sigma_1 \le \sigma_2$) if
$T_1$ can be instantiated to a subtype of $T_2$ in the environment $\Delta_2$.

The generic declarations $f_1 = \hdeclg{f}{\Delta_1}{S_1}{T_1}$ 
and $f_2 = \hdeclg{f}{\Delta_2}{S_2}{T_2}$ with $f_1 \ms f_2$ satisfy the Return Type Rule
whenever:
\renewcommand{\theequation}{\fnsymbol{equation}}
\begin{equation} \label{eq:rtr}
  \forall\ob{\Delta_1}S_1 \rightarrow T_1 \; \le \; \forall\ob{\Delta_1, \Delta_2}(S_1 \cap S_2) \rightarrow T_2
\end{equation}
To see this, suppose $W$ is a type to which $f_1$ is applicable and the instance $\decl{f_2'}{S_2'}{T_2'}$ of $f_2$
is also applicable to $W$.
Let $f_\meet = \hdeclg{f}{\Delta_1, \Delta_2}{S_1 \cap S_2}{T_2}$ be the
computed meet of $f_1$ and $f_2$;
clearly, there is an instance 
$\decl{f_\meet'}{U'}{V'}$ of $f_\meet$ 
that is applicable to $W$ with $V' = T_2'$. 
If $\textit{arrow}(f_1) \le \textit{arrow}(f_\meet)$,
then there must be an instantiation $\decl{f_1'}{S_1'}{T_1'}$ of $f_1$
with $U' <: S_1'$ and $T_1' <: V'$. Moreover, since $V'=T_2'$,
such an $f_1'$ would satisfy the Return Type Rule for $f_1$ and $f_2$.
Finally, observe that $\textit{arrow}(f_1) \le \textit{arrow}(f_\meet)$ is identical to the
condition (\ref{eq:rtr}), and so the verification of (\ref{eq:rtr}) implies
the verification of the Return Type Rule.



\subsection{Incorporating Constraints into Subtyping}
%%%%%%%%%%%%%%% BEGIN REDUCTION FIGURE %%%%%%%%%%%%%%%%
\begin{figure}[t]
\begin{tabular}{c}
\begin{minipage}{.45\textwidth}
\newjudge{Existential reduction}{\jtreds{\delta}{\delta}{\phi}}
\infrule
  {\jcequiv{T}{\BottomType}{\TRUE}}
  {\jtreds{\exttype{T}}{\BottomType}{[]}}

% \infrule
%   {\jcnonequiv{T}{\BottomType}{\C} \\
%    \jcsolve{\C}{\NONE}}
%   {\jtred{\exttype{T}}{\exttype{T}}}
% 
% \infrule
%   {\jcnonequiv{T}{\BottomType}{\C} \\
%    \jcsolve{\C}{\phi} \andalso \phi[\Delta] = \NONE}
%   {\jtred{\exttype{T}}{\exttype{T}}}

\infrule
  {\jcnonequiv{T}{\BottomType}{\C} \\
   \jcsolve{\C}{\phi} \andalso \phi[\Delta] = \Delta'}
  {\jtreds{\exttype{T}}{\exttype[\Delta']{\phi(T)}}{\phi}}


%\vspace*{-.5em}
\newjudge{Bounds substitution}{\phi[\Delta] = \Delta}
%\vspace*{-.8em}
\infrule
  {\Delta = \bds{X}{M}
      \andalso \phi(\bar{X}) = \bar{Y}, \bar{T}
      \andalso \bar{\bar{N}} = \bar{\phi^{-1}[Y,\Delta]} \\[.2em]
    \forall i. \quad \jcsub[\bds{Y}{N}]{\phi(X_i)}{\bar{\phi(M_i)}}{\TRUE}}
  {\phi[\Delta] = \bds{Y}{N}}

% \infrule
%   {\Delta = \bds{X}{M}
%       \andalso \phi(\bar{X}) = \bar{Y} \sqcup \bar{T}
%       \andalso \bar{N} = \phi^{-1}[\bar{Y},\Delta] \\
%     \exists i. \quad \jcsub[\bds{Y}{N}]{\phi(X_i)}{\bd{\phi(M_i)}}{\C_i}
%       \andalso \C_i \neq \FALSE}
%   {\phi[\Delta] = \NONE}

%\vspace*{-.5em}
\newjudge{Bounds transfer}{\phi^{-1}[X,\Delta] = \bar{T}}
%\vspace*{-.8em}
\infrule
  {\{\bar{T}\} = \{\phi(\Delta(X)) \mid X \in \textit{dom}(\Delta),\; \phi(X) = Y\}}
  {\phi^{-1}[Y,\Delta] = \textit{conjuncts}(\bigcap \bar{T})}

\vspace*{-1em}
%% CONJUNCTS DEF
\[
\begin{array}{l}
\textit{conjuncts}(T)\\
\syndef
  \begin{cases}
    \textit{conjuncts}(T_1), \textit{conjuncts}(T_2)\quad & \text{if } T = T_1 \cap T_2 \\
    T & \text{otherwise}
  \end{cases}
\end{array}
\]
\end{minipage}
\end{tabular}
  \caption{Reduction of existential types}
  \label{fig:exred}
\end{figure}
%%%%%%%%%%%%%%% END REDUCTION FIGURE %%%%%%%%%%%%%%%%

We can recover the usual subtyping judgment and define the exclusion judgment as follows:\\[1em]
\begin{tabular}{cc}
\begin{minipage}{0.23\textwidth}
\infrule
{\jcsub{S}{T}{\TRUE}}
{\jgsub{S}{T}}
\end{minipage}
&
\begin{minipage}{0.23\textwidth}
\infrule
{\jexc{S}{T}{\TRUE}}
{\jgtemplate{S}{\exc}{T}}
\end{minipage}
\\[1.5em]
\end{tabular}
\noindent which state that, under assumptions $\Delta$, if the predicate generates the constraint $\TRUE$ (or some constraint that simplifies to $\TRUE$), then that predicate is proved.

% We now use the non-equivalence judgment to make rigorous our intuition about
% how polymorphic exclusion affects the mechanical verification of the overloading rules.
To allow more overloaded functions such as \VAR{tail} and \VAR{foo} as valid overloadings,
we adjust the subtype relation to take into account the relationships between type variables:\\[1em]
%described by the constraints defined in Section~\ref{sec:constraints}:
\begin{tabular}{c}
\begin{minipage}{0.45\textwidth}
\infrule
  {\jtred{\delta}{\delta'} \andalso \jqsub{\delta'}{T}}
  {\jqsub{\delta}{T}}
\end{minipage}
\\[1.5em]
\begin{minipage}{0.45\textwidth}
\infrule
  {\jtreds{\exttype[\Delta_1]{T}}{\delta'}{\phi} \\[.3em]
   \jqsub{S}{\unitype[{\phi[\Delta_1]}]{\left(\arrowtype{\phi(T)}{\phi(U)}\right)}}}
  {\jqsub{S}{\unitype[\Delta_1]{\arrowtype{T}{U}}}}
\end{minipage}
\\[2em]
\end{tabular}
A reduction judgment on existential types $\vdash\delta\eqred\delta',\phi$
defined in Figure~\ref{fig:exred} reduces $\delta = \exttype{T}$ to
$\delta' = \exttype[\Delta']{T'}$ with the substitution $\phi$
under the assumption that $T$ is not equivalent to \BottomType; or,
if $T$ is equivalent to \BottomType, the reduced existential $\delta'$ is \BottomType.
When the substitution is unnecessary we omit it.
The $solve$ operation is like that defined in Smith and Cartwright~\cite{smith08}.
In fact, an existential type $\delta$ reduces to $\delta'$
such that $T \ni \delta$ if and only if $T \ni \delta'$; therefore,
we have $\delta' \le \delta$.
Reducing an existential type in this fashion involves the same kind of
type analysis required for type checking generalized algebraic data types
\cite{simonet07,jones09}.



As an example, consider the following reduction:\\[.8em]
\begin{tabular}{l}
$\vdash$ \EXP{\exists\llbracket{}X \SHORTCUT{<} \TYP{Any}, Y \SHORTCUT{<} \mathbb{Z}\rrbracket\bigl(\TYP{ArrayList}\llbracket{}X\rrbracket \cap \TYP{List}\llbracket{}Y\rrbracket\bigr)}\\[.3em]
\andalso~~~~~~$\eqred$ \EXP{\exists\llbracket{}W \SHORTCUT{<} \mathbb{Z}\rrbracket\TYP{ArrayList}\llbracket{}W\rrbracket}
\end{tabular}
\\[.8em]
with substitution $[W/X,\, W/Y]$. We first check under what constraints $\C$
the intersection \EXP{\TYP{ArrayList}\llbracket{}X\rrbracket \cap \TYP{List}\llbracket{}Y\rrbracket} is not equivalent to \BottomType:
with polymorphic exclusion (specifically, the absence of multiple instantiation inheritance)
we know that \EXP{\TYP{ArrayList}\llbracket{}X\rrbracket} excludes \EXP{\TYP{List}\llbracket{}Y\rrbracket},
which makes their intersection equivalent
to \BottomType, unless $X \equiv Y$ is true. Solving the constraint $X
\equiv Y$ yields a type substitution like $\phi = [W/X,\, W/Y]$. The
judgment $\phi[\Delta] = \Delta'$ lets us construct reduced bounds
from $\phi$ and the original bounds $\Delta$. To do this, we first
partition $\phi(\bar{X})$ into a list of type variables $\bar{Y}$ and a
list of other types $\bar{T}$. In our example, $\phi(X, Y) = W$ gets
partitioned into $W$ and $\emptyset$. Then we need to construct a new
bound $\phi^{-1}[Y_i, \Delta]$ for each $Y_i$ in $\bar{Y}$ by
conjoining the bounds for every type variable in $\phi^{-1}(Y)$. In
our example, $X$ and $Y$ map to $W$, so the bounds for $W$ are \EXP{\lbrace\TYP{Any}, \mathbb{Z}\rbrace},
which we take as the new bounds environment $\Delta'$. We must
ensure that the substitution does not produce invalid bounds, so we
check $\bar{\Delta' \vdash \phi(X) <: \bar{\phi(M)}}$. In our example,
\EXP{W \SHORTCUT{<} \lbrace\mathbb{Z}, \TYP{Any}\rbrace} easily proves that \EXP{W \SHORTCUT{<} \mathbb{Z}} and \EXP{W \SHORTCUT{<} \TYP{Any}}. With
$\Delta'$ the reduced existential type is simply
$\exists\ob{\Delta'}\phi(T)$, where $T$ is the constituent type of the
original existential. In our example, the final, reduced existential
type is \EXP{\exists\llbracket{}W \SHORTCUT{<} \mathbb{Z}\rrbracket\TYP{ArrayList}\llbracket{}W\rrbracket}.
%
Once we have augmented the subtyping relation with existential
reduction, we can finally check that the declarations $\D(\VAR{foo})$ from
Section~\ref{sec:problems} satisfy the Meet Rule.


Similiarly, to check that the declarations $\D(\VAR{tail})$ from Section~\ref{sec:problems}
satisfy the Return Type Rule, we need to show the following:\\[.8em]
\begin{tabular}{l}
$\vdash$
\EXP{\forall\llbracket{}X\SHORTCUT{<}\TYP{Any}\rrbracket\TYP{ArrayList}\llbracket{}X\rrbracket \rightarrow \TYP{ArrayList}\llbracket{}X\rrbracket}\\
$\le$
\EXP{\forall\llbracket{}X\SHORTCUT{<}\TYP{Any},Y\SHORTCUT{<}\TYP{Any}\rrbracket\bigl(\TYP{ArrayList}\llbracket{}X\rrbracket \cap \TYP{List}\llbracket{}Y\rrbracket\bigr)}\\
\hspace*{11.2em}
\EXP{\rightarrow \TYP{List}\llbracket{}Y\rrbracket}
\end{tabular}
\\[.8em]
%EXISTS[\X <: Any, Y <: ZZ\](ArrayList[\X\] CAP List[\Y\])
%\andalso~~~$\eqred$ \EXP{\exists\llbracket{}W \SHORTCUT{<} \mathbb{Z}\rrbracket\TYP{ArrayList}\llbracket{}W\rrbracket}
By the adjusted subtype relation in this section and the rules in Figure~\ref{fig:exred},
we can show the following:\\[.8em]
\begin{tabular}{l}
$\vdash$
\EXP{\exists\llbracket{}X\SHORTCUT{<}\TYP{Any}, Y\SHORTCUT{<}\TYP{Any}\rrbracket\TYP{ArrayList}\llbracket{}X\rrbracket \cap \TYP{List}\llbracket{}Y\rrbracket}\\[.3em]
\andalso~~~$\eqred$ \EXP{\exists\llbracket{}W \SHORTCUT{<} \TYP{Any}\rrbracket\TYP{ArrayList}\llbracket{}W\rrbracket}
\end{tabular}
\\[.8em]
with substitution $[W/X, W/Y]$.  Because the substitution satisfies the following:\\[.8em]
\begin{tabular}{l}
$\vdash$
\EXP{\forall\llbracket{}X\SHORTCUT{<}\TYP{Any}\rrbracket\TYP{ArrayList}\llbracket{}X\rrbracket \rightarrow \TYP{ArrayList}\llbracket{}X\rrbracket}\\
$\le$
\EXP{\forall\llbracket{}X\SHORTCUT{<}\TYP{Any}\rrbracket\TYP{ArrayList}\llbracket{}X\rrbracket \rightarrow \TYP{List}\llbracket{}X\rrbracket}\\
\end{tabular}
\\[.8em]
we can verify that the declarations $\D(\VAR{tail})$ satisfy the Return Type Rule.


% A similar analysis shows that if an instance of a universal arrow has
% the domain \TYP{BottomType}, then it is irrelevant for the purposes of
% guaranteeing Progress. Therefore we can use our reduction rule for
% existential types to aid in the verification of the Return Type Rule
% by augmenting the subtype rules for universal arrows:
% \\[-1.5em]
% \infrule
%   {\jtreds{\exttype[\Delta_1]{T}}{\delta'}{\phi} \andalso
%    \jqsub{S}{\unitype[{\phi[\Delta_1]}]{\left(\arrowtype{\phi(T)}{\phi(U)}\right)}}}
%   {\jqsub{S}{\unitype[\Delta_1]{\arrowtype{T}{U}}}}
%  % {\Delta \vdash S <: reduce(\EXP{\forall\llbracket\bar{X} \SHORTCUT{<} \bar{M}\rrbracket{}S \hbox{\char'134}\VAR{to}\, T})}
%  % {\Delta \vdash S <: \EXP{\forall\llbracket\bar{X} \SHORTCUT{<} \bar{M}\rrbracket{}S \hbox{\char'134}\VAR{to}\, T}}
% It is easy to see from the previous existential reduction instance that this subtype judgment can be proven,
% thus allowing us to verify the Return Type Rule for the function $\D(tail)$ from Section~4.

%% Because this subtype judgment holds for the previous existential
%% reduction instance, we can verify the Return Type Rule for the
%% function $\D(tail)$ from Section~4.
%% %% It is easy to see from the previous existential reduction instance
%% %% that this subtype judgment holds, thus allowing us to verify
%% %% the Return Type Rule for the function $\D(tail)$ from Section~4. 



\section{Discussion}\label{sec:discussion}
\subsection{Type Inference}

Before describing a system for ensuring progress and preservation,
it is important to discuss some implications of these conditions on
type inference in a programming language.
The application of a function declaration to a type requires
instantiation of the type parameters of the declaration.
In most programming languages with parametric polymorphism,
a type inference mechanism automatically instantiates type parameters
based on the types of the arguments and the enclosing context.
But note that our progress and preservation conditions do not require
that the type parameters of the function declaration 
that is (dynamically) most specific 
of those applicable to the ilks of the argument values 
are the same as the type parameters
of the function declaration that is (statically) most specific 
of those applicable to the static types of the argument expressions.
Thus, the results of static type inference do not tell us how to
instantiate the type parameters of a most specific function
declaration at run time.  
In the Fortress programming language,
type inference is performed statically, 
and the results of that inference 
are passed to the run-time system 
to ensure that run-time type inference at a function call is sound.
The rules for overloaded function declarations
introduced in Section~\ref{sec:rules} ensure that
the declaration of the dynamically most applicable function declaration,
when instantiated with whatever we infer at run time,
is more specific than the declaration of the statically most applicable
function declaration, instantiated with what was inferred at compile time.
%% For the dynamically most applicable function declaration,
%% the instantiated declaration with whatever instantiation we infer
%% must be more specific than the instantiated declaration of the
%% statically most applicable function declaration with what was inferred
%% at compile time.


%% For the dynamically most applicable function declaration,
%% the instantiated declaration with whatever instantiation we infer
%% must be more specific than the instantiated declaration of the
%% statically most applicable function declaration with what was inferred
%% at compile time.
%% The rules on the overloaded function declarations
%% introduced in Section~\ref{sec:rules} ensure that
%% dynamically inferred types satisfy this requirement.



Aside from this caveat, our system for checking overloaded declarations 
is largely independent of how a specific type inference engine would choose 
instantiations\footnote{Type inference manifests itself as the choice of instantiation of type variables in existential and universal subtyping; specifically, $\bar{V}$ in the inference rules for subtyping in Figure~\ref{fig:sub}. Mitchell \cite{mitchell88} first showed how type inference interacts with polymorphic subtyping.}. Thus we do not discuss the specific features of a type inference
system further in this paper.

% Section 2.1 last para: We should comment that this definition of
% well-formedness agrees with that of the old system. If we view
% monomorphic function declarations as polymorphic declarations with no
% type parameters, then for each monomorphic declaration there is
% exactly one instantiation of it (with an empty vector of type
% arguments).  Then Preservation specifically says that if (monomorphic)
% declaration f U: V is applicable to T, then for the most specific
% declaration that is applicable to T, f' U': V', we have that V' <: V.
% Thus if the polymorphic well-formedness is satisfied, so is the
% monomorphic well-formedness.

\subsection{Modularity}

To demonstrate the modularity of our design,
we present a lightweight modeling of program modules,
and show how applying our rules to each module separately
suffices to guarantee the soundness of the entire program.
In our model, 
a program is a \emph{module}, which may be either \emph{simple} or \emph{compound}.
A \emph{simple module} consists of 
(i) a class table and 
(ii) a collection of function declarations.
That is, a simple module is just a program as described in the rest of this paper.
It is well-formed if it satisfies the well-formedness conditions of a whole program,
as described in previous sections.

A \emph{compound module} combines multiple modules, 
possibly renaming members (i.e., classes and functions) of its constituents.
More precisely, a compound module is a collection of \emph{filters},
where a filter consists of a module 
and a complete mapping from names of members of the module to names.
The name of a member that is not renamed is simply mapped to itself.

The semantics of a compound module is the semantics of the simple module
that results from recursively \emph{flattening} the compound module as follows:
\begin{itemize}

\item
Flattening a simple module simply yields the same module.

\item Flattening a compound module \VAR{C}
consisting of filters (module/mapping pairs) $(c_1, m_1), \ldots, (c_N, m_N)$
yields a simple module whose class table and collection of function declarations
is the concatenation of the class tables and collections of function declarations
of $s_1, \ldots, s_N$, where $s_i$ is the simple module resulting from
first flattening $c_i$ and then renaming all members
of the resulting simple module according to the mapping $m_i$.

\end{itemize}
A compound module is well-formed if its flattened version is well-formed.
This requirement implies that the type hierarchies in each constituent component 
are consistent with the type hierarchy in the flattened version.

We can now use this model of modularity to see 
that we can separately compile and combine modules.

First consider the case of a collection of modules with no overlapping function names
such that each module has been checked separately 
to ensure that the overloaded functions within them satisfy the overloading rules.
Because the type hierarchies of each constituent of a compound module 
must be consistent with that of the compound module, 
all overloaded functions in the resulting compound module
also obey the overloading rules.

Now consider the case of a collection of separately checked modules 
with some overlapping function names.
When overloaded functions from separate modules are combined, 
there are three rules that might be violated
by the resulting overloaded definitions: 
(1) the Meet Rule, (2) the No Duplicates Rule, (3) the Return Type Rule.
If the Meet Rule is violated, 
the programmer need only define yet another module to combine 
that defines the missing meets of the various overloaded definitions.
If the No Duplicates Rule or the Return Type Rule is violated, 
the programmer can fix the problem by renaming functions 
from one or more combined components to avoid the clash; 
the programmer can then define another component 
with more overloadings of the same function name 
that dispatch to the various renamed functions in the manner the programmer intends.

Consider the following example:\footnote{Suggested by 
an anonymous reviewer of a previous version of this paper.}
Suppose we have a type Number in module \VAR{A}, with a
function:
\begin{FortressCode}
{\tt ~~}\+\VAR{add} \COLONOP (\TYP{Number}, \TYP{Number}) \rightarrow \TYP{Number}\-
\end{FortressCode}
Suppose we have the type and function:
\begin{FortressCode}
{\tt ~~}\+\TYP{BigNum} \SHORTCUT{<} \TYP{Number} \\
  \VAR{add} \COLONOP (\TYP{BigNum}, \TYP{BigNum}) \rightarrow \TYP{BigNum}\-
\end{FortressCode}
in module \VAR{B} and the type and function:
\begin{FortressCode}
{\tt ~~}\+\TYP{Rational} \SHORTCUT{<} \TYP{Number} \\
  \VAR{add} \COLONOP (\TYP{Rational}, \TYP{Rational}) \rightarrow \TYP{Rational}\-
\end{FortressCode}
in module \VAR{C}.

Each of modules \VAR{B} and \VAR{C} satisfy the No Duplicates and Meet rules.
Now, suppose we define two compound modules \VAR{D} and \VAR{E}, 
each of which combines modules \VAR{B} and \VAR{C} without renaming \VAR{add}.
In each of \VAR{D} and \VAR{E}, 
we have an ambiguity 
in dispatching calls to \VAR{add} with types \EXP{(\TYP{BigNum}, \TYP{Rational})} or \EXP{(\TYP{Rational}, \TYP{BigNum})}.
Our rules require adding two declarations in each of \VAR{D} and \VAR{E}
to resolve these ambiguities.

Now let us suppose we wish to combine \VAR{D} and \VAR{E} into a compound component \VAR{F}.
Without renaming, this combination would violate the No Duplicates Rule;
each of \VAR{D} and \VAR{E} has an implementation of \EXP{\VAR{add}(\TYP{Bignum}, \TYP{Rational})}. 
To resolve this conflict, 
the program can rename \VAR{add} from both \VAR{D} and \VAR{E}, 
and define a new \VAR{add} in \VAR{F}. 
This new definition could dispatch to either of the renamed functions from \VAR{D} or \VAR{E}, 
or it could do something entirely different, 
depending on the programmer's intent.


\section{Related Work}\label{sec:related}
% * multiple dispatch
%   * fortress
%   * CLOS
%   * multijava
%   * cecil
% * type classes
%   * wadler 89
%   * qualified types (mpj)
%   * concepts (siek)
%   * inability to add ad-hoc overloaded functions
% * GADTs
%   * GADT inference (spj)
%   * HMG(X) (pottier)
%   * with OOP (russo)
%
\subsection{Overloading and dynamic dispatch.} 

\TODO{Add discussion of Castagna et al.\ here?}

Primarily, our system
strictly extends our previous effort \cite{allen07} with parametric polymorphism;
all previous properties and results remain intact. The inclusion of parametric
functions and types represents a shift in the research literature on overloading
and multiple dynamic dispatch.

Millstein and Chambers \cite{millstein02,millstein03} devised the language
Dubious to study overloaded functions with symmetric multiple dynamic dispatch
(\emph{multimethods}), and with Clifton and Leavens they developed MultiJava
\cite{multijava}, an extension of Java with Dubious' semantics for multimethods.
In \cite{feml}, Lee and Chambers presented F(E\textsc{ml}), a language with
classes, symmetric multiple dispatch, and parameterized modules, but without
parametricity at the level of functions or types. None of these systems support
polymorphic functions or types. F(E\textsc{ml})'s parameterized modules
(\emph{functors}) constitute a form of parametricity but they cannot be implicitly
applied; the functions defined therein cannot be \emph{overloaded} with those
defined in other functors. For a more detailed comparison of modularity and
dispatch between our system and these, we refer to the related work section of
our previous paper \cite{allen07}.

% I took out discussion of modularity here; it's charged and unnecessary
% in order to distinguish our work. EricAllen 7/15/2011
Overloading and multiple dispatch in the context of polymorphism 
has previously been studied by Bourdoncle and Merz \cite{bourdoncle97}. 
Their system, ML$_\le$, integrates parametric polymorphism, 
class-based object orientation, and multimethods,
but lacks multiple inheritance. 
Each multimethod (overloaded set) requires a unique specification (principal type), 
which prevents overloaded functions defined on disjoint domains; 
% the domains of the multimethod branches must partition the specification domain, 
% which eliminates subtype-based specialization;
and link-time checks are performed to ensure that multimethods are fully
implemented across modules. 
On the other hand, ML$_\le$ allows variance annotations on type constructors---% 
something we attribute to future work.

Litvinov~\cite{litvinov98} developed a type system for the Cecil language,
which supports bounded parametric polymorphism and multimethods.
Because Cecil has a type-erasure semantics, 
statically checked parametric polymorphism has no effect on run-time dispatch.

\subsection{Type classes.} Wadler and Blott \cite{wadler89} introduced
\emph{type class} as a means to specify and implement overloaded
functions like equality and arithmetic operators in Haskell. Other authors
have translated type classes to languages besides Haskell \cite{dreyer07,siek05,wehr07}.
Type classes encapsulate overloaded function declarations, with separate
\emph{instances} that define the behavior of those functions (called \emph{class methods})
for any particular type schema. Parametric polymorphism is then augmented to
express type class constraints, providing a way to quantify a type variable --- and
thus a function definition --- over all types that instantiate the type class. 

% In his thesis \cite{jonesbook} Jones generalized Haskell's underlying type
% system as \emph{qualified types}, in which the satisfaction of type predicates
% must be proved with constructed \emph{evidence}. Qualified type systems (such
% as Haskell) exhibit the \emph{principal types} property necessary for full
% Damas-Milner style type inference \cite{dm82,jonesbook}; our system conservatively
% assumes only \emph{local type inference} \cite{pierce00} --- implicit type
% instantiation for polymorphic function calls.

In systems with type classes, overloaded functions must be contained in some
type class, and their signatures must vary in exactly the same structural
position. This uniformity is necessary for an overloaded function call to
admit a principal type; with a principal type for some function call's context,
the type checker can determine the constraints under which a correct overloaded
definition will be found. Because of this requirement, type classes are ill-suited
for fixed, \emph{ad hoc} sets of overloaded functions like:
\begin{FortressCode}
{\tt ~~~~}\+\VAR{println}(\ultrathin)\COLON (\ultrathin) = \VAR{println}(\hbox{\rm\usefont{T1}{ptm}{m}{n}``\verythin''}) \\
    \VAR{println}(s\COLON \TYP{String})\COLON (\ultrathin) = \ldots\-
\end{FortressCode}
or functions lacking uniform variance in the domain and range\footnote{With the
\emph{multi-parameter type class} extension, one could define functions as these.
A reference to the method \mono{bar}, however, would require an explicit type
annotation like \mono{:: Int -> Bool} to apply to an \mono{Int}.} like:
\begin{FortressCode}
{\tt ~~~~}\+\VAR{bar}(x\COLON \mathbb{Z})\COLON \TYP{Boolean} = (x = 0) \\
    \VAR{bar}(x\COLON \TYP{Boolean})\COLON \mathbb{Z} =\; \KWD{if} x \KWD{then} 1 \KWD{else} 2 \KWD{end} \\
    \VAR{bar}(x\COLON \TYP{String})\COLON \TYP{String} = x\-
\end{FortressCode}
With type classes one can write overloaded functions with identical domain
types. Such behavior is consistent with the \emph{static}, \emph{type-based}
dispatch of Haskell, but it would lead to irreconcilable ambiguity in the
\emph{dynamic}, \emph{value-based} dispatch of our system.
%% In Appendix~\ref{app:haskell}, we present a further discussion of how our overloading resolution differs from that of Haskell and how our system might translate to that language, thereby addressing an existing inconsistency in modern type class extensions.

A broader interpretation of Wadler and Blott's \cite{wadler89} sees type
classes as program abstractions that quotient the space of ad-hoc polymorphism
into the much smaller space of class methods. Indeed, Wadler and Blott's title
suggests that the unrestricted space of ad-hoc polymorphism should be tamed,
whereas our work embraces the possible expressivity achieved from mixing ad-hoc
and parametric polymorphism by specifying the requisites for determinism and type safety.


\section{Conclusion and Discussion}\label{sec:conclusion}
We have shown how to statically ensure safety of overloaded, polymorphic functions while imposing relatively minimal restrictions solely on function definition sites. We provide rules on definitions that can be checked modularly, irrespective of call sites, and we show how to mechanically verify that a program satisfies these rules. The type analysis required for implementing these checks involves subtyping on universal and existential types, which adds complexity not required for similar checks on monomorphic functions. We have defined an object-oriented language to explain our system of static checks, and we have implemented them as part of the open-source Fortress compiler \cite{Fortress}.

Further, we show that in order to check many ``natural'' overloaded functions with our system in the context of a generic, object-oriented language, richer type relations must be available to programmers---the subtyping relation prevalent among such languages does not afford enough type analysis alone. We have thus introduced an explicit, nominal exclusion relation to check safety of more interesting overloaded functions.

Variance annotations have proven to be a convenient and expressive addition to languages based on nominal subtyping \cite{bourdoncle97,kennedy07,scala}. They add additional complexity to polymorphic exclusion checking, so we leave them to future work.


\bibliographystyle{plain}
% The bibliography should be embedded for final submission.
\bibliography{paper}
% \begin{thebibliography}{}
% \softraggedright
% 
% \input{biblio.tex}
% 
% \end{thebibliography}

% \appendix
% \section{Application to Haskell}
% \label{app:haskell}
% \input{haskell}

\end{document}
