\section{Overloading Rules Checking}
\label{sec:checking}

This section describes how to mechanically check 
the three overloading rules described in Section~\ref{sec:threerules}.
Because the rules use the notions of applicability and specificity for generic functions,
we first formalize them which in turn need formalization of
the ``arrow type of $f$'' for a generic function declaration $f$.

% For a generic function declaration \declg{f}{X}{N}{S}{T},
% its arrow type (written $\textit{arrow}(f)$) is
% the \emph{universal type} $\forall\tplist{X}{N}S \rightarrow T$.
For a generic function declaration $\hdeclg{f}{\Delta}{S}{T}$,
its arrow type (written $\textit{arrow}(f)$) is
the \emph{universal type} \mbox{$\forall\ob{\Delta}S \rightarrow T$}.
A universal type binds type parameter declarations over some type and
can be instantiated by any types valid for those type parameters.
We write $\forall\tplist{X}{N}T$ to quantify each type variable $X_i$ with bounds
$\{\bar{N_i}\}$ over the type $T$, and we use the metavariable $\sigma$ to range over universal types.
The meta-level function \OPR{FV} maps a type to all free variables contained in that type. 

%applicability of a generic function
Recall that a generic function declaration $\hdeclg{f}{\Delta}{S}{T}$ 
is applicable to a type $U$
if and only if some instance of $f$ is applicable to $U$.
A monomorphic function declaration $\decl{f}{S'}{T'}$ is applicable to a type $U$
if and only if $U <: \textit{dom}(f)$,
where $\textit{dom}(f)$ is the domain $S'$ of $S' \rightarrow T'$, the arrow type of $f$.
%We must then extend our notion of the ``arrow type of $f$'' for a generic function declaration $f$,
%but first we need to characterize the type of a generic function.
% In particular, we need a higher-level notion of universally
% quantified types separate from the language of types $W$.
This existential quantification over possible
instantiations of the domain directly corresponds to an existentially
quantified type as in \cite{bourdoncle97}. An \emph{existential type}
$\exists\tplist{X}{N}T$ also binds type parameter declarations over a type, but
unlike a universal type, it cannot be instantiated; instead, it represents
some hidden type instantiation $\bar{W}$ and the corresponding instantiated
type $\substb{W}{X}T$. Therefore, we say that the domain of the universal arrow type for $f$,
$\forall\ob{\Delta}S \rightarrow T$ (again written $\textit{dom}(f)$ as an abuse of notation),
is the existential type $\exists\ob{\Delta}{S}$.
We use the metavariable $\delta$ to range over existential types.
Note that we abbreviate both the universal type $\forall\ob{}T$ and
the existential type $\exists\ob{}T$ as simply $T$ when the meaning is clear from context.
Figure~\ref{fig:sub} presents the formal definition of applicability for a generic function $f$
to a type $T$, $f \ni T$.

%%%%%%%%%%%%%%%%%%%%% BEGIN SUBTYPING %%%%%%%%%%%%%%%%%%%%%%
\begin{figure*}[ht]
\begin{center}
  \begin{tabular}{c|c}

  %% MINIPAGE FOR LEFT COLUMN
  \begin{minipage}{.55\textwidth}

    %%% UNIVERSAL SUBTYPING RULE
    \begin{center}
      \fbox{\textbf{Universal subtyping:} \jqsub{\sigma}{\sigma}}
    \end{center}

    \infrule
      {\fresh{\bar{Z}}
         \andalso \Delta' = \Delta, \bds{Z}{N} \\
       \jgsub[\Delta']{\substb{V}{X}T}{\substb{Z}{Y}U} \andalso
           \forall i\,.\;\jgsub[\Delta']{V_i}{\substb{V}{X}\bd{M_i}}
           % \bar{\jgsub[\Delta']{V}{\substb{V}{X}\bar{M}}}
      }
      {\jqsub{\forall\obb{X <: \bd{M}}T}{\forall\obb{Y <: \bd{N}}U}}

\TODO{Perhaps $\bar{Z}$ should be defined in a ``where'' side condition.}

    %%% SYNTAX DEF
    \begin{tabularx}{\textwidth}{RcX}
    $\sigma_1 \le \sigma_2$ &
      \syndef &
      $\jqsub[\emptyset]{\sigma_1}{\sigma_2}$ \\
    % $\textit{dom}(\unitype{\arrowtype{T}{U}})$ &
    %   \syndef &
    %   $\exttype{T}$
    \end{tabularx}

    %%% DOMAIN SUBTYPING RULE
    \begin{center}
      \fbox{\textbf{Existential subtyping:} \jqsub{\delta}{\delta}}
    \end{center}
    \infrule
      {\fresh{\bar{Z}}
         \andalso \Delta' = \Delta, \bds{Z}{N} \\
       \jgsub[\Delta']{\substb{Z}{X}T}{\substb{V}{Y}U} 
          \andalso \forall i\,.\;\jgsub[\Delta']{V_i}{\substb{V}{Y}\bd{N_i}}
                   % \bar{\jgsub[\Delta']{V}{\substb{V}{Y}\bar{N}}}
          }
      {\jqsub{\exists\obb{X <: \bd{M}}T}{\exists\obb{Y <: \bd{N}}U}}
    %%% SYNTAX DEF
    \begin{tabularx}{\textwidth}{RcX}
    $\delta_1 \le \delta_2$ & \syndef & $\jqsub[\emptyset]{\delta_1}{\delta_2}$
    \end{tabularx}
  \end{minipage}

  %%% RIGHT COLUMN
  &

  %% MINIPAGE FOR RIGHT COLUMN
  \begin{minipage}{.40\textwidth}
    %%% APPLICABILITY
    \begin{center}
%      \fbox{\textbf{Applicability:} \japp{\delta}{T}~~$f \ni T$}
      \fbox{
\begin{tabular}{lc}
\textbf{Applicability:}& \japp{\delta}{T}\\
&$f \ni T$
\end{tabular}
}
    \end{center}
    \infrule
      {\jqsub{\exists\ob{}T}{\delta}}
      {\japp{\delta}{T}}
    %%% SYNTAX DEF

    \begin{tabularx}{\textwidth}{RcX}
    $\delta \ni T$ & \syndef & $\japp[\emptyset]{\delta}{T}$ \\
    $f \ni T$ & \syndef & $\japp[\emptyset]{\textit{dom}(f)}{T}$\\
    \end{tabularx}

    %%% MORE SPECIFIC
    \begin{center}
      \fbox{\textbf{Specificity:} \jms{f}{f}}
    \end{center}
    \infrule
      {\jqsub{\textit{dom}(f_1)}{\textit{dom}(f_2)}}
      {\jms{f_1}{f_2}}
    %%% SYNTAX DEF
    \begin{tabularx}{\textwidth}{RcX}
    $f_1 \ms f_2$ & \syndef & $\jms[\emptyset]{f_1}{f_2}$
    \end{tabularx}
  \end{minipage} \\
  \hline
  \end{tabular}

  %%% SECOND ROW
  \begin{tabular}{c}
  \begin{minipage}{0.95\textwidth}
%  \begin{center}
\vspace*{.5em}
    \fbox{\textbf{Existential meet:} \quad $\delta_1 \;\meet\; \delta_2$}
    \[
\begin{array}{c}
      \left(\exists\tplist{X}{M}T\right) \; \meet \;
          \left(\exists\tplist{Y}{N}U\right)
      \quad \syndef \quad
      \exists\ob{\bds{X'}{M}, \bds{Y'}{N}}(\substb{X'}{X}T \cap \substb{Y'}{Y}U)
\\[.5em]
\text{where}\quad \fresh{\bar{X'}, \bar{Y'}}
\end{array}
    \] % \vspace{-1em}
    %     %%% SYNTAX DEF
    %     \[
    %       f_1 \;\meet\; f_2
    %       \quad \syndef \quad
    %       \textit{dom}(f_1) \;\meet\; \textit{dom}(f_2)
    %     \]
%  \end{center}
\end{minipage}
  \end{tabular}
\end{center}
  \caption{Subtyping on universal and existential types, applicability and specificity on generic function declarations,
and meet of existential types}
  \label{fig:sub}
\end{figure*}
%%%%%%%%%%%%%%%%%%%%% END SUBTYPING %%%%%%%%%%%%%%%%%%%%%%


%specificity for a generic function
A generic function declaration is \emph{more specific} than another generic function declaration
if and only if the domain of the former is a subtype of the latter as presented in Figure~\ref{fig:sub}.
Figure~\ref{fig:sub} also presents the subtype relation on existential types
as originally given by Mitchell \cite{mitchell88}.
Roughly, an existential type $\delta_1 = \exists \ob{\Delta_1} T_1$ is a
subtype of another existential type $\delta_2 = \exists\ob{\Delta_2}T_2$
(written $\delta_1 \le \delta_2$)
if $T_2$ can be instantiated to a supertype of $T_1$ in the environment $\Delta_1$.


Now, we can mechanically check the No Duplicates Rule
by mechanically determining if two declarations are equally specific.
To check whether one declaration is more specific than another,
we check whether the domain of the former is a subtype of the latter.
Checking whether one existential type is a subtype of another
is described in Figure~\ref{fig:sub}.


To check the Meet Rule, for every pair of declarations $f_1, f_2 \in \D(f)$,
we must find a function declaration $f_0 \in \D(f)$ that is
applicable to a type $T$ if and only if both $f_1$ and $f_2$ are applicable to $T$.
In other words, we need to find $f_0$ that is equivalent under specificity to
the meet of $f_1$ and $f_2$.
To mechanically find the meet of two generic function declarations,
we define the \emph{computed meet} of $f_1$ and $f_2$ as
a declaration $f_\meet$, not necessarily in $\D(f)$,
such that $\textit{dom}(f_\meet) \equiv \textit{dom}(f_1) \meet \textit{dom}(f_2)$\footnote{Note that the computed meet, as defined,
is not actually unique since the return type is unspecified.
By an abuse of notation, we refer to ``the'' computed meet to mean any such computed meet.}: \begin{align*}
  \textit{dom}(f_0) &\;\le\; \textit{dom}(f_1) \meet \textit{dom}(f_2) \\
  \textit{dom}(f_1) \meet \textit{dom}(f_2) &\;\le\; \textit{dom}(f_0)
\end{align*}
Figure~\ref{fig:sub} defines the meet of two existential types $\delta_1 \meet \delta_2$,
which may require alpha renaming on the existential types' parameters.
The following lemma shows that the definition of the meet of two existential types is correct:
\begin{lemma}\label{lem:meet}
$\delta_1 \meet \delta_2$ is the meet of $\delta_1$ and $\delta_2$ under $\le$.
\end{lemma}
\begin{proof}
That $\delta_1 \meet \delta_2$ is a subtype of both $\delta_1$ and $\delta_2$ is obvious.
For any $\delta_0$, if $\bar{U}$ and $\bar{V}$ are instantiations that prove
$\delta_0$ is a subtype of $\delta_1$ and $\delta_2$, respectively,
then we can use $\bar{U},\bar{V}$ to prove that $\delta_0$ is a subtype of $\delta_1 \meet \delta_2$.
\end{proof}


We can check the Return Type Rule using the subtype relation on universal types.
Figure~\ref{fig:sub} presents the subtype relation on universal types, again,
as originally given by Mitchell \cite{mitchell88}.
Roughly, a universal type $\sigma_1 = \forall\ob{\Delta_1} T_1$ is a subtype of another universal
type $\sigma_2 = \forall\ob{\Delta_2} T_2$ (written $\sigma_1 \le \sigma_2$) if
$T_1$ can be instantiated to a subtype of $T_2$ in the environment $\Delta_2$.

The generic declarations $f_1 = \hdeclg{f}{\Delta_1}{S_1}{T_1}$ 
and $f_2 = \hdeclg{f}{\Delta_2}{S_2}{T_2}$ with $f_1 \ms f_2$ satisfy the Return Type Rule
whenever:
\renewcommand{\theequation}{\fnsymbol{equation}}
\begin{equation} \label{eq:rtr}
  \forall\ob{\Delta_1}S_1 \rightarrow T_1 \; \le \; \forall\ob{\Delta_1, \Delta_2}(S_1 \cap S_2) \rightarrow T_2
\end{equation}
To see this, suppose $W$ is a type to which $f_1$ is applicable and the instance $\decl{f_2'}{S_2'}{T_2'}$ of $f_2$
is also applicable to $W$.
Let $f_\meet = \hdeclg{f}{\Delta_1, \Delta_2}{S_1 \cap S_2}{T_2}$ be the
computed meet of $f_1$ and $f_2$;
clearly, there is an instance 
$\decl{f_\meet'}{U'}{V'}$ of $f_\meet$ 
that is applicable to $W$ with $V' = T_2'$. 
If $\textit{arrow}(f_1) \le \textit{arrow}(f_\meet)$,
then there must be an instantiation $\decl{f_1'}{S_1'}{T_1'}$ of $f_1$
with $U' <: S_1'$ and $T_1' <: V'$. Moreover, since $V'=T_2'$,
such an $f_1'$ would satisfy the Return Type Rule for $f_1$ and $f_2$.
Finally, observe that $\textit{arrow}(f_1) \le \textit{arrow}(f_\meet)$ is identical to the
condition (\ref{eq:rtr}), and so the verification of (\ref{eq:rtr}) implies
the verification of the Return Type Rule.

