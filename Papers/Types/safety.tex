\begin{lemma}[Progress]
\label{lem:progress}
If some declaration in $\D(f)$ is applicable to $W$
then there is a unique most specific declaration $f_W \in \D(f)$ that is applicable to $W$.
\end{lemma}
\begin{proof}
Our proof strategy for satisfying the Progress condition
makes use of the old idea from Castagna \emph{et al.} \cite{castagna95} that
for each name $f$ the set of function declarations $\D(f)$ should form a meet
semilattice under the \emph{specificity} order defined in
Section~\ref{terms}.
If any declaration of name $f$ is applicable to a type $W$,
then the set $\D_W(f) \subseteq \D(f)$
of all declarations named $f$ and applicable to $W$
also forms a nonempty meet semilattice under specificity. Therefore
$\D_W(f)$ must have a least element.
\end{proof}

\begin{lemma}[Preservation]
\label{lem:preservation}
If $\decl{f}{S}{T}$ is an instance of some declaration 
and $\decl{f}{S}{T}$ is applicable to $W$, 
then there exists some instance $\decl{f_W}{U}{V}$ of $f_W$ such that
$\decl{f_W}{U}{V}$ is applicable to $W$ and  $V \extends T$.
\end{lemma}
\begin{proof}
Note that subtyping is preserved under class table extension,
so if $\D(f)$ satisfies the No Duplicates Rule and the Meet Rule
with respect to the class table $\T$ then it satisfies them
with respect to $\T'$ for any $\T' \ctext \T$.
Therefore, we can be certain that adding more types will not
invalidate the Progress guarantee.
Just as in our discussion of the Meet Rule, the fact that subtyping is preserved
under class table extension makes sure that the property that $\D(f)$ satisfies
the Return Type Rule is preserved under class table extension. Therefore, the
Return Type Rule and the rules from the last section are sufficient to ensure safety.
\end{proof}

\begin{theorem}[Overloading Resolution Safety]
\label{thm:safety}
There exists always a unique function to call at run time among a collection of overloaded function declarations.
%There are no ambiguous calls at run time due to a collection of overloaded function declarations.
\end{theorem}
\begin{proof}
Lemma~\ref{lem:progress} ensures that we never get stuck resolving an application;
every call to a function declaration is unambiguous.
Lemma~\ref{lem:preservation} ensures that the static type of an application
is a supertype of the ilk of each value produced by the application at run time,
provided that the ilks for the argument values in function applications 
are always subtypes of the static types of the argument expressions
(as they are in any language with a sound type system).
\end{proof}
Note that as a monomorphic function declaration is a special case of a generic function declaration,
where the sequence of the type parameters is empty,
these conditions apply also for monomorphic function declarations.

