\documentclass[10pt]{sigplanconf}
\usepackage{amsmath,graphicx,url,color,alltt,fortify,verbatim,bcprules,tabularx,theorem}
\advance \textheight by 4pt

% make a big red TODO label
\newcommand{\TODO}[1]{\textbf{\emph{\textcolor{red}{TODO}}}: \textsf{\footnotesize #1}}
% \newcommand{\TODO}[1]{}

\newcommand{\kwd}[1]{\mathtt{#1}}

\newcommand{\emptyseq}{\bullet}
\newcommand{\whennonempty}[2]{{\def\nonemptytempA{#1}\def\nonemptytempB{}\ifx\nonemptytempA\nonemptytempB\else#2\fi}}
\newcommand{\turnstile}{\vdash}
%newcommand
\newcommand{\ms}{\preceq}
\renewcommand{\bar}{\overline}
\newcommand{\meet}{\wedge}
\newcommand{\C}{\mathcal{C}}
\newcommand{\quoted}[1]{\begin{quote}#1\end{quote}}
\newcommand{\exc}{\mathrel{\lozenge}}
\newcommand{\nexc}{\mathrel{\hbox to 0pt{$\mskip -1.4mu\not$\hss}\lozenge}}
\newcommand{\smalllozenge}{\vcenter{\hbox{\scalebox{.5}{$\lozenge$}}}}
\newcommand{\normallozenge}{\vcenter{\hbox{$\lozenge$}}}
% \newcommand{\altlozenge}{\ooalign{\hfil$\normallozenge$\hfil\cr\hfil$\smalllozenge$\hfil}}
\newcommand{\altlozenge}{\ooalign{\hfil$\vcenter{\hbox{$\lozenge$}}$\hfil\cr\hfil$\cdot$\hfil}}
\newcommand{\bexc}{\mathrel{\altlozenge}}
\newcommand{\bexcp}{\mathrel{\altlozenge}_\textrm{m}}
\newcommand{\bnexc}{\mathrel{\hbox to 0pt{$\mskip -1.4mu\not$\hss}\altlozenge}}

\newcommand{\countof}{\hbox{\tt\#}}
\newcommand{\apply}{\hbox{\tt\char'100}}
\newcommand{\comprises}{\mathrel{\mathord{\equiv}\mathord{\bigcup}}}

\newcommand{\fresh}[1]{\textit{fresh}({#1})}
\newcommand{\freeVar}[1]{\textit{freeVars({#1})}}

\newcommand{\dontcare}{\_\!\_}
\newcommand{\excr}{\triangleright}
\newcommand{\excl}{\triangleleft}
\newcommand{\excre}{\excr_\textrm{x}}
\newcommand{\excle}{\excl_\textrm{x}}
\newcommand{\excrc}{\excr_\textrm{c}}
\newcommand{\exclc}{\excl_\textrm{c}}
\newcommand{\excro}{\excr_\textrm{o}}
\newcommand{\exclo}{\excl_\textrm{o}}
% \newcommand{\excrp}{\excr_\textrm{m}}
% \newcommand{\exclp}{\excl_\textrm{m}}
\newcommand{\excrx}{\excr_*}
\newcommand{\exclx}{\excl_*}
\newcommand{\excx}{\exc_*}

\newcommand{\exce}{\exc_\textrm{x}}
\newcommand{\excc}{\exc_\textrm{c}}
\newcommand{\exco}{\exc_\textrm{o}}
\newcommand{\excp}{\exc_\textrm{m}}

\newcommand{\propop}{\ensuremath{\mathrel{\ast}}}

\newcommand{\ancestors}{\textit{ancestors}}
\newcommand{\ancexcludes}{\textit{excludes}^*}
\newcommand{\myexcludes}[1]{{#1}.\textit{excludes}}
\newcommand{\mycomprises}[1]{{#1}.\textit{comprises}}
\newcommand{\myextends}[1]{{#1}.\textit{extends}}

\newcommand{\extends}{\ensuremath{<:}}
\newcommand{\subtypeof}{\ensuremath{<:}}
\newcommand{\nsubtypeof}{\not \subtypeof}
\newcommand{\supertypeof}{\ensuremath{:>}}
\newcommand{\leinner}{\ensuremath{\lesssim}}

\newcommand{\alphaequiv}{\ensuremath{\stackrel{\alpha}{\sim}}}
\newcommand{\cequiv}{\ensuremath{\sim}}

\newcommand{\arrowtype}[2]{\mbox{\ensuremath{({#1} \rightarrow {#2})}}}
\newcommand{\tuple}[1]{\ensuremath{(#1)}}
\newcommand{\tupleb}[1]{\ensuremath{(\bar{#1})}}
\newcommand{\bigtuple}[1]{\ensuremath{\big(#1\big)}}
\newcommand{\bigtupleb}[1]{\ensuremath{\big(\bar{#1}\big)}}

\newcommand{\uniontype}[2]{\mbox{\ensuremath{({#1} \cup {#2})}}}
\newcommand{\intersectiontype}[2]{\mbox{\ensuremath{({#1} \cap {#2})}}}

\newcommand{\dom}{\ensuremath{\mathit{dom}}}
\newcommand{\arrow}{\ensuremath{\mathit{arrow}}}
\newcommand{\FV}{\ensuremath{\mathit{FV}}}

% indented code block
\newenvironment{ttquote}%
{\begin{quote}\begin{alltt}}
{\end{alltt}\end{quote}}

\newcommand{\verythinmathspace}{\mskip0.5\thinmuskip}

\newcommand{\set}[1]{\ensuremath{\{{#1}\}}}
\newcommand{\setb}[1]{\ensuremath{\{\bar{#1}\}}}
\newcommand{\bigset}[1]{\ensuremath{\big\{{#1}\big\}}}
\newcommand{\bigsetb}[1]{\ensuremath{\big\{\bar{#1}\big\}}}

% put in oxford brackets
\newcommand{\ob}[1]{\ensuremath{\llbracket\verythinmathspace {#1} \verythinmathspace\rrbracket}}
\newcommand{\bigob}[1]{\ensuremath{\big\llbracket\verythinmathspace {#1} \verythinmathspace\big\rrbracket}}
% put in oxford brackets and an overbar
\newcommand{\obb}[1]{\ensuremath{\llbracket\verythinmathspace \bar{#1} \verythinmathspace\rrbracket}}
\newcommand{\bigobb}[1]{\ensuremath{\big\llbracket\verythinmathspace \bar{#1} \verythinmathspace\big\rrbracket}}
% make a type param bound with the given name
\newcommand{\bd}[1]{\ensuremath{\{{#1}\}}}
\newcommand{\bigbd}[1]{\ensuremath{\big\{{#1}\big\}}}
\newcommand{\bdb}[1]{\ensuremath{\{\bar{#1}\}}}
\newcommand{\bigbdb}[1]{\ensuremath{\big\{\bar{#1}\big\}}}
% syntactic definition
\newcommand{\syndef}{\ensuremath{\overset{\mathrm{def}}{=}}}
% make a substitution
\newcommand{\subst}[2]{\ensuremath{[#1/#2]}}
% make a substitution with bars
\newcommand{\substb}[2]{\ensuremath{[\bar{#1}/\bar{#2}]}}
% type parameter list with bounds and oxford brackets
\newcommand{\tplist}[2]{\ensuremath{\ob{\bds{#1}{#2}}}}
% monomorphic fn decl
\newcommand{\decl}[3]{\mbox{\ensuremath{{#1}\,{#2}\!:\!{#3}}}}
% a generic function declaration 
\newcommand{\declg}[5]{\mbox{\ensuremath{#1 \tplist{#2}{#3}\, #4\!:\!#5}}}
\newcommand{\hdeclg}[4]{\mbox{\ensuremath{#1 \ob{#2}\, #3\!:\!#4}}}
% a class table T
\newcommand{\T}{\ensuremath{\mathcal{T}}}
% class table extension
\newcommand{\ctext}{\ensuremath{\supseteq}}
% a declaration set D
\newcommand{\D}{\ensuremath{\mathcal{D}}}
% a declaration set restricted to a function name
\newcommand{\Df}[1][f]{\D_{\!#1}}
% existential type
\newcommand{\existstype}[2]{\ensuremath{\exists\bigob{#1}{#2}}}
\newcommand{\existstypeb}[2]{\ensuremath{\exists\bigobb{#1}{#2}}}
% universal type
\newcommand{\foralltype}[2]{\ensuremath{\forall\bigob{#1}{#2}}}
\newcommand{\foralltypeb}[2]{\ensuremath{\forall\bigobb{#1}{#2}}}
% reduced existential type
\newcommand{\reduce}[1]{\ensuremath{{#1}_r}}

%%%%% Any and Bottom %%%%

\newcommand{\Any}{\TYP{Any}}
\newcommand{\Bottom}{\TYP{Bottom}}

\newcommand{\FALSE}{\textrm{false}}
\newcommand{\TRUE}{\textrm{true}}

\newcommand{\NONE}{\bullet}

\newcommand{\eqred}{\overset{\equiv}{\longrightarrow}}

%%%%%%% JUDGMENTS %%%%%%%%

%%% NEW SYNTACTIC JUDGMENT
%\newcommand{\newjudge}[2]{\fbox{\textbf{#1:} \quad \ensuremath{#2}}}
\newcommand{\newjudge}[2]{\hbox{{#1:} \quad \fbox{\ensuremath{#2}}}}

% non constrained judgements
\newcommand{\jgtemplate}[4][\Delta]{\ensuremath{\whennonempty{#1}{{#1}\,}\turnstile\,{#2}\;{#3}\;{#4}}}
\newcommand{\jgbtemplate}[4][\Delta]{\ensuremath{\whennonempty{#1}{{#1}\,}\turnstile\,\bar{{#2}\;{#3}\;{#4}}}}
\newcommand{\jgTHREEtemplate}[8]{\ensuremath{\whennonempty{#1}{{#1}\,}\turnstile\,{#2}\;{#3}\;{#4}, {#5}\;{#3}\;{#6}, {#7}\;{#3}\;{#8}}}

\newcommand{\jgshorttemplate}[3][\Delta]{\ensuremath{\whennonempty{#1}{{#1}\,}\turnstile\,{#2}\;{#3}}}
\newcommand{\jgbshorttemplate}[3][\Delta]{\ensuremath{\whennonempty{#1}{{#1}\,}\turnstile\,\bar{{#2}\;{#3}}}}

% ground subtyping
\newcommand{\jgsub}[3][\Delta]{\jgtemplate[#1]{#2}{\subtypeof}{#3}}
\newcommand{\jgnequiv}[3][\Delta]{\jgtemplate[#1]{#2}{\not \equiv}{#3}}

% evaluation
\newcommand{\jevalstep}[3][\Delta]{\jgtemplate[#1]{#2}{\longrightarrow}{#3}}
\newcommand{\jevalstepTWO}[5][\Delta]{\ensuremath{\whennonempty{#1}{{#1}\,}\turnstile\,{#2}\;{\longrightarrow}\;{#3}, {#4}\;{\longrightarrow}\;{#5}}}
\newcommand{\jbevalstep}[3][\Delta]{\jgbtemplate[#1]{#2}{\longrightarrow}{#3}}
\newcommand{\jevalstar}[3][\Delta]{\jgtemplate[#1]{#2}{\longrightarrow^*}{#3}}

% typing
\newcommand{\jtype}[3][\Delta;\Gamma]{\jgtemplate[#1]{#2}{\mathrel{:}}{#3}}
\newcommand{\jbtype}[3][\Delta;\Gamma]{\jgbtemplate[#1]{#2}{\mathrel{:}}{#3}}

%subtyping
\newcommand{\jsubtype}[3][\Delta]{\jgtemplate[#1]{#2}{\subtypeof}{#3}}
\newcommand{\jbsubtype}[3][\Delta]{\jgbtemplate[#1]{#2}{\subtypeof}{#3}}
\newcommand{\jnotsubtype}[3][\Delta]{\jgtemplate[#1]{#2}{\not\subtypeof}{#3}}
\newcommand{\jequivtype}[3][\Delta]{\jgtemplate[#1]{#2}{\equiv}{#3}}
\newcommand{\jbequivtype}[3][\Delta]{\jgbtemplate[#1]{#2}{\equiv}{#3}}
\newcommand{\jnotequivtype}[3][\Delta]{\jgtemplate[#1]{#2}{\not\equiv}{#3}}

% well-formed types
\newcommand{\jwftype}[2][\Delta]{\jgshorttemplate[#1]{#2}{\mathsf{ok}}}
\newcommand{\jbwftype}[2][\Delta]{\jgbshorttemplate[#1]{#2}{\mathsf{ok}}}

\newcommand{\jwfdecl}[2][\Delta]{\jgshorttemplate[#1]{#2}{\mathsf{ok}}}
\newcommand{\jbwfdecl}[2][\Delta]{\jgbshorttemplate[#1]{#2}{\mathsf{ok}}}

\newcommand{\jwfmeth}[2][\Delta;\Gamma]{\jgshorttemplate[#1]{#2}{\mathsf{ok}}}
\newcommand{\jbwfmeth}[2][\Delta;\Gamma]{\jgbshorttemplate[#1]{#2}{\mathsf{ok}}}


% subtyping on quantified types
\newcommand{\jle}[3][\Delta]{\jgtemplate[#1]{#2}{\le}{#3}}
\newcommand{\jleinner}[3][\Delta]{\jgtemplate[#1]{#2}{\leinner}{#3}}

% constrained judgments
\newcommand{\jgconstrtemplate}[5][\Delta]{\ensuremath{\whennonempty{#1}{{#1}\,}\turnstile\,{#2}\;{#3}\;{#4}\,\Leftarrow\,{#5}}}
% ground subtyping with constraints
\newcommand{\jsub}[4][\Delta]{\jgconstrtemplate[#1]{#2}{\subtypeof}{#3}{#4}}
% not subtype
\newcommand{\jnsub}[4][\Delta]{\jgconstrtemplate[#1]{#2}{\not \subtypeof}{#3}{#4}}
% type exclusion
\newcommand{\jexc}[4][\Delta]{\jgconstrtemplate[#1]{#2}{\exc}{#3}{#4}}
% type non-exclusion
\newcommand{\jnexc}[4][\Delta]{\jgconstrtemplate[#1]{#2}{\nexc}{#3}{#4}}
% equivalence
\newcommand{\jequiv}[4][\Delta]{\jgconstrtemplate[#1]{#2}{\equiv}{#3}{#4}}
% nonequivalence
\newcommand{\jnequiv}[4][\Delta]{\jgconstrtemplate[#1]{#2}{\not\equiv}{#3}{#4}}


% contrapositive judgements
\newcommand{\jgcontratemplate}[5][\Delta]{\ensuremath{\whennonempty{#1}{{#1}\,}\turnstile\,{#2}\;{#3}\;{#4}\,\Rightarrow\,{#5}}}
% ground subtyping with constraints
\newcommand{\jcsub}[4][\Delta]{\jgcontratemplate[#1]{#2}{\subtypeof}{#3}{#4}}
% not subtype
\newcommand{\jcnsub}[4][\Delta]{\jgcontratemplate[#1]{#2}{\not \subtypeof}{#3}{#4}}
% type exclusion
\newcommand{\jcexc}[4][\Delta]{\jgcontratemplate[#1]{#2}{\exc}{#3}{#4}}
% type non-exclusion
\newcommand{\jcnexc}[4][\Delta]{\jgcontratemplate[#1]{#2}{\nexc}{#3}{#4}}
% equivalence
\newcommand{\jcequiv}[4][\Delta]{\jgcontratemplate[#1]{#2}{\equiv}{#3}{#4}}
% nonequivalence
\newcommand{\jcnequiv}[4][\Delta]{\jgcontratemplate[#1]{#2}{\not\equiv}{#3}{#4}}


% applicability of a domain or fndecl to a type
\newcommand{\japp}[3][\Delta]{\jgtemplate[#1]{#2}{\ni}{#3}}
% specificity between fndecls
\newcommand{\jms}[3][\Delta]{\jgtemplate[#1]{#2}{\ms}{#3}}

% constraints
% convert a bound environment into a constraint
\newcommand{\toConstraint}[2]{\ensuremath{\textit{toConstraint}({#1})\,=\,{#2}}}
% convert a constraint into a bound environment
\newcommand{\toBounds}[2]{\ensuremath{\textit{toBounds}({#1})\,=\,{#2}}}

% apply substitution to constraint
\newcommand{\japply}[4][\Delta]{\ensuremath{\whennonempty{#1}{{#1}\,}\turnstile\,\textit{apply}({#2}, {#3})\,=\,{#4}}}
% solve constraint to get a substitution and the residual constraints
\newcommand{\jsolve}[4][\Delta]{\ensuremath{\whennonempty{#1}{{#1}\,}\turnstile\,\textit{unify}({#2})\,=\,{#3}\,,\;{#4}}}



% type reduction
\newcommand{\jtred}[2]{\ensuremath{\Delta \turnstile\,{#1} \eqred {#2}}}
\newcommand{\jtreds}[3]{\ensuremath{\Delta \turnstile\,{#1} \eqred {#2}\,,\;{#3}}}


% for tabularx environments to have a right-aligned, stretched col
\newcolumntype{R}{>{\raggedleft\arraybackslash}X}%

\theorembodyfont{\rm}
\newtheorem{lemma}{Lemma}
\newtheorem{theorem}{Theorem}
% Our proofs are more like proof sketches!! EricAllen 7/15/2011
\newenvironment{proof}{\noindent \textbf{Proof:} }{\hfill $\Box$}
\newenvironment{psketch}{\noindent \textbf{Proof sketch:} }{\hfill $\Box$}

\begin{document}

\conferenceinfo{OOPSLA '11}{October 22--27, 2011, Portland, Oregon, USA.}
\CopyrightYear{2011}
\copyrightdata{978-1-4503-0940-0/11/10}

\titlebanner{draft}        % These are ignored unless
\preprintfooter{draft}     % 'preprint' option specified.

\title{Welterweight Fortress DRAFT}
\subtitle{}
\authorinfo{David Chase}{Oracle Labs}{david.r.chase@oracle.com}
\authorinfo{Justin Hilburn}{Oracle Labs}{justin.hilburn@oracle.com}
\authorinfo{Victor Luchangco}{Oracle Labs}{victor.luchangco@oracle.com}
\authorinfo{Karl Naden}{Oracle Labs}{karl.naden@oracle.com}
\authorinfo{Sukyoung Ryu}{KAIST}{sryu.cs@kaist.ac.kr}
\authorinfo{Guy L. Steele Jr.}{Oracle Labs}{guy.steele@oracle.com}
\authorinfo{John Tristan}{Oracle Labs}{jean.baptiste.tristan@oracle.com}

\makeatletter
\def \@maketitle {%
 \begin{center}
 \@settitlebanner
 \let \thanks = \titlenote
 \noindent \LARGE \bfseries \@titletext \par
 %\vskip 6pt
 %\noindent \Large \@subtitletext \par
 \vskip 6pt
   \noindent \@setauthor{9pc}{i}{\@false}\hspace{1.5pc}%
             \@setauthor{9pc}{ii}{\@false}\hspace{1.5pc}%
             \@setauthor{10pc}{iii}{\@false}\hspace{1.5pc}%
             \@setauthor{9pc}{iv}{\@true}\par
\vspace{12pt plus 2pt}
 \noindent \@setauthor{9pc}{v}{\@false}\hspace{1.5pc}%
           \@setauthor{9pc}{vi}{\@false}\hspace{1.5pc}%
           \@setauthor{11pc}{vii}{\@false}\par
\vspace{10pt plus 2pt}
 \end{center}}
\makeatother
\maketitle


\begin{abstract}
%
\begin{abstract}

The Fortress programming language integrates traditional mathematical
notation into an object-oriented framework based on traits with
multiple inheritance, overloading (of both methods and functions)
resolved by symmetric dynamic dispatch, static types, and separately
compiled modules.  One innovation is
\emph{functional methods}, which (like conventional ``dotted methods'')
are declared within traits and may be inherited, but are invoked by
ordinary function calls (or mathematical operator syntax) rather
than conventional ``dotted method calls,'' and therefore compete
in overloading resolution with ordinary function declarations.
A component/API system governs visibility of traits, objects, and
functions, and allows separate compilation of components.

A longstanding problem with multiple inheritance is what to do when
methods inherited from several parents conflict.  Many approaches have
been explored in the literature; most fail to obey the
intuitively desirable requirement that the function or method invoked
be the uniquely most specific one that is both accessible and
applicable.  Fortress requires that the signatures in every overload
set form a meet-bounded lattice; therefore it is impossible for any
function or method call to be ambiguous.  This idea goes back nearly
two decades, but Fortress appears to be the first programming language
to adopt and statically enforce it.  Because this rule guarantees
confluence, it enables a distributed implementation of dispatching
that allows selective export and selective optimization.

We exhibit a source-to-source rewrite from a source language
(a stripped-down version of Fortress) to a related target language
that is simpler than the Java\texttrademark\ programming language and is readily
supported by the Java Virtual Machine.  The demonstrated rewriting is
a practical basis for separate compilation and is easily extended to
explicitly type-parameterized methods and functions.

\end{abstract}






Fortress~\cite{fortress}

\end{abstract}

\category{D.3.3}{Programming Languages}{Language Constructs and Features---classes and objects, inheritance, modules, packages, polymorphism}

\terms{Languages}

\keywords{object-oriented programming, multiple dispatch, 
symmetric dispatch, multiple inheritance, overloading, ilks, run-time types, static types,
components, modularity, meet rule,
methods, multimethods, separate compilation, Fortress}

\section{Introduction}
\label{sec:introduction}
%A key feature of object-oriented languages is \emph{dynamic dispatch}: 
there may be multiple definitions of a function (or method) with the same name---%
we say the function is \emph{overloaded}---%
and a call to a function of that name is resolved
based on the ``run-time types''---we use the term \emph{ilks}---of the arguments. 
With \emph{single dispatch}, 
a particular argument is designated as the \emph{receiver}, 
and the call is resolved only with respect to that argument.
With \emph{multiple dispatch}, 
the run-time types of  all arguments to a call are used to resolve the call.
\emph{Symmetric multiple dispatch} is a special case of multiple dispatch 
in which all arguments are considered equally when resolving a call.

Multiple dispatch provides great expressivity.
In particular, 
mathematical operators such as $+$ and $\leq$ and $\cup$
and especially $\cdot$ and $\times$
have different definitions depending on the types of the arguments
to an application of the operator
(even the number of arguments may vary between calls); 
in a language with multiple dispatch, 
it is natural to define these operators as overloaded functions. 
Similarly, 
many binary operations on collections such as \VAR{append} and \VAR{zip} 
have different definitions 
depending on the types of both arguments. 
\TODO{Add (reference to) argument for symmetric multiple dispatch?}

% \TODO{Do we want something like the following paragraph here?}
% In this paper,
% we present rules for defining overloaded functions 
% to ensure type soundness and non-ambiguity of function calls
% under symmetric multiple dispatch
% in an object-oriented language 
% that supports parametric polymorphism and multiple inheritance.
% \TODO{If we do, trim text on next page.}

% \TODO{Alternative is to have a shorter intro, 
% which mostly mimics the abstract, but with a bit more elaboration,
% and I would probably leave the discussion of our prior work till later.
% We may want to mention Fortress early as a context for this work.
% Then have a long ``background'' section 
% containing the discussion starting from Castagna to Bourdoncle and Merz,
% and including the discussion of our prior work and former thoughts.}

% To preserve type soundness and avoid ambiguous function calls 
% while incorporating multiple dispatch 
% into an object-oriented language with a static semantics, 
% the sets of valid overloaded definitions must be restricted.
% For example, to avoid ambiguous function calls,
% we must ensure that for every call site 
% (knowing only the static types of the arguments),
% there exists a unique ``best'' function to dispatch to at run time.\footnote{
% In languages with static overloading, 
% such as Scala, C\#, and the Java\texttrademark\ programming language 
% \cite{scala,CSharpSpec,JavaSpec}, 
% it is possible to simply reject ambiguous call sites of overloaded functions.
% However, as Millstein and Chambers have observed, 
% it is impossible to statically forbid ambiguity 
% in the presence of multiple dynamic dispatch 
% without imposing constraints at the definition sites of overloaded functions
% \cite{millstein02,millstein03}.
% \TODO{Is this true for asymmetric multiple dispatch?}}

In an object-oriented language with symmetric multiple dispatch,
some restrictions must be placed on overloaded function definitions
to guarantee type soundness 
and avoid ambiguous function calls.
% \cite{castagna95,millstein02,millstein03}.
For example, 
consider the following overloaded function definitions:
\small
\begin{FortressCode}
{\tt ~~}\+f(b\COLONOP{}B, a\COLONOP{}A)\COLON \mathbb{Z} = 1 \\
  f(a\COLONOP{}A, b\COLONOP{}B)\COLON \mathbb{Z} = 2\-
\end{FortressCode}
\normalsize
If $A$ is a subtype of $B$ (we write this as \EXP{A \SHORTCUT{<} B}),
to which of these definitions do we dispatch 
when $f$ is called with two arguments of type $A$? 
% Note that the ambiguity is inherent in these definitions:
% there is a real question as to what behavior the programmer intended
% in this case.  

Castagna \textit{et al.} \cite{castagna95} address this problem 
in the context of a type system 
without parametric polymorphism or multiple inheritance
by requiring every pair of overloaded function definitions 
to satisfy the following properties:
(\emph{i}) whenever the parameter type of one 
is a subtype of the parameter type of the other, 
the return type of the first
must also be a subtype of the return type of the second; 
and 
(\emph{ii}) whenever the parameter types of the two definitions 
have a common lower bound (i.e., nontrivial subtype), 
there is a unique definition for the same function 
whose parameter type is the greatest lower bound 
of the parameter types of the two definitions.
Thus, for the example above, 
% the solution of Castagna {\it et~al}.\ is to require the programmer to
the programmer must provide a third definition
to satisfy the latter property:
 \small
\begin{FortressCode}
{\tt ~~}\+f(a\COLONOP{}A,a'\COLONOP{}A)\COLON \mathbb{Z} = \ldots\-
\end{FortressCode}
 \normalsize

This latter property is equivalent to requiring
that the definitions for each overloaded function form a meet semilattice 
partially ordered by the subtype relation on their parameter types.
We call this partial order the \emph{more specific than} relation,\!\footnote{%
Despite its name,
this relation, like the subtype relation, is reflexive: 
two function definitions with the same parameter type 
are each more specific than the other.
In that case, we say the definitions are equally specific.}
and we call the property the \emph{meet rule}.
We call the first property above the \emph{return type rule} or \emph{subtype rule}.

In this paper, 
we give variants of these rules for ensuring safe overloaded functions 
in a language that supports symmetric multiple dispatch, 
multiple inheritance, and parametric polymorphism 
(that is, generic types \emph{and} generic functions).
We prove that these rules guarantee type soundness
and that there are no ambiguous calls at run time 
(see Section~\ref{sec:safety}).
We do this by extending our earlier rules 
for a core of the Fortress programming language 
that did not support generics \cite{allen07,Fortress},
for which we proved the analogous theorems.



The type system considered by Castagna \emph{et al.} 
assumed knowledge of the entire type hierarchy 
(to determine whether two types have a common subtype), 
and the type hierarchy was assumed to be a meet semilattice 
(to ensure that any two types have a greatest lower bound).
In previous work~\cite{allen07},
we applied variants of the meet and return type rules
to a simplified version of the Fortress programming language~\cite{Fortress}, 
which supports multiple inheritance 
and does not require that types have expressible meets 
(i.e., the types that can be expressed in the language 
need not form a meet semilattice).
We showed that we could check these rules in a modular way, 
so that the type hierarchy could be extended safely by new modules
without rechecking old modules.

Because the type hierarchy defined by a module may be extended,
and because Fortress supports multiple inheritance,
two types may have a common nontrivial subtype 
even if no declared type extends them both.
Thus,
for any pair of overloaded function definitions with incomparable parameter types
(i.e., neither definition is more specific than the other),
the meet rule requires some other definition to resolve the potential ambiguity.
Because explicit intersection types cannot be expressed in Fortress, 
it is not always possible to provide such a function definition.
However, 
Fortress defines an \emph{exclusion relation} on types, 
such that types related by exclusion must have no common nontrivial subtypes,
and thus definitions with such types as parameter types 
need not be disambiguated.

In this paper, we extend our prior rules 
to handle parametric polymorphism, 
where both types and functions may be parameterized by type variables, 
which Fortress also supports.
To do so, 
it is helpful to have an interpretation for parametric types 
and type-parametric functions.

One way to think about a parametric type such as \EXP{\TYP{List}\llbracket{}T\rrbracket}
(a list with elements of type \VAR{T}---type parameter lists 
in Fortress are delimited by white square brackets) 
is that it represents an infinite set of ground types 
\EXP{\TYP{List}\llbracket\TYP{Object}\rrbracket} (lists of objects),
\EXP{\TYP{List}\llbracket\TYP{String}\rrbracket} (lists of strings), 
\EXP{\TYP{List}\llbracket\mathbb{Z}\rrbracket} (lists of integers), 
and so on.
An actual type checker must have rules 
for working with uninstantiated (non-ground) parametric types, 
but for many purposes this model of ``an infinite set of ground types'' 
is adequate for explanatory purposes.
Not so, however, for type-parametric functions.  

For some time during the development of Fortress, 
one of us (Steele) pushed for an interpretation of type-parametric functions
analogous to the one above for parametric types;
that is, 
that the type-parametric function definition\footnote{%
The first pair of white square brackets delimits the declaration of a type parameter \VAR{T},
but the other pairs of white brackets indicate 
that this type variable \VAR{T} is the static argument to the parametric type \TYP{List}.}
\small
\begin{FortressCode}
{\tt ~~}\+\VAR{append}\llbracket{}T\rrbracket\bigl(x\COLON \TYP{List}\llbracket{}T\rrbracket, y\COLON \TYP{List}\llbracket{}T\rrbracket\bigr)\COLON \TYP{List}\llbracket{}T\rrbracket = e\-
\end{FortressCode}
\normalsize
should be understood as if it stood for an infinite set of monomorphic definitions:
\small
\begin{FortressCode}
{\tt ~~}\+\VAR{append}\bigl(x\COLON \TYP{List}\llbracket\TYP{Object}\rrbracket, y\COLON \TYP{List}\llbracket\TYP{Object}\rrbracket\bigr)\COLON \TYP{List}\llbracket\TYP{Object}\rrbracket = e \\
  \VAR{append}\bigl(x\COLON \TYP{List}\llbracket\TYP{String}\rrbracket, y\COLON \TYP{List}\llbracket\TYP{String}\rrbracket\bigr)\COLON \TYP{List}\llbracket\TYP{String}\rrbracket = e \\
  \VAR{append}\bigl(x\COLON \TYP{List}\llbracket\mathbb{Z}\rrbracket, y\COLON \TYP{List}\llbracket\mathbb{Z}\rrbracket\bigr)\COLON \TYP{List}\llbracket\mathbb{Z}\rrbracket = e \\
  \ldots\-
\end{FortressCode}
\normalsize
The intuition was that for any specific function call,
the usual rule for dispatch would then choose 
the appropriate most specific definition 
for this (infinitely) overloaded function.

Although that intuition worked well enough 
for a single polymorphic function definition,
it failed utterly when we considered multiple function definitions.
For example, 
a programmer might want to provide definitions 
for specific monomorphic special cases, as in:
\small
\begin{FortressCode}
{\tt ~~}\+\VAR{append}\llbracket{}T\rrbracket\bigl(x\COLON \TYP{List}\llbracket{}T\rrbracket, y\COLON \TYP{List}\llbracket{}T\rrbracket\bigr)\COLON \TYP{List}\llbracket{}T\rrbracket = e_1 \\
  \VAR{append}\bigl(x\COLON \TYP{List}\llbracket\mathbb{Z}\rrbracket, y\COLON \TYP{List}\llbracket\mathbb{Z}\rrbracket\bigr)\COLON \TYP{List}\llbracket\mathbb{Z}\rrbracket = e_2\-
\end{FortressCode}
\normalsize
But if the interpretation above is taken seriously, 
this would be equivalent to:
\small
\begin{FortressCode}
{\tt ~~}\+\VAR{append}\bigl(x\COLON \TYP{List}\llbracket\TYP{Object}\rrbracket, y\COLON \TYP{List}\llbracket\TYP{Object}\rrbracket\bigr)\COLON \TYP{List}\llbracket\TYP{Object}\rrbracket = e_1 \\
  \VAR{append}\bigl(x\COLON \TYP{List}\llbracket\TYP{String}\rrbracket, y\COLON \TYP{List}\llbracket\TYP{String}\rrbracket\bigr)\COLON \TYP{List}\llbracket\TYP{String}\rrbracket = e_1 \\
  \VAR{append}\bigl(x\COLON \TYP{List}\llbracket\mathbb{Z}\rrbracket, y\COLON \TYP{List}\llbracket\mathbb{Z}\rrbracket\bigr)\COLON \TYP{List}\llbracket\mathbb{Z}\rrbracket = e_1 \\
  \ldots \\
  \VAR{append}\bigl(x\COLON \TYP{List}\llbracket\mathbb{Z}\rrbracket, y\COLON \TYP{List}\llbracket\mathbb{Z}\rrbracket\bigr)\COLON \TYP{List}\llbracket\mathbb{Z}\rrbracket = e_2\-
\end{FortressCode}
\normalsize
and we can see that there is an ambiguity 
when the arguments are both of type \EXP{\TYP{List}\llbracket\mathbb{Z}\rrbracket}.

It gets worse if the programmer wishes to handle an infinite set of cases specially.  
It would seem natural to write
\small
\begin{FortressCode}
{\tt ~~}\+\VAR{append}\llbracket{}T\rrbracket\bigl(x\COLON \TYP{List}\llbracket{}T\rrbracket, y\COLON \TYP{List}\llbracket{}T\rrbracket\bigr)\COLON \TYP{List}\llbracket{}T\rrbracket = e_1 \\
  \VAR{append}\llbracket{}T \SHORTCUT{<} \TYP{Number}\rrbracket\bigl(x\COLON \TYP{List}\llbracket{}T\rrbracket, y\COLON \TYP{List}\llbracket{}T\rrbracket\bigr)\COLON \TYP{List}\llbracket{}T\rrbracket = e_2\-
\end{FortressCode}
\normalsize
to handle specially all cases where \VAR{T} is a subtype of \TYP{Number}.
But the model would regard this as an overloading 
with an infinite number of ambiguities.


To resolve this problem, 
we had to develop an alternate model 
and an associated type system 
that could handle overloaded type-parametric functions 
in a manner that would accord with programmer intuition 
and support the plausible examples shown above.

Two other authors of this paper 


The key insight 


Credit for championing key insights---regarding each polymorphic definition
as a single definition (rather than an infinite set of definitions)
competing in the overload set, and using universal and existential types
to describe them in the type system (an idea reported by
Bourdoncle and Merz~\cite{bourdoncle97})---%
belongs to two other authors of this paper (Hilburn and Kilpatrick).
Adopting this new approach has made
overloaded polymorphic functions both tractable and effective
for writing Fortress code.

In this paper, 
we give rules for ensuring safe overloaded functions 
in a language that supports symmetric multiple dispatch, 
multiple inheritance, and parametric polymorphism 
(that is, generic types \emph{and} generic functions),
and we prove that these rules guarantee 
that there are no ambiguous calls at run time 
(see Section~\ref{sec:safety}).
We do this by extending our earlier rules 
for a core of the Fortress programming language 
that did not support generics \cite{allen07,Fortress},
for which we proved the analogous theorem.
To minimize syntactic overhead 
and avoid having to translate 
between a concrete language syntax 
and a formal semantics, 
we present these rules (see Section~\ref{sec:rules}) 
in the context of a straightforward formalization 
of a type system supporting multiple inheritance 
and parametric polymorphism, 
which we define in Section~\ref{sec:pre}.

The problem of dynamic dispatch 
in the presence of overloaded \emph{generic} functions 
is challenging 
because the overloaded definitions
might have not only distinct argument types, 
but also distinct type parameters 
(even different numbers of type parameters), 
so the type values of these parameters 
make sense only in distinct type environments. 
For example, consider the following overloaded function definitions in Fortress:
\small
\begin{FortressCode}
{\tt ~~}\+\VAR{combine}\llbracket{}T\rrbracket\bigl(\VAR{xs}\COLON \TYP{List}\llbracket{}T\rrbracket, \VAR{ys}\COLON \TYP{List}\llbracket{}T\rrbracket\bigr)\COLON \TYP{List}\llbracket{}T\rrbracket \\
  \VAR{combine}\llbracket{}S,T\rrbracket\bigl(s\COLON \TYP{Table}\llbracket{}S,T\rrbracket, t\COLON \TYP{Table}\llbracket{}S,T\rrbracket\bigr)\COLON \TYP{Table}\llbracket{}S,T\rrbracket\-
\end{FortressCode}
\normalsize
The first definition declares a single
type parameter denoting the types of the elements of the two
list arguments $xs$ and $ys$. The second definition declares two 
type parameters corresponding to the domains and ranges of the two
table arguments $s$ and $t$. But the type parameter of the first
definition bears no relation to the type parameters of the second.
How should we compare such function definitions 
to determine which is the best to dispatch to?
How can we ensure that there even is a best one in all cases?
Furthermore, the rules must be compatible with type inference, 
since instantiation of type parameters at a call site 
is typically done automatically.
So even determining which definitions are applicable 
to a particular call is not always obvious.

In providing rules to ensure 
that any valid set of overloaded function definitions 
guarantees that there is always a unique function to call at run time, 
we strive to be maximally permissive: 
A set of overloaded definitions should be disallowed 
only if it permits ambiguity
that cannot be resolved at run time.  
Nonetheless, 
we show in Section~\ref{sec:problems} 
that some seemingly valid sets of overloaded functions are rejected by our rules, 
and rightly so: 
although intuitively appealing, 
these overloaded functions admit ambiguous calls.

Many of these overloaded functions can, 
and we believe should, 
be allowed 
if the type system supports an \emph{exclusion relation},
which asserts that two types have no common instances.
If the domains of two function definitions exclude each other, 
then these definitions can never be applicable to the same call,
and so no ambiguity can arise between them.
Many languages provide a way of declaring some exclusion relations
implicitly. For example, single inheritance ensures that, for any 
two types, if one is not a subtype of the other, then the two types exclude each other.
Fortress enables programmers to declare ``nominal exclusion''
in addition to determining many exclusions implicitly, 
and in Section~\ref{sec:exclusion}, 
we formalize how Fortress does this, 
and show how this exclusion relation is used 
to improve expressivity 
by accommodating overloadings that would otherwise be rejected.
The proof of safety in Section~\ref{sec:safety} 
covers the rules under this extended type system.

% The remainder of this paper is organized thus:
% In Section~\ref{sec:pre}, we define the concepts and notation necessary
% to explain our formal rules for checking overloaded function definitions,
% which we present using universal and existential types in Section~\ref{sec:rules}.
% In Section~\ref{sec:problems}, 
% we explain why some apparently valid overloadings
% are (correctly) rejected by our rules 
% and why a multiple-inheritance language
% should include features for ``nominal exclusion'' (as Fortress does)
% to improve expressiveness and accommodate such overloadings.
% In Section~\ref{sec:exclusion}, we formalize the exclusion
% relation and use it to extend the overloading rules of Section~\ref{sec:rules}.
% %use it to augment the subtyping relation for universal and existential types.
% Section~\ref{sec:safety} explains that the overloading rules are
% sufficient to guarantee no undefined or ambiguous calls at run time.
In Section~\ref{sec:discussion}, we discuss type inference and modularity.
We discuss related work in Section~\ref{sec:related} and
conclude in Section~\ref{sec:conclusion}.



\newpage

\section{Notation}
\label{sec:notation}
We use the term \emph{monogram} to refer to a single letter (Latin or Greek)  that, rather than being used for decorative purposes, is itself possibly ``decorated'' with one or more prime marks and/or a sequence of one or more subscripts.  Examples of monograms are $x$, $\beta$, $e'$, $\alpha_2$, and $\tau'_{15\,27}$.

We write $\bar{x}$ as shorthand for a possibly empty comma-separated sequence $x_1, x_2, \ldots, x_n$ for some freely chosen nonnegative integer $n$;
thus $\bar{x}$ may expand to `` '' or ``$x_1$'' or ``$x_1, x_2$'' or ``$x_1, x_2, x_3$'' or ``$x_1, x_2, x_3, x_4$'' and so on.
More generally, for any expression, that same expression with an overbar is shorthand for a possibly empty comma-separated sequence
of copies of that expression with two transformations applied to each copy: (a) any subexpression that is underlined one or more times
is replaced by a copy of that subexpression with one underline removed, and (b) any subexpression that is a monogram that is not underlined
is replaced by a copy of that monogram with an additional subscript $i$ appended, where $i$ is the number of the copy (starting from the left with 1).
Thus $\bar{\underline{[\tau/P]}\tau'}$ means $[\tau/P]\tau'_1, [\tau/P]\tau'_2, \dots, [\tau/P]\tau'_n$,
so one possible concrete expansion is $[\tau/P]\tau'_1, [\tau/P]\tau'_2, [\tau/P]\tau'_3, [\tau/P]\tau'_4, [\tau/P]\tau'_5$.
If overbar constructions are nested, they are expanded outermost first.
Therefore the shorthand
$f\bigobb{P \extends \bdb{\tau}}$ means $f\bigob{P_1 \extends \bdb{\tau_1}, P_2 \extends \bdb{\tau_2}, \ldots, P_n \extends \bdb{\tau_n}}$, which in turn means:
\[ \begin{array}{l@{}l}
    f \big\llbracket \mskip0.5\thinmuskip & P_1 \extends \bd{\tau_{1\,1}, \tau_{1\,2}, \ldots, \tau_{1\,{n_1}}}, \\
                                          & P_2 \extends \bd{\tau_{2\,1}, \tau_{2\,2}, \ldots, \tau_{2\,{n_2}}}, \\
                                          & \ldots, \\
                                          & P_n \extends \bd{\tau_{m\,1}, \tau_{m\,2}, \ldots, \tau_{m\,{n_m}}} \mskip 0.5\thinmuskip \big\rrbracket
\end{array} \]
so one possible concrete expansion is:
\[ \begin{array}{l@{}l}
    f \big\llbracket \mskip0.5\thinmuskip & P_1 \extends \bd{\tau_{1\,1}, \tau_{1\,2}, \tau_{1\,3}}, \\
                                          & P_2 \extends \bd{\tau_{2\,1}}, \\
                                          & P_3 \extends \bd{\,}, \\
                                          & P_4 \extends \bd{\tau_{4\,1}, \tau_{4\,2}, \tau_{4\,3}, \tau_{4\,4}, \tau_{4\,5}, \tau_{4\,6}} \mskip 0.5\thinmuskip \big\rrbracket
\end{array} \]
Note the use of whitespace between subscripts so that $\tau_{4\,12}$ is clearly different from $\tau_{41\,2}$.

The function $\countof$ returns an integer saying how many arguments it was given; thus $\countof(\bar{x})$ tells the length of the sequence into which $\bar{x}$ has expanded.

After all occurrences of the overbar construction have been expanded, three other shorthand substitutions take place:
\begin{itemize}
\item Every monogram whose base letter has been described as \emph{ranging over} a {\sc bnf} nonterminal of a specified grammar is replaced by a token sequence generated by the grammar from that nonterminal.
\item Each occurrence of the symbol ``$\dontcare$'' is replaced by any sequence of tokens that is correctly balanced within respect to parentheses, braces, and brackets of all kinds,
such that every comma or semicolon in the sequence is contained within at least one matched pair of parentheses, braces, or brackets. (This is used as a ``don't care'' indication when asking whether any of a set of constructs matches a certain syntactic pattern.)
\item Each occurrence of the symbol ``$\emptyseq$'' is deleted.  (This symbol is used as an explicit indication that an empty sequence of symbols is intended.)
\end{itemize}

When any of these shorthands is used in an overall context (such as a {\sc bnf} rule, inference rule, axiom, or expository sentence or paragraph), it is as if there were an infinite number of instantiations of that context, one for each possible expansion of the shorthand.  Three consistency constraints must be obeyed in performing the substitutions for any single such context:
\begin{itemize}
\item
If the same monogram (with identical decorations) has an additional subscript attached to it
by more than one overbar construction, then all such overbar constructions are constrained to produce the same number of copies in any given instantiation of the rule;
otherwise the choices for the number of copies produced by each overbar construction is free and independent.
\item
If the base letter of a monogram ranges over a {\sc bnf} nonterminal, then multiple identical occurrences of the monogram must be replaced by identical copies of a single generated token sequence.
\item
If two distinct monograms each have base letters that range over a {\sc bnf} nonterminal that expands to simply ``identifier'', then they must be replaced with different identifiers.
\end{itemize}
The last two constraints rely on metavariable declarations such as those in Figure~\ref{fig:metavariables}.  A declaration such as
``$e$ ranges over expressions $e$'' means 
``monograms with base letter $e$ expand into expressions generated by {\sc bnf} nonterminal $e$'';
this may seem redundant, but only because by convention we frequently use a single-letter identifier as a {\sc bnf} nonterminal
and then go on to use that same single-letter identifier as a base letter for monograms.
A declaration such as
``$\alpha, \gamma, \rho, \chi, \eta$ range over lattice types $\alpha$'' means 
``monograms with base letter $\alpha$ or $\gamma$ or $\rho$ or $\chi$ or $\eta$ expand into expressions generated by {\sc bnf} nonterminal $\alpha$''
which is more clearly not a redundant statement.

As an additional convenience using these shorthands, we adopt these conventions:
\begin{itemize}
\item If a judgment has several comma-separated expressions to the right of the turnstile ``$\turnstile$'', it is
as if there were several distinct judgments, one containing each of the expressions to the right of the turnstile.
Thus the judgment $\jgTHREEtemplate{\Gamma}{\tau_1}{\extends}{\tau'_1}{\tau_2}{\tau'_2}{\tau_3}{\tau'_3}$
means the same as three separately written judgments:
\[\jgtemplate[\Gamma]{\tau_1}{\extends}{\tau'_1} \andalso \jgtemplate[\Gamma]{\tau_2}{\extends}{\tau'_2} \andalso \jgtemplate[\Gamma]{\tau_3}{\extends}{\tau'_3} \]
\item If a judgment has nothing to the right of the turnstile, it is
as if there were no judgment written at all.
\item If an inference rule has several comma-separated expressions or judgments as consequents, it is
as if there were several distinct inference rules, one containing each of the consequents.
\item If an inference rule has no consequents, it is
as if there were no inference rule written at all.
\end{itemize}

As an extreme (but useful) example of the application of these conventions, consider this axiom:

\infax{ \jbevalstep[\Delta]{\underline{E}\big[\pi(\underline{\bar{v}})\big]}{\underline{E}[v]} }

\noindent In order to apply this axiom to a particular case, we may freely choose to expand the largest overbar construction to produce, say, two copies:

\infax{ \jevalstepTWO[\Delta]{E\big[\pi_1(\bar{v})\big]}{E[v_1]}{E\big[\pi_2(\bar{v})\big]}{E[v_2]} }

\noindent Note that both the $\pi$ symbol and the second occurrence of $v$ receive subscripts in each copy, but the underlines (which are removed as part of the expansion process) prevent the first occurrence of $v$ (which happens to have a second overbar) and the two occurrences of $E$ from receiving subscripts.  Now we expand the remaining overbars, but because they will attach subscripts to the symbol $v$, and $v$ has already had subscripts attached by the larger overbar, we must choose the same number of copies (two) for each of these overbars:

\infax{ \jevalstepTWO[\Delta]{E\big[\pi_1(v_1,v_2)\big]}{E[v_1]}{E\big[\pi_2(v_1,v_2)\big]}{E[v_2]} }

\noindent Then this judgment with two comma-separated expressions to the right of the turnstile is understood to mean two distinct judgments:

\infax{ \jevalstep[\Delta]{E\big[\pi_1(v_1,v_2)]}{E[v_1]} \\[2pt]
        \jevalstep[\Delta]{E\big[\pi_2(v_1,v_2)]}{E[v_2]} }

A final note: we sometimes use parentheses or braces or brackets of different sizes within an expression purely to enhance readability;
the size of such a symbol does not affect its meaning in the formalism.


\section{Grammar}
\label{sec:grammar}

\begin{figure}

\begin{array}[t]{@{}l@{\;}c@{\;}l@{\hskip 2em}l@{}}
\alpha   & ::= &  P                                              & \hbox{\rm type parameter name} \\
         &  |  &  T\obb{\alpha}                                  & \hbox{\rm trait type} \\
         &  |  &  O\obb{\alpha}                                  & \hbox{\rm object type} \\
         &  |  &  (\bar{\alpha})                                 & \hbox{\rm tuple type} \\
         &  |  &  \arrowtype{\alpha}{\alpha}                     & \hbox{\rm arrow type} \\
         &  |  &  \Any                                           & \hbox{\rm special \Any\ type} \\
         &  |  &  \Object                                        & \hbox{\rm special \Object\ type} \\
         &  |  &  \Bottom                                        & \hbox{\rm special \Bottom\ type} \\
         &  |  &  \uniontype{\alpha}{\alpha}                     & \hbox{\rm union type} \\
         &  |  &  \intersectiontype{\alpha}{\alpha}              & \hbox{\rm intersection type} \\[4pt]
\kappa   & ::= &  \alpha                                         & \hbox{\rm lattice type} \\
         &  |  &  \Xi                                            & \hbox{\rm existentially quantified type} \\
         &  |  &  \Upsilon                                       & \hbox{\rm universally quantified type} \\[4pt]
\Xi      & ::= &  \existstypeb{\lambda}{\alpha}                  & \hbox{\rm existentially quantified type} \\[4pt]
\Upsilon & ::= &  \foralltypeb{\lambda}{\alpha}                  & \hbox{\rm universally quantified type} \\[4pt]
\lambda  & ::= &  \bdb{\alpha} \extends P \extends \bdb{\alpha}  & \hbox{\rm lattice type parameter binding} \\[4pt]
\psi     & ::= &  \delta                                         & \hbox{\rm program declaration} \\
         &  |  &  \lambda                                        & \hbox{\rm lattice type parameter binding} \\[4pt]
\var     & ::= &  x                                              & \hbox{\rm variable name} \\
         &  |  &  z                                              & \hbox{\rm field name} \\
         &  |  &  \kwd{self}                                     & \hbox{\rm self keyword} \\[4pt]
\Delta   & ::= &  \bar{\psi}                                     & \hbox{\rm type-declaration environment} \\[4pt]
\Gamma   & ::= &  \bar{\mathit{var}\COLON\alpha}                 & \hbox{\rm variable-type environment} \\[4pt]
\end{array}

\caption{Symbols Not Used in the Concrete Syntax}
\label{fig:internalsymbols}
\end{figure}


\begin{figure}
\typicallabel{W-Object}

\newjudge{Well-formed types}{\jwftype{\kappa}}
\bigskip

% Stuff in \Delta is assumed to be well-formed

\infrule[W-Param]
  { \bd{\dontcare} \extends P \extends \bd{\dontcare} \in \set{\Delta} }
  { \jwftype{P} }

\bigskip

\infrule[W-Trait]
  { \kwd{trait} \; T\bigobb{V\;P \extends \bdb{\xi}} \; \dontcare \; \kwd{end} \in \set{\Delta} \\[2pt]
    \countof(\bar{\alpha}) = \countof(\bar{P})  \andalso  \jbwftype{\alpha} \\[3pt]
    \jbsubtype{\alpha}{\underline{\Big[\bar{\alpha/P}\Big]}\xi} }
  { \jwftype{T\obb{\alpha}} }

\bigskip

\infrule[W-Object]
  { \kwd{object} \; O\bigobb{P \extends \bdb{\xi}} \; \dontcare \; \kwd{end} \in \set{\Delta} \\[2pt]
    \countof(\bar{\alpha}) = \countof(\bar{P})  \andalso  \jbwftype{\alpha} \\[3pt]
    \jbsubtype{\alpha}{\underline{\Big[\bar{\alpha/P}\Big]}\xi} }
  { \jwftype{O\obb{\alpha}} }

\bigskip

\infrule[W-Tuple]
  { \jbwftype{\alpha} }
  { \jwftype{(\bar{\alpha})} }

\bigskip

\infrule[W-Arrow]
  { \jwftype{\alpha}  \andalso  \jwftype{\rho} }
  { \jwftype{\arrowtype{\alpha}{\rho}} }

\bigskip

\infax[W-Any-Type]
  { \jwftype{\Any} }

\bigskip

\infax[W-Object-Type]
  { \jwftype{\Object} }

\bigskip

\infax[W-Bottom-Type]
  { \jwftype{\Bottom} }

\bigskip

\infrule[W-Union]
  { \jwftype{\alpha}  \andalso  \jwftype{\gamma} }
  { \jwftype{\uniontype{\alpha}{\gamma}} }

\bigskip

\infrule[W-Intersection]
  { \jwftype{\alpha}  \andalso  \jwftype{\gamma} }
  { \jwftype{\intersectiontype{\alpha}{\gamma}} }

\bigskip

\infrule[W-Exists]
  { \jbwftype{\chi}  \andalso  \jbwftype{\eta} \\[4pt]
    \jwftype[\Delta, \bar{\bdb{\chi} \extends P \extends \bdb{\eta}}]{\alpha} }
  { \jwftype{\existstypeb{\bdb{\chi} \extends P \extends \bdb{\eta}}{\alpha}} }

\bigskip

\infrule[W-Forall]
  { \jbwftype{\chi}  \andalso  \jbwftype{\eta} \\[4pt]
    \jwftype[\Delta, \bar{\bdb{\chi} \extends P \extends \bdb{\eta}}]{\alpha} }
  { \jwftype{\foralltypeb{\bdb{\chi} \extends P \extends \bdb{\eta}}{\alpha}} }

\medskip
\caption{Well-formed Types}
\label{fig:wellformedtypes}
\end{figure}



The metavariables used throughout this paper are listed in Figure~\ref{fig:metavariables}.

The grammar for Welterweight Fortress (hereafter called ``WF'') is given in Figure~\ref{fig:grammar}.
A program consists of declarations and an expression to be evaluated i the contex of those declarations.
Each declaration defines a trait, an object, or a top-level function.

A trait is similar to a Java interface, but can contain method declarations.
It also has an \emph{extends clause} \hbox{$\extends\;\bigsetb{t}$}, an \emph{excludes clause} \hbox{$\exc\;\bigsetb{t}$},
and optionally a \emph{comprises clause} \hbox{$\comprises\;\setb{c}$}.  The extends clause indicates what other trait instances are supertypes
of the trait; the excludes clause indicates what other trait instances cannot have values in common with this trait.
The comprises clause, if present, indicates that no value can belong to the trait unless it also belongs
to one of the comprised types.

An object is similar to a Java class.  Rather than having a separately declared constructor method,
it has a parameter list containing names of fields and their types; an object creation expression
provides a set of arguments that are simply used to initialize the fields, with no further action taken.
An object declaration also has an extends clause, indicating what trait instances are supertypes of the object type.

A top-level function has a parameter list containing names of parameters and their types, and also a return type and a body expression.
Top-level functions may be overloaded; that is, more than one function declaration may have the same name $f$.

Method declarations appearing in traits or objects are similar in form to function declarations; however, the keyword $\kwd{self}$
may be used within its body expression to refer to a value (the \emph{target object}) for which the method was invoked.

\begin{figure}
\typicallabel{T-Field}

\newjudge{Static types of expressions}{\jtype{e}{\alpha}}
\bigskip

\infrule[T-Variable]
  { \Gamma = \dontcare, \var\COLON\alpha, \bar{x'\COLON\dontcare}  \andalso  x \not\in \bigsetb{x'} }
  { \jtype{x}{\alpha} }

\bigskip

\infrule[T-Tuple]
  { \jbtype{e}{\alpha} }
  { \jtype{(\bar{e})}{(\bar{\alpha})} }

\bigskip

\infrule[T-Project]
  { \jtype{e}{\tupleb{\alpha}} }
  { \jbtype{\pi(\underline{e})}{\alpha} }

\bigskip

\infrule[T-Func]
  { \jtype[\Delta;\Gamma,\bar{x\COLON\tau}]{e}{\omega} }
  { \jtype{((\bar{x\COLON\tau})\COLON\omega \Rightarrow e)}{\arrowtype{(\bar{\tau})}{\omega}} }

\bigskip

\infrule[T-Apply]
  { \jtype{e}{\arrowtype{(\bar{\alpha})}{\rho}}  \\
    \jtype{(\bar{e'})}{(\bar{\chi})} \andalso
    \jbsubtype{\chi}{\alpha}  }
  { \jtype{e\apply(\bar{e'})}{\rho} }

\bigskip

\infrule[T-Field]
  { \jtype{e}{O\obb{\alpha}} \\
    \kwd{object} \; O\bigobb{P \extends \bd{\dontcare}} (\bar{z\COLON\tau}) \; \dontcare \; \kwd{end} \in \set{\Delta} }
  { \jbtype{\underline{e}.z}{\underline{\Big[\bar{\alpha/P}\Big]}\tau} }

\bigskip

\infrule[T-Object]
  { TBD }
  { \jtype{O\obb{\tau}(\bar{e})}{} }

\bigskip

\infrule[T-Func-SA]
  { TBD }
  { \jtype{f\obb{\tau}(\bar{e})}{} }

\bigskip

\infrule[T-Func-NSA]
  { TBD }
  { \jtype{f(\bar{e})}{} }

\bigskip

\infrule[T-Method-SA]
  { TBD }
  { \jtype{e.m\obb{\tau}(\bar{e})}{} }

\bigskip

\infrule[T-Method-NSA]
  { TBD }
  { \jtype{e.m(\bar{e})}{} }

\bigskip

\infrule[T-Match]
  { \jtype{e}{\alpha}  \\ \jtype[\Delta;\Gamma,x\COLON(\alpha\cap\tau)]{e'}{\eta}  \andalso \jtype{e''}{\chi} }
  { \jtype{(e \; \kwd{match} \; x\COLON\tau \Rightarrow e' \; \kwd{else}\; e'')}{(\eta\cup\chi)} }

\medskip
\caption{Static Types of Expressions}
\label{fig:expressiontypes}
\end{figure}


Traits, objects, functions, and methods all have a (possibly empty) list of \emph{static type parameters}.
Associated with each static type parameter is (possibly empty) set of \emph{upper bounds}; any type used to instantiate the
type parameter must be a subtype of each of the declared upper bounds.
Each static type parameter of a method also has a (possibly empty) set of \emph{lower bounds}; any type used to instantiate the
type parameter must be a supertype of each of the declared lower bounds.
Each static type parameter of a trait has an associated \emph{variance}, which indicates ways in which different
instances of the same trait may extend or exclude each other.

Most of the forms of expression are fairly conventional.  The function creation expression
might be called a `typed lambda expression'' in other languages.  We use the symbol $\apply$ to
distinguish application of function-typed values from invocation of (possibly overloaded) top-level functions.
Function invocations and method invocations each come in two forms, depending on whether static type arguments
are provided explicitly or are to be inferred.  The \emph{match expression} is a kind of compromise between
a conventional %\kwd{if}$ expression and a conventional $\kwd{typecase}$ expression.  One can get the effect
of $\kwd{if}\;e_1\;\kwd{then}\;e_2\;\kwd{else}\;e_3$ by declaring objects $\mathrm{True}\ob{\,}$ and $\mathrm{False}\ob{\,}$,
using them as values of predicates, and then writing
$e_1\;\kwd{match}\;x'\COLON \mathrm{true}\ob{\,} \Rightarrow e_2 \;\kwd{else}\;e_3$.

The type $\Any$ is a supertype of all types. The type $\Object$ is a supertype of every constructed type, that is, every trait type and every object type.


\section{Wellformedness}
\label{sec:wellformedness}
%\input{wellformedness}

\section{Examples}
\label{sec:examples}
%\input{examples}

\section{Related Work}\label{sec:related}
%% * multiple dispatch
%   * fortress
%   * CLOS
%   * multijava
%   * cecil
% * type classes
%   * wadler 89
%   * qualified types (mpj)
%   * concepts (siek)
%   * inability to add ad-hoc overloaded functions
% * GADTs
%   * GADT inference (spj)
%   * HMG(X) (pottier)
%   * with OOP (russo)
%
\subsection{Overloading and dynamic dispatch.} 

\TODO{Add discussion of Castagna et al.\ here?}

Primarily, our system
strictly extends our previous effort \cite{allen07} with parametric polymorphism;
all previous properties and results remain intact. The inclusion of parametric
functions and types represents a shift in the research literature on overloading
and multiple dynamic dispatch.

Millstein and Chambers \cite{millstein02,millstein03} devised the language
Dubious to study overloaded functions with symmetric multiple dynamic dispatch
(\emph{multimethods}), and with Clifton and Leavens they developed MultiJava
\cite{multijava}, an extension of Java with Dubious' semantics for multimethods.
In \cite{feml}, Lee and Chambers presented F(E\textsc{ml}), a language with
classes, symmetric multiple dispatch, and parameterized modules, but without
parametricity at the level of functions or types. None of these systems support
polymorphic functions or types. F(E\textsc{ml})'s parameterized modules
(\emph{functors}) constitute a form of parametricity but they cannot be implicitly
applied; the functions defined therein cannot be \emph{overloaded} with those
defined in other functors. For a more detailed comparison of modularity and
dispatch between our system and these, we refer to the related work section of
our previous paper \cite{allen07}.

% I took out discussion of modularity here; it's charged and unnecessary
% in order to distinguish our work. EricAllen 7/15/2011
Overloading and multiple dispatch in the context of polymorphism 
has previously been studied by Bourdoncle and Merz \cite{bourdoncle97}. 
Their system, ML$_\le$, integrates parametric polymorphism, 
class-based object orientation, and multimethods,
but lacks multiple inheritance. 
Each multimethod (overloaded set) requires a unique specification (principal type), 
which prevents overloaded functions defined on disjoint domains; 
% the domains of the multimethod branches must partition the specification domain, 
% which eliminates subtype-based specialization;
and link-time checks are performed to ensure that multimethods are fully
implemented across modules. 
On the other hand, ML$_\le$ allows variance annotations on type constructors---% 
something we attribute to future work.

Litvinov~\cite{litvinov98} developed a type system for the Cecil language,
which supports bounded parametric polymorphism and multimethods.
Because Cecil has a type-erasure semantics, 
statically checked parametric polymorphism has no effect on run-time dispatch.

\subsection{Type classes.} Wadler and Blott \cite{wadler89} introduced
\emph{type class} as a means to specify and implement overloaded
functions like equality and arithmetic operators in Haskell. Other authors
have translated type classes to languages besides Haskell \cite{dreyer07,siek05,wehr07}.
Type classes encapsulate overloaded function declarations, with separate
\emph{instances} that define the behavior of those functions (called \emph{class methods})
for any particular type schema. Parametric polymorphism is then augmented to
express type class constraints, providing a way to quantify a type variable --- and
thus a function definition --- over all types that instantiate the type class. 

% In his thesis \cite{jonesbook} Jones generalized Haskell's underlying type
% system as \emph{qualified types}, in which the satisfaction of type predicates
% must be proved with constructed \emph{evidence}. Qualified type systems (such
% as Haskell) exhibit the \emph{principal types} property necessary for full
% Damas-Milner style type inference \cite{dm82,jonesbook}; our system conservatively
% assumes only \emph{local type inference} \cite{pierce00} --- implicit type
% instantiation for polymorphic function calls.

In systems with type classes, overloaded functions must be contained in some
type class, and their signatures must vary in exactly the same structural
position. This uniformity is necessary for an overloaded function call to
admit a principal type; with a principal type for some function call's context,
the type checker can determine the constraints under which a correct overloaded
definition will be found. Because of this requirement, type classes are ill-suited
for fixed, \emph{ad hoc} sets of overloaded functions like:
\begin{FortressCode}
{\tt ~~~~}\+\VAR{println}(\ultrathin)\COLON (\ultrathin) = \VAR{println}(\hbox{\rm\usefont{T1}{ptm}{m}{n}``\verythin''}) \\
    \VAR{println}(s\COLON \TYP{String})\COLON (\ultrathin) = \ldots\-
\end{FortressCode}
or functions lacking uniform variance in the domain and range\footnote{With the
\emph{multi-parameter type class} extension, one could define functions as these.
A reference to the method \mono{bar}, however, would require an explicit type
annotation like \mono{:: Int -> Bool} to apply to an \mono{Int}.} like:
\begin{FortressCode}
{\tt ~~~~}\+\VAR{bar}(x\COLON \mathbb{Z})\COLON \TYP{Boolean} = (x = 0) \\
    \VAR{bar}(x\COLON \TYP{Boolean})\COLON \mathbb{Z} =\; \KWD{if} x \KWD{then} 1 \KWD{else} 2 \KWD{end} \\
    \VAR{bar}(x\COLON \TYP{String})\COLON \TYP{String} = x\-
\end{FortressCode}
With type classes one can write overloaded functions with identical domain
types. Such behavior is consistent with the \emph{static}, \emph{type-based}
dispatch of Haskell, but it would lead to irreconcilable ambiguity in the
\emph{dynamic}, \emph{value-based} dispatch of our system.
%% In Appendix~\ref{app:haskell}, we present a further discussion of how our overloading resolution differs from that of Haskell and how our system might translate to that language, thereby addressing an existing inconsistency in modern type class extensions.

A broader interpretation of Wadler and Blott's \cite{wadler89} sees type
classes as program abstractions that quotient the space of ad-hoc polymorphism
into the much smaller space of class methods. Indeed, Wadler and Blott's title
suggests that the unrestricted space of ad-hoc polymorphism should be tamed,
whereas our work embraces the possible expressivity achieved from mixing ad-hoc
and parametric polymorphism by specifying the requisites for determinism and type safety.


\section{Conclusion and Discussion}\label{sec:conclusion}
%We have shown how to statically ensure safety of overloaded, polymorphic functions while imposing relatively minimal restrictions solely on function definition sites. We provide rules on definitions that can be checked modularly, irrespective of call sites, and we show how to mechanically verify that a program satisfies these rules. The type analysis required for implementing these checks involves subtyping on universal and existential types, which adds complexity not required for similar checks on monomorphic functions. We have defined an object-oriented language to explain our system of static checks, and we have implemented them as part of the open-source Fortress compiler \cite{Fortress}.

Further, we show that in order to check many ``natural'' overloaded functions with our system in the context of a generic, object-oriented language, richer type relations must be available to programmers---the subtyping relation prevalent among such languages does not afford enough type analysis alone. We have thus introduced an explicit, nominal exclusion relation to check safety of more interesting overloaded functions.

Variance annotations have proven to be a convenient and expressive addition to languages based on nominal subtyping \cite{bourdoncle97,kennedy07,scala}. They add additional complexity to polymorphic exclusion checking, so we leave them to future work.


\section*{Acknowledgments}
% This work is supported in part by the Engineering Research Center of Excellence Program of Korea Ministry of Education,
% Science and Technology(MEST) / National Research Foundation of Korea(NRF)
% (Grant 2011-0000974).

\bibliographystyle{plain}
% The bibliography should be embedded for final submission.
\bibliography{paper}
% \begin{thebibliography}{}
% \softraggedright
% 
% \input{biblio.tex}
% 
% \end{thebibliography}

% \appendix
\end{document}
