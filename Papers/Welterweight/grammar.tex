
\begin{figure}

\begin{array}[t]{@{}l@{\;}c@{\;}l@{\hskip 2em}l@{}}
\alpha   & ::= &  P                                              & \hbox{\rm type parameter name} \\
         &  |  &  T\obb{\alpha}                                  & \hbox{\rm trait type} \\
         &  |  &  O\obb{\alpha}                                  & \hbox{\rm object type} \\
         &  |  &  (\bar{\alpha})                                 & \hbox{\rm tuple type} \\
         &  |  &  \arrowtype{\alpha}{\alpha}                     & \hbox{\rm arrow type} \\
         &  |  &  \Any                                           & \hbox{\rm special \Any\ type} \\
         &  |  &  \Object                                        & \hbox{\rm special \Object\ type} \\
         &  |  &  \Bottom                                        & \hbox{\rm special \Bottom\ type} \\
         &  |  &  \uniontype{\alpha}{\alpha}                     & \hbox{\rm union type} \\
         &  |  &  \intersectiontype{\alpha}{\alpha}              & \hbox{\rm intersection type} \\[4pt]
\kappa   & ::= &  \alpha                                         & \hbox{\rm lattice type} \\
         &  |  &  \Xi                                            & \hbox{\rm existentially quantified type} \\
         &  |  &  \Upsilon                                       & \hbox{\rm universally quantified type} \\[4pt]
\Xi      & ::= &  \existstypeb{\lambda}{\alpha}                  & \hbox{\rm existentially quantified type} \\[4pt]
\Upsilon & ::= &  \foralltypeb{\lambda}{\alpha}                  & \hbox{\rm universally quantified type} \\[4pt]
\lambda  & ::= &  \bdb{\alpha} \extends P \extends \bdb{\alpha}  & \hbox{\rm lattice type parameter binding} \\[4pt]
\psi     & ::= &  \delta                                         & \hbox{\rm program declaration} \\
         &  |  &  \lambda                                        & \hbox{\rm lattice type parameter binding} \\[4pt]
\var     & ::= &  x                                              & \hbox{\rm variable name} \\
         &  |  &  z                                              & \hbox{\rm field name} \\
         &  |  &  \kwd{self}                                     & \hbox{\rm self keyword} \\[4pt]
\Delta   & ::= &  \bar{\psi}                                     & \hbox{\rm type-declaration environment} \\[4pt]
\Gamma   & ::= &  \bar{\mathit{var}\COLON\alpha}                 & \hbox{\rm variable-type environment} \\[4pt]
\end{array}

\caption{Symbols Not Used in the Concrete Syntax}
\label{fig:internalsymbols}
\end{figure}


\begin{figure}
\typicallabel{W-Object}

\newjudge{Well-formed types}{\jwftype{\kappa}}
\bigskip

% Stuff in \Delta is assumed to be well-formed

\infrule[W-Param]
  { \bd{\dontcare} \extends P \extends \bd{\dontcare} \in \set{\Delta} }
  { \jwftype{P} }

\bigskip

\infrule[W-Trait]
  { \kwd{trait} \; T\bigobb{V\;P \extends \bdb{\xi}} \; \dontcare \; \kwd{end} \in \set{\Delta} \\[2pt]
    \countof(\bar{\alpha}) = \countof(\bar{P})  \andalso  \jbwftype{\alpha} \\[3pt]
    \jbsubtype{\alpha}{\underline{\Big[\bar{\alpha/P}\Big]}\xi} }
  { \jwftype{T\obb{\alpha}} }

\bigskip

\infrule[W-Object]
  { \kwd{object} \; O\bigobb{P \extends \bdb{\xi}} \; \dontcare \; \kwd{end} \in \set{\Delta} \\[2pt]
    \countof(\bar{\alpha}) = \countof(\bar{P})  \andalso  \jbwftype{\alpha} \\[3pt]
    \jbsubtype{\alpha}{\underline{\Big[\bar{\alpha/P}\Big]}\xi} }
  { \jwftype{O\obb{\alpha}} }

\bigskip

\infrule[W-Tuple]
  { \jbwftype{\alpha} }
  { \jwftype{(\bar{\alpha})} }

\bigskip

\infrule[W-Arrow]
  { \jwftype{\alpha}  \andalso  \jwftype{\rho} }
  { \jwftype{\arrowtype{\alpha}{\rho}} }

\bigskip

\infax[W-Any-Type]
  { \jwftype{\Any} }

\bigskip

\infax[W-Object-Type]
  { \jwftype{\Object} }

\bigskip

\infax[W-Bottom-Type]
  { \jwftype{\Bottom} }

\bigskip

\infrule[W-Union]
  { \jwftype{\alpha}  \andalso  \jwftype{\gamma} }
  { \jwftype{\uniontype{\alpha}{\gamma}} }

\bigskip

\infrule[W-Intersection]
  { \jwftype{\alpha}  \andalso  \jwftype{\gamma} }
  { \jwftype{\intersectiontype{\alpha}{\gamma}} }

\bigskip

\infrule[W-Exists]
  { \jbwftype{\chi}  \andalso  \jbwftype{\eta} \\[4pt]
    \jwftype[\Delta, \bar{\bdb{\chi} \extends P \extends \bdb{\eta}}]{\alpha} }
  { \jwftype{\existstypeb{\bdb{\chi} \extends P \extends \bdb{\eta}}{\alpha}} }

\bigskip

\infrule[W-Forall]
  { \jbwftype{\chi}  \andalso  \jbwftype{\eta} \\[4pt]
    \jwftype[\Delta, \bar{\bdb{\chi} \extends P \extends \bdb{\eta}}]{\alpha} }
  { \jwftype{\foralltypeb{\bdb{\chi} \extends P \extends \bdb{\eta}}{\alpha}} }

\medskip
\caption{Well-formed Types}
\label{fig:wellformedtypes}
\end{figure}



The metavariables used throughout this paper are listed in Figure~\ref{fig:metavariables}.

The grammar for Welterweight Fortress (hereafter called ``WF'') is given in Figure~\ref{fig:grammar}.
A program consists of declarations and an expression to be evaluated i the contex of those declarations.
Each declaration defines a trait, an object, or a top-level function.

A trait is similar to a Java interface, but can contain method declarations.
It also has an \emph{extends clause} \hbox{$\extends\;\bigsetb{t}$}, an \emph{excludes clause} \hbox{$\exc\;\bigsetb{t}$},
and optionally a \emph{comprises clause} \hbox{$\comprises\;\setb{c}$}.  The extends clause indicates what other trait instances are supertypes
of the trait; the excludes clause indicates what other trait instances cannot have values in common with this trait.
The comprises clause, if present, indicates that no value can belong to the trait unless it also belongs
to one of the comprised types.

An object is similar to a Java class.  Rather than having a separately declared constructor method,
it has a parameter list containing names of fields and their types; an object creation expression
provides a set of arguments that are simply used to initialize the fields, with no further action taken.
An object declaration also has an extends clause, indicating what trait instances are supertypes of the object type.

A top-level function has a parameter list containing names of parameters and their types, and also a return type and a body expression.
Top-level functions may be overloaded; that is, more than one function declaration may have the same name $f$.

Method declarations appearing in traits or objects are similar in form to function declarations; however, the keyword $\kwd{self}$
may be used within its body expression to refer to a value (the \emph{target object}) for which the method was invoked.

\begin{figure}
\typicallabel{T-Field}

\newjudge{Static types of expressions}{\jtype{e}{\alpha}}
\bigskip

\infrule[T-Variable]
  { \Gamma = \dontcare, \var\COLON\alpha, \bar{x'\COLON\dontcare}  \andalso  x \not\in \bigsetb{x'} }
  { \jtype{x}{\alpha} }

\bigskip

\infrule[T-Tuple]
  { \jbtype{e}{\alpha} }
  { \jtype{(\bar{e})}{(\bar{\alpha})} }

\bigskip

\infrule[T-Project]
  { \jtype{e}{\tupleb{\alpha}} }
  { \jbtype{\pi(\underline{e})}{\alpha} }

\bigskip

\infrule[T-Func]
  { \jtype[\Delta;\Gamma,\bar{x\COLON\tau}]{e}{\omega} }
  { \jtype{((\bar{x\COLON\tau})\COLON\omega \Rightarrow e)}{\arrowtype{(\bar{\tau})}{\omega}} }

\bigskip

\infrule[T-Apply]
  { \jtype{e}{\arrowtype{(\bar{\alpha})}{\rho}}  \\
    \jtype{(\bar{e'})}{(\bar{\chi})} \andalso
    \jbsubtype{\chi}{\alpha}  }
  { \jtype{e\apply(\bar{e'})}{\rho} }

\bigskip

\infrule[T-Field]
  { \jtype{e}{O\obb{\alpha}} \\
    \kwd{object} \; O\bigobb{P \extends \bd{\dontcare}} (\bar{z\COLON\tau}) \; \dontcare \; \kwd{end} \in \set{\Delta} }
  { \jbtype{\underline{e}.z}{\underline{\Big[\bar{\alpha/P}\Big]}\tau} }

\bigskip

\infrule[T-Object]
  { TBD }
  { \jtype{O\obb{\tau}(\bar{e})}{} }

\bigskip

\infrule[T-Func-SA]
  { TBD }
  { \jtype{f\obb{\tau}(\bar{e})}{} }

\bigskip

\infrule[T-Func-NSA]
  { TBD }
  { \jtype{f(\bar{e})}{} }

\bigskip

\infrule[T-Method-SA]
  { TBD }
  { \jtype{e.m\obb{\tau}(\bar{e})}{} }

\bigskip

\infrule[T-Method-NSA]
  { TBD }
  { \jtype{e.m(\bar{e})}{} }

\bigskip

\infrule[T-Match]
  { \jtype{e}{\alpha}  \\ \jtype[\Delta;\Gamma,x\COLON(\alpha\cap\tau)]{e'}{\eta}  \andalso \jtype{e''}{\chi} }
  { \jtype{(e \; \kwd{match} \; x\COLON\tau \Rightarrow e' \; \kwd{else}\; e'')}{(\eta\cup\chi)} }

\medskip
\caption{Static Types of Expressions}
\label{fig:expressiontypes}
\end{figure}


Traits, objects, functions, and methods all have a (possibly empty) list of \emph{static type parameters}.
Associated with each static type parameter is (possibly empty) set of \emph{upper bounds}; any type used to instantiate the
type parameter must be a subtype of each of the declared upper bounds.
Each static type parameter of a method also has a (possibly empty) set of \emph{lower bounds}; any type used to instantiate the
type parameter must be a supertype of each of the declared lower bounds.
Each static type parameter of a trait has an associated \emph{variance}, which indicates ways in which different
instances of the same trait may extend or exclude each other.

Most of the forms of expression are fairly conventional.  The function creation expression
might be called a `typed lambda expression'' in other languages.  We use the symbol $\apply$ to
distinguish application of function-typed values from invocation of (possibly overloaded) top-level functions.
Function invocations and method invocations each come in two forms, depending on whether static type arguments
are provided explicitly or are to be inferred.  The \emph{match expression} is a kind of compromise between
a conventional %\kwd{if}$ expression and a conventional $\kwd{typecase}$ expression.  One can get the effect
of $\kwd{if}\;e_1\;\kwd{then}\;e_2\;\kwd{else}\;e_3$ by declaring objects $\mathrm{True}\ob{\,}$ and $\mathrm{False}\ob{\,}$,
using them as values of predicates, and then writing
$e_1\;\kwd{match}\;x'\COLON \mathrm{true}\ob{\,} \Rightarrow e_2 \;\kwd{else}\;e_3$.

The type $\Any$ is a supertype of all types. The type $\Object$ is a supertype of every constructed type, that is, every trait type and every object type.
