\documentclass[11pt]{article}

% THIS REQUIRES XELATEX OR LUATEX, AND SOME FONTS.

\usepackage{geometry}                % See geometry.pdf to learn the layout options. There are lots.
\geometry{letterpaper}                   % ... or a4paper or a5paper or ... 
%\geometry{landscape}                % Activate for for rotated page geometry
%\usepackage[parfill]{parskip}    % Activate to begin paragraphs with an empty line rather than an indent
%\usepackage{graphicx}
%\usepackage{amssymb}
%\usepackage{epstopdf}
\usepackage{fontspec}
%\DeclareGraphicsRule{.tif}{png}{.png}{`convert #1 `dirname #1`/`basename #1 .tif`.png}
\title{Fortress function and method encodings}

\def\LOBR{\mbox{\fontspec{STIXGeneral}⟦\,}} %spacing better if ROBR is followed by a thin space
\def\TLOBR{\mbox{\fontspec{STIXGeneral}\,⟦\,}} % LOBR preceded by thin space, useful in math mode
\def\ROBR{\mbox{\fontspec{STIXGeneral}\,⟧}} %spacing better if ROBR is preceded by a thin space
\def\LROB{\mbox{\fontspec{STIXGeneral}⟦⟧}}
\def\LOPR{\mbox{\fontspec{Zapf Dingbats}❮}} % brackets for opr parameters in RTTI
\def\ROPR{\mbox{\fontspec{Zapf Dingbats}❯}}
\def\SNOW{\mbox{\fontspec{Code2000}☃}}
\def\HVYX{\mbox{\fontspec{Zapf Dingbats}✖}}
\def\HVYP{\mbox{\fontspec{Zapf Dingbats}✚}}
\def\GEAR{\mbox{\fontspec{Apple Symbols}⚙}}
\def\UNDX{\mbox{\fontspec{Apple Symbols}☝}}
\def\RNDX{\mbox{\fontspec{Zapf Dingbats}☞}}
\def\NVLP{\mbox{\fontspec{STIXGeneral}✉}}
\def\DNGR{\mbox{\fontspec{STIXGeneral}⚠}}

\def\PAWN{\mbox{\fontspec{Code2000}♙}}

\def\RARW{\mbox{\fontspec{STIXGeneral}→}}
\def\CPYRT{\mbox{\fontspec{STIXGeneral}©}} % \copyright blank in luatex, why???

\newcommand{\fit}[1]{\mbox{\it #1}}
\newcommand{\jvm}[1]{{\tt #1}}
\newcommand{\ftt}[1]{{\tt\slshape{#1}}}

\def\SELF{\fit{self}}
\def\OBJ{\fit{Object}}
\def\obj{\fit{object}}
\def\ext{\fit{extends}}
\def\zz32{\mathbb{Z}_{32}}
\def\BS{\textbackslash}

\makeatletter
\newcommand{\slantverbatim}{\def\verbatim@font{\slshape\ttfamily\hyphenchar\font\m@ne\@noligs}}
\makeatother

\author{David Chase and Karl Naden}
%\date{}                                           % Activate to display a given date or no date
%\setmainfont{Georgia}
\begin{document}
\maketitle
%\section{}
%\subsection{}


% Footnote hack from http://www.latex-community.org/forum/viewtopic.php?f=5&t=2768
\makeatletter{\renewcommand*{\@makefnmark}{}
\footnotetext{Copyright {\normalsize\CPYRT}\ 2010, 2011 Sun Microsystems, Inc.}\makeatother}

The following documents the naming conventions used in the Fortress compiler and explains the meaning
and rationale behind the design.

\section{Closures}

Top level functions can be passed around as first-class values.  In order to support this functionality, the compiler
generates a reference to an oddly-named class that triggers dynamic code generation of a wrapper class that forwards to the 
referenced function.

The general form of the class name is $\mbox{\it packageClass}\,\$\mbox{\it function}\,\NVLP\$\mbox{Arrow}\LOBR{}{\mbox{\it domain}};{\mbox{\it range}}\ROBR$ where
\begin{itemize}
\item {\it packageClass} is the Java package and class derived from the Api or Component containing the top level function.
\item {\it function} is the name of the function itself.
\item {\it domain} is the translation of the function's domain; \SNOW if it is void "()", a single translated type for a single input, or a tuple
	type if there are multiple parameters.
\item {\it range} is the translation of the function's range
\end{itemize}
For example, a non-call reference to
\begin{verbatim}
printlnZZ(n:ZZ32):()
\end{verbatim}
defined in component C6 results in a load of static field ``closure" from
\begin{center}
\small{\verb+C6$printlnZZ+\NVLP\verb+$Arrow⟦com/sun/fortress/compiler/runtimeValues/FZZ32,+\SNOW\ROBR}
\end{center}
Dangerous-characters-mangled, this is
\begin{center}
\small{\verb+C6$printlnZZ+\NVLP\verb+$\=Arrow⟦com\|sun\|fortress\|compiler\|runtimeValues\|FZZ32,+\SNOW\ROBR}
\end{center}
Here, the package is empty (top level) because the Api name is undotted.  The class name is \jvm{C6}, and the function name is \jvm{println}.
\jvm{com/sun/fortress/compiler/runtimeValues/ZZ32}\ is the Java/JVM type for Fortress \ftt{ZZ32}, and the function returns void, so the range is ``\SNOW".

The Fortress classloader checks for an occurrence of ``\NVLP" in a requested class, and if one is found, then a closure class is synthesized based on the information in the name itself.

\section{Generic functions}
Generic function references are also compiled into references to an oddly named class.  The generic function class is also a closure.
The form of the generated name is
%
\begin{center}
{\it packageClass}\,\GEAR\${\it function}\LOBR{\it static\_arguments}\ROBR\NVLP\$\HVYP{}Arrow\LOBR{}{\it domain},{\it range}\ROBR
\end{center}
%
Unlike closures, the class resulting from a reference is not synthesized from just its name, 
but is instead created by expanding a template ``class"  with the {\it static arguments} which
represent the instantiation of the function's static type parameters.
The template is stored in a ``.class" file with basename 
%
\begin{center}
{\it packageClass}\,\GEAR\${\it function}\LROB\NVLP\$\HVYP{}Arrow\LOBR{}{\it domain},{\it range}\ROBR
\end{center}
%
The \fit{packageClass}\ and \fit{function}\ are the same as in closures.
The \HVYP\ character serves a special purpose; it blocks further translation within a name during generic expansion by the classloader.  
The arrow schema serves to disambiguate different generic functions overloading the same name.  

Because function names can be overloaded the {\it domain} and {\it range} must be formulated in order to ensure
that each function declaration is given a unique name. The domain and range as defined for closures could include static
parameters which are not necessarily unique between function declarations.  For instance the declarations from a package P
%
\begin{verbatim}
f[\T extends Object\](x:T):() = ...
f[\T extends ZZ32\](x:T):() = ...
\end{verbatim}
%
constitute a valid overloading, but both would generate the a template file with name
%
\begin{center}
P\,\GEAR\$f\LROB\NVLP\$\HVYP{}Arrow\LOBR{}T,\SNOW\ROBR
\end{center}
%
Therefore, mentions of static parameters are replaced with their upper bounds in domain and range.  The overloading
rules ensure that no two functions can have the same arrow type using this translation as otherwise they would both be equally specific.
For instance, the second declaration above would generate a template file with the non-mangled name

\begin{center}
P\,\GEAR\$f\LROB\NVLP\$\HVYP{}Arrow\LOBR{}com/sun/fortress/compiler/runtimeValues/FZZ32,\SNOW\ROBR
\end{center}

A second file with suffix ``.xlation" contains info about the static parameters, including their names, whether they are types, oprs, or nats, and their variance relation with the generic type.

****TODO: update the following 2 paragraphs****

The purpose of the oddly named container file and the non-rewritten schema is to help work around the lack of a linker in the current implementation, and to simplify linking when one is finally needed.  Ideally (for the code generator and classloader, if not for error reporting and humans trying to read generated code) static parameter names would be normalized at both declaration and reference, but currently they are not.  Lacking that, the Fortress classloader needs a way to find the template for a given generic function.  It does this by stripping out the static parameters and using the name that results, together with the schema for the function.  The schema is not rewritten, and thus matches between definition and use sites (assuming some consistency in static parameter names between the api's declaration and the components definition of the same generic function).  Note that the schema has no particular semantic meaning; it merely serves to disambiguate multiple generic functions with the same name, and it could in principle be the source location in the api declaring the function (or component, if it is not exported to an api).

The Fortress classloader checks for an occurrence of ``\GEAR'' to detect generic functions.  This check occurs before the closure test.

Note that closure bits for the generic function are not synthesized at run time, but instead come from part of the generic template.
(However, the same code stamps out templates and synthesizes closures, it is simply called from different contexts).

\subsection{Modifications to incorporate overloading of generics}
When an overloaded function includes dispatch to (and instantiation of) a generic function, that will require two things.  First, there will need to be (1) a symbolic instantiator (similar to what is done for generic methods), (2) a dispatch cache for the particular function (like generic methods), and (3) the generic that is instantiated must include an \jvm{apply}\ method in which type parameters have been erased, followed by casts to the instantiated types.  This is so because the symbolic reference, being symbolic, cannot include a cast to a non-constant (symbolic) type.  Because generic functions are also code generated as closures, the erased version of apply exists.

\section{Functional methods}

A functional method in a trait or object is promoted to a top-level trampoline function, and the method name is rewritten by appending $\mbox{\RNDX}K$, where $K$ is the index of \SELF\ in the parameter list of the functional method.  For example, the method f in trait T below
\begin{verbatim}
trait T
  f(x:String, self, y:ZZ32):String
end
\end{verbatim}
is rewritten to the new method \verb+f+\RNDX\verb+2(x:String, y:ZZ32):String+ and the top level function is almost-as-if\\
\\
\verb+f(x:String, renamed_self:T, y:ZZ32):String = renamed_self.f+\RNDX\verb+2(x,y)+\\

What prevents this from being the exact translation, is the possibility of Fortress ASIF being applied to arguments to the functional method; for dispatch purposes, this alters the apparent type of a value, but the alteration only lasts for a ``single" dispatch.  In the case of functional methods, the altered type is in force until the body of the final method (after any method overloading) is reached.

\section{Renaming of overloaded functions}

Within any component, there may be overloaded functions.  In some cases, the overloaded function may have the same domain as one of the functions in the overload.  When this occurs, the overloaded function gets the canonical name that the function will be known by if it is exported, and the name of the function in the overload that clashes is mangled by appending \jvm{\PAWN}\ (Unicode White Chess Pawn) to its name.  It is correct, though not necessarily efficient, to refer to the overloaded function even when it is known that only the single function is applicable, because dispatch will still arrive at the proper place.  Within the overloaded dispatch itself, however, the distinction matters.

Note that, because no top-level function in an overloading is allowed to be more specific than a functional method, this single-mangling will never apply to a trampoline function for a functional method.  (The overloading can be resolved at the method level.)  {\bf
This is not true for the new overloading story; this needs to be corrected.}

{\it Note that there may be a problem with methods and ASIF, if we lazily generate dispatch code for ASIF references, because we cannot lazily add methods to a class. }

\section{Functional methods of generic traits and objects}




\section{Generic methods}

Because the JVM does not allow new methods to be added to an already-loaded class, generic methods cannot be template-expanded
in the same fashion as generic traits can be.  Instead, the body of a generic method is compiled into a separate template class
similar to a generic function.  The method is first converted into a function by prepending a parameter for self.  Since the actual type
of self will be determined dynamically, a static parameter for the type of self is also added to the type parameter list.

A lookup method is created that returns a particular instantiation of the generic method.
The lookup method has the same name as the generic method, but always takes two parameters, the first a hashcode of a string, 
and the second parameter that string which represents the static arguments for the instantiation.  The lookup
method will load the class representing the particular instantiation and return an instance of it.  In addition, a table is used to cache previously loaded
instantiations.  In Java, the generated code would follow the form:
\begin{verbatim}
private static BAlongTree sampleLookupTable = new BAlongTree();
public Object sampleLookup(long hashcode, String signature) {
    Object o = sampleLookupTable.get(hashcode);
    if (o == null) 
         o = RTHelpers.
             findGenericMethodClosure(
                  hashcode,
                  sampleLookupTable,
                  "template_class_name",
                  signature);
    return o;
}
\end{verbatim}

To make the JVM happy when using the returned closure object, it must be cast to an arrow type with an apply method.
Since it is possible that the runtime instantiation may be different (more specific) than the static instantiation,
the exact type of the arrow is not known statically.  However, since the number of parameters
(including the receiver) that the closure must take is know, the closure can be cast to the arrow type
with the correct number of parameters of type {\tt Object} that also returns an {\tt Object}.  Then the apply
method will take care of casting the parameters correctly and all that is required from the client
is to cast the result to the expected static type (which is guaranteed by the type system to succeed).\footnote{A potential
optimization is to make the lookup method return the Objectified Arrow.}

The calling convention for a generic method closure is as follows:
\begin{enumerate}
\item evaluate the object expression
\item \jvm{DUP}\ the object value
\item push the hashcode and string for the static-self-prepended generic arguments (Note that in a generic context, these generic arguments may mention arguments bound in an outer instantiation; in that case special methods are used that are replaced with the appropriate instructions when the outer generic is template-instantiated)
\item invoke the lookup method name
\item cast the returned Java Object to the appropriate Objectified Arrow type (one additional input parameter)
\item \jvm{SWAP}\ closure and object
\item evaluate arguments
\item invoke closure
\item cast result to return type
\end{enumerate}

In order to find the closure template class, the helper method \verb+findGenericMethodClosure+ method of \verb+RTHelpers+ must be given
the {\it template\verb+_+class\verb+_+name}.  The name generated by the compiler follows the same pattern as for generic functions.  The only difference
is that the function name is replaced by the string \verb+T+\UNDX\verb+m+.  
As an example, suppose that component \verb+C16+ contains a trait \verb+T+ with a 
generic method \verb+f+\LOBR\verb+S+\ROBR\ mapping \verb+S+ to \verb+Foo+.
\begin{verbatim}
component C16
trait Foo end
trait T
  f[\S\](s:S):Foo
end
\end{verbatim}
The name of the template file\footnote{The current compiler passes the class name of the template,
which includes the names of the static parameters so that each instantiation will have a unique class name, eg 
{\tt C16}\GEAR{\tt \$T}\UNDX{\tt f}\LOBR\UNDX{\tt ,S}\ROBR\NVLP{\tt \$}\HVYP{\tt Arrow}\LOBR{\tt S,C16/Foo}\ROBR.
However, this extra information is immediately discarded and could be removed as an optimization.}
 will be
\\
\verb+C16+\GEAR\verb+$T+\UNDX\verb+f+\LROB\NVLP\verb+$+\HVYP\verb+Arrow+\LOBR\verb+S,C16/Foo+\ROBR
\\
or with dangerous-characters mangling
\\
\verb+C16+\GEAR\verb+$T+\UNDX\verb+f+\LROB\NVLP\verb+$\=+\HVYP\verb+Arrow+\LOBR\verb+S,C16\|Foo+\ROBR

{\it This would be a great place for an example invocation.}

\section{Special methods}

Certain methods are rewritten to loads of long and String constants when the templates containing them are instantiated.  This is necessary to correctly rewrite invocations of generic methods occurring in a template (generic) class.

These are encoded by use of a magic class name (``CONST'') and whatever constant encodings we decide we need; in principle, we could embed a Lisp interpreter here.  Currently, the encoded method names are \verb+String.+{\it stringValue} and \verb+hash.+{\it stringValue} (yes, the capitalization is inconsistent, sigh).

\section{Generic methods of generic traits and objects}

The calling convention for a generic method of a generic trait/object is the same as the convention for calling a generic method of a plain trait/object.  It has to be, since a generic trait could extend or be extended by a non-generic trait supplying the same generic method.  The (generic) trait provides a lookup method, and the lookup method provides an appropriate closure, which must be cast in the same way before invocation.

The only interesting difference is that the generated closure will have a concatenated set of generic parameters; first the static self type, then the method static parameters, finally the trait static parameters (this is different from the order used for functional method declarations in a double-generic context, the goal is to cut down on annoying parsing of generic parameter lists).  The hash code and generic method parameters passed to the lookup method will be the same, so the extra static parameter concatenation occurs within an extended closure lookup method.

\begin{verbatim}
private static BAlongTree sampleLookupTable = new BAlongTree();
public Object sampleLookup(long hashcode, String signature) {
    Object o = sampleLookupTable.get(hashcode);
    if (o == null) 
         o = InstantiatingClassloader.
             findGenericMethodClosure(
                  hashcode,
                  sampleLookupTable,
                  "template_class_name",
                  signature,
                  trait_signature);
    return o;
}
\end{verbatim}

Note that the hashcode and index strings do not depend on the trait signature; it is used only in locating the correct closure template to instantiate.

\section{Functional generic methods}

The forwarding function created for a functional generic method is itself a generic function, following the same naming conventions for a top level generic function of that name.  The dotted method invocation within the forwarding function, follows the recipe for invocation of a dotted generic method.

{\it It appears that the code generator, as of revision 4668, is failing to add the finger-index annotation to generic functional methods.}

\section{Overloaded function dispatch including generic functions}

Initially, the (revised) model for overloaded function dispatch is to topologically sort the member functions from more-to-less specific, and iteratively attempt to match the functions, choosing the first winner.  Static analysis ensures that the last function in the list needs no testing; it must match.  This approach is taken so that the details of ``more specific than" are centralized in static analysis.  The first attempts to optimize this will merely look for CSEs among the constraint checks, and perhaps modify the order of constraint checking in order to perform common tests first.

The generated code for an overloaded function must perform three subtasks.  First, it must verify constraints.  This is one place where mistakes can be made; the code generator needs to be sure that no constraint is overlooked.  In this overloading, the first function is more specific.  
{\slantverbatim
\begin{verbatim}
f[\T, U extends ZZ32\](t:T, u:U):U
f[\T, U \](t:T, u:U):U
\end{verbatim}
}
This can get more complex.  Consider this overloading
{\slantverbatim
\begin{verbatim}
f[\T, U extends ZZ32\](t1:T, t2:T, u1:U, u2:U):U
f[\T, U \](t1:T, t2:T, u1:U, u2:U):U
\end{verbatim}
}
Here, $T$ is the join (more general) of \ftt{t1} and \ftt{t2}, and \ftt{U} is the join of \ftt{u1} and \ftt{u2}.  The first function applies whenever the second (\ftt{U}) join is more-specific-than-or-equal to \ftt{ZZ32}.  Note, however, that determining applicability does not require computing the joins; if the types of \ftt{u1} and \ftt{u2} are both extensions of \ftt{ZZ32}, then their join is also an extension of \ftt{ZZ32}.

The second task (itself a subtask of the first, sometimes) is to infer static parameters.  Different contexts will be either invariant, covariant, or contravariant; invariant is easiest (if a parameter $x$ has declared type \ftt{Invar[\BS T\BS]}, then if $x$ is some sort of \ftt{Invar}, then $T$ must be exactly that \ftt{Invar}'s static parameter).  Top level parameter types are covariant, arrow ranges are covariant, and arrow domains are contravariant.

The third task (for generic targets) is to symbolically instantiate and invoke the targeted generic.  This is similar to the technique used for generic methods, but slightly different.  Generic function implementations must be enhanced in two ways; they must provide an entrypoint supporting symbolic instantiation with type parameters and hashcode (similar to generic methods), and the closure returned must supply an apply method where the type parameters of the function have been erased (because the invoking can only cast as far as the erased type).  That apply method will contain the necessary casts, forwarding to the ``real" apply method.

To infer static parameters, it will be necessary to add additional Fortress-specific type reflection operations.
For any generic trait or object \ftt{TO[\BS U1, U2, U3\BS]+}, the implemented ``Fortress type" (distinct from the Java class of its implementation) must supply methods to obtain the first, second, and third static parameter types.  The Fortress type must also supply methods, indexed by erased type, of the Fortress types that it extends (these will be constants if the extended type is not generic).  (Implementations of) Fortress objects must supply a method to return their Fortress type.

What methods should each Fortress type $T$ support?
\begin{description}
\item[\jvm{asSUPER\#N()}] For each generic trait {\it SUPER} that $T$ extends, retrieve the type for the $N$th static parameter of {\it SUPER} 
\end{description}
Each generic type will include a table-backed static factory called \verb+factory+\ (parameters are Fortress types for the static parameters of the generic).  Non-generic types will have a static member \verb+ONLY+.  

For purposes of dispatch, we need additional methods, and we need a system of naming these types.  Suppose that the function whose contraints we are verifying has this signature, where List and Set are invariant in T.
{\slantverbatim
\begin{verbatim}
f[\T extends Number\](l:List[\Set[\T\]\], a:T, b:T):List[\T\]
\end{verbatim}
}
Assuming that the arguments l, a, and b have actual types L, A, and B, the following conditions must all be satisfied:
\begin{itemize}
\item \verb+L <: List[\?\]+
\item \verb+L.asList#1() = Set[\?\]+
\item \verb+T <: Number+ where \verb+T = L.asList#1().asSet#1()+
\item \verb+A <: T+ where \verb+T = L.asList#1().asSet#1()+
\item \verb+B <: T+ where \verb+T = L.asList#1().asSet#1()+
\end{itemize}
One problem here is that this sequence of tests does not follow obviously from the signature.
If the first (l) parameter were missing, the constraint on T is that it be equal to the join of A and B; however, because the first parameter produces an invariant constraint, we know the type that T must be.  Note that the actual constraints are that both A and B extend T, and that T be the most-specific type with this property.

To compile the queries, note that sometimes no value of a type is available for querying in the style of Java \verb+instanceof+ (this is the motivation for \ftt{List[\BS Set[\BS ...\BS]\BS]}\ -- it is possible to ask parameter l if it is an instance of List-something, but to what should the instance-of-Set-something query be directed?)

Another thing to note is that if we pursue this implementation, it bakes in the restriction that a generic stem can only appear once in the transitive-extends of a class.  If \jvm{asList\#1} returns a single type instead of a set of types, then \jvm{List} can only appear once in the transitive extends.

Use of instanceof for stem queries suggests that the type system will take the form of two parallel interface and interface-implementation type hierarchies.

For sake of example, suppose that \ftt{Array[\BS T\BS]} extends both \ftt{Map[\BS NN, T\BS]} and \ftt{List[\BS T\BS]}.  In the generated Java, the run-time type information will include three interfaces
\begin{description}
\item[\jvm{Array\$RTTIi}] extends \jvm{RTTI}, \jvm{Map\$RTTIi} and \jvm{List\$RTTIi}. Declares methods:\\
\jvm{asArray\#1}
\item[\jvm{Map\$RTTIi}] extends \jvm{RTTI}.  Declares methods:\\
\jvm{asMap\#1}\\
\jvm{asMap\#2}
\item[\jvm{List\$RTTIi}] extends \jvm{RTTI}.  Declares methods\\
\jvm{asList\#1}
\end{description}

and three interface implementations
\begin{description}
\item[\jvm{Array\$RTTIc}] implements \jvm{Array\$RTTIi}
{
\begin{verbatim}
constructor and fields Array$RTTIc(t1 RRTI)
asArray#1() = t1
final Map$RTTIc map = new Map$RTTIc(ZZ32$RTTIc.only(), t1);
final List$RTTIc list = List$RTTIc(t1);
asMap#1() = map.asMap#1()
asMap#2() = map.asMap#2()
asList#1() = list.asList#1()
\end{verbatim}
}
\item[\jvm{Map\$RTTIc}] implements \jvm{Map\$RTTIi}
{
\begin{verbatim}
constructor and fields Map$RTTIc(t1 RRTI, t2 RTTI)
asMap#1=t1
asMap#2=t2
\end{verbatim}
}
\item[\jvm{List\$RTTIc}] implements \jvm{List\$RTTIi}
{
\begin{verbatim}
constructor and fields List$RTTIc(t1 RRTI)
asList#1() = t1
\end{verbatim}
}
\end{description}

It appears that the extends list for a type must be initialized lazily, because the type itself can appear there, and also that each generic type instantiation must be unique (because of cycles through the extends list).  All of this may occur in a multi-threaded context, so (in order to obey Java memory model rules) some of these fields (in particular, the extends items) must be volatile so that double-checked locking will work for initialization, and no locking will work for reading.

\subsection{Implementation notes}
It might be simpler to dispense with the FRTTI base type; if it has no methods, what's the point?

It looks like the names of the interface and class will be derived from the name, followed by ``\$RTTIi" and ``\$RTTIc".

The Java class created for each object must include a static member containing the type and instance method returning that type.

What do we do about self types?  We need to be sure that cycles are ok.  Consider
{\slantverbatim
\begin{verbatim}
ZZ32 extends BinOp[\ZZ32, PLUS\]
\end{verbatim}
}
What this means is that the naively specified constructors simply will not work; the constructor needs to take a thunk that will evaluate to a type, instead.  Or, the type needs to defer creation of its supertypes.
\begin{verbatim}
class ZZ32.RttiC implements ZZ32.RttiI {
  static public ZZ32.RttiC only = new ZZ32.RttiC();
  Map.RttiC _BinOp;
  Map.RttiC asBinOp() {
     if 
  }
}
\end{verbatim}

Note, also, that we think we DO allow extension of multiple instantiations of the same generic with different opr parameters, for example
{\slantverbatim
\begin{verbatim}
ZZ32 extends { BinOp[\ZZ32, PLUS\], BinOp[\ZZ32, TIMES\] }
\end{verbatim}
}
A corollary of this is that it should not be necessary to infer an opr parameter dynamically (because this would be ambiguous, at least using the current approach).
The names of the RTTI interface and class for \ftt{BinOp[\ T, PLUS\BS]} will be \jvm{BinOP\LOPR PLUS\ROPR\$RTTIi} and \jvm{BinOP\LOPR PLUS\ROPR\$RTTIc}.  (The heavy angle brackets are Unicode 276e and 276f in the Dingbats character group; they are not mathematical operators.)  The class and interface will declare a method \jvm{asBinOp\LOPR PLUS\ROPR\#1} to return their type parameter \jvm{T}.  Thus, \jvm{ZZ32\$RTTIc} will implement both \jvm{BinOP\LOPR PLUS\ROPR\$RTTIi} and \jvm{BinOP\LOPR TIMES\ROPR\$RTTIi} and supply methods \jvm{asBinOp\LOPR PLUS\ROPR\#1} and \jvm{asBinOp\LOPR TIMES\ROPR\#1}.

\subsection{Fortress object getters}

Each Fortress Object (perhaps not Any, meaning not Tuples, not Arrows) will have a method \verb+getRTTI()+.  For an object with type O, this will simply push 


\section{Odds and ends}

One thing to be aware of is that covariant type relationships (not part of Fortress yet except for Arrow and Tuple types) may be costly if mapped directly onto Java types.  Covariant generics with Any-bounded static parameters (for example,  \ftt{AnyList[\BS\ T extends Any \BS]} will create special headaches because Tuple and Arrow type relationions are far too large to encode into the Java type system.  One possibility will simply be to forbid that case; otherwise, it can be implemented by wrapping values, and using a method to implement strict equality, not just a pointer comparison. 

\end{document}

