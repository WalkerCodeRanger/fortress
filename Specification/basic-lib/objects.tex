%%%%%%%%%%%%%%%%%%%%%%%%%%%%%%%%%%%%%%%%%%%%%%%%%%%%%%%%%%%%%%%%%%%%%%%%%%%%%%%%
%   Copyright 2009, Oracle and/or its affiliates.
%   All rights reserved.
%
%
%   Use is subject to license terms.
%
%   This distribution may include materials developed by third parties.
%
%%%%%%%%%%%%%%%%%%%%%%%%%%%%%%%%%%%%%%%%%%%%%%%%%%%%%%%%%%%%%%%%%%%%%%%%%%%%%%%%

\chapter{Objects}
\chaplabel{lib:object}

\Trait{Fortress.Core.Any}

\tracingcommands=1\tracingmacros=1

The trait \TYP{Any} is the single root of the type hierarchy; every
object in Fortress has trait \TYP{Any} and therefore every object
implements the methods of this trait.  The immediate subtypes of the trait
\TYP{Any} are \TYP{Object}, \TYP{Tuple}, and \TYP{()}.

%trait Any
%  getter hashCode(): NN64
%  opr ===(self, other: Any): Boolean
%  opr IDENTITY(self): Any
%  hash(maxval: NN64): NN64
%  hash(maxval: NN32): NN32
%  toString(): String
%  property FORALL (x, y, n: NN64) x === y IMPLIES x.hash(n) === y.hash(n)
%  property FORALL (x, y, n: NN32) x === y IMPLIES x.hash(n) === y.hash(n)
%  property FORALL (x) x.hashCode === x.hash(2^64-1)
%  property FORALL (x, y) x === y IMPLIES x.toString() = y.toString()
%  property FORALL (a) (IDENTITY a) === a
%  property FORALL (a) a===a
%  property FORALL (a,b) a===b IFF b===a
%  property FORALL (a,b,c) (a===b AND: b===c) IMPLIES: a===c
%end
\begin{Fortress}
\(\KWD{trait}\mskip 4mu plus 4mu\TYP{Any}\)\\
{\tt~~}\pushtabs\=\+\(  \KWD{getter}\mskip 4mu plus 4mu\VAR{hashCode}()\COLON \mathbb{N}64\)\\
\(  \KWD{opr} \mathord{\sequiv}(\KWD{self}, \VAR{other}\COLON \TYP{Any})\COLON \TYP{Boolean}\)\\
\(  \KWD{opr} \mathord{\OPR{IDENTITY}}(\KWD{self})\COLON \TYP{Any}\)\\
\(  \VAR{hash}(\VAR{maxval}\COLON \mathbb{N}64)\COLON \mathbb{N}64\)\\
\(  \VAR{hash}(\VAR{maxval}\COLON \mathbb{N}32)\COLON \mathbb{N}32\)\\
\(  \VAR{toString}()\COLON \TYP{String}\)\\
\(  \KWD{property} \forall (x, y, n\COLON \mathbb{N}64)\; x \sequiv y \rightarrow x.\VAR{hash}(n) \sequiv y.\VAR{hash}(n)\)\\
\(  \KWD{property} \forall (x, y, n\COLON \mathbb{N}32)\; x \sequiv y \rightarrow x.\VAR{hash}(n) \sequiv y.\VAR{hash}(n)\)\\
\(  \KWD{property} \forall (x)\; x.\VAR{hashCode} \sequiv x.\VAR{hash}(2^{64}-1)\)\\
\(  \KWD{property} \forall (x, y)\; x \sequiv y \rightarrow x.\VAR{toString}() = y.\VAR{toString}()\)\\
\(  \KWD{property} \forall (a)\; (\OPR{IDENTITY}\mskip 4mu plus 4mu{a}) \sequiv a\)\\
\(  \KWD{property} \forall (a)\; a\sequiv{}a\)\\
\(  \KWD{property} \forall (a,b)\; a\sequiv{}b \leftrightarrow b\sequiv{}a\)\\
\(  \KWD{property} \forall (a,b,c)\; (a\sequiv{}b \wedge\COLON b\sequiv{}c) \rightarrow\COLON a\sequiv{}c\)\-\\\poptabs
\(\KWD{end}\)
\end{Fortress}


%  getter hashCode(): NN64
\Method{\EXP{\KWD{getter} \VAR{hashCode}()\COLON \mathbb{N}64}}

Every object has associated with it a 64-bit unsigned integer value called its \emph{hash code};
this is the value returned by the \VAR{hash} method when given the argument \EXP{2^{64}-1}.
Hash codes are not necessarily consistent from one Fortress application to another, nor from
one execution of a Fortress application to another execution of the same application,
but remain fixed during the execution of a single Fortress application.
It is permitted for two objects to have the same hashCode, but Fortress programmers and
implementors should be aware that assigning distinct hash codes to distinct objects
may improve the performance of hash tables.

The trait \TYP{Any} defines its \VAR{hash} methods in terms of \VAR{hashCode};
therefore it suffices for a subtrait to override \VAR{hashCode} to get the benefit
of the \VAR{hash} methods as well.


%  opr ===(self, other: Any): Boolean
\Method{\EXP{\KWD{opr} \mathord{\sequiv}(\KWD{self}, \VAR{other}\COLON \TYP{Any})\COLON \TYP{Boolean}}}

The infix operator \EXP{\sequiv} (object equivalence) is used to
decide whether two objects are ``the same object'' in the strictest
sense possible; this is described in detail in \secref{equivalence}.
Note that the properties of trait \TYP{Any} assert that \EXP{\sequiv}
is an equivalence relation.  In \TYP{Object} this property is asserted abstractly using the algebraic constraints of \chapref{lib:algebraic-constraints}.

For \EXP{\not\sequiv} \see{operator-NSEQV}.

%  opr IDENTITY(self): Any
\Method{\EXP{\KWD{opr} \mathord{\OPR{IDENTITY}}(\KWD{self})\COLON \TYP{Any}}}

The operator \OPR{IDENTITY} simply returns its argument.  (This may not be
terribly useful for applications programming, but it has technical uses for
specifying contracts and algebraic properties in libraries as described in
\secref{opr-properties}.)


%  hash(maxval: NN64): NN64
%  hash(maxval: NN32): NN32
\Method{\EXP{\VAR{hash}(\VAR{maxval}\COLON \mathbb{N}64)\COLON \mathbb{N}64}}
\Method*{\EXP{\VAR{hash}(\VAR{maxval}\COLON \mathbb{N}32)\COLON \mathbb{N}32}}

The \VAR{hash} method returns a \emph{hash value} for the object as an
unsigned integer that is less than or equal to the \VAR{maxval} argument.
This hash value is not necessarily consistent from one Fortress
application to another, nor from one execution of a Fortress
application to another execution of the same application, but the hash
value produced for a given value of the \VAR{maxval} argument remains
fixed during the execution of a single Fortress application.  There is
no defined relationship between hash values produced for the same
object but with different \VAR{maxval} values.  Fortress programmers and
implementors should be aware that the performance of hash tables is
likely to be improved if, for any given collection of objects and
given value for the \VAR{maxval} argument, the \VAR{hash} method
assigns hash values to those objects with relatively uniform distribution.


%  toString(): String
\Method{\EXP{\VAR{toString}()\COLON \TYP{String}}}

The general contract of \VAR{toString} is that it returns a string that
``textually represents'' this object. The idea is to provide a concise but
informative representation that will be useful to a person reading it.


\Trait{Fortress.Core.Object}

The trait \TYP{Object} is the root of the type hierarchy for all
user-constructed objects and functions.
The \TYP{Object} type does not
have any additional methods, but uses algebraic constraints (see
\chapref{lib:algebraic-constraints}) to describe the properties of the
existing operators more abstractly.

%trait Object extends { Any, EquivalenceRelation[\Object,===\], IdentityOperator[\Object\] }
%    excludes { Tuple }
%end
\begin{Fortress}
\(\KWD{trait}\mskip 4mu plus 4mu\TYP{Object} \KWD{extends} \{\,\TYP{Any}, \TYP{EquivalenceRelation}\llbracket\TYP{Object},\sequiv\rrbracket, \TYP{IdentityOperator}\llbracket\TYP{Object}\rrbracket\,\}\)\\
{\tt~~~~}\pushtabs\=\+\(    \KWD{excludes} \{\,\TYP{Tuple}\,\}\)\-\\\poptabs
\(\KWD{end}\)
\end{Fortress}


\Trait{Fortress.Core.Tuple}
\seclabel{core-tuple}

The trait \TYP{Tuple} is the root of the type hierarchy for all tuple types (see \secref{tuple-types}).  No user-defined object can be a subtype of \TYP{Tuple}.  As with the type \TYP{Object}, the type \TYP{Tuple} uses algebraic constraints to describe its existing methods.

%trait Tuple extends { Any, EquivalenceRelation[\Tuple,===\], IdentityOperator[\Tuple\] }
%    excludes { Object }
%  toString(): String
%end
\begin{Fortress}
\(\KWD{trait}\mskip 4mu plus 4mu\TYP{Tuple} \KWD{extends} \{\,\TYP{Any}, \TYP{EquivalenceRelation}\llbracket\TYP{Tuple},\sequiv\rrbracket, \TYP{IdentityOperator}\llbracket\TYP{Tuple}\rrbracket\,\}\)\\
{\tt~~~~}\pushtabs\=\+\(    \KWD{excludes} \{\,\TYP{Object}\,\}\)\-\\\poptabs
{\tt~~}\pushtabs\=\+\(  \VAR{toString}()\COLON \TYP{String}\)\-\\\poptabs
\(\KWD{end}\)
\end{Fortress}


\Method{\EXP{\VAR{toString}()\COLON \TYP{String}}}

The \VAR{toString} method returns a string \EXP{\hbox{\rm``\STR{}''}}
concatenated as follows:
\begin{itemize}
\item If the tuple has just one element, concatenate
\EXP{\hbox{\rm``\STR{tuple}''}}.
\item Concatenate \EXP{\hbox{\rm``\STR{(}''}}.
\item For each element of the tuple, in left-to-right order,
     taking all plain elements before any varargs or keyword
     element, taking any varargs element before any keyword
     element, and taking keyword elements in the order defined
     by the type of the tuple (which may not be the sorted order):
     \begin{itemize}
     \item If this is a plain element,
 concatenate the \VAR{toString} of the element.
     \item If this is a varargs element ``\EXP{e\ldots}'',
 concatenate the \VAR{toString} of \VAR{e} and
 concatenate \EXP{\hbox{\rm``\STR{...}''}}.
     \item If this is a keyword element ``\EXP{\VAR{id} = e}'',
 concatenate the \VAR{toString} of \VAR{id},
 concatenate \EXP{\hbox{\rm``\STR{~=~}''}}, and
 concatenate the \VAR{toString} of \VAR{e}.
     \item If this is the last element of the tuple,
 concatenate \EXP{\hbox{\rm``\STR{)}''}};
          otherwise concatenate ", ".
     \end{itemize}
\end{itemize}
