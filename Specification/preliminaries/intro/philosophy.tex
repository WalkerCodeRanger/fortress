%%%%%%%%%%%%%%%%%%%%%%%%%%%%%%%%%%%%%%%%%%%%%%%%%%%%%%%%%%%%%%%%%%%%%%%%%%%%%%%%
%   Copyright 2009 Sun Microsystems, Inc.,
%   4150 Network Circle, Santa Clara, California 95054, U.S.A.
%   All rights reserved.
%
%   U.S. Government Rights - Commercial software.
%   Government users are subject to the Sun Microsystems, Inc. standard
%   license agreement and applicable provisions of the FAR and its supplements.
%
%   Use is subject to license terms.
%
%   This distribution may include materials developed by third parties.
%
%   Sun, Sun Microsystems, the Sun logo and Java are trademarks or registered
%   trademarks of Sun Microsystems, Inc. in the U.S. and other countries.
%%%%%%%%%%%%%%%%%%%%%%%%%%%%%%%%%%%%%%%%%%%%%%%%%%%%%%%%%%%%%%%%%%%%%%%%%%%%%%%%

The Fortress programming language is a
general-purpose,
statically typed,
component-based
programming language
designed for producing robust high-performance software
with high programmability.

In many ways, Fortress is intended to be a ``growable language'',
i.e., a language that can be gracefully extended
and applied in new and unanticipated contexts.
Fortress supports state-of-the-art compiler optimization techniques,
scaling to unprecedented levels of parallelism and of addressable memory.
Fortress has an extensible component system,
allowing separate program components
to be independently developed, deployed, and linked
in a modular and robust fashion.
Fortress also supports modular and extensible parsing,
allowing new notations and static analyses to be added to the language.

The name ``Fortress'' is derived from the intent
to produce a ``secure Fortran'',
i.e., a language for high-performance computation
that provides abstraction and type safety
on par with modern programming language principles.
Despite this etymology,
the language is a new language
with little relation to Fortran
other than its intended domain of application.
No attempt has been made to support
backward compatibility with existing versions of Fortran;
indeed, many new language features were invented
during the design of Fortress.
Many aspects of Fortress were inspired by other object-oriented and
functional programming languages, including
the Java\texttrademark\ Programming Language~\cite{JLS},
NextGen~\cite{NextGen}, Scala~\cite{Scala}, Eiffel~\cite{Eiffel},
Self~\cite{Self}, Standard ML~\cite{SML}, Objective Caml~\cite{OCaml},
Haskell~\cite{Haskell}, and Scheme~\cite{Scheme}.
The result is a language that employs cutting-edge features
from the programming-language research community
to achieve an unprecedented combination of performance and programmability.

Fortress is an open source project. An initial
interpreter, implementing a core of
the language features presented in this specification, is available
at the Fortress project website:

\begin{verbatim}
http://projectfortress.sun.com
\end{verbatim}

There you will find source code, supporting documents,
and access to discussion groups related to the Fortress project.
