%%%%%%%%%%%%%%%%%%%%%%%%%%%%%%%%%%%%%%%%%%%%%%%%%%%%%%%%%%%%%%%%%%%%%%%%%%%%%%%%
%   Copyright 2009, Oracle and/or its affiliates.
%   All rights reserved.
%
%
%   Use is subject to license terms.
%
%   This distribution may include materials developed by third parties.
%
%%%%%%%%%%%%%%%%%%%%%%%%%%%%%%%%%%%%%%%%%%%%%%%%%%%%%%%%%%%%%%%%%%%%%%%%%%%%%%%%

\chapter{Default Libraries}
\chaplabel{default-libraries}

Two sets of libraries are imported into every Fortress program by
default.  The first, \TYP{FortressLibrary} (\secref{FortressLibrary}),
implements the high-level functionality that will be used by most
Fortress programmers.  It will eventually be possible to selectively
exclude portions of this library from a component if desired.  By
contrast, the second set of libraries are the \emph{builtins}
(\secref{builtins}).  These libraries are intended to encompass
primitive functionality that must be visible in every Fortress
component.

\section{FortressLibrary}
\seclabel{FortressLibrary}
\input{\home/library/apis/FortressLibrary.tex}

\section{Builtins}
\seclabel{builtins}

There is a single library containing builtins:
\TYP{FortressBuiltin} (\secref{FortressBuiltin}).
Every Fortress API and component will unconditionally see the names
exported by this builtin API.

\subsection{FortressBuiltin}
\seclabel{FortressBuiltin}
\input{\home/library/apis/FortressBuiltin.tex}
