%%%%%%%%%%%%%%%%%%%%%%%%%%%%%%%%%%%%%%%%%%%%%%%%%%%%%%%%%%%%%%%%%%%%%%%%%%%%%%%%
%   Copyright 2009 Sun Microsystems, Inc.,
%   4150 Network Circle, Santa Clara, California 95054, U.S.A.
%   All rights reserved.
%
%   U.S. Government Rights - Commercial software.
%   Government users are subject to the Sun Microsystems, Inc. standard
%   license agreement and applicable provisions of the FAR and its supplements.
%
%   Use is subject to license terms.
%
%   This distribution may include materials developed by third parties.
%
%   Sun, Sun Microsystems, the Sun logo and Java are trademarks or registered
%   trademarks of Sun Microsystems, Inc. in the U.S. and other countries.
%%%%%%%%%%%%%%%%%%%%%%%%%%%%%%%%%%%%%%%%%%%%%%%%%%%%%%%%%%%%%%%%%%%%%%%%%%%%%%%%

\chapter{Structure of the Fortress Libraries}
\chaplabel{library-structure}

The Fortress libraries are divided into two basic categories.  The
first, covered in \chapref{default-libraries}, are the libraries that
are automatically imported by every Fortress component and API,
chiefly \TYP{FortressLibrary} and \TYP{FortressBuiltin}.  These
libraries provide the basic numeric types, generators, arrays,
booleans, exceptions, and the like.  The second, covered in
\chapref{optional-libraries}, are libraries that must be explicitly
imported by the programmer.  These include \TYP{List}, \TYP{Set}, and
\TYP{Map}.

The documentation in \partref{library} is largely automatically
generated from the API code for the libraries themselves, and
describes the state of the libraries as of the release date of this
specification.  The libraries are presently in a state of flux, and
programmers using a more recent version of the Fortress implementation
may find differences between the APIs presented here and those found
in the actual implementation.
