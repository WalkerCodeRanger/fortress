%%%%%%%%%%%%%%%%%%%%%%%%%%%%%%%%%%%%%%%%%%%%%%%%%%%%%%%%%%%%%%%%%%%%%%%%%%%%%%%%
%   Copyright 2009, Oracle and/or its affiliates.
%   All rights reserved.
%
%
%   Use is subject to license terms.
%
%   This distribution may include materials developed by third parties.
%
%%%%%%%%%%%%%%%%%%%%%%%%%%%%%%%%%%%%%%%%%%%%%%%%%%%%%%%%%%%%%%%%%%%%%%%%%%%%%%%%

\chapter{Operator Declarations}
\chaplabel{operatordefs}

\note{Multifix operators and keyword parameters are not yet supported.}

An operator declaration may appear anywhere a top-level function or
method declaration may appear.  Operator declarations are like other
function or method declarations in all respects except that an
operator declaration has \KWD{opr} and has
an operator name (see \secref{operator-names} for a discussion of
valid operator names) instead of an identifier.  The precise placement
of the operator name within the declaration depends on the fixity of
the operator.  Like other functionals, operators may have overloaded
declarations (see \chapref{multiple-dispatch} for a discussion of
overloading).  These overloadings may be of the same or differing
fixities.

\begin{Grammar}
\emph{FnDecl}
&::=& \option{\emph{FnMods}} \emph{FnHeaderFront} \emph{FnHeaderClause}
\options{\EXP{=} \emph{Expr}} \\

\emph{FnHeaderFront}  &::=& \emph{OpHeaderFront} \\

\emph{OpHeaderFront}
&::=& \KWD{opr} \option{\KWD{BIG}}
(\{\EXP{\mapsto} $|$ \emph{LeftEncloser} $|$ \emph{Encloser}) \option{\emph{StaticParams}}
\option{\emph{Params}} \\
&&(\emph{RightEncloser} $|$ \emph{Encloser}) \\
&$|$& \KWD{opr} \emph{ValParam}
(\emph{Op} $|$ \emph{ExponentOp}) \option{\emph{StaticParams}} \\
&$|$& \KWD{opr} \option{\KWD{BIG}}
(\emph{Op} $|$ \texttt{\^} $|$ \emph{Encloser} $|$ $\sum$ $|$ $\prod$)
 \option{\emph{StaticParams}} \emph{ValParam} \\

\emph{MdDecl} &::=& \emph{MdDef} \\
&$|$& \option{\KWD{abstract}}
\option{\emph{MdMods}} \emph{MdHeaderFront}
\emph{FnHeaderClause} \\

\emph{MdHeaderFront}  &::=& \emph{OpMdHeaderFront} \\

\emph{OpMdHeaderFront}
&::=& \KWD{opr} \option{\KWD{BIG}}
(\{\EXP{\mapsto} $|$ \emph{LeftEncloser} $|$ \emph{Encloser}) \option{\emph{StaticParams}}
\option{\emph{Params}} \\
&&(\emph{RightEncloser} $|$ \emph{Encloser}) \\
&&\options{\EXP{\ASSIGN} \texttt( \emph{SubscriptAssignParam} \texttt)}
\\
&$|$& \KWD{opr} \emph{ValParam}
(\emph{Op} $|$ \emph{ExponentOp}) \option{\emph{StaticParams}} \\
&$|$& \KWD{opr} \option{\KWD{BIG}}
(\emph{Op} $|$ \texttt{\^} $|$ \emph{Encloser} $|$ $\sum$ $|$ $\prod$)
 \option{\emph{StaticParams}} \emph{ValParam} \\

\emph{SubscriptAssignParam}
&::=& \emph{Varargs}\\
&$|$& \emph{Param}\\
\end{Grammar}

An operator declaration has one of eight
forms:
multifix operator declaration,
infix operator declaration,
prefix operator declaration, postfix operator declaration,
nofix operator declaration,
bracketing operator declaration,
subscripting operator method declaration,
and subscripted assignment operator method declaration.
Each is invoked according to specific rules of syntax.
An operator method declaration must be a functional method declaration,
a subscripting operator method declaration, or a subscripted assignment
operator method declaration.

\section{Multifix Operator Declarations}

A multifix operator declaration has \KWD{opr}
and then an operator name where a
functional declaration would have an identifier.
The declaration must not have any keyword parameters, and must
be capable of accepting at least two arguments.  It is permissible
to use a varargs parameter; in fact, this is a good way to
define a multifix operator.
Static parameters (described in \chapref{trait-parameters})
 may also be present, between the operator and
the parameter list.

An expression consisting of a multifix operator applied to
an expression will invoke a multifix operator declaration.
The compiler considers all multifix operator declarations for that
operator that are both
accessible and applicable, and the most specific operator declaration
is chosen according to the usual rules for overloaded functionals.
The invocation will pass more than two arguments.

A multifix operator declaration may also be invoked
by an infix, a prefix, or nofix (but not a postfix) operator application if
the declaration is applicable.

Example:
\note{Example is commented out because it is not supported nor run by the interpreter yet.}
%\input{\home/advanced/examples/OprDecl.Multifix.tex}

\section{Infix Operator Declarations}

An infix operator declaration has \KWD{opr}
and then an operator name where a
functional declaration would have an identifier.
The declaration must have two value parameter,
which must not be a keyword parameter or varargs parameter.
Static parameters may also be present,
between the operator and the parameter list.

An expression consisting of an infix operator applied to
an expression will invoke an infix operator declaration.
The compiler considers all infix and multifix
operator declarations for that
operator that are both
accessible and applicable, and the most specific operator declaration
is chosen according to the usual rules for overloaded functionals.

Note that superscripting (\txt{{\char'136}}) may be defined using
an infix operator declaration even though it has very high precedence
and cannot be used as a multifix operator.

Example:
\input{\home/advanced/examples/OprDecl.Infix.tex}

\section{Prefix Operator Declarations}

A prefix operator declaration has \KWD{opr}
and then an operator name where a
functional declaration would have an identifier.
The declaration must have one value parameter, which must not be a keyword parameter
or varargs parameter.
Static parameters may also be present, between the operator and the parameter list.

An expression consisting of a prefix operator applied to
an expression will invoke a prefix operator declaration.
The compiler considers all prefix and multifix
operator declarations for that
operator that are both
accessible and applicable, and the most specific operator declaration
is chosen according to the usual rules for overloaded functionals.

Example:
\input{\home/advanced/examples/OprDecl.Prefix.tex}

\section{Postfix Operator Declarations}
\seclabel{postfix-opr-decl}

A postfix operator declaration has \KWD{opr} where a
functional declaration would have an identifier;
the operator name itself \emph{follows} the parameter list.
The declaration must have one value parameter, which must not be a keyword parameter
or varargs parameter.
Static parameters may also be present, between \KWD{opr}
and the parameter list.

An expression consisting of a postfix operator applied to
an expression will invoke a postfix operator declaration.
The compiler considers all postfix operator declarations for that
operator that are both
accessible and applicable, and the most specific operator declaration
is chosen according to the usual rules for overloaded functionals.

Example:
\input{\home/advanced/examples/OprDecl.Postfix.tex}


\section{Nofix Operator Declarations}

A nofix operator declaration has \KWD{opr}
and then an operator name where a functional declaration would have an
identifier.  The declaration must have no parameters.

An expression consisting only of a nofix operator
will invoke a nofix operator declaration.
The compiler considers all nofix and multifix
operator declarations
for that operator that are both
accessible and applicable, and the most specific operator declaration
is chosen according to the usual rules for overloaded functionals.

Uses for nofix operators are rare, but those rare examples are very useful.
For example, if the \EXP{@} operator is used to construct subscripting ranges,
and it is the nofix declaration of \EXP{@} that allows a lone \EXP{@} to be
used as a subscript:
\input{\home/advanced/examples/OprDecl.Nofix.tex}


\section{Bracketing Operator Declarations}
\seclabel{bracketing-opr-decl}

A bracketing operator declaration has \KWD{opr}
where a functional declaration would have an identifier.
The value parameter list, rather than being surrounded by parentheses,
is surrounded by the brackets being defined.  A bracketing operator
declaration may have any number of parameters, keyword parameters,
and varargs parameters in the value parameter list.  Static parameters may
also be present, between \KWD{opr} and the parameter list.  Any paired
Unicode brackets
may be so defined \emph{except} ordinary parentheses
and white square brackets.

An expression consisting of zero or more comma-separated expressions
surrounded by a  bracket pair will invoke a bracketing operator
declaration.  The compiler considers all bracketing operator
declarations for that type of bracket pair that are both accessible and applicable, and the most
specific operator declaration is chosen according to the usual rules for
overloaded functionals.
For example, the expression \EXP{\langle p,q \rangle} might invoke the
following bracketing method declaration:
\input{\home/advanced/examples/OprDecl.Bracketing.tex}
