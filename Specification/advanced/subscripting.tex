%%%%%%%%%%%%%%%%%%%%%%%%%%%%%%%%%%%%%%%%%%%%%%%%%%%%%%%%%%%%%%%%%%%%%%%%%%%%%%%%
%   Copyright 2009, Oracle and/or its affiliates.
%   All rights reserved.
%
%
%   Use is subject to license terms.
%
%   This distribution may include materials developed by third parties.
%
%%%%%%%%%%%%%%%%%%%%%%%%%%%%%%%%%%%%%%%%%%%%%%%%%%%%%%%%%%%%%%%%%%%%%%%%%%%%%%%%

\section{Subscripting Operator Method Declarations}
\seclabel{subscripting-ops}

A subscripting operator method declaration has
\KWD{opr} where a method declaration would have an identifier.
The value parameter list, rather than being surrounded by parentheses,
is surrounded by a pair of brackets.  A subscripting operator method
declaration may have any number of value parameters, keyword parameters,
and varargs  parameters in that
value parameter list.  Static parameters may also be present, between
\KWD{opr} and the left bracket.  Any paired
Unicode brackets may be so defined \emph{except} ordinary
parentheses and white square brackets; in particular, the square
brackets ordinarily used for indexing may be used.

An expression consisting of a subexpression immediately
followed (with no intervening whitespace) by zero or more
comma-separated expressions surrounded by brackets will
invoke a subscripting operator method declaration.  Methods
for the expression preceding the bracketed expression list are
considered.  The compiler considers all subscripting operator method
declarations that are both accessible and applicable, and the most
specific method declaration is chosen according to the usual overloading
rules.  For example, the expression \EXP{\VAR{foo}_p} might invoke the
following subscripting method declaration because expressions in the square
brackets are rendered as subscripts:
\input{\home/advanced/examples/OprDecl.Subscripting.tex}



\section{Subscripted Assignment Operator Method Declarations}
\seclabel{subscripted-assignment}

A subscripted assignment operator method declaration has
\KWD{opr} where a method declaration would have an identifier.  The value
parameter list, rather than being surrounded by parentheses, is
surrounded by a pair of brackets; this is then followed by the
operator \txt{:=} and then a second value parameter list in parentheses,
which must contain exactly one non-keyword
value parameter.  A subscripted
assignment operator method declaration may have any number of value
parameters within the brackets; these value parameters may include
keyword parameters and
varargs parameters.  A result type
may appear after the second value parameter list, but it must be \TYP{()}.
Static parameters may also be present, between \KWD{opr}
and the left bracket.  Any paired Unicode brackets may be so
defined \emph{except} ordinary parentheses and white square brackets; in
particular, the square brackets ordinarily used for indexing may
be used.

An assignment expression consisting of an expression immediately
followed (with no intervening whitespace) by zero or more
comma-separated expressions surrounded by brackets, followed by the
assignment operator \txt{:=}, followed by another expression, will
invoke a subscripted assignment operator method declaration.  Methods
for the expression preceding the bracketed expression list are
considered.  The compiler considers all subscript operator method
declarations that are both accessible and applicable, and the most
specific method declaration is chosen according to the usual overloading
rules.  When a compound assignment operator (described in
\secref{operator-app-expr}) is used with a subscripting operator and
a subscripted assignment operator, for example \EXP{a_{3} \mathrel{+}= k},
both a subscripting operator declaration and a subscripted assignment
operator declaration are required.
For example, the assignment \EXP{\VAR{foo}_p \ASSIGN \VAR{myWidget}} might
invoke the following subscripted assignment method declaration:
\input{\home/advanced/examples/OprDecl.SubscriptedAssignment.tex}




\section{Conditional Operator Declarations}
\seclabel{conditional-operators-impl}

A \emph{conditional operator} is a binary operator (other than `\txt{:}' and
`\txt{::}') that
is immediately followed by `\txt{:}'; see \secref{conditional-operators}.
A conditional operator expression \EXP{x@\COLONOP{}y} is syntactic sugar for
\EXP{x@(\KWD{fn} () \Rightarrow y)}; that is, the right-hand operand
is converted to a ``thunk''
(zero-parameter function) that then becomes the right-hand operand of the
corresponding unconditional operator.  Therefore a conditional
operator is simply implemented as an overloading of the operator that
accepts a thunk.


It is also permitted for a conditional operator to have a preceding as
well as a following colon.  A conditional operator expression
\EXP{x\COLONOP@\COLONOP{}y} is syntactic sugar for \EXP{(\KWD{fn} ()
  \Rightarrow x)@(\KWD{fn} () \Rightarrow y)};
that is, each operand is converted to a thunk.  This mechanism  is
used, for example, to define the results-comparison operator
\txt{:$\sim$:}, which takes exceptions into account.


The conditional $\wedge$ and $\vee$ operators for
boolean values, for example, are implemented as follows:
\input{\home/library/examples/ConditionalOps.tex}



\section{Big Operator Declarations}
\seclabel{big-operators-impl}

A \emph{big operator} such as $\sum$ or $\prod$ is declared as a usual
operator declaration.
See \secref{defining-generators} for an example declaration of a big operator.
A big operator application is either a \emph{reduction expression}
described in \secref{reduction-expr} or a \emph{comprehension}
described in \secref{comprehensions}.
