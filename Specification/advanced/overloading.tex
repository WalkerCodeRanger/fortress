%%%%%%%%%%%%%%%%%%%%%%%%%%%%%%%%%%%%%%%%%%%%%%%%%%%%%%%%%%%%%%%%%%%%%%%%%%%%%%%%
%   Copyright 2009,2010, Oracle and/or its affiliates.
%   All rights reserved.
%
%
%   Use is subject to license terms.
%
%   This distribution may include materials developed by third parties.
%
%%%%%%%%%%%%%%%%%%%%%%%%%%%%%%%%%%%%%%%%%%%%%%%%%%%%%%%%%%%%%%%%%%%%%%%%%%%%%%%%

\chapter{Overloaded Functional Declarations}
\chaplabel{overloaded-declarations}

\note{Keyword and varargs parameters are not yet supported.}

\note{Return types are required for overloaded functional declarations.}

Fortress allows multiple functional declarations to be in scope of a
particular program point.  We call this overloading.
\chapref{multiple-dispatch} describes
how to determine which overloaded declarations are
applicable to a particular functional call, and when several
are applicable, how to select the most specific one among them.
In this chapter, we give a
set of rules on overloaded declarations that guarantee there
exists a most specific declaration for any given functional call.
These rules are complicated by the presence of coercion, which may
enlarge the set of declarations that are applicable to a functional
call, as discussed in \chapref{conversions-coercions}.



\section{Principles of Overloading}
\seclabel{principles-overloading}

Fortress allows multiple functional declarations of the same name to
be declared in a single scope.  However, recall from \chapref{names}
the following shadowing rules:
\begin{itemize}
\item
dotted method declarations shadow top-level function declarations with
the same name, and
\item
dotted method declarations provided by a trait or object declaration
or object expression
shadow functional method declarations with the same name that are
provided by a different trait or object declaration
or object expression.
\end{itemize}
Also, note that a trait or object declaration
or object expression must not have a
functional method declaration and a dotted method declaration with the
same name, either directly or by inheritance.
Therefore, top-level functions can overload
with other top-level functions and functional methods, dotted methods with
other dotted methods, and functional methods with other functional methods
and top-level functions.

Overloading functional declarations allows the benefits of polymorphic
declarations.  However, with these benefits comes the potential for
ambiguous calls at run time.  Fortress provides overloading rules for the
\emph{declarations} of functionals to eliminate the \emph{possibility}
of ambiguous calls at run time, whether or not these calls actually
appear in the program.

For each functional application expression, it should be a static error
if there does not exist a statically most applicable declaration for
that expression.
Furthermore, these rules are checked statically.
In fact, the overloading rules
in Fortress allow the compiler to identify the
statically most specific declaration for a particular call.  Therefore
an implementation strategy may be used in which the statically most
specific declaration is identified statically, and the runtime
dispatch mechanism need only consider dispatching among that
declaration plus declarations that are more specific than that
declaration (proof of this is given in \secref{proof-overloading-resolution}).

This chapter outlines several criteria for valid functional
overloading.  At any given program point, there is a set of overloaded
declarations that are in scope.  Fortress determines whether there is
a possibility for ambiguous calls from this set by comparing
declarations pairwise.  The following three sections each describes a
rule to accept a pair of overloaded functional declarations.  If a
pair of overloaded declarations satisfies any one of the three rules,
it is considered a valid overloading.

The overloading rules given in this chapter are disjoint.  That is, at
most one rule applies to each pair of overloaded declarations.  It is
possible for new rules to be added which allow additional
overloadings.

\label{adv-over-kwd}
Overloaded declarations must have static parameters that are identical
(up to $\alpha$-equivalence).  Also, valid overloadings for
declarations that contain keyword or
varargs parameters are determined
by analyzing the expansion of these declarations as described in
\secref{applicability-with-special-params}.  Therefore, static
parameters, keyword parameters,
and varargs parameters are ignored in
the remainder of this chapter.

An operator method declaration whose name is one of the operator parameters
(described in \secref{opr-ident})
of its enclosing trait or object may be overloaded with other operator
declarations in the same component.  Therefore, such an operator method
declaration must satisfy the overloading rules
with every operator declaration in the same component.



In addition, a functional which takes a single parameter of type
a naked type parameter of bound \TYP{Any} cannot be
overloaded.  This restriction prevents ambiguous overloadings such as
the following:
%makeSet[\T extends Any\](x: T): Set[\T\]
%makeSet[\T extends Any\](x: T, y: T): Set[\T\]
\begin{Fortress}
\(\VAR{makeSet}\llbracket{}T \KWD{extends}\mskip 4mu plus 4mu\TYP{Any}\rrbracket(x\COLON T)\COLON \TYP{Set}\llbracket{}T\rrbracket\)\\
\(\VAR{makeSet}\llbracket{}T \KWD{extends}\mskip 4mu plus 4mu\TYP{Any}\rrbracket(x\COLON T, y\COLON T)\COLON \TYP{Set}\llbracket{}T\rrbracket\)
\end{Fortress}
The type parameter of \EXP{\VAR{makeSet}} may be instantiated with
\TYP{(\mathbb{Z}, \mathbb{Z})} or \TYP{\mathbb{Z}}.  If this is the
case then the call \EXP{\VAR{makeSet}(3, 4)}, where 3 and 4 have type
\TYP{\mathbb{Z}}, is ambiguous.

\secref{subtype-rule} states the \emph{Subtype Rule}, which stipulates
that the parameter type of one declaration be a subtype of the
parameter type of the other.  In this case, there is no possibility of
ambiguous calls, because one declaration is more specific than the
other.  This section also places a restriction on the return types of
the overloaded declarations to ensure that type safety is not
violated.  \secref{incompatibility-rule} defines the \emph{
Incompatibility Rule} that, if satisfied by a pair of declarations,
guarantees that neither declaration is applicable to the same
functional call.  In \secref{more-specific-rule}, the \emph{Meet Rule}
requires the existence of a declaration that is more specific than
both overloaded declarations in the situation that both are applicable
to a given call.

In the remainder of this chapter, we build on the terminology and
notation defined in \chapref{multiple-dispatch}
and \chapref{conversions-coercions}.



\section{Subtype Rule}
\seclabel{subtype-rule}

If the parameter type of one declaration is a subtype of the parameter
type of another (and they are not the same) then there is no
ambiguity between these two declarations: for every call to which both
are applicable, the first is more specific.  This is the basis of the
Subtype Rule.

The Subtype Rule also requires a relationship between the return types
of the two declarations.  Without such a requirement, a program may
violate type safety.

\paragraph{The Subtype Rule for Functions and Functional Methods:}
Suppose that $\f(\Ps) : \Us$ and $\f(\Qs) : \Vs$ are two distinct
function or functional method declarations in a single scope.
If $\Ps \strictsubtype \Qs$ and $\Us \Ovrsubtype \Vs$ then $\f(\Ps)$ and
$\f(\Qs)$ are a valid overloading.

\paragraph{The Subtype Rule for Dotted Methods:}
Suppose that $\Ps_0.\f(\Ps) : \Us$ and $\Qs_0.\f(\Qs) : \Vs$ are two
distinct dotted method declarations provided by a trait or object $C$.
If $\Ps_0 \strictsubtype \Qs_0$, $\Ps \strictsubtype \Qs$,
and $\Us \Ovrsubtype
\Vs$ then $\Ps_0.\f(\Ps)$ and $\Qs_0.\f(\Qs)$ are a valid overloading.

\section{Incompatibility Rule}
\seclabel{incompatibility-rule}

The basic idea behind the Incompatibility Rule is that if there is no
call to which two overloaded declarations are both applicable
then there is no potential for ambiguous calls.  In such a case, we say
that the declarations are incompatible.
Without coercion, incompatibility is equivalent to exclusion.
However, the presence of coercion complicates the definition of
incompatibility.  To formally define incompatibility we first define the
following notation.
%
For types $T$ and $U$,
we say that $T$ and $U$ \emph{do not share coercions}, and write
$T \noshare U$, if any type that coerces to $T$ excludes any type
that coerces to $U$:
\[
T \noshare U \iff \forall A,B: A \coercedefto T \logicand B \coercedefto U
                               \implies A \excludes B.
\]
%
We say that $T$ \emph{is incompatible with} $U$, and write $T
\incompatible U$, if $T$ and $U$ exclude, reject each other, and do
not share coercions:
\begin{align*}
T \incompatible U \iff
&T \excludes U \logicand T \rejects U \logicand U \rejects T \logicand  T \noshare U \\
\iff
&\phantom{\logicand\ } T \excludes U \\
&\logicand\ (\forall A: A \coercedefto T \implies A \excludes U) \\
&\logicand\ (\forall B: B \coercedefto U \implies B \excludes T) \\
&\logicand\ (\forall A,B: A \coercedefto T \logicand B \coercedefto U
                          \implies A \excludes B)
\end{align*}
Note that if $T \incompatible U$ then no type is substitutable for
both $T$ and $U$.

\paragraph{The Incompatibility Rule for Functions and Functional Methods:}

Suppose that $\f(\Ps)$ and $\f(\Qs)$ are two distinct function or
functional method declarations in a single scope.
If $\Ps \incompatible \Qs$ then $\f(\Ps)$ and $\f(\Qs)$ are a valid overloading.

\paragraph{The Incompatibility Rule for Dotted Methods:}
Suppose that $\Ps_0.\f(\Ps)$ and $\Qs_0.\f(\Qs)$ are two distinct
dotted method declarations provided by a trait or object $C$.  If
$\Ps \incompatible \Qs$ then $\Ps_0.\f(\Ps)$ and $\Qs_0.\f(\Qs)$ are
a valid overloading.

\section{Meet Rule}
\seclabel{more-specific-rule}

If neither the Subtype Rule nor the Incompatibility Rule holds 
for a pair of overloaded declarations 
then the declarations introduce the possibility of ambiguity.  
To avoid this ambiguity, 
we require a \emph{disambiguating declaration}:
for any possible arguments to which both declarations are applicable, 
there must be another more specific declaration that is also applicable.  
Thus, 
if the functional is called with such arguments, 
neither of the pair of declarations is executed 
because the disambiguating declaration is also applicable, 
and it is more specific than both.

\note{What is "an intersection of two declarations"?
-- Sukyoung

For function declarations we require that the disambiguating
declaration be the intersection of the pair of overloaded
declarations.}

\paragraph{The Meet Rule for Functions:}
Suppose that $\f(\Ps)$ and $\f(\Qs)$ are two function
declarations in a single scope
such that neither $\Ps$ nor $\Qs$ is a subtype of the other and $\Ps$ and
$\Qs$ are not incompatible with one another.
Let $\mathpzc{S}$ be the set of types that $\Ps$ defines coercions from
and $\mathpzc{T}$ be the set of types that $\Qs$ defines coercions from.
$\f(\Ps)$ and $\f(\Qs)$ are a valid overloading if
there is a declaration $\f(\Ps \inter \Qs)$ in the scope.

all of the following hold:
\begin{itemize}
\item
either $\Ps \excludes \Qs$ or there is a declaration $\f(\Ps \inter
\Qs)$ in the scope,
and
\item
either $\Ps \morespecific \Qs$ or $\Qs \morespecific \Ps$ or for all
$\Ps' \in \mathpzc{S}$ and $\Qs' \in \mathpzc{T}$ one of the two
conditions holds:
\begin{itemize}
\item
$\Ps' \excludes \Qs'$, or
\item
there is a declaration $\f(\Ps' \inter \Qs')$ in the scope.
\end{itemize}
\end{itemize}

Recall that $\Ps \inter \Qs$ is the intersection of types $\Ps$ and $\Qs$
as defined in \secref{internal-types}.
We write $\Ps \inter \Qs$ to denote the intersection of types $\Ps$ and $\Qs$.
If for some type $S$ we have $S \Ovrsubtype P$ and $S \Ovrsubtype Q$ then
$S \Ovrsubtype (P \intersect Q)$, but it is not necessarily the
case that $S = (P \intersect Q)$ since another type may be more
specific than both $P$ and $Q$.
\label{adv-over-comprises}
For example, suppose the following:
%trait S comprises {U,V} end
%trait T comprises {V,W} end
%trait U extends S excludes W end
%trait V extends {S,T} end
%trait W extends T end
%
%f(s:S) = 1
%f(t:T) = 1
%f(v:V) = 1
\begin{Fortress}
\(\KWD{trait} S \KWD{comprises} \{U,V\} \KWD{end}\)\\
\(\KWD{trait} T \KWD{comprises} \{V,W\} \KWD{end}\)\\
\(\KWD{trait} U \KWD{extends} S \KWD{excludes} W \KWD{end}\)\\
\(\KWD{trait} V \KWD{extends} \{S,T\} \KWD{end}\)\\
\(\KWD{trait} W \KWD{extends} T \KWD{end}\)\\[4pt]
\(f(s\COLONOP{}S) = 1\)\\
\(f(t\COLONOP{}T) = 1\)\\
\(f(v\COLONOP{}V) = 1\)
\end{Fortress}
Because of the \KWD{comprises} clauses of $S$ and $T$ and the \KWD{excludes}
clause of $U$, any subtype of both $S$ and $T$ must be a subtype of
$V$.  Thus, $V = S \inter T$, and the declaration $f(V)$
``disambiguates'' $f(S)$ and $f(T)$, i.e., it is applicable to and more
specific for any call to which both $f(S)$ and $f(T)$ are applicable.

The Meet Rule shall not be difficult to obey, especially because
the compiler can give useful feedback.  For example:

%  foo(x:Number, y:ZZ64) = ...
%  foo(x:ZZ64, y:Number) = ...
\begin{Fortress}
{\tt~~}\pushtabs\=\+\(  \VAR{foo}(x\COLONOP\TYP{Number}, y\COLONOP\mathbb{Z}64) = \ldots\)\\
\(  \VAR{foo}(x\COLONOP\mathbb{Z}64, y\COLONOP\TYP{Number}) = \ldots\)\-\\\poptabs
\end{Fortress}

Assuming that \(\hbox{\ZZ64} \prec \hbox{\TYP{Number}}\), the compiler
reports that these two declarations are a problem because of ambiguity
and suggests that a new declaration for \EXP{\VAR{foo}({\ZZ64},
{\ZZ64})} would resolve the ambiguity. As another example, consider:

% bar(x:Printable) = ...
% bar(x:Throwable) = ...
\begin{Fortress}
{\tt~}\pushtabs\=\+\( \VAR{bar}(x\COLONOP\TYP{Printable}) = \ldots\)\\
\( \VAR{bar}(x\COLONOP\TYP{Throwable}) = \ldots\)\-\\\poptabs
\end{Fortress}

Assuming that $\hbox{\TYP{Printable}}$ and $\hbox{\TYP{Throwable}}$
are neither comparable by the subtyping relation nor disjoint, the
compiler reports that these two declarations are a problem because
\TYP{Printable} and \TYP{Throwable} are incomparable but possibly
overlapping types.
As a result, these two declarations are statically rejected.

Unlike for functions, the Meet Rule for dotted methods only applies to
dotted methods that are provided by the same trait or object.  This is
possible because two dotted methods are applicable to a given call
$\As_0.f(\As)$ only if they are both provided by the trait or object
$\As_0$.

\paragraph{The Meet Rule for Dotted Methods:}
Suppose that $\Ps_0.\f(\Ps)$ and $\Qs_0.\f(\Qs)$ are two dotted method
declarations provided by a trait or object $C$ such that neither
$(\Ps_0, \Ps)$ nor $(\Qs_0, \Qs)$ is a subtype of the other and $\Ps$
and $\Qs$ are not incompatible with one another.
Let $\mathpzc{S}$ be the set of
types that $\Ps$ defines coercions from and $\mathpzc{T}$ be the set
of types that $\Qs$ defines coercions from.
$\Ps_0.\f(\Ps)$ and
$\Qs_0.\f(\Qs)$ are a valid overloading if
there is a declaration $\Rs_0.\f(\Ps
\inter \Qs)$ provided by $C$ with $\Rs_0 \Ovrsubtype (\Ps_0 \inter
\Qs_0)$.

all of the following hold:
\begin{itemize}
\item
either $\Ps \excludes \Qs$ or there is a declaration $\Rs_0.\f(\Ps
\inter \Qs)$ provided by $C$ with $\Rs_0 \Ovrsubtype (\Ps_0 \inter
\Qs_0)$, and
\item
either $\Ps \morespecific \Qs$ or $\Qs \morespecific \Ps$ or
for all $\Ps' \in \mathpzc{S}$ and
$\Qs' \in \mathpzc{T}$ one of the two conditions holds:
\begin{itemize}
\item
$\Ps' \excludes \Qs'$, or
\item
there is a declaration $\Rs_0.\f(\Ps' \inter \Qs')$ provided by $C$
with $\Rs_0 \Ovrsubtype (\Ps_0 \inter \Qs_0)$.
\end{itemize}
\end{itemize}

Recall that functional methods can be viewed semantically as top-level
functions, as described in \secref{methods}.  However, treating
functional methods as top-level functions for determining valid
overloading is too restrictive.  In the following example:
\input{\home/advanced/examples/Overloading.tex}
if the functional methods were interpreted as top-level functions then
this program would define two top-level functions with parameter types
\EXP{\mathbb{Z}} and \EXP{\mathbb{R}}.
These declarations would be
statically rejected as
an invalid overloading because there is no
relation between \EXP{\mathbb{Z}} and \EXP{\mathbb{R}}; another trait
may extend them both without declaring its own version of the
functional method which may lead to an ambiguous call at run time.
However, notice that declaraions are ambiguous only for calls on
arguments that extend both \EXP{\mathbb{Z}} and \EXP{\mathbb{R}}, and
any type that extends both can include a new declaration that
disambiguates them.  We use this intuition to allow such overloadings.

\paragraph{The Meet Rule for Functional Methods:}
Suppose that $\f(\Ps)$ and $\f(\Qs)$ are two functional method
declarations occurring in trait or object declarations or object expressions
such that neither $\Ps$ nor $\Qs$ is a subtype of the other and $\Ps$ and $\Qs$
are not incompatible with one another.  Let $\f(\Ps)$ and $\f(\Qs)$
have self parameters at $i$ and $j$ respectively.
$\f(\Ps)$ and $\f(\Qs)$ are a valid
overloading if all of the following hold:
\begin{itemize}
\item
$i = j$
\item if there exists a trait or object $C$
that provides both $\f(\Ps)$ and $\f(\Qs)$ then $\Ps \neq \Qs$ and
there is a declaration $\f(\Ps \inter \Qs)$ provide by $C$
having self parameter at $i$.
\end{itemize}

Also, let $\mathpzc{S}$ be the set of types that $\Ps$
defines coercions from and $\mathpzc{T}$ be the set of types that
$\Qs$ defines coercions from.  $\f(\Ps)$ and $\f(\Qs)$ are a valid
overloading if all of the following hold:
\begin{itemize}
\item
$i = j$
\item
either $\Ps \excludes \Qs$ or if there exists a trait or object $C$
that provides both $\f(\Ps)$ and $\f(\Qs)$ then $\Ps \neq \Qs$ and
there is a declaration $\f(\Ps \inter \Qs)$ provide by $C$, and
\item
either $\Ps \morespecific \Qs$ or $\Qs \morespecific \Ps$ or
for all $\Ps' \in \mathpzc{S}$ and
$\Qs' \in \mathpzc{T}$ one of the two conditions holds:
\begin{itemize}
\item
$\Ps' \excludes \Qs'$, or
\item
there is a declaration $\f(\Ps' \inter \Qs')$ provided by $C$.
\end{itemize}
\end{itemize}

Notice that the Meet Rule for functional methods requires the self
parameters of two overloaded declarations to be in the same position.
This requirement guarantees that no ambiguity is caused by the
position of the self parameter.  Two declarations which differ in the
position of the self parameter must satisfy either the Subtype Rule or
the Incompatibility Rule to be a valid overloading.

Functional method declarations can overload with function declaraions.
A valid overloading between a functional method declaration and
a function declaration is
determined by applying the (more restrictive) Meet Rule for functions
to both declarations.

\section{Coercion and Overloading Resolution}

The overloading rules given in this chapter are
sufficient to prove the following two facts:
\begin{enumerate}
\item
If no declaration is applicable to a static call but there is a
declaration that is applicable with coercion then there exists a
single most specific declaration that is applicable with coercion to
the static call.
\item
If any declaration is applicable to a static call then there exists a
single most specific declaration that is applicable to the static call
and a single most specific declaration that is applicable to the
corresponding dynamic call.
\end{enumerate}

Moreover, we can prove that the most specific declaration that is
applicable to a dynamic call is more specific than the most specific
declaration that is applicable to the corresponding static call.

\appref{overloaded-functional} formally proves that
the rules discussed in the previous sections guarantee the static resolution of
coercion (described in \secref{resolving-coercion}) is well defined
for functions (the case for methods is analogous).  Also in
\appref{overloaded-functional} is a proof that
the overloading rules for
 function declarations are sufficient to guarantee no
undefined nor ambiguous calls at run time (again,
the case for methods is analogous).
