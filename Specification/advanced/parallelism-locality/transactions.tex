%%%%%%%%%%%%%%%%%%%%%%%%%%%%%%%%%%%%%%%%%%%%%%%%%%%%%%%%%%%%%%%%%%%%%%%%%%%%%%%%
%   Copyright 2009, Oracle and/or its affiliates.
%   All rights reserved.
%
%
%   Use is subject to license terms.
%
%   This distribution may include materials developed by third parties.
%
%%%%%%%%%%%%%%%%%%%%%%%%%%%%%%%%%%%%%%%%%%%%%%%%%%%%%%%%%%%%%%%%%%%%%%%%%%%%%%%%

\section{Abortable Atomicity}
\seclabel{transactions}

Fortress provides a user-level \EXP{\VAR{abort}()} function which
abandons execution of the innermost \KWD{atomic} expression and rolls back its
changes, requiring the \KWD{atomic} expression to execute again from
the beginning.  This permits an atomic section to perform consistency
checks as it runs.
Invoking the \VAR{abort} function not within an \KWD{atomic} expression has
no effect.
The functionality provided by
\EXP{\VAR{abort}()} can be abused; it is possible to induce deadlock
or livelock by creating an atomic section that always fails.  Here is
a simple example of a program using \EXP{\VAR{abort}()} which is
incorrect because Fortress does not guarantee that the two implicit
threads (created by evaluating the two elements of the tuple) will
always run in parallel; it is possible for the first element of the
tuple to continually abort without ever running the second element of
the tuple:
\note{The example is commented out because it is not yet supported nor run by the interpreter.}
%% \input{\home/advanced/examples/Parallel.Abort.a.tex}

Fortress also includes a \KWD{tryatomic} expression, which attempts to
run its body expression atomically.  If it succeeds, the result is returned;
if it aborts due to a call to \VAR{abort} or due to conflict
(as described in \secref{atomic}),
the checked exception \TYP{TryAtomicFailure} is thrown.  In addition, \KWD{tryatomic} is permitted to throw \TYP{TryAtomicFailure} in the absence of conflict; however, it is not permitted to fail unless some other thread performs an access to shared state while the body is being run.
Conceptually \KWD{atomic} can be defined in terms of \KWD{tryatomic} as follows:
\input{\home/advanced/examples/Parallel.Abort.b.tex}
Unlike the above definition, an implementation may choose to suspend a
thread running an \KWD{atomic} expression which invokes \VAR{abort},
re-starting it at a later time when it may be possible to make further
progress.  The above definition restarts the body of the \KWD{atomic}
expression immediately without suspending.
