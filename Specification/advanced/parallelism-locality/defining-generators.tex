%%%%%%%%%%%%%%%%%%%%%%%%%%%%%%%%%%%%%%%%%%%%%%%%%%%%%%%%%%%%%%%%%%%%%%%%%%%%%%%%
%   Copyright 2009, Oracle and/or its affiliates.
%   All rights reserved.
%
%
%   Use is subject to license terms.
%
%   This distribution may include materials developed by third parties.
%
%%%%%%%%%%%%%%%%%%%%%%%%%%%%%%%%%%%%%%%%%%%%%%%%%%%%%%%%%%%%%%%%%%%%%%%%%%%%%%%%

\section{Use and Definition of Generators}
\seclabel{defining-generators}

Several expressions in Fortress make use of \emph{generator clause
  lists} to express parallel iteration (see \secref{generators}).  A
generator clause list binds a series of variables to the values produced by a
series of objects that extend the trait \TYP{Generator}.  A generator clause
list is simply syntactic sugar for a nested series of invocations of
methods on these objects.

A type that extends \EXP{\TYP{Generator}\llbracket{}E\rrbracket} acts
as a generator of elements of type \VAR{E}.  An instance of
\EXP{\TYP{Generator}\llbracket{}E\rrbracket} only needs to define the
\VAR{generate} method:
%    generate[\R\](r:Reduction[\R\], body: E->R): R
\begin{Fortress}
{\tt~~~~}\pushtabs\=\+\(    \VAR{generate}\llbracket{}R\rrbracket(r\COLONOP\TYP{Reduction}\llbracket{}R\rrbracket, \VAR{body}\COLON E\rightarrow{}R)\COLON R\)\-\\\poptabs
\end{Fortress}
The mechanics of object generation are embedded entirely in the
\VAR{generate} method.  This method takes two arguments.  The \VAR{generate} method invokes \VAR{body} once for each
object which is to be generated, passing the generated object as an
argument.  Each call to \VAR{body} returns a result of type \VAR{R}; these results are combined using the reduction \VAR{r}, which encapsulates a monoidal operator on objects of type \VAR{R}.    \emph{All the parallelism provided by a
  particular generator is specified by definition of the \VAR{generate} method.}

In practice, calls to \VAR{generate} are produced by desugaring
expressions with generator clause lists, as described below
(\secref{desugaring-generators}).  However it is possible to call the
\VAR{generate} method directly, as in the following example:
\input{\home/advanced/examples/Generators.ReductionClass.tex}

Any reduction must define two methods: an associative binary operation
\VAR{join}, and \VAR{empty}, a method that returns the identity of
this operation.  Here we define reduction \TYP{SumZZ32} representing integer
addition.  We use this to compute the sum of \EXP{3 x + 2} for \VAR{x}
drawn from the range \EXP{1\mathinner{\hbox{\tt\char'43}}100},
yielding the expected answer of 15350.

\begin{figure}
\input{\home/advanced/examples/Generators.GeneratorDefn.tex}
\caption{\figlabel{generatorDefn}Sample \TYP{Generator} definition: blocked integer ranges.}
\end{figure}

For non-commutative reductions such as the \TYP{Concat} reduction used for
list comprehensions in \figref{generatedExpressions}, it is important to
note that results must be combined in the natural order of the
generator.  If \VAR{join} is not associative, or \VAR{empty} is not
the identity of \VAR{join}, passing the reduction to \VAR{generate}
will produce unpredictable results.  A generator is permitted to group
reduction operations in any way it likes consistent with its natural
order, and insert an arbitrary number of \VAR{empty} elements.

\figref{generatorDefn} defines a generator that generates the integers
between \VAR{lo} and \VAR{hi} in sequential blocks of size at most
\VAR{b}.  In this example, we divide the range in half if it is larger
than the block size \VAR{b}; these two halves are computed in parallel
(recall that the arguments to the method call
\EXP{\VAR{reduction}.\VAR{join}} are evaluated in parallel).  If the
range is smaller than \VAR{b}, then it is enumerated serially using a
\KWD{while} loop, accumulating the result \VAR{r} as it goes.  Observe
that the parallelism obtained from a generator is \emph{dictated by
  the code of the \VAR{generate} method}.  While programmers using a
generator should assume that calls to \VAR{body} may occur in
parallel, the library programmer is free to choose the amount of
parallelism that is actually exposed.

This example uses \emph{recursive subdivision} to divide a blocked
range into approximately equal-sized chunks of work.  Recursive
subdivision is the recommended technique for exposing large amounts of
parallelism in a Fortress program because it adjusts easily to varying
machine sizes and can be dynamically load balanced.

\TYP{Generator} defines the functional method
\EXP{\VAR{seq}(\KWD{self})} that returns an equivalent generator that
runs iterations of the body sequentially in natural order.  In most
cases (such as in this example), it is prudent to override the default
definition of this method; the default implementation of \VAR{seq}
effectively collects the generated elements together in parallel and
traverses the result sequentially.

\note{
Note that at the moment there is no way to tell the compiler for
performance reasons that we really mean it when we ask for
sequentiality, as opposed to saying that we should preserve sequential
semantics.  Future versions of this specification may use the
\TYP{Local} distribution for this purpose, or provide additional
functions on generators which guarantee serial execution (rather than
simply providing sequential semantics).
}

The remainder of this section describes in detail the desugaring of
expressions with generator clause lists into invocations of the
\VAR{generate} methods of the generators in the
generator clause list.
\note{
It then outlines how method overloading may be used
to specialize the behavior of particular combinations of generators
and reductions.}

\subsection{Simple Desugaring of Expressions with Generators}
\seclabel{desugaring-generators}

\note{NOTE: tweaked by hand by Jan, as above.}

\begin{figure}
\begin{tabular}{l@{\:\emph{body}\;\;=\;\;}l}
%BIG OP[ ]
\(\mathcal{C}[\;\;]\)
&
%u(body)
\(u(\emph{body})\)
\\
%BIG OP[ x <- g, gs ]
\(\mathcal{C}[\,x \leftarrow g, \emph{gs}\,]\)
&
%g.generate(r, fn (x) => BIG OP [gs] body)
\(g.\VAR{generate}(r, \KWD{fn} (x) \Rightarrow \mathcal{C}[\emph{gs}]\:\emph{body})\)
\\
%BIG OP[ p, gs ]
\(\mathcal{C}[\,p, \emph{gs}\,]\)
&
%p.generate(r, fn () => BIG OP [gs] body)
\(p.\VAR{generate}(r, \KWD{fn} () \Rightarrow \mathcal{C} [\emph{gs}]\:\emph{body})\)
\end{tabular}
\caption{\figlabel{simpleDesugar}Simple syntax-directed desugaring of
  a generator clause list.  Here the reduction $r$ and unit $u$ are
  variables chosen by the desugarer to be distinct from the variables
  in \emph{gs} and \emph{body}. }
\end{figure}

\begin{figure}
%%\begin{tabular}{ll|ll>{$\;\;\llbracket$}c<{$\rrbracket$}}
\begin{tabular}{ll|llc}
expr & type & \VAR{wrapper} & \EXP{u(\VAR{body})} & $r$ \\
\hline

% SUM [ gs ] e
\EXP{\sum\limits_{\VAR{gs}} e}
&
$N$
&
\EXP{\OPR{SUM}\llbracket{}N\rrbracket}
&
\EXP{\VAR{identity}\llbracket{}N\rrbracket(e)}
&
\EXP{\TYP{SumReduction}\llbracket{}N\rrbracket}
\\

% <| e | gs |>
\( \langle\,e \mid \VAR{gs}\,\rangle\)
&
% List[\E\]
\( \TYP{List}\llbracket{}E\rrbracket\)
&
\EXP{\OPR{BIG} \langle\llbracket{}E\rrbracket\,\rangle}
&
\( \VAR{singleton\llbracket{}E\rrbracket}(e)\)
&
\EXP{\TYP{Concat}\llbracket{}E\rrbracket}
\\

% lv := e, gs
\( \VAR{lv} \ASSIGN e, \VAR{gs}\)
&
\EXP{()}
&
built in
&
$\VAR{ignore}(\VAR{lv} \ASSIGN e)$
&
\TYP{NoReduction}

\end{tabular}
\caption{Examples of wrappers for expressions with generator clause
  lists.  Top to bottom: big operators (here \EXP{\sum} is used as an
  example; the appropriate library function is called on the
  right-hand side), comprehensions (here list comprehensions are
  shown; other comprehensions are similar to list comprehensions) and
  generated assignment.}
\figlabel{generatedExpressions}
\end{figure}

An expression with a generator clause list \emph{gs} and body expression \emph{body} is desugared into the following general form:
\note{
NOTE: tweaked by hand by Jan.  Replaced BIG OP by $\mathcal{C}$,
emph'd gs and body.}
%wrapper(fn (r, u) => BIG OP[gs] body)
\begin{Fortress}
\( \VAR{wrapper}(\KWD{fn} (r, u) \Rightarrow \mathcal{C}[\emph{gs}]\:\emph{body})\)
\end{Fortress}
The generator clause list and body can be desugared using the
syntax-directed desugaring $\mathcal{C}$ defined in
\figref{simpleDesugar}.  This yields a function that is in turn passed
as an argument to \VAR{wrapper}.  The particular choice of the
function \VAR{wrapper} depends upon the construct that is being
desugared.  For a reduction or a comprehension, the wrapper function
is the corresponding big operator definition; see
\secref{reduction-expr} and \secref{comprehensions}.  For a \KWD{for}
loop (\secref{for-expr}) or a generated expression
(\secref{generated}), a special built-in wrapper is used.
Examples are shown in \figref{generatedExpressions}.
A wrapper function always has the following type:
%% wrapper( g:(Reduction[\R0\],T->R0)->R0): R
\begin{Fortress}
\(\VAR{wrapper}( g\COLONOP(\TYP{Reduction}\llbracket{}R_{0}\rrbracket,T\rightarrow{}R_{0})\rightarrow{}R_{0})\COLON R\)
\end{Fortress}
Here the type $T$ is the type of values returned by the body expression,
and \EXP{R_{0}} and \VAR{R} are arbitrary types chosen by the wrapper function.

Note that the function $g$ passed to the wrapper has essentially the
same type signature as the \VAR{generate} method itself.  It is
instructive to think of \VAR{wrapper} as having the following similar
type signature:\footnote{In future, it is likely that Fortress will use
  a desugaring that in fact yields a \TYP{Generator} rather than a
  higher-order function.  This permits type-directed nesting and
  composition of generators.}
%% wrapper'( g: Generator[\T\] ): R  (* NOT THE ACTUAL TYPE *)
\begin{Fortress}
\(\VAR{wrapper}'( g\COLON \TYP{Generator}\llbracket{}T\rrbracket )\COLON R  \mathtt{(*}\;\hbox{\rm  NOT THE ACTUAL TYPE \unskip}\;\mathtt{*)}\)
\end{Fortress}

\note{From here till the end of this chapter, copied from F1.0$\beta$.}

An array comprehension simply desugars into a factory function call
and a series of assignments:
\begin{center}
\begin{tabular}[c]{@{}l@{}}
% [ i1 = e1 | gs1
%   i2 = e2 | gs2
%   ...
%   i_n = e_n | gs_n ]
\EXP{[\,\,\,i_{1} = e_{1} \mid {gs}_{1}}\\
~~~\EXP{i_{2} = e_{2} \mid {gs}_{2}} \\
~~~$\ldots$ \\
~~~\EXP{i_n = e_n \mid \VAR{gs}_n\,]}
\end{tabular}
\hspace{6em}
$\longrightarrow$
\hspace{6em}
\begin{tabular}[c]{@{}l@{}}
\( \KWD{do}\;a = \VAR{array}()\)\\
~~~~~~\(    a[i_{1}] \ASSIGN e_{1}, {gs}_{1}\)\\
~~~~~~\(    a[i_{2}] \ASSIGN e_{2}, {gs}_{2}\)\\
~~~~~~\(    \ldots\)\\
~~~~~~\(    a[i_n] \ASSIGN e_n, \VAR{gs}_n\)\\
~~~~~~\(    a\) \\
\( \KWD{end}\)
\end{tabular}
\end{center}

The desugaring of a \KWD{for} loop depends upon the set of reduction
variables.
We conceptually desugar the \KWD{for} loop with reduction variables,
$r_1$, $r_2$, $\ldots$, $r_n$, reduced using the reduction operator
\EXP{\oplus} for type \EXP{(T_{1}, T_{2}, \ldots T_n)}
%% with the identity \EXP{(z_{1}, z_{2}, \ldots z_n)}
as follows:
\begin{center}
\EXP{\KWD{for} \VAR{gs} \KWD{do} \VAR{block} \KWD{end}}
\\
$\longrightarrow$
\\[2em]
\begin{tabular}[c]{l@{}l@{}l@{}l}
\EXP{(r_{1}, r_{2}, \ldots r_n)\; \mathord{\oplus}\!\!= }&
\EXP{\mathcal{T}
\llbracket(T_{1}, T_{2}, \ldots T_n), \oplus
\rrbracket[\VAR{gs}]} &\EXP{(\KWD{do}}&\\
&&&\EXP{(r_{1}, r_{2}, \ldots r_{\mathrm{n}}) \mathrel{\mathtt{:}} (T_{1}, T_{2}, \ldots T_{\mathrm{n}}) \ASSIGN \TYP{Identity}\llbracket\oplus\rrbracket} \\
&&&\EXP{\VAR{block}} \\
&&&\EXP{(r_{1}, r_{2}, \ldots r_n)}\\
&&\EXP{\;\KWD{end})}\\
\end{tabular}
\end{center}
In practice, a tuple type is not a monoid.  If there is only one
reduction variable, this is not a problem.  If there are no reduction
variables, we simply use the type \TYP{NoReduction} used in desugaring
assignments.  When there are multiple reduction variables, we use
nested applications of types extending the trait \TYP{ReductionPair}.
These types encode the common properties of the variables being
reduced.  Recall that every reduction variable must at least have type
\TYP{Monoid}, so it is not difficult to guarantee that
\TYP{ReductionPair} itself also extends \TYP{Monoid}.

\subsection{Accounting for Dependencies among Generators}
\seclabel{generators-dependencies}

The naive desugaring for generator lists in
\figref{simpleDesugar} assumes there are always data
dependencies among generators.  The actual desugaring makes use of the
\VAR{join} method in the \TYP{Generator} trait to group together
generators that have no data dependencies.  The goal is to permit
library code to define more efficient merged generators for generator
pairs.  For example, it is possible for the \VAR{join} method to take
the generator list \EXP{i \leftarrow
  1\mathinner{\hbox{\tt\char'43}}100, j \leftarrow
  2\mathinner{\hbox{\tt\char'43}}200} and generate a blocked
two-dimensional traversal.  This could then be joined with \EXP{k
  \leftarrow 3\mathinner{\hbox{\tt\char'43}}300} to obtain a
three-dimensional blocked traversal.

However, most generators will simply make use of the default
definition of \VAR{join} which calls \TYP{SimplePairGenerator}:
% object SimplePairGenerator[\ A extends Any, B extends Any \]
%         (outer : Generator[\ A \], inner : Generator[\ B \]) extends Generator[\ (A, B) \]
%   size : ZZ64 = outer.size DOT inner.size
%   generate[\ R extends Monoid[\ R, OPLUS \], opr OPLUS \](body : (A, B) -> R):R =
%     outer.generate(fn (a : A) => inner.generate(fn (b : B) => body (a,b)))
%   join[\ N extends Any \](other : Generator[\ N \]) : Generator[\ ((A,B), N) \] =
%     SimpleMapGenerator(outer.join(inner.join(other)),
%                        (fn (a,(b,n)) => ((a,b),n)))
% end
\begin{Fortress}
{\tt~}\pushtabs\=\+\( \KWD{object}\mskip 4mu plus 4mu\TYP{SimplePairGenerator}\llbracket\,A \KWD{extends}\mskip 4mu plus 4mu\TYP{Any}, B \KWD{extends}\mskip 4mu plus 4mu\TYP{Any}\,\rrbracket\)\\
{\tt~~~~~~~~}\pushtabs\=\+\(         (\VAR{outer} \mathrel{\mathtt{:}}\mskip 4mu plus 4mu\TYP{Generator}\llbracket\,A\,\rrbracket, \VAR{inner} \mathrel{\mathtt{:}}\mskip 4mu plus 4mu\TYP{Generator}\llbracket\,B\,\rrbracket) \KWD{extends}\mskip 4mu plus 4mu\TYP{Generator}\llbracket\,(A, B)\,\rrbracket\)\-\\\poptabs
{\tt~~}\pushtabs\=\+\(   \VAR{size} \mathrel{\mathtt{:}}\mskip 4mu plus 4mu\mathbb{Z}64 = \VAR{outer}.\VAR{size} \cdot \VAR{inner}.\VAR{size}\)\\
\(   \VAR{generate}\llbracket\,R \KWD{extends}\mskip 4mu plus 4mu\TYP{Monoid}\llbracket\,R, \oplus\,\rrbracket, \KWD{opr} \mathord{\oplus}\,\rrbracket(\VAR{body} \mathrel{\mathtt{:}} (A, B) \rightarrow R)\COLONOP{}R =\)\\
{\tt~~}\pushtabs\=\+\(     \VAR{outer}.\VAR{generate}(\KWD{fn} (a \mathrel{\mathtt{:}}\mskip 4mu plus 4mu{A}) \Rightarrow \VAR{inner}.\VAR{generate}(\KWD{fn} (b \mathrel{\mathtt{:}}\mskip 4mu plus 4mu{B}) \Rightarrow \VAR{body} (a,b)))\)\-\\\poptabs
\(   \VAR{join}\llbracket\,N \KWD{extends}\mskip 4mu plus 4mu\TYP{Any}\,\rrbracket(\VAR{other} \mathrel{\mathtt{:}}\mskip 4mu plus 4mu\TYP{Generator}\llbracket\,N\,\rrbracket) \mathrel{\mathtt{:}}\mskip 4mu plus 4mu\TYP{Generator}\llbracket\,((A,B), N)\,\rrbracket =\)\\
{\tt~~}\pushtabs\=\+\(     \TYP{SimpleMapGenerator}( \null\)\pushtabs\=\+\(\VAR{outer}.\VAR{join}(\VAR{inner}.\VAR{join}(\VAR{other})),\)\\
\(                        (\KWD{fn} (a,(b,n)) \Rightarrow ((a,b),n)))\)\-\-\-\\\poptabs\poptabs\poptabs
\( \KWD{end}\)\-\\\poptabs
\end{Fortress}
Note how \TYP{SimplePairGenerator} itself overrides the \VAR{join}
method.  When we attempt to join an existing pair of joined
generators, we first attempt to \VAR{join} the \VAR{inner} generator
of the pair with the \VAR{other} generator (the new innermost
generator) passed in.  This means that every generator will have the
opportunity to combine with both its left and right neighbors if
neither has a dependency which prevents it.  Note that we use a
\TYP{SimpleMapGenerator}, which simply applies a function to the
result of another generator, to re-nest the tuples produced by the
nested \VAR{join} operation.

Which pairs of adjacent traversals are combined using \VAR{join}?
This question is complicated by examples such as
% i<-1:100, j<-1:100, k<-i:100, l<-j:100.
\EXP{i\leftarrow{}1\COLONOP{}100, j\leftarrow{}1\COLONOP{}100,
  k\leftarrow{}i\COLONOP{}100, l\leftarrow{}j\COLONOP{}100}.  We can
either combine \VAR{i} and \VAR{j} traversals, or we can combine
\VAR{j} and \VAR{k} traversals.  In the former case we can also
combine \VAR{k} and \VAR{l} traversals.  The Fortress compiler is free
to choose any grouping subject to the following constraints:
\begin{itemize}
\item Two generators may not be combined using \VAR{join} if the second is data dependent upon the first.
\item Generator order must be preserved when invoking \VAR{join}.
\item When a chain of three or more generators is joined, the
  traversals must be combined left-associatively.
\end{itemize}
We can obtain a simple greedy desugaring which joins together
traversals in accordance with the above rules by simply adding the
following desugaring rule which takes precedence over those given in
\figref{simpleDesugar} when each variable bound in $v_1$ does not occur free in $g_2$.
\begin{center}
%T[\R,BOXPLUS\][ v1 <- g1, v2 <- g2, gs ] body
\EXP{\mathcal{T}\llbracket{}R,\boxplus\rrbracket[\,v_{1} \leftarrow g_{1}, v_{2} \leftarrow g_{2}, \VAR{gs}\,]\;\VAR{body}
\;\;=\;\;
%T[\R,BOXPLUS\][ (v1,v2) <- g1.join(g2), gs ] body
\mathcal{T}\llbracket{}R,\boxplus\rrbracket[\,(v_{1},v_{2}) \leftarrow g_{1}.\VAR{join}(g_{2}), \VAR{gs}\,]\;\VAR{body}}
%%\\
%%\hfill when $\VAR{BV}[v_{1}]$ does not intersect $\VAR{FV}[g_2]$
\end{center}
%where the set of bound variables in $v_1$ does not intersect with the set of
%free variables in $g_2$.


\subsection{Using Overloading to Adapt Generators and Traversals}

Overloaded instances of the \VAR{generate} method can be used to adapt
a generator to the particular properties of the reduction being
performed.  For example, a commutative monoid need only maintain a
single variable \VAR{result} containing the reduced value so far:
% value object BlockedRange(lo: ZZ64, hi: ZZ64, b: ZZ64) extends Generator[\ZZ64\]
%   ...
%   generate[\R extends CommutativeMonoid[\ R, OPLUS\], opr OPLUS\](body : ZZ64 -> R) : R = do
%     result : R = Identity[\OPLUS\]
%     traverse(l,u) =
%       if u-l+1 <= max(b,1) then
%         i : ZZ64 = l
%         while i <= u do
%           t = body(i)
%           atomic result := result OPLUS t
%           i += 1
%         end
%         ()
%       else
%         mid = LC l/2 RC + LF u/2 RF
%         (traverse(l,mid), traverse(mid+1,u))
%         ()
%       end
%     traverse(lo, hi)
%     result
%   end
% end
\begin{Fortress}
{\tt~}\pushtabs\=\+\( \KWD{value}\;\;\KWD{object} \TYP{BlockedRange}(\VAR{lo}\COLON \mathbb{Z}64, \VAR{hi}\COLON \mathbb{Z}64, b\COLON \mathbb{Z}64) \KWD{extends} \TYP{Generator}\llbracket\mathbb{Z}64\rrbracket\)\\
{\tt~~}\pushtabs\=\+\(   \ldots\)\\
\(   \VAR{generate}\llbracket{}R \KWD{extends} \TYP{CommutativeMonoid}\llbracket\,R, \oplus\rrbracket, \KWD{opr} \mathord{\oplus}\rrbracket(\VAR{body} \mathrel{\mathtt{:}} \mathbb{Z}64 \rightarrow R) \mathrel{\mathtt{:}} R = \;\KWD{do}\)\\
{\tt~~}\pushtabs\=\+\(     \VAR{result} \mathrel{\mathtt{:}} R = \TYP{Identity}\llbracket\mathord{\oplus}\rrbracket\)\\
\(     \VAR{traverse}(l,u) =\)\\
{\tt~~}\pushtabs\=\+\(       \KWD{if} u-l+1 \leq \max(b,1) \KWD{then}\)\\
{\tt~~}\pushtabs\=\+\(         i \mathrel{\mathtt{:}} \mathbb{Z}64 = l\)\\
\(         \KWD{while} i \leq u \KWD{do}\)\\
{\tt~~}\pushtabs\=\+\(           t = \VAR{body}(i)\)\\
\(           \KWD{atomic} \VAR{result} \ASSIGN \VAR{result} \oplus t\)\\
\(           i \mathrel{+}= 1\)\-\\\poptabs
\(         \KWD{end}\)\\
\(         ()\)\-\\\poptabs
\(       \KWD{else}\)\\
{\tt~~}\pushtabs\=\+\(         \VAR{mid} = \lceil l/2 \rceil + \lfloor u/2 \rfloor\)\\
\(         (\VAR{traverse}(l,\VAR{mid}), \VAR{traverse}(\VAR{mid}+1,u))\)\\
\(         ()\)\-\\\poptabs
\(       \KWD{end}\)\-\\\poptabs
\(     \VAR{traverse}(\VAR{lo}, \VAR{hi})\)\\
\(     \VAR{result}\)\-\\\poptabs
\(   \KWD{end}\)\-\\\poptabs
\( \KWD{end}\)\-\\\poptabs
\end{Fortress}

The choice of whether to apply this transformation is left up to the
author of the generator; when many iterations run in parallel the
\VAR{result} variable becomes a scalability bottleneck and this
technique should not be used.

Various other properties of the reduction operator can be exploited:
\begin{itemize}
\item Idempotent reductions permit redundant computation.  For
  example, when computing the maximum element of a set it might be
  simpler to enumerate set elements more than once.

\item On the other hand, sometimes a more efficient non-idempotent
  operator can be used for a reduction if the generator promises never
  to produce duplicates---this fact can be used to advantage in set,
  multiset, or map comprehensions.

\item If the reduction operator has a zero, this can be used to exit
  early from a partial computation.  This requires that the body
  expression have no visible side effects such as writes or \KWD{io} actions.
\end{itemize}

At the moment, the author of a \TYP{Generator} is responsible for
taking advantage of opportunities such as these.  In future, we expect
some standardized support for efficient versions of various traversals
based on experience with the definitions provided here.

\subsection{Making a Serial Version of a Generator or Distribution}
\seclabel{serial-generator}

\note{
TODO: the mechanism for generator serialization is once again magic.
Formerly we constrained generator structure so the serialization was
straightforward (but this probably made things slow).}

A generator \VAR{g} can be made sequential simply by calling the builtin function \VAR{sequential} as follows:
%v <- sequential(g)
\begin{Fortress}
\(v \leftarrow \VAR{sequential}(g)\)
\end{Fortress}
The resulting generator yields its elements in the natural order
specified by the original generator.  Several builtin generators (such
as those for array indices) have an associated distribution.  For
these generators, \VAR{sequential} function simply re-distributes the
underlying object as follows:
%sequential(r) = Sequential.distribute(r)
\begin{Fortress}
\(\VAR{sequential}(r) = \TYP{Sequential}.\VAR{distribute}(r)\)
\end{Fortress}
As a convenient shorthand, the \VAR{sequential} function is also
defined to work for distributions themselves.  The complete declarations of
the overloaded \VAR{sequential} function are as follows:
%sequential[\ E extends Any \](g : Generator[\E\]) : Generator[\E\]
%sequential[\ E extends Any \](d : Distribution) : Distribution
\begin{Fortress}
\(\VAR{sequential}\llbracket\,E \KWD{extends}\mskip 4mu plus 4mu\TYP{Any}\,\rrbracket(g \mathrel{\mathtt{:}}\mskip 4mu plus 4mu\TYP{Generator}\llbracket{}E\rrbracket) \mathrel{\mathtt{:}}\mskip 4mu plus 4mu\TYP{Generator}\llbracket{}E\rrbracket\)\\
\(\VAR{sequential}\llbracket\,E \KWD{extends}\mskip 4mu plus 4mu\TYP{Any}\,\rrbracket(d \mathrel{\mathtt{:}}\mskip 4mu plus 4mu\TYP{Distribution}) \mathrel{\mathtt{:}}\mskip 4mu plus 4mu\TYP{Distribution}\)
\end{Fortress}
The \VAR{sequential} function has special meaning to the Fortress
implementation; there is no need to distinguish reduction variables in
loops for which generator is surrounded by a direct call to
\VAR{sequential}.
