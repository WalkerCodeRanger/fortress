%%%%%%%%%%%%%%%%%%%%%%%%%%%%%%%%%%%%%%%%%%%%%%%%%%%%%%%%%%%%%%%%%%%%%%%%%%%%%%%%
%   Copyright 2009, Oracle and/or its affiliates.
%   All rights reserved.
%
%
%   Use is subject to license terms.
%
%   This distribution may include materials developed by third parties.
%
%%%%%%%%%%%%%%%%%%%%%%%%%%%%%%%%%%%%%%%%%%%%%%%%%%%%%%%%%%%%%%%%%%%%%%%%%%%%%%%%

\section{Distributions}
\seclabel{distributions}

\note{Distributions are not yet supported.
Examples in this section are not tested.}

Most of the heavy lifting in mapping threads and arrays to regions is performed
by \emph{distributions}.  An instance of the trait
\TYP{Distribution} describes the parallel structure of ranges and
other numeric generators (such as the generators for the index space
of an array), and provides for the allocation and distribution of
arrays on the machine:
%trait Distribution
%  distribute[\E, B extends ArrayIndex\](Range[\B\]):Range[\B\]
%  distribute[\E, B extends ArrayIndex\](a:Array[\E,B\]):Array[\E,B\] =
%    distributeFromTo[\E,B\](a,a.distribution,self)
%end
\begin{Fortress}
\(\KWD{trait}\mskip 4mu plus 4mu\TYP{Distribution}\)\\
{\tt~~}\pushtabs\=\+\(  \VAR{distribute}\llbracket{}E, B \KWD{extends}\mskip 4mu plus 4mu\TYP{ArrayIndex}\rrbracket(\TYP{Range}\llbracket{}B\rrbracket)\COLONOP\TYP{Range}\llbracket{}B\rrbracket\)\\
\(  \VAR{distribute}\llbracket{}E, B \KWD{extends}\mskip 4mu plus 4mu\TYP{ArrayIndex}\rrbracket(a\COLONOP\TYP{Array}\llbracket{}E,B\rrbracket)\COLONOP\TYP{Array}\llbracket{}E,B\rrbracket =\)\\
{\tt~~}\pushtabs\=\+\(    \VAR{distributeFromTo}\llbracket{}E,B\rrbracket(a,a.\VAR{distribution},\KWD{self})\)\-\-\\\poptabs\poptabs
\(\KWD{end}\)
\end{Fortress}

Abstractly, a \TYP{Distribution} acts as a transducer for generators
and arrays.  The \VAR{distribute} method applied to a multidimensional
\TYP{Range} organizes its indices into the leaves of a tree whose
inner nodes correspond to potential levels of parallelism and locality
in the underlying computation, producing a fresh \TYP{Range} whose
behavior as a \TYP{Generator} may differ from that of the passed-in
\TYP{Range}.  The \VAR{distribute} method applied to an array creates
a copy of that array distributed according to the given distribution.
This is specified in terms of a call to the overloaded function
\VAR{distributeFromTo}.  This permits the definition of specialized
versions of this function for particular pairs of distributions.

The intention of distributions is to separate the task of data
distribution and program correctness.  That is, it should be possible
to write and debug a perfectly acceptable parallel program using only
the default data distribution provided by the system.  Imposing a
distribution on particular computations, or designing and implementing
distributions from scratch, is a task best left for performance
tuning, and one which should not affect the correctness of a working
program.

There is a \TYP{DefaultDistribution} which is defined by the
underlying system.  This distribution is designed to be reasonably
adaptable to different system scales and architectures, at the cost of
some runtime efficiency.  Arrays and generators that are not
explicitly allocated through a distribution are given the
\TYP{DefaultDistribution}.

There is a generator, \VAR{indices},
associated with every array.  This generator is distributed in the
same way as the array itself.  When we re-distribute an array, we also
re-distribute the generator; thus
\EXP{d.\VAR{distribute}(a.\VAR{indices})} is equivalent to
\EXP{(d.\VAR{distribute}(a)).\VAR{indices}}.

There are a number of built-in distributions:
\begin{tabbing}
\begin{tabular}{ll}
\TYP{DefaultDistribution}        &       Name for distribution chosen by system. \\
\TYP{Sequential}
&       Sequential distribution.
       Arrays are allocated in one contiguous piece of memory. \\
\TYP{Local}          &       Equivalent to \TYP{Sequential}. \\
\TYP{Par}            &       Blocked into chunks of size 1. \\
\TYP{Blocked}        &       Blocked into roughly equal chunks. \\
\EXP{\TYP{Blocked}(n)}   &   Blocked into \VAR{n} roughly equal chunks. \\
\TYP{Subdivided}     &       Chopped into $2^k$-sized chunks, recursively. \\
\EXP{\TYP{Interleaved}(d_{1}, d_{2},\ldots d_{n})}
&       The first \VAR{n} dimensions are distributed according to \EXP{d_{1} \ldots d_{n}}, \\
&       with subdivision alternating among
dimensions. \\
\EXP{\TYP{Joined}(d_{1}, d_{2},\ldots d_{n})}
&       The first \VAR{n} dimensions are distributed according to \EXP{d_{1} \ldots d_{n}}, \\
&       subdividing completely in each
dimension before proceeding to the next.
\end{tabular}
\end{tabbing}
From these, a number of composed distributions are provided:
\begin{tabbing}
\begin{tabular}{ll}
\EXP{\TYP{Morton}}
&       Bit-interleaved Morton order~\cite{morton}, recursive subdivision
       in all dimensions.\\
\EXP{\TYP{Blocked}(x_{1}, x_{2}, \ldots x_{n})}
&       Blocked in \VAR{n} dimensions into chunks of size \EXP{x_{i}} in dimension \VAR{i}; \\
&       remaining dimensions (if any) are local.
\end{tabular}
\end{tabbing}

To allocate an array which is local to a single thread (and most
likely allocated in contiguous storage), the \TYP{Local}
distribution can be used:
%a = Local.distribute [ 1 0 0 ; 0 1 0 ; 0 0 1 ]
\begin{Fortress}
\(a = \TYP{Local}.\VAR{distribute} [\,1\;0\;0 ; 0\;1\;0 ; 0\;0\;1\,]\)
\end{Fortress}
Other distributions can be requested in a similar way.


Distributions can be constructed and given names:
%spatialDist = Blocked(n,n,1)     (* Pencils along the $z$ axis *)
\begin{Fortress}
\(\VAR{spatialDist} = \TYP{Blocked}(n,n,1)     \mathtt{(*}\;\hbox{\rm  Pencils along the $z$ axis \unskip}\;\mathtt{*)}\)
\end{Fortress}
The system will lay out arrays with the same distribution in the same
way in memory (as much as this is feasible), and will run loops with
the same distribution in the same way (as much as this is feasible).
By contrast, if we replace every occurrence of \VAR{spatialDist} by
\EXP{\TYP{Blocked}(n,n,1)}, this code will likely divide up arrays and
ranges into the same-sized pieces as above, but these pieces need not
be collocated.
