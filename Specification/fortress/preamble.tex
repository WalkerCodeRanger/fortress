%%%%%%%%%%%%%%%%%%%%%%%%%%%%%%%%%%%%%%%%%%%%%%%%%%%%%%%%%%%%%%%%%%%%%%%%%%%%%%%%
%   Copyright 2009, Oracle and/or its affiliates.
%   All rights reserved.
%
%
%   Use is subject to license terms.
%
%   This distribution may include materials developed by third parties.
%
%%%%%%%%%%%%%%%%%%%%%%%%%%%%%%%%%%%%%%%%%%%%%%%%%%%%%%%%%%%%%%%%%%%%%%%%%%%%%%%%

\begin{figure}[p]
The Fortress Language Specification, Version 1.0\cite{Fortress} was the first to be
released in tandem with a compliant interpreter (as of March 31, 2008),
available as open source and online at:\\

\begin{center}
{\tt http://projectfortress.sun.com}\\[1em]
\end{center}
Our tandem release of a specification and matching interpreter was a major
milestone for the project.
Our reference implementation has been evolving gradually,
in parallel with the evolution of the language specification and
the development of the core libraries.  In order to synchronize
the specification with the implementation, it was necessary both to add
features to the implementation and to drop features from the specification.
All Fortress source code appearing in the specification has been
tested by executing it with the open source Fortress implementation. Moreover,
all code has been rendered automatically with the tool \emph{Fortify}, also
included with the standard Fortress distribution. Fortify is an open source tool
for converting Fortress source code to \LaTeX.\\


Since the release, we have incrementally added back
features taken out of the specification as they have been implemented.
In particular, we have been building a Fortress compiler with a static
type checker and various static constraints checker.
We also made considerable amount of modification and evolution of the
language design.\\


This specification is a working draft;
it is meant to include all language features included in
the Fortress Language Specification, Version 1.0\EXP{\beta{}}
(perhaps with additional ) and any language changes since then.
This version of the specification is not yet for a release;
it may include wild ideas that require additional research before they can be
implemented reliably.
This specification includes descriptions of unimplemented features in boxes
at the beginning of sections,
and informal comments and notes in boxes in the margin.
\end{figure}


\begin{figure}[p]
The not-yet-supported features include:
\begin{itemize}
\item widening coercion
\item multifix operators
\item type coverage check
\item generic overloading check
\item definite assignment check
\item keyword and varargs parameters
\item varargs expressions
\item some modifiers (\KWD{override}, \KWD{io}, \KWD{atomic}, \KWD{value}, ...)
\item qualified names
\item identifiers including connecting punctuations
\item ASCII conversion
\item Unicode canonicalization
\item array comprehensions
\item reduction variables
\item object copy
\item unboxed types
\item mutual recursion in value objects
\item thread fairness guarantee
\item distributions
\item matrix unpasting
\item Non-type static parameters
\item dimensions and units
\item tests and properties
\item type aliases
\item where clauses and conditional extension
\item abstract function declarations
\item checked exceptions, deferred exceptions, and chained exceptions
\item contract checking (overloaded functionals and method overriding)
\item static expressions, static range types, and some range operations
\item some types (\TYP{HeapSequence},
\TYP{LinearSequence}, $\mathbb{Q}$,
\TYP{Numeral}, \EXP{\mathbb{Z}}, \EXP{\mathbb{N}}, \EXP{\mathbb{Q}}, ...)
\item some operators on aggregate expressions
\item some functions in \secref{other-expressions}
\item tail-call optimization
\item compound APIs check
\item import API names
\item components linking
\end{itemize}
\end{figure}

%% Description in F1.0
%% -- Sukyoung
% This release of the Fortress Language Specification is the first to be
% released in tandem with a compliant interpreter, available as open source
% and online at:\\

% \begin{center}
% {\tt http://projectfortress.sun.com}\\[1em]
% \end{center}
% Our tandem release of a specification and matching interpreter is a major
% milestone for the project; it is a goal we have been working toward for
% some time.  All Fortress source code appearing in this specification has been
% tested by executing it with the open source Fortress implementation. Moreover,
% all code has been rendered automatically with the tool \emph{Fortify}, also
% included with the standard Fortress distribution. Fortify is an open source tool
% for converting Fortress source code to \LaTeX.

% Our reference implementation has been evolving gradually,
% in parallel with the evolution of the language specification and
% the development of the core libraries.  In order to synchronize
% the specification with the implementation, it was necessary both to add
% features to the implementation and to drop features from the specification.
% Most significantly, most static checks in the implementation are currently
% turned off, as we are in the process of completing the static
% type checker and the type inference engine.
% Static constraints are still included in the specification as documentation.
% Contrary to the Fortress Language Specification, Version 1.0\EXP{\beta{}},
% inference of static parameter instantiations
% is based on the runtime types of the arguments to a functional call.
% Support for syntactic abstraction is not included in this release.
% We do not yet support nontrivial distributions, nor parallel nested
% transactions.  Moreover, many other minor language features defined in
% the Fortress Language Specification, Version 1.0\EXP{\beta{}} have been elided.
% All of these features require additional research before they can be
% implemented reliably; this research and development is a high priority
% for the Fortress team.\\

% With this release, our goal in moving forward is to incrementally add back
% features taken out of the specification as they are implemented.
% In particular, all language features included in
% the Fortress Language Specification, Version 1.0\EXP{\beta{}}
% remain goals for eventual inclusion in the language
% (perhaps with additional modification and evolution of their design).
% By proceeding in this manner, we believe that our implementation will be
% useful for more tasks more quickly, as it will comply with the public
% specification.  Moreover, the Fortress community will be better able to
% evaluate the design of new features, as users will be able to use them
% immediately, and developers will be able to contribute to the implementation
% effort more easily, as they will be able to build off of a relatively stable
% and well-specified base.\\

% Moving forward with the implementation, in concert with our open source
% community, our goal is to build off of the infrastructure of our interpreter
% to construct the first optimizing Fortress compiler and to achieve our
% long-standing goal of constructing a new programming language with high
% performance and high programmer productivity, owned by the community that
% uses it, and able to grow gracefully with the tasks it is applied to.
