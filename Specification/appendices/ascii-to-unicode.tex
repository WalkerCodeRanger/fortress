%%%%%%%%%%%%%%%%%%%%%%%%%%%%%%%%%%%%%%%%%%%%%%%%%%%%%%%%%%%%%%%%%%%%%%%%%%%%%%%%
%   Copyright 2009, Oracle and/or its affiliates.
%   All rights reserved.
%
%
%   Use is subject to license terms.
%
%   This distribution may include materials developed by third parties.
%
%%%%%%%%%%%%%%%%%%%%%%%%%%%%%%%%%%%%%%%%%%%%%%%%%%%%%%%%%%%%%%%%%%%%%%%%%%%%%%%%

\chapter{Support for Unicode Input in ASCII}
\applabel{ascii-unicode}

\note{Only \secref{preprocessing-unicode-names} is supported.}

\note{QUESTIONS/NOTES from Victor:
\begin{itemize}
\item does micro convert to MICRO SIGN by itself,
or only as part of a restricted word?  (The latter is currently specified.)
\item Is '' ASCII shorthand for double prime (and the analogous qns
for ''' `` and ```)?  If so, then we need to fix the example in
 the character literal section, which says ''' is a character
 literal that evaluates to the APOSTROPHE character.
\item The section on conversion of ASCII symbol sequences still needs
 to be fixed up.  I think it should list all the ASCII shorthand
 (certainly it needs to list all the shorthand for non-operators),
 and then it needs to deal with not converting sequences that are
 subsequences of multicharacter enclosing operators.
\end{itemize}
}

To facilitate the writing of Fortress programs
using ASCII-based tools,
Fortress programs are subject to \emph{ASCII conversion},
an idempotent process consisting of three steps
described in this appendix.
We expect that IDEs that support Unicode
may apply ASCII conversion immediately,
saving programs as converted Unicode character sequences.


\section{Word Pasting across Line Terminators}
\seclabel{identifier-pasting}

Consider any line terminator in the program
for which the last non-whitespace character before the line terminator
is an ampersand (\txt{\&}) that is immediately preceded by a word character,
and the first non-whitespace character after the line terminator
is an ampersand that is immediately followed by a word character.
All the characters between the ampersands inclusive
are removed from the program.
Note that the removed characters other than the two ampersands
must be whitespace characters
(including one or more line terminators).
This step permits very long names and numerals
to be split across line boundaries.
For example:

\begin{verbatim}
supercalifragilisticexpiali&
   &docious = 0.142857142857142857&
   &142857 TIMES &
       GREEK_SMALL_LETTER_&

   &UPSILON_WITH_DIALYTICA_AND_TONOS
\end{verbatim}
becomes
\begin{verbatim}
supercalifragilisticexpialidocious = 0.142857142857142857142857 TIMES &
       GREEK_SMALL_LETTER_UPSILON_WITH_DIALYTICA_AND_TONOS
\end{verbatim}

To ensure the idempotence of ASCII conversion,
it is a static error if
an ampersand is immediately followed by another ampersand,
or if it is immediately followed
by one or more whitespace characters,
not including a line terminator,
followed by an ASCII letter or ampersand.
In other words,
if an ampersand is not the last non-whitespace character on a line
then the next non-whitespace character must not be an ampersand,
nor can it be an ASCII letter
unless it is adjacent to the ampersand.


\section{Converting Names of Unicode Characters and ASCII Shorthand}
\seclabel{preprocessing-unicode-names}

After word pasting across line terminators,
certain ASCII character sequences are converted
into single Unicode characters.
These character sequences fall into two categories:
restricted words (or parts of restricted words)
and ASCII shorthand for operators and special characters.
In some cases,
ampersands adjacent to converted sequences are removed.

\subsection{Determining Potentially Convertible Character Sequences}
\seclabel{determining-conversion-chunks}

We first identify nonoverlapping contiguous subsequences
of the program
that may be subject to conversion in this step.
In particular,
names of characters and ASCII shorthand consist of
only ASCII characters,
and they are \emph{not} converted within string literals.

The boundaries of string literals (and comments)
are determined follows:
There are three modes of processing:
outside any comment or string literal,
inside a string literal
and inside a comment.
Within a comment,
we also keep track of ``nesting depth''
(this is 0 when not within a comment).
Processing proceeds from left to right.
Outside any comment or string literal,
encountering an unescaped string literal delimiter
(i.e., a string literal delimiter not immediately
preceded by an unescaped backslash)
changes processing to the mode for within a string literal
(however, it is a static error if the string literal delimiter
is the right double quotation mark),
and encountering
the opening comment delimiter ``\txt{(*}''
changes processing to the mode for within a comment,
incrementing the nesting depth (to 1).
All other characters are ignored,
except to note they are not within a string literal.
In particular,
because character literal delimiters are ignored,
string literal delimiters must be escaped within
character literals.

Within a string literal,
all characters other than an unescaped string literal delimiter
are ignored
(except to note that they are within a string literal).
An unescaped string literal delimiter switches processing
to the mode for outside any comment or string literal.
However,
it is a static error if a line terminator appears
within a string literal
(Fortress provides \emph{escape sequences}
to include line terminators in strings;
see~\secref{preprocessing-string-literals}),
or if the opening and closing string literal delimiters
do not ``match''
(again see~\secref{preprocessing-string-literals}).

Within a comment,
all characters, including unescaped string literal delimiters, are ignored
(except to note that they are not within a string literal)
other than the two-character opening and closing comment delimiters
``\txt{(*}'' and ``\txt{*)}''.
Whenever the opening comment delimiter is encountered,
the nesting depth is incremented,
and each time the closing comment delimiter is encountered,
the nesting depth is decremented,
until it becomes 0.
At that point, processing is changed again to the mode
for outside any comment or string literal.

A contiguous subsequence of ASCII characters
not within any string literal
is a candidate for conversion in this step
if it is either a restricted word,
or it is a maximal such subsequence
that does not contain any restricted-word characters,
control characters,
whitespace characters
or ampersands
(i.e., it must be delimited by whitespace,
a control or non-ASCII characters,
string literals,
restricted words,
ampersands,
or the beginning or end of the file).
We call the latter kind of subsequence
an \emph{ASCII symbol sequence}.
The conversion of restricted words and ASCII symbol sequences
are described in \secref{convert-restricted-words}
and \secref{convert-symbol-sequences} respectively.
Whitespace characters and ampersands are not converted,
but some ampersands may be eliminated,
as described in \secref{convert-ampersand}.


\subsection{Converting Restricted Words}
\seclabel{convert-restricted-words}

A restricted word that ``names'' a Unicode character
that is not protected\footnote{Protecting
the backslash (an ASCII character)
and string literal delimiters
preserves the boundaries of string literals
determined as described above;
protecting the other ASCII characters
ensures that the ASCII conversion process is idempotent.
}
(i.e., not an ASCII or control character
nor a string literal delimiter)
is replaced by that character.
A restricted word may name a character in various ways,
described below.
In a few cases,
a restricted word may name more than one character
(via different ways),
so the order in which we consider these ways is important.
Also,
in some cases,
only part of the restricted word is replaced by a Unicode character.

Fortress explicitly provides short ASCII names
for many characters,
especially ones that programmers might most commonly want.
Some characters have more than one short name.
For operators,
these names are given in \appref{operators},
including the following common ones:

\begin{tabular}{rcl@{\hspace{5ex}}rcl@{\hspace{5ex}}rcl}
        \txt{LE} & \emph{becomes} & $\leq$ &
        \txt{GE} & \emph{becomes} & $\geq$ &
        \txt{NE} & \emph{becomes} & $\neq$ \\
        \txt{BY}  & \emph{becomes} & $\times$ &
        \txt{TIMES}  & \emph{becomes} & $\times$ &
        \txt{CROSS} & \emph{becomes} & $\times$ \\
        \txt{DOT}  & \emph{becomes} & $\csdot$ &
        \txt{PROD} & \emph{becomes} & $\prod$ &
        \txt{SUM} & \emph{becomes} & $\sum$ \\
        \txt{CUP} & \emph{becomes} & $\cup$ &
        \txt{CAP} & \emph{becomes} & $\cap$ &
        \txt{SUBSET} & \emph{becomes} & $\subset$ \\
        \txt{EMPTYSET} & \emph{becomes} & $\emptyset$ &
        \txt{AND} & \emph{becomes} & $\wedge$ &
        \txt{OR} & \emph{becomes} & $\vee$ \\
\end{tabular}

To provide a level of compatibility with Fortran,
\appref{operators} also gives short names
for the following ASCII (and thus protected) characters:

\begin{tabular}{rcl@{\hspace*{5ex}}rcl@{\hspace*{5ex}}rcl}
        \txt{LT} & \emph{becomes} & $<$ &
        \txt{GT} & \emph{becomes} & $>$ &
        \txt{EQ} & \emph{becomes} & $=$
\end{tabular}

These names are the only ones converted to a protected character.
Furthermore,
they are \emph{not} replaced by the corresponding character
unless they are delimited by whitespace characters
(\emph{not} ampersands).
Thus, they cannot participate in further conversion,
maintaining the idempotence of ASCII conversion.


The following non-operator characters also have short names:

\begin{tabular}{rcl@{\hspace{5ex}}rcl@{\hspace{5ex}}rcl}
        \txt{ALPHA} & \emph{becomes} & A &
        \txt{alpha} & \emph{becomes} & $\alpha$ &
        \txt{BOTTOM} & \emph{becomes} & $\bot$ \\
        \txt{BETA} & \emph{becomes} & B &
        \txt{beta} & \emph{becomes} & $\beta$ &
        \txt{TOP} & \emph{becomes} & $\top$ \\
        \txt{GAMMA} & \emph{becomes} & $\Gamma$ &
        \txt{gamma} & \emph{becomes} & $\gamma$ &
        \txt{INF} & \emph{becomes} & $\infty$ \\
        \txt{DELTA} & \emph{becomes} & $\Delta$ &
        \txt{delta} & \emph{becomes} & $\delta$ &
        \txt{FORALL} & \emph{becomes} & $\forall$ \\
        \txt{EPSILON} & \emph{becomes} & E &
        \txt{epsilon} & \emph{becomes} & $\epsilon$ &
        \txt{EXISTS} & \emph{becomes} & $\exists$ \\
        \txt{ZETA} & \emph{becomes} & Z &
        \txt{zeta} & \emph{becomes} & $\zeta$ \\
        \txt{ETA} & \emph{becomes} & H &
        \txt{eta} & \emph{becomes} & $\eta$ \\
        \txt{THETA} & \emph{becomes} & $\Theta$ &
        \txt{theta} & \emph{becomes} & $\theta$ \\
        \txt{IOTA} & \emph{becomes} & I &
        \txt{iota} & \emph{becomes} & $\iota$ \\
        \txt{KAPPA} & \emph{becomes} & K &
        \txt{kappa} & \emph{becomes} & $\kappa$ \\
        \txt{LAMBDA} & \emph{becomes} & $\Lambda$ &
        \txt{lambda} & \emph{becomes} & $\lambda$ \\
        \txt{MU} & \emph{becomes} & M &
        \txt{mu} & \emph{becomes} & $\mu$ \\
        \txt{NU} & \emph{becomes} & N &
        \txt{nu} & \emph{becomes} & $\nu$ \\
        \txt{XI} & \emph{becomes} & $\Xi$ &
        \txt{xi} & \emph{becomes} & $\xi$ \\
        \txt{OMICRON} & \emph{becomes} & O &
        \txt{omicron} & \emph{becomes} & $o$ \\
        \txt{PI} & \emph{becomes} & $\Pi$ &
        \txt{pi} & \emph{becomes} & $\pi$ \\
        \txt{RHO} & \emph{becomes} & P &
        \txt{rho} & \emph{becomes} & $\rho$ \\
        \txt{SIGMA} & \emph{becomes} & $\Sigma$ &
        \txt{sigma} & \emph{becomes} & $\sigma$ \\
        \txt{TAU} & \emph{becomes} & T &
        \txt{tau} & \emph{becomes} & $\tau$ \\
        \txt{UPSILON} & \emph{becomes} & $\Upsilon$ &
        \txt{upsilon} & \emph{becomes} & $\upsilon$ \\
        \txt{PHI} & \emph{becomes} & $\Phi$ &
        \txt{phi} & \emph{becomes} & $\phi$ \\
        \txt{CHI} & \emph{becomes} & X &
        \txt{chi} & \emph{becomes} & $\chi$ \\
        \txt{PSI} & \emph{becomes} & $\Psi$ &
        \txt{psi} & \emph{becomes} & $\psi$ \\
        \txt{OMEGA} & \emph{becomes} & $\Omega$ &
        \txt{omega} & \emph{becomes} & $\omega$
\end{tabular}

A restricted word that is not a short name specified by Fortress
is compared to names of characters
specified by the Unicode Standard~\cite{Unicode}.
Because restricted words consist only of
letters, digits and underscores,
while Unicode names may include hyphens and spaces
(but not underscores),
hyphens and spaces in the Unicode names
are replaced by underscores for this comparison.
Specifically,
if a restricted word is the official \unicode\ name of a character
with hyphens and spaces replaced by underscores,
it is replaced by that character.
For any Unicode character other than the control characters,
there is a unique official \unicode\ name
not shared by any other Unicode character.
Since control characters are protected,
they do not present a problem in this regard.

If there is no matching official \unicode\ name,
the restricted word is compared to the alternative names for characters
specified by the Unicode Standard,
again with hyphens and spaces replaced by underscores,
and converted to a character with such a matching name.
If more than one character has a matching alternative name,
then the restricted word is converted to the first such non-protected character
(i.e., the one with the smallest code point).
If there is no matching alternative name,
then the restricted word is compared to
official \unicode\ names
and any alternative names specified in the Unicode Standard,
with underscores in place of hyphens and spaces,
where any of the following substrings may be omitted:
\begin{verbatim}
  "LETTER_"
  "DIGIT_"
  "RADICAL_"
  "NUMERAL_"
\end{verbatim}
If there are multiple such substrings in a given name,
any combination of them may be omitted.
Again,
if there are multiple characters with matching abridged names,
the restricted word is converted to the first such non-protected character.

If the restricted word is not converted to a single Unicode character
by any of the above rules,
parts of it may be converted as follows:
If the restricted word begins with the short name
(i.e., the name in the table above)
of a Greek letter followed by an underscore or a digit,
or ends with the short name of a Greek letter
that is preceded by an underscore,
or contains the short name of a Greek letter
with an underscore on each side of it,
then the short name of the Greek letter
is replaced by the Greek letter itself.
In the same manner,
the character sequence ``micro'' is replaced
with the character \txt{MICRO SIGN} $\mu$
(U+00B5,
which looks just like the Greek lowercase mu $\mu$ but is different).
If a word-part being thus replaced
has an underscore to each side,
and the underscore on the right is the last character of the restricted word,
then the underscore on the left is removed as the name is replaced;
this ad hoc rule is for
the sake of the abbreviations of certain dimensional units,
so that, for example, \txt{micro\_OMEGA\_}
is converted to \txt{$\mathrm{\mu\Omega\_}$},
signifying micro-ohms,
and \txt{G\_OMEGA\_} is converted to \txt{$\mathrm{G\Omega\_}$},
signifying gigaohms.
Here are some other examples:

\begin{tabular}{rcl@{\hspace{5ex}}rcl}
        \txt{alpha} & \emph{becomes} & $\alpha$ &
        \txt{OMEGA13} & \emph{becomes} & $\Omega13$ \\
        \txt{alpha\_hat} & \emph{becomes} & $\alpha\_hat$ &
        \txt{theta\_elephant} & \emph{becomes} & $\theta\_elephant$ \\
        \txt{OMEGA\_} & \emph{becomes} & $\Omega\_$ &
        \txt{\_XI} & \emph{becomes} & $\_\Xi$
\end{tabular}


\subsection{Converting ASCII Symbol Sequences}
\seclabel{convert-symbol-sequences}




For the sequences of characters other than restricted words,
each is converted from left to right,
with the longest possible substring being converted at once,
with one exception:
The sequence ``\txt{(<}'' is not converted
if it is immediately followed by any of the
following characters:
`\txt{<}', '\txt{|}', `\txt{/}', `\txt{\char'134}', `\txt{*}', or `\txt{.}'.
That is, with this one exception,
the longest shorthand beginning from the first character,
if any, is converted first.
Then, the longest shorthand beginning from the next character
of the string after replacement is converted, and so on.
Here are the ASCII shorthands
for some of the characters we expect to be most frequently used:

\begin{tabular}{rcl@{\hspace{5ex}}rcl}
        \txt{[{\char'134}} & \emph{becomes} & $\bTPl$ &
        \txt{{\char'134}]} & \emph{becomes} & $\bTPr$ \\
        \txt{->}  & \emph{becomes} & $\rightarrow$ &
        \txt{=>}  & \emph{becomes} & $\Rightarrow$ \\
        \txt{$\sim$>} & \emph{becomes} & $\leadsto$ &
        \txt{|->} & \emph{becomes} & $\mapsto$ \\
        \txt{>=} & \emph{becomes} & $\geq$ &
        \txt{<=} & \emph{becomes} & $\leq$ \\
        \txt{=/=} & \emph{becomes} & $\neq$ &
        \text{...} & \emph{becomes} & $\ldots$ \\
        \txt{:=} & \emph{becomes} & $:=$ &
\end{tabular}

Although the characters with string literals
are generally not subject to this step of ASCII conversion,
They are if they are part of a restricted-word escape sequence
or a quoted-character escape sequence.
See \secref{preprocessing-string-literals} for details.

\subsection{Eliminating Ampersands}
\seclabel{convert-ampersand}

Finally,
if an ampersand is adjacent to a sequence of characters
that is changed by this step of ASCII conversion
(even if the sequence was only partly changed,
as long as the character adjacent to the ampersand is changed),
the ampersand is removed after the transformation,
\emph{unless} the ampersand is the first
or last non-whitespace character on the line.
\marginnote{Why leave the first?}
Thus, the expression
\begin{verbatim}
(GREEK_SMALL_LETTER_PHI&GREEK_SMALL_LETTER_PSI +
GREEK_SMALL_LETTER_OMEGA&GREEK_SMALL_LETTER_LAMBDA)
\end{verbatim}

is converted to:

\begin{ttt}
($\phi\psi$ +
$\omega\lambda$)
\end{ttt}

in which there are two identifiers,
each consisting of two Greek letters.
On the other hand,

\begin{verbatim}
(GREEK_SMALL_LETTER_PHI GREEK_SMALL_LETTER_PSI +
GREEK_SMALL_LETTER_OMEGA GREEK_SMALL_LETTER_LAMBDA)
\end{verbatim}

is converted to:

\begin{ttt}
($\phi$ $\psi$ +
$\omega$ $\lambda$)
\end{ttt}

in which there are four identifiers.

Because we never convert to protected characters,
and converted sequences consist only of protected characters,
this step is idempotent:
repeated application has no effect beyond the first.
\note{Commented out text:
However,
to ensure that the entire ASCII conversion process
is idempotent,
it is a static error
if an ampersand is immediately followed by a word character
after this step.
This restriction prevents any word pasting
in the converted program.
}


\section{Apostrophe Replacement within Numerals}
\seclabel{digit-separator}

The last step of ASCII conversion simply replaces
apostrophes with digit-group separators
when they appear in words that would otherwise be numerals.
Specifically,
consider any word that is not within a string literal
(determined as described in \secref{determining-conversion-chunks}),
has only digits, letters, underscores
and apostrophes,
and neither begins nor ends with an apostrophe or an underscore.
Call such a word a \emph{proto-numeral}.
If a proto-numeral begins with a digit
or ends with a radix specifier
(see \secref{numerals}),
or if it is immediately followed by a `.'
and then a proto-numeral that ends with a radix specifier,
or it is immediately preceded by a `.',
which is immediately preceded by a proto-numeral
that begins with a digit,
then replace each apostrophe in that proto-numeral
with a digit-group separator.

\section{Equivalence under ASCII Conversion}
\seclabel{ascii-conversion-equivalence}

Two (valid) Fortress programs are equivalent under ASCII conversion
if they are identical after ASCII conversion
except for string literals
(determined as described in \secref{determining-conversion-chunks}),
the converted programs have string literals
in exactly corresponding places,
and corresponding string literals evaluate to the same string.
This exception and additional rule
are necessary because string literals
are not subject to ASCII conversion.
