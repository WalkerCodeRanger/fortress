%%%%%%%%%%%%%%%%%%%%%%%%%%%%%%%%%%%%%%%%%%%%%%%%%%%%%%%%%%%%%%%%%%%%%%%%%%%%%%%%
%   Copyright 2009, Oracle and/or its affiliates.
%   All rights reserved.
%
%
%   Use is subject to license terms.
%
%   This distribution may include materials developed by third parties.
%
%%%%%%%%%%%%%%%%%%%%%%%%%%%%%%%%%%%%%%%%%%%%%%%%%%%%%%%%%%%%%%%%%%%%%%%%%%%%%%%%

\section{Proof of Coercion Resolution for Functions}
\seclabel{proof-coercion-resolution}

This section proves that the overloading rules discussed in the previous
sections guarantee the static resolution of coercion
(described in \secref{resolving-coercion}) is well defined for functions
(the case for methods is analogous).

Consider a static function call $\f(\As)$ at some program point $Z$
and its corresponding dynamic function call $\f(\Xs)$.  Let $\Sigma$
be the set of parameter types of function declarations of $\f$ that
are visible at $Z$ and applicable to the static call $\f(\As)$.  Let
$\Sigma'$ be the set of parameter types of function declarations of
$\f$ that are visible at $Z$ and applicable with coercion to the
static call $\f(\As)$.  Moreover, let $\sigma'$ be the subset of
$\Sigma'$ for which no type in $\Sigma'$ is more specific:
\[
\begin{array}{lclcll}
\sigma' & = & \set{ S \in\Sigma' & | & \lnot \exists S'\in\Sigma': S' \morespecific S & }.
\end{array}
\]

We prove the following:
\[
\begin{array}{l}
|\Sigma| = 0 \ \text{and} \ |\Sigma'| \neq 0 \ \text{imply} \ |\sigma'| = 1.
\end{array}
\]
Informally, if no declaration is applicable to a static call but there
is a declaration that is applicable with coercion then there exists a
single most specific declaration that is applicable with coercion to
the static call.
\\
\begin{lemma}
\label{lem:nonempty}
Given an acyclic, irreflexive binary relation $R$ on a set $S$, and a
finite nonempty subset $A$ of $S$, the set $\set{a \in A| \lnot
\exists a' \in A : (a',a) \in R}$ is nonempty.
\end{lemma}

\begin{proof}
Consider the relation $R$ on $S$ as a directed acyclic graph.  Let $A$
represent a subgraph.  Then the Lemma amounts to proving that there
exists a node in the graph represented by $A$ with no edges pointing
to it.  This follows from the fact that $A$ is finite and the graph is
acyclic.
\end{proof}

\begin{lemma}
\label{lem:dgesge}
%If $|\Delta| \ge 1$ then $|\delta| \ge 1$.  Also, i
%If $|\Sigma| \ge 1$ then $|\sigma| \ge 1$.
If $|\Sigma'| \ge 1$ then $|\sigma'| \ge 1$.
\end{lemma}

\begin{proof}
Follows from Lemma \ref{lem:nonempty} where $S$ is the set of all
types, $A$ is %$\Delta$, $\Sigma$ and
$\Sigma'$, %respectively,
and the relation %$\strictsubtype$ and
$\morespecific$ is acyclic and irreflexive.
\end{proof}

\begin{lemma}
\label{lem:le'}
If $|\Sigma| = 0$ then $|\sigma'| \le 1$.
\end{lemma}

\begin{proof}
For the purpose of contradiction suppose there are two declarations
$\f(\Ps)$ and $\f(\Qs)$ in $\sigma'$.  Since both $\f(\Ps)$ and
$\f(\Qs)$ are applicable with coercion to $\f(\As)$ and $|\Sigma| =
0$ there must exist a coercion from some type $\Ps'$ to $\Ps$ and a
coercion from some type $\Qs'$ to $\Qs$ such that $\As \Ovrsubtype
\Ps' \inter \Qs'$.  Therefore it is not the case that $\Ps
\incompatible \Qs$.  By the overloading rules, $\Ps \neq \Qs$
and either $\Ps \morespecific \Qs$ or $\Qs \morespecific \Ps$ or
for all $\Ps' \in \mathpzc{S}$ and $\Qs' \in \mathpzc{T}$ either $\Ps'
\excludes \Qs'$, or there is a declaration $\f(\Ps' \inter \Qs')$
visible at $Z$.  If $\Ps \morespecific \Qs$ or $\Qs \morespecific
\Ps$ then we contradict our assumption.  Otherwise, if there exists a
declaration $\f(\Ps' \inter \Qs')$ visible at $Z$ then this
declaration is applicable to $\f(\As)$ without coercion.  This
contradicts $|\Sigma| = 0$.  If such a declaration does not exist then
it must be the case that $\Ps' \excludes \Qs'$.  Then both $\f(\Ps)$
and $\f(\Qs)$ can not be applicable to the call $\f(\As)$ which is a
contradiction.
\end{proof}

\begin{theorem}
\label{thm:overloading-coercion}
If $|\Sigma| = 0$ and $|\Sigma'| \neq 0$ then $|\sigma'| = 1$.
\end{theorem}

\begin{proof}
Follows from Lemmas \ref{lem:dgesge} and \ref{lem:le'}.
\end{proof}
