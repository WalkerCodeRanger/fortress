%%%%%%%%%%%%%%%%%%%%%%%%%%%%%%%%%%%%%%%%%%%%%%%%%%%%%%%%%%%%%%%%%%%%%%%%%%%%%%%%
%   Copyright 2009, Oracle and/or its affiliates.
%   All rights reserved.
%
%
%   Use is subject to license terms.
%
%   This distribution may include materials developed by third parties.
%
%%%%%%%%%%%%%%%%%%%%%%%%%%%%%%%%%%%%%%%%%%%%%%%%%%%%%%%%%%%%%%%%%%%%%%%%%%%%%%%%

\subsection{Syntax}\label{basic-syntax}

A syntax for \basiccore\ is provided in Figure \ref{fig:basic-syntax}.
We use the following notational conventions:
\begin{itemize}
\item For brevity, we omit separators such as \comma\ and \semicolon\ from
  \basiccore.

\item \seq{\ty} is a shorthand for a (possibly empty) sequence \tyn1,
 \more, \tyn n.

\item Similarly, we abbreviate a sequence of relations
 \sub\tvone 1 \extends \sub\tappone 1, \more,
 \sub\tvone n \extends \sub\tappone n to
 \seq{\tvone \extends \tappone}

\item We use \tyn i to denote the $i$th element of \seq\ty.

\item For simplicity, we assume that every name (type variables,
  field names, and parameters) is different and every trait/object
  declaration declares unique name.

\item We prohibit cycles in type hierarchies.

\end{itemize}

\begin{figure*}[htbp!]
\[
\begin{array}{llll}
\tvone, \tvtwo&& &  \mbox{type variables}\\
\fname&& &  \mbox{method name}\\
\vname&& &  \mbox{field name}\\
\tname&& &  \mbox{trait name}\\
\oname&& &  \mbox{object name}\\
\ty, \prm\ty, \prm{\prm\tau} &::=& \tvone &\mbox{type}\\
         &|& \tynontv\\
\tynontv&::=& \tappone & \mbox{type that is not a type variable}\\
         &|& \oapp \\
\tappone, \tapptwo, \tappthree &::=& \tapp
 & \mbox{type that can be a type bound}\\
         &|& \obj\\
\exp   &::=& \vname & \mbox{expression}\\
         &|& \self\\
         &|& \objexp\\
         &|& \fldacc\\
         &|& \mthinvk\\
\fd &::=& \fdsyntax& \mbox{method definition}\\
\td &::=& \tdsyntax& \mbox{trait definition}\\
\od &::=& \odsyntax& \mbox{object definition}\\
\d &::=& \td &\mbox{definition}\\
   &|  & \od \\
\pgm &::=& \pd & \mbox{program}\\
\end{array}
\]
\caption{Syntax of \basiccore}
\label{fig:basic-syntax}
\end{figure*}

The syntax of \basiccore\ allows only a small subset of the Fortress
language to be formalized.  \basiccore\ includes trait and object
definitions, method and field invocations, and \KWD{self} expressions.
The types of \basiccore\ include type variables, instantiated traits,
instantiated objects, and the distinguished trait \TYP{Object}.
Note that we syntactically prohibit extending objects.
Among other features, \basiccore\ does \emph{not} include top-level
variable and function definitions, overloading, \KWD{excludes} clauses,
\KWD{comprises} clauses, \KWD{where} clauses, object expressions, and function
expressions.  \basiccore\ will be extended to formalize a larger set
of Fortress programs in the future.
