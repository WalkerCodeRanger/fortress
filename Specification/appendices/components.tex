%%%%%%%%%%%%%%%%%%%%%%%%%%%%%%%%%%%%%%%%%%%%%%%%%%%%%%%%%%%%%%%%%%%%%%%%%%%%%%%%
%   Copyright 2009, Oracle and/or its affiliates.
%   All rights reserved.
%
%
%   Use is subject to license terms.
%
%   This distribution may include materials developed by third parties.
%
%%%%%%%%%%%%%%%%%%%%%%%%%%%%%%%%%%%%%%%%%%%%%%%%%%%%%%%%%%%%%%%%%%%%%%%%%%%%%%%%

\renewcommand{\api}{a}
\chapter{Components and APIs}
\applabel{components}

\note{This chapter might not be up to date.}

As mentioned in \chapref{components}, 
we formally specify key functionality of the Fortress component system,
and illustrate how we can reason about the correctness of the system.

\paragraph{Components}
One important restriction on components is that no \apiN\ may be both
imported and exported by the same component.
Formally, we introduce two functions on components,
$\imports$ and $\exports$,
that return the imported and exported \apisN\ of the component, respectively.
For any component $\comp$, %% \in \components$,
$\imports(\comp) \intersect \exports(\comp) = \emptyset$.
This restriction is required throughout to ground the semantics of
operations on components, as discussed in \secref{basicops}.

\paragraph{APIs}
Other than its identity,
the only relevant characteristic of an \apiN\ $\api$
is the set of \apisN\ that it uses,
denoted by $\uses(\api)$.
Because an \apiN\ $\api$ might expose types
defined in $\uses(\api)$,
we require that a component that exports $\api$
also exports all \apisN\ in $\uses(\api)$ that it does not import.
Formally, the following condition holds on the exported \apisN\
of a component $\comp$:
\[
\api \in \exports(\comp) \logicand \api' \in \uses(\api)
        \implies \api' \in \imports(\comp) \union \exports(\comp)
\]

\paragraph{Link}
Given a set $\compset = \set{\comp_1,\ldots,\comp_k}$ of components,
we define a partial function $\linkfn(\compset)$
that returns the component resulting from $\comp_1$ through $\comp_k$.
If $\comp = \linkfn(\compset)$,
then $\exports(\comp) = \Union_{\comp' \in \compset} \exports(\comp')$
and $\imports(\comp) = \Union_{\comp' \in \compset} \imports(\comp')
                      - \exports(\comp)$.

The function $\linkfn$ is partial
because we do not allow arbitrary sets of components to be linked.
In particular,
Two components cannot be linked if they export the same
\apiN.\!\footnote{There is one exception to this rule:
the special \apiN\ \upgapi, which is used during upgrades discussed below.}
This restriction is made for the sake of simplicity;
it allows programmers to link a set of components
without having to specify explicitly
which constituent exporting an \apiN\ $A$
provides the implementation exported by the linked component,
and which constituent connects to the constituents that import $A$:
only one component exports $A$, so there is only one choice.
Although we lose expressiveness with this design,
the user interface to link is vastly simplified,
and it is rare that including multiple components that export a given \apiN\
in a set of linked components is even desirable.
We discuss how even such rare cases can be supported in
\secref{advancedops}.

For a compound component,
in addition to the exported and imported \apisN,
we want to know what its constituents are.
So we introduce another function $\constituents$,
which takes a component and returns the set of its constituents.
That is,
$\constituents(\linkfn(\compset)) = \compset$.
It is an invariant of the system
that for any compound component $C$
(i.e., $\constituents(\comp) \neq \emptyset$),
any \apiN\ imported by any of its constituents
is either imported by $C$ or exported by one of its constituents
(i.e., $\Union_{\comp' \in \constituents(\comp)} \imports(\comp')
          \subseteq \imports(\comp) \union \Union_{\comp' \in
          \constituents(\comp)} \exports(\comp')$). 
This property is crucial for executing components, as we discuss below.
A simple component (i.e., one produced directly by compilation)
has no constituents (i.e., $\constituents(\comp) = \emptyset$).

\paragraph{Upgrade}
A predicate \replacerest\ takes two components
and indicates whether the first can be upgraded with the second;
that is, $\replacerest(\targcomp,\upgcomp)$
returns true if and only if $\targcomp$ can be upgraded with $\upgcomp$.
This predicate captures both the constraints imposed by a component's
\VAR{isValidUpgrade} function and
the conditions that guarantee the well-formedness of the result.
That is,
\begin{align*}
\replacerest(\targcomp,\upgcomp) \implies
        &\targcomp\VAR{.isValidUpgrade}(\upgcomp)  \\
        &\logicand \imports(\upgcomp) \subseteq \exports(\targcomp)
                                        \union \imports(\targcomp) \\
        &\logicand \exports(\upgcomp) \subset \exports(\targcomp) \\
        &\logicand \forall \comp \in \constituents(\targcomp).
                (\exports(\comp) \subseteq \exports(\upgcomp)
                 \logicor \exports(\comp) \intersect
        \exports(\upgcomp) = \emptyset
                 \logicor \replacerest(\comp,\upgcomp))
\end{align*}

\paragraph{Visible and Provided}
We introduce two new functions on components:
\visibleAPI, which returns the \apisN\ of a component that have not been
hidden; 
and \provided, which returns those visible \apisN\ that are exported
by some top-level constituent of the component
(or all the exported \apisN\ of a simple component);
we say these \apisN\ are \emph{provided} by the component.
We need to distinguish provided \apisN\
because they can be imported by the top-level constituents of a component,
and thus by a replacement component in an upgrade,
while other visible \apisN\ cannot be.
Thus, for a compound component $\comp$,
$\provided(\comp) = \visibleAPI(\comp) \intersect
                        \Union_{\comp' \in \constituents(\comp)} \exports(\comp')$.
For a simple component $\comp$,
$\provided(\comp) = \visibleAPI(\comp) = \exports(\comp)$.

\paragraph{Constrain}
If $\comp$ is a compound component
and $\apiset \subset \exports(\comp)$ is a set of \apisN\
such that
$\api \in \exports(\comp) \logicand \api' \in \uses(\api) \intersect \apiset
        \implies \api \in \apiset$,
we define $\comp' = \constrainfn(\comp,\apiset)$
such that $\exports(\comp') = \exports(\comp) - \apiset$
and for any component $\comp''$,
$\replacerest(\comp',\comp'')
  \iff \replacerest(\comp,\comp'') \logicand
                \exports(\comp') \not\subseteq \exports(\comp'')$.
The $\imports$, $\visibleAPI$, $\provided$ and $\constituents$ functions
all have the same values for $\comp$ and $\comp'$.
The extra condition on the upgrade compatibility
simply captures the restriction we mentioned above,
that a replacement component should not export every \apiN\
exported by the target.

\paragraph{Hide}
If $\comp$ is a compound component
and $\apiset \subset \visibleAPI(\comp)$ is a set of \apisN\
such that $\exports(\comp) \not\subseteq \apiset$
and $\api \in \visibleAPI(\comp) \logicand \api'\in\uses(\api)
 \intersect \apiset        \implies \api \in \apiset$,
we define $\comp' = \hidefn(\comp,\apiset)$
such that $\visibleAPI(\comp') = \visibleAPI(\comp) - \apiset$,
$\provided(\comp') = \provided(\comp) - \apiset$,
$\exports(\comp') = \exports(\comp) - \apiset$,
and for any component $\comp''$,
$\replacerest(\comp',\comp'')
  \iff \replacerest(\comp,\comp'') \logicand
                \exports(\comp') \not\subseteq \exports(\comp'')
        \logicand \visibleAPI(\comp'') \subseteq \visibleAPI(\comp')$.
The additional clause in $\replacerest(\comp',\comp'')$
(compared with that of $\constrainfn$)
reflects the hiding of the \apisN:
we can no longer upgrade \apisN\ that are hidden.


\paragraph{Upgrade}
The interplay between imported, exported, visible and provided \apisN\
introduces subtleties.  In particular,
the last of the three conditions imposed for well-formedness of upgrades
is modified to state
that for any constituent that is not subsumed by a replacement component,
either it can be upgraded with the replacement,
or its \emph{visible} \apisN\ are disjoint
from the \apisN\ exported by the replacement
(i.e., it is unaffected by the upgrade).
To maintain the invariant that no two constituents export the same \apiN,
we need another condition,
which was implied by the previous condition
when no \apisN\ were constrained or hidden:
if the replacement subsumes any constituents of the target,
then its exported \apisN\ must exactly match
the exported \apisN\ of some subset of the constituents of the target.
That is,
if $\replacerest(\targcomp,\upgcomp)
        \logicand \exists \comp \in \constituents(\targcomp).\
                        \exports(\comp) \subseteq \exports(\upgcomp)$
then $\exports(\upgcomp) = \Union_{\comp \in \compset} \exports(\comp)$
for some $\compset \subset \constituents(\targcomp)$.
In practice, this restriction is rarely a problem;
in most cases, a user wishes to upgrade a target
with a new version of a single constituent component,
where the \apisN\ exported by the old and new versions
are either an exact match,
or there are new \apisN\ introduced by the new component
that have no implementation in the target.
