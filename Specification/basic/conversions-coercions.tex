%%%%%%%%%%%%%%%%%%%%%%%%%%%%%%%%%%%%%%%%%%%%%%%%%%%%%%%%%%%%%%%%%%%%%%%%%%%%%%%%
%   Copyright 2009, Oracle and/or its affiliates.
%   All rights reserved.
%
%
%   Use is subject to license terms.
%
%   This distribution may include materials developed by third parties.
%
%%%%%%%%%%%%%%%%%%%%%%%%%%%%%%%%%%%%%%%%%%%%%%%%%%%%%%%%%%%%%%%%%%%%%%%%%%%%%%%%

\chapter{Conversions and Coercions}
\chaplabel{conversions-coercions}

\note{Widening and where clauses are not yet supported.

The examples in this chapter are not tested nor run by the interpreter.}

\note{
\begin{itemize}
 \item ``has a type'' vs ``is a subtype of the type or could be coerced to the type''
\item Coercion between tuple types (Victor's email titled ``Coercion between tuple types'' on 07/16/07) and arrow types

\item Vectors and matrices: \secref{aggregate-expr}

\item Do we allow coercion declarations in object expressions?

\item If tests and coercion

Suppose a test in an expression has a type that is not \TYP{Boolean}
(or a subtype of \TYP{Boolean}) but is coercible to \TYP{Boolean}.
Presumably, that is not a static error? But note that F1.0 beta
makes no mention of this case; all tests in an if expression must be of
type \TYP{Boolean}. There are similar problems with other expression types;
we need to revisit them all in light of the coercion facility...

\end{itemize}
}

Fortress provides a mechanism, called \emph{coercion}, to allow a
value of one type to be automatically converted to a value of another
type.  Programmers can define coercions (described in
\secref{coercion-declarations}) and coercions may change the set of
applicable declarations for a call (as described in
\secref{applicability-with-coercion}).  If multiple coercions can be
applied to a particular functional call then the most appropriate
coercion is chosen statically.  Restrictions on coercion declarations
(described in \secref{restrictions-coercion}) guarantee that the
static resolution of coercion (described in
\secref{resolving-coercion}) is well-defined.  Fortress provides
implicit coercions for tuple types and arrow types (described in
\secref{coercion-tuple-arrow}).  Fortress also provides an additional
feature of coercion that allows ``widest-need'' evaluation of
numerical expressions (described in \secref{widening}).

\section{Principles of Coercion}
\seclabel{coercion}

In certain situations it is convenient to be able to use a value of
one type \VAR{T} as if it were a value of another type \VAR{U} even
though \VAR{T} is not a subtype of \VAR{U}.  For example, it is
convenient to be able to use an integer-valued expression
in a floating-point expression even though its
type is not a subtype of any floating-point type.  Fortress supports
the automatic conversion of integer values to floating-point values (the
technical term for this kind of automatic conversion is coercion).  In
this way one can write \EXP{(x + 1)/2} rather than \EXP{(x +
1.0)/2.0}, for example.  Such coercion applies generally to function
and method (collectively called as functional) calls as well as to
operators; one can write \EXP{\ln 2} rather than \EXP{\ln 2.0}, or
\EXP{\arctan(1,1)} rather than \EXP{\arctan(1.0,1.0)}, for example.

One way to think about coercion is that if type \VAR{T} can be coerced
to type \VAR{U}, then a value of type \VAR{T} can be used to ``stand
in'' for a value of type \VAR{U} in a functional call, for example.  This is
different from actually \emph{being} of type \VAR{U}; it means only
that, given any value of type \VAR{T}, an appropriate value of type
\VAR{U} can be computed to substitute for it.  Also, coercion occurs
only when the declared type of the corresponding parameter in the
functional declaration is exactly the type \VAR{U} being coerced to
 not if it is a supertype of \VAR{U}.  Note, however, that if type
 \VAR{T} can be coerced to type \VAR{U} then any subtype of \VAR{T}
 can also be coerced to type \VAR{U}.

Coercion from \VAR{T} to \VAR{U} may occur in an expression context
where the context expects the expression to have type \VAR{U} and
the type of the expression is a subtype of \VAR{T} except for the following
cases:
\begin{itemize}
\item type ascriptions (see \secref{type-ascription})
\item type assumptions (see \secref{type-assumption})
\end{itemize}
For example, a coercion may occur in a variable definition when the
declared type of the variable is \VAR{U} and the type of the expression on
the right-hand side is a subtype of \VAR{T}.


The expression contexts where a coercion can occur are:
\begin{itemize}
\item the right-hand sides of variable and field declarations where the
  left-hand sides have declared types
\item arguments to functionals and constructors where the corresponding
  parameters have declared types
\item body expressions of functionals and constructors where the return
  types are declared
\item the right-hand sides of test and property declarations
\item expressions in contracts
\item expressions whose enclosing expressions constrain the types of
  their subexpressions
\end{itemize}
\chapref{expressions} describes the type constraints for each expression in
Fortress.

\note{
  \begin{itemize}
\item assignment expressions with type on the left-hand sides
\item filter in a generator
\item spawn
\item throw
\item while Expr
\item if Expr
\item if . then Expr end
\item at Expr
  \end{itemize}
}

Coercion is \emph{not} automatically chained in Fortress, unlike some
other programming languages: even if type \VAR{T} can be coerced to
type \VAR{U} and type \VAR{U} can be coerced to type \VAR{V}, it is
not necessarily the case that type \VAR{T} can be coerced to type
\VAR{V}.  Type \VAR{T} can be coerced to type \VAR{V} only if this
coercion is explicitly defined.

The basic principles behind coercion are (1) a decision by the
designer of a trait \VAR{U} that, in all circumstances where a
functional is to be called, an actual argument of some other type
\VAR{T} may be used to satisfy a parameter of type
\VAR{T}, and (2) a language mechanism for putting this decision into
effect automatically and concisely.

Example 1: For any floating-point parameter, a decimal integer literal
argument may be used.

Example 2: For any floating-point parameter, a floating-point argument
of a shorter format may be used.

Example 3: For any floating-point interval parameter, a non-interval
floating-point argument (of the same or shorter floating-point format) may be used.



\section{Coercion Declarations}
\seclabel{coercion-declarations}

\begin{Grammar}
\emph{Coercion} &::=&
\KWD{coerce}\option{\emph{StaticParams}}\texttt(\emph{BindId} \emph{IsType}\texttt)\emph{CoercionClauses} \option{\KWD{widens}} \EXP{=} \emph{Expr}\\

\emph{CoercionClauses} &::=&
\option{\emph{CoercionWhere}}
\option{\emph{Ensures}} \option{\emph{Invariant}}\\

\emph{CoercionWhere}
&::=& \KWD{where} \bTPl\ \emph{WhereBindingList} \bTPr\
\options{\{ \emph{CoercionWhereConstraintList} \}}\\
&$|$& \KWD{where} \{ \emph{CoercionWhereConstraintList} \}\\

\emph{CoercionWhereConstraintList} &::=& \emph{CoercionWhereConstraint}(\EXP{,}
\emph{CoercionWhereConstraint})$^*$ \\

\emph{CoercionWhereConstraint} &::=& \emph{WhereConstraint}\\
&$|$& \emph{Type} \KWD{widens} \KWD{or} \KWD{coerces} \emph{Type}\\
\end{Grammar}


To declare that trait \VAR{U} allows a coercion from type \VAR{T},
the declaration of trait \VAR{U} must provide a coercion declaration
whose parameter type is \VAR{T}.
Coercion declarations are like functional declarations, except that
\KWD{coerce} is actually a reserved word,
a coercion declaration may have special \KWD{where}-clause constraints
(described below),
it is not allowed to have a \KWD{throws} clause or a \KWD{requires} clause, and
it is not permitted to specify a return type, because the
return type must be the very trait in whose declaration the coercion
declaration appears.  The coercion body is required; there is
no such thing as an abstract coercion declaration.
Within a trait or object declaration, coercion declarations appear before any
field declarations.


Coercions may have static parameters (described in
\chapref{trait-parameters}) and a \KWD{where} clause (described in
\secref{where-clauses}), just like functionals.  The \KWD{where}
clause may contain one of the following special constraints:
%% Type1 coerces Type2
%% Type1 widens  Type2
%% Type1 widens or coerces Type2
\begin{Fortress}
\(\TYP{Type}_1 \KWD{coerces} \TYP{Type}_2\)\\
\(\TYP{Type}_1 \KWD{widens}  \TYP{Type}_2\)\\
\(\TYP{Type}_1 \KWD{widens}\;\;\KWD{or}\;\;\KWD{coerces} \TYP{Type}_2\)
\end{Fortress}
The first is true if trait \EXP{\TYP{Type}_{1}} has a coercion from
\EXP{\TYP{Type}_{2}}.  The second is true if trait
\EXP{\TYP{Type}_{1}} has a widening coercion (described in \secref{widening})
from \EXP{\TYP{Type}_{2}}.  The third may be used only in a coercion with
the ``\KWD{widens}'' reserved word; it is true if
\EXP{\TYP{Type}_{1}} has a coercion from \EXP{\TYP{Type}_{2}}, and
furthermore the coercion declaration to which the \KWD{where} clause
belongs is widening only if \EXP{\TYP{Type}_{1}} has a widening
coercion from \EXP{\TYP{Type}_{2}}.  For example:
%trait Vector[\T extends Number\]
%  coerce[\S extends Number\](x: Vector[\S\])
%    where {T widens or coerces S} = ...
%end
\begin{Fortress}
\(\KWD{trait}\:\TYP{Vector}\llbracket{}T \KWD{extends}\:\TYP{Number}\rrbracket\)\\
{\tt~~}\pushtabs\=\+\(  \KWD{coerce}\llbracket{}S \KWD{extends}\:\TYP{Number}\rrbracket(x\COLON \TYP{Vector}\llbracket{}S\rrbracket)\)\\
{\tt~~}\pushtabs\=\+\(    \KWD{where} \{T \KWD{widens}\;\;\KWD{or}\;\;\KWD{coerces}\:S\} = \ldots\)\-\-\\\poptabs\poptabs
\(\KWD{end}\)
\end{Fortress}

Coercions, unlike methods, are not inherited.  If trait \VAR{V}
extends trait \VAR{U}, and trait \VAR{U} has a coercion from type
\VAR{T}, then \VAR{V} does \emph{not} thereby have a coercion from
type \VAR{T}.  (It may, however, have its own coercion from type
\VAR{T}, separately defined within the body of \VAR{V}.)  For example,
given the following declarations,
%trait A
%  coerce(b:B) = ...
%end
%object B end
%trait C extends A end
%
%f(c:C) = 5
\begin{Fortress}
\(\KWD{trait} A\)\\
{\tt~~}\pushtabs\=\+\(  \KWD{coerce}(b\COLONOP{}B) = \ldots\)\-\\\poptabs
\(\KWD{end}\)\\
\(\KWD{object} B \KWD{end}\)\\
\(\KWD{trait} C \KWD{extends} A \KWD{end}\)\\[4pt]
\(f(c\COLONOP{}C) = 5\)
\end{Fortress}
a call to \EXP{f(B)} is considered a static error because trait
\VAR{C} does not inherit a coercion from type \VAR{B}.



\section{Coercion Invocations}
\seclabel{using-coercion}

One way to think about coercion is that explicit calls to the
\EXP{\mathrm{coerce}}
function are automatically inserted into the expression in a context
where a coercion can occur.  For example, if trait \VAR{U} has a coercion
from type \VAR{T}, then whenever a functional, \VAR{f}, is declared
with a parameter of type \VAR{U}, a call \EXP{f(t)} where \VAR{t} has
type \VAR{T} can be rewritten to
%f(coerce_[\U\](t))
\EXP{f(\mathrm{coerce}\llbracket{}U\rrbracket(t))}
making the declaration of \VAR{f} applicable to the call.  However, this
rewriting does not occur if there exists a declaration that is
applicable to the call before the rewriting (see
\secref{resolving-coercion} for further discussion).

If coercion is possible for more than one element of a tuple
argument, a cross-product effect is obtained.  For example, in this code:
%object X
%  coerce(kiki:Y) = ...
%  coerce(tutu:Q) = ...
%  hack(other:X) = 1
%end
%object Y end
%object Q end

%foo(dodo:X) = 2

%bar(hyar:X, yon:X) = 3
%bar(hyar:Q, yon:Q) = 4
\begin{Fortress}
\(\KWD{object}\:X\)\\
{\tt~~}\pushtabs\=\+\(  \KWD{coerce}(\VAR{kiki}\COLONOP{}Y) = \ldots\)\\
\(  \KWD{coerce}(\VAR{tutu}\COLONOP{}Q) = \ldots\)\\
\(  \VAR{hack}(\VAR{other}\COLONOP{}X) = 1\)\-\\\poptabs
\(\KWD{end}\)\\
\(\KWD{object}\:Y \KWD{end}\)\\
\(\KWD{object}\:Q \KWD{end}\)\\[4pt]
\(\VAR{foo}(\VAR{dodo}\COLONOP{}X) = 2\)\\[4pt]
\(\VAR{bar}(\VAR{hyar}\COLONOP{}X, \VAR{yon}\COLONOP{}X) = 3\)\\
\(\VAR{bar}(\VAR{hyar}\COLONOP{}Q, \VAR{yon}\COLONOP{}Q) = 4\)
\end{Fortress}
the following method invocations are valid:
%X.hack(Y)
%X.hack(Q)
\begin{Fortress}
\(X.\VAR{hack}(Y)\)\\
\(X.\VAR{hack}(Q)\)
\end{Fortress}
Because there is no declaration of \VAR{hack} applicable to \VAR{Y} and
\VAR{Q}, these invocations are rewritten to:
%X.hack(X.coercion(Y))
%X.hack(X.coercion(Q))

%X.hack(coerce_[\X\](Y))
%X.hack(coerce_[\X\](Q))
\begin{Fortress}
\(X.\VAR{hack}(\mathrm{coerce}\llbracket{}X\rrbracket(Y))\)\\
\(X.\VAR{hack}(\mathrm{coerce}\llbracket{}X\rrbracket(Q))\)
\end{Fortress}
Similarly, the function calls in the left-hand side of the following are
valid and rewritten to the right-hand side:

\begin{tabular}{rcl}
\EXP{\VAR{foo}(Y)} & \emph{is rewritten to} & \EXP{\VAR{foo}(\mathrm{coerce}\llbracket{}X\rrbracket(Y))}\\
\EXP{\VAR{foo}(Q)} & \emph{is rewritten to} & \EXP{\VAR{foo}(\mathrm{coerce}\llbracket{}X\rrbracket(Q))}\\[4pt]
\EXP{\VAR{bar}(Y, X)} & \emph{is rewritten to} & \EXP{\VAR{bar}(\mathrm{coerce}\llbracket{}X\rrbracket(Y), X)}\\
\EXP{\VAR{bar}(Q, X)} & \emph{is rewritten to} & \EXP{\VAR{bar}(\mathrm{coerce}\llbracket{}X\rrbracket(Q), X)}\\
\EXP{\VAR{bar}(X, Y)} & \emph{is rewritten to} & \EXP{\VAR{bar}(X, \mathrm{coerce}\llbracket{}X\rrbracket(Y))}\\
\EXP{\VAR{bar}(X, Q)} & \emph{is rewritten to} & \EXP{\VAR{bar}(X, \mathrm{coerce}\llbracket{}X\rrbracket(Q))}\\
\EXP{\VAR{bar}(Y, Y)} & \emph{is rewritten to} & \EXP{\VAR{bar}(\mathrm{coerce}\llbracket{}X\rrbracket(Y), \mathrm{coerce}\llbracket{}X\rrbracket(Y))}\\
\EXP{\VAR{bar}(Q, Q)} & \emph{is rewritten to} & \EXP{\VAR{bar}(Q, Q)}\\
\EXP{\VAR{bar}(Q, Y)} & \emph{is rewritten to} & \EXP{\VAR{bar}(\mathrm{coerce}\llbracket{}X\rrbracket(Q), \mathrm{coerce}\llbracket{}X\rrbracket(Y))}\\
\EXP{\VAR{bar}(Y, Q)} & \emph{is rewritten to} & \EXP{\VAR{bar}(\mathrm{coerce}\llbracket{}X\rrbracket(Y), \mathrm{coerce}\llbracket{}X\rrbracket(Q))}\\
\end{tabular}

Note that the call \EXP{\VAR{bar}(Q, Q)} remains unchanged because there
exists a declaration for \VAR{bar} that is applicable without
coercion.

To continue the example, let us illustrate a point about type parameters.
If we add the following definitions:
%trait Frobboz
%  frob(item:X) = 5
%  coerce(m:Matrix) = ...  (* irrelevant! *)
%end
%
%object Bozo[\T extends Frobboz\](t:T)  ... end
\begin{Fortress}
\(\KWD{trait} \TYP{Frobboz}\)\\
{\tt~~}\pushtabs\=\+\(  \VAR{frob}(\VAR{item}\COLONOP{}X) = 5\)\\
\(  \KWD{coerce}(m\COLONOP\TYP{Matrix}) = \ldots  \mathtt{(*}\;\hbox{\rm  irrelevant! \unskip}\;\mathtt{*)}\)\-\\\poptabs
\(\KWD{end}\)\\[4pt]
\(\KWD{object} \TYP{Bozo}\llbracket{}T \KWD{extends} \TYP{Frobboz}\rrbracket(t\COLONOP{}T)  \ldots \KWD{end}\)
\end{Fortress}
then this says nothing about whether the type parameter \VAR{T} of \TYP{Bozo}
has coercions, because coercions are not inherited.  (In fact, we can
make a stronger statement: types named by type parameters \emph{do not have coercions}!)
But it is guaranteed that the following method invocations are valid
if they occur within the declaration of \TYP{Bozo}:
\note{(if $t$ has type \TYP{Bozo}:[???]) -- Sukyoung}
%% frob(item:X)
%% frob(item:Y)
%% frob(item:Q)
%\begin{Fortress}
%\(\VAR{frob}(\VAR{item}\COLONOP{}X)\)\\
%\(\VAR{frob}(\VAR{item}\COLONOP{}Y)\)\\
%\(\VAR{frob}(\VAR{item}\COLONOP{}Q)\)
%\end{Fortress}
%t.frob(X)
%t.frob(Y)
%t.frob(Q)
\begin{Fortress}
\(t.\VAR{frob}(X)\)\\
\(t.\VAR{frob}(Y)\)\\
\(t.\VAR{frob}(Q)\)
\end{Fortress}
because \VAR{T} inherits the method declaration
\EXP{\VAR{frob}(\VAR{item}\COLONOP{}X)}
from \TYP{Frobboz} and \VAR{X} contains coercions from \VAR{Y} and \VAR{Q}.



\section{Applicability with Coercion}
\seclabel{applicability-with-coercion}

As discussed in the previous section, coercion in Fortress alters the
set of applicable declarations for a particular functional call.  In
this section we formally define the applicability of declarations with
coercion.  (This level of formality is necessary to discuss the
interaction of overloading and coercion.)
We build on the terminology and notation defined in
\chapref{multiple-dispatch}.

A \emph{coercion} from $T$ to $U$ is defined in the declaration of $U$.
We write $T \coercedefto U$ if $U$ defines a coercion from $T$.
We say that $T$ \emph{can be coerced to} $U$, and we write $T \coercesto
U$, if $U$ defines a coercion from $T$ or any supertype of $T$; that
is, $T \coercesto U \iff \exists T': T \Ovrsubtype T' \logicand T'
\coercedefto U$.

We define coercion between tuple types and arrow types in
\secref{coercion-tuple-arrow}.

Because of automatic coercion, a declaration may be applicable even if
its parameter type is not a supertype of the argument type of the
call.  We define a new relation, \emph{substitutability}, that takes
coercion into account.

We say that a type $T$ is \emph{substitutable} for type $U$, and we
write $T \substitutable U$, if $T$ is a subtype of $U$ or $T$ can be
coerced to $U$; that is,
\[
T \substitutable U \iff T \Ovrsubtype U \logicor T \coercesto U.
\]

Note that $\substitutable$ need not be transitive.  This is a result
of the fact that coercion does not chain.
However, if $T$ is substitutable for $U$ then so are $T$'s subtypes;
that is, $A \Ovrsubtype T \logicand T \substitutable U \implies A
\substitutable U$.

A declaration $\f(\Ps)$ is \emph{applicable with coercion} to a
call $\f(\Cs)$ if the call is in the scope of the declaration and
$\Cs \substitutable \Ps$.
(See \chapref{names} for the definition of scope.)

Recall a rewriting of
named functional calls into dotted method calls when no declarations
are applicable to the named functional call.  This rewriting is also
used to determine applicability with coercion.  If there is no
declaration applicable with coercion to a named functional call then
the call is rewritten into a dotted method call.
Applicable declarations are then
determined by the rules for applicability with coercion for dotted
method calls.

Similarly, a dotted method declaration $\Ps_0.\f(\Ps)$ is
\emph{applicable with coercion} to a dotted method call
$\Cs_0.\f(\Cs)$ if it $\Cs_0 \Ovrsubtype \Ps_0$ and $\Cs
\substitutable \Ps$.  Notice that the receiver is compared using the
subtype relation instead of the substitutable relation.  This reflects
the restriction that receivers of dotted methods cannot be
coerced.  However, \KWD{self} parameters of functional methods can be
coerced.  This distinction is consistent with the intuition that
functional methods are rewritten into top-level functions.

Note that the only difference between applicability and applicability
with coercion is that substitutability is used instead of subtyping
to compare the parameter types of the declarations.

Also note that the definition of applicability with coercion does not
take keyword parameters or varargs parameters into account.  Such
declarations are conceptually rewritten into numerous declarations
which cannot be called with elided arguments nor with variable number of
arguments (as described in \secref{applicability-with-special-params}).  If
one of the expanded declarations is applicable with coercion to a call,
then the original declaration (with keyword or varargs parameters) is
applicable with coercion to the call.



\section{Coercion Resolution}
\seclabel{resolving-coercion}
For a given functional call, we first determine whether there exists a
declaration that is applicable without coercion.  If so, the most
specific declaration is selected; if not, then coercions are
explicitly added to the functional call.  If more than one coercion is
possible then the coercion that yields the most specific type is
chosen.  \secref{restrictions-coercion} and
\chapref{overloaded-declarations} describe overloading rules for the
declarations of coercions and overloaded functionals that guarantee
there exists always a most specific coercion when two or more are possible.

We first define a notion of more specific types in the presence of coercion.
Recall from \secref{type-relations} that Fortress defines an
\emph{exclusion} relation between types which is denoted by $\excludes$.

For types $T$ and $U$, we say that $T$ \emph{rejects} $U$,
and write $T \rejects U$, if for all coercions to $T$, the type coerced
from excludes $U$:
\[
T \rejects U \iff \forall A: A \coercedefto T \implies A \excludes U.
\]
Note that $T \rejects U$ implies $U \not\coercesto T$.

We say that $T$ is \emph{no less specific} than $U$, and write $T
\morespecificeq U$, if $T$ is a subtype of $U$ or $T$ excludes, can be
coerced to, and rejects $U$:
\[
T \morespecificeq U \iff
T \Ovrsubtype U \logicor
(T \excludes U \logicand T \coercesto U \logicand T \rejects U).
\]

The $\morespecificeq$ relation is reflexive and antisymmetric but not
necessarily transitive.  It is possible, for example, for
three types, $T$, $U$ and $V$, each excluding the other two,
to have coercions from $T$ to $U$ and from $U$ to $V$.  In
this case, $T \morespecificeq U$ and $U \morespecificeq V$ but
$T \not\morespecificeq V$.

Notice that if two tuple types have a bijective correspondence
between their element types then the $\morespecificeq$ relationship between
the tuple types is equivalent to applying the $\morespecificeq$
relation elementwise.

We say that $T$ is \emph{more specific} than $U$, and write $T
\morespecific U$, if $T \morespecificeq U \logicand T \neq U$.

We extend the definitions of no less specific and more specific to
declarations.  We say that a declaration $\f(\Ps)$ is \emph{no less
specific} than a declaration $\f(\Qs)$ if $\Ps \morespecificeq \Qs$.
Similarly, we say that a declaration $\Ps_0.\f(\Ps)$ is \emph{no less
specific} than a declaration $\Qs_0.\f(\Qs)$ if $(\Ps_0, \Ps)
\morespecificeq (\Qs_0, \Qs)$.  A declaration $\f(\Ps)$ is \emph{more
specific} than a declaration $\f(\Qs)$ if $\Ps \morespecific \Qs$.
Similarly, a declaration $\Ps_0.\f(\Ps)$ is \emph{more specific} than
a declaration $\Qs_0.\f(\Qs)$ if $(\Ps_0, \Ps) \morespecific (\Qs_0,
\Qs)$.

Now we describe how to determine which coercion is applied to a given
call in terms of rewriting functional calls.  The rewriting is
for pedagogical purposes; implementation techniques may vary.

Consider a static call $\f(\As)$ or $\As_0.\f(\As)$.  Let $\Sigma$ be
the set of parameter types of functional declarations of $\f$ that are
applicable to the call.
Let $\Sigma'$ be the set of parameter types of
functional declarations of $\f$ that are
applicable with coercion to the call.
If $\Sigma$ is not empty then we use the overloading resolution
described in \secref{resolving-overloading} to determine which
element of $\Sigma$ is called.
If $\Sigma$ is empty but $\Sigma'$ is not, then we rewrite the call as
follows.  Define: \[\mathpzc{C} = \{T \in \Sigma' \mid S \coercedefto T
\logicand \As \Ovrsubtype S\}.\] Let $T \in \mathpzc{C}$ be the most
specific element of $\mathpzc{C}$.  In other words, there does not exist
$T' \in \mathpzc{C}$ such that $T' \morespecific T$.  The restrictions
given in \secref{restrictions-coercion} and
\chapref{overloaded-declarations} guarantee that such a $T$
exists and that it is unique.  The call $\f(\As)$ or $\As_0.\f(\As)$ is
rewritten to \EXP{f(\mathrm{coercion}\llbracket{T}\rrbracket(\TYP{A}))} or
\EXP{A_{0}.f(\mathrm{coerce}\llbracket{T}\rrbracket(A))}
and the declaration with parameter
type $T$ is applied to the call.

As an example, consider the following definitions:
%trait ZZ32 end
%trait ZZ64
%  coerce(z:ZZ32) = ...
%end
%trait ZZ128
%  coerce(z:ZZ32) = ...
%  coerce(z:ZZ64) = ...
%end
%
%f(z:ZZ64) = 1
%f(z:ZZ128) = 2
\begin{Fortress}
\(\KWD{trait} \mathbb{Z}32 \KWD{end}\)\\
\(\KWD{trait} \mathbb{Z}64\)\\
{\tt~~}\pushtabs\=\+\(  \KWD{coerce}(z\COLONOP\mathbb{Z}32) = \ldots\)\-\\\poptabs
\(\KWD{end}\)\\
\(\KWD{trait} \mathbb{Z}128\)\\
{\tt~~}\pushtabs\=\+\(  \KWD{coerce}(z\COLONOP\mathbb{Z}32) = \ldots\)\\
\(  \KWD{coerce}(z\COLONOP\mathbb{Z}64) = \ldots\)\-\\\poptabs
\(\KWD{end}\)\\[4pt]
\(f(z\COLONOP\mathbb{Z}64) = 1\)\\
\(f(z\COLONOP\mathbb{Z}128) = 2\)
\end{Fortress}
Then the call \EXP{f(\mathbb{Z}32)} resolves to the
declaration \EXP{f(\mathbb{Z}64)} because $\EXP{\mathbb{Z}64}
\morespecific \EXP{\mathbb{Z}128}$, which follows from the fact that
\EXP{\mathbb{Z}64} coerces to \EXP{\mathbb{Z}128}.

Notice that coercion is resolved statically.  Once a coercion is
statically chosen for a call, this coercion is conceptually inserted
into the call.
This means that the statically chosen coercion is applied at run time.

For example, in the following program:
%trait A
%  coerce(c:C) = ...
%end
%trait B excludes A end
%trait C end
%object D extends {B,C} end
%
%f(a:A) = 3
%f(b:B) = 4
%
%c : C = D
%f(c)
\begin{Fortress}
\(\KWD{trait} A\)\\
{\tt~~}\pushtabs\=\+\(  \KWD{coerce}(c\COLONOP{}C) = \ldots\)\-\\\poptabs
\(\KWD{end}\)\\
\(\KWD{trait} B \KWD{excludes} A \KWD{end}\)\\
\(\KWD{trait} C \KWD{end}\)\\
\(\KWD{object} D \KWD{extends} \{B,C\} \KWD{end}\)\\[4pt]
\(f(a\COLONOP{}A) = 3\)\\
\(f(b\COLONOP{}B) = 4\)\\[4pt]
\(c \mathrel{\mathtt{:}} C = D\)\\
\(f(c)\)
\end{Fortress}
the call to \EXP{f(C)} resolves to the declaration
%\EXP{f(a\COLONOP{}A)}
\EXP{f(A)} despite the fact that the declaration
%\EXP{f(b\COLONOP{}B)}
\EXP{f(B)} is
%dynamically directly
applicable to the dynamic call \EXP{f(D)} and does not require
coercion.




\section{Restrictions on Coercion Declarations}
\seclabel{restrictions-coercion}
We place two restrictions on coercion declarations to ensure that it is
always possible to resolve coercions.

It is a static error for a type to define a coercion from any of its
subtypes.  For example, the following program is statically rejected:
%trait Number
%  coerce(z:ZZ) = ...
%end

%trait ZZ extends Number end
\begin{Fortress}
\(\KWD{trait} \TYP{Number}\)\\
{\tt~~}\pushtabs\=\+\(  \KWD{coerce}(z\COLONOP\mathbb{Z}) = \ldots\)\-\\\poptabs
\(\KWD{end}\)\\[4pt]
\(\KWD{trait} \mathbb{Z} \KWD{extends} \TYP{Number} \KWD{end}\)
\end{Fortress}
Instead, \VAR{z} of type \EXP{\mathbb{Z}} can be coerced to type
\TYP{Number} by
%coerce_[\Number\](z)
``\EXP{\mathrm{coerce}\llbracket\TYP{Number}\rrbracket(z)}''
where \EXP{\mathrm{coerce}} is defined in \library\
(as described in \secref{coercion-expr}).


It is also a static error for cycles to exist in the type hierarchy
produced by extension and coercion declarations.  Such cycles may be
composed of solely subtype relationships or solely coercion or a
mixture of the two.  In all cases the cycles are statically rejected.
An example showing a cycle that is a mixture of subtype and coercion
relationships follows:
%trait A end

%trait B extends A end

%trait C extends B
%  coerce(a:A) = ...
%end
\begin{Fortress}
\(\KWD{trait} A \KWD{end} \)\\[4pt]
\(\KWD{trait} B \KWD{extends} A \KWD{end}\)\\[4pt]
\(\KWD{trait} C \KWD{extends} B\)\\
{\tt~~}\pushtabs\=\+\(  \KWD{coerce}(a\COLONOP{}A) = \ldots\)\-\\\poptabs
\(\KWD{end}\)
\end{Fortress}



\section{Coercions for Tuple and Arrow Types}
\seclabel{coercion-tuple-arrow}

Unlike other types, tuple types (described in \secref{tuple-types}) and
arrow types (described in \secref{arrow-types}) are
not defined explicitly.  Therefore coercions to these types are also
not defined explicitly.  Instead, the following rules describe when
these coercions are implicitly defined.

There is a coercion from a tuple type $\VAR{X}$ to a tuple type
$\VAR{Y}$ if all the following conditions hold:
\begin{enumerate}
\item
$\VAR{X}$ is not a subtype of $\VAR{Y}$;
\item
for every plain type $\VAR{T}$ in $\VAR{Y}$, there is a corresponding
plain type in $\VAR{X}$ whose type is substitutable for $\VAR{T}$;
\item
if neither $\VAR{X}$ nor $\VAR{Y}$ has a varargs type, then they have
the same number of plain types;
\item
if $\VAR{X}$ has a varargs type, then $\VAR{Y}$ has a varargs type,
the type $\VAR{S}$ of the varargs type $\VAR{S}...$ in $\VAR{X}$ is
substitutable for the type $\VAR{T}$ of the varargs type
$\VAR{T}...$ in $\VAR{Y}$, and $\VAR{X}$ and $\VAR{Y}$ have the same
number of plain types;
\item
if $\VAR{X}$ has no varargs type and $\VAR{Y}$ has a varargs type
$\VAR{T}...$, then every plain type in $\VAR{X}$ that has no
corresponding plain type in $\VAR{Y}$ is substitutable for $\VAR{T}$; and
\item
the correspondence between keyword-type pairs in $\VAR{X}$ and
$\VAR{Y}$ is bijective, and the type of each such pair in $\VAR{X}$ is
substitutable for the type in the corresponding pair in $\VAR{Y}$.
\end{enumerate}

Tuple type coercions are invoked by distributing the coercion
elementwise.  However, if an element type of the tuple type being coerced
from is a subtype of the corresponding element type of the tuple type being
coerced to, the coercion is ignored.  For example, the following coercions:
%coerce_[\(A, B)\](x, y)
%coerce_[\(kwd=A)\](kwd=x)
\begin{Fortress}
\(\mathrm{coerce}\llbracket(A, B)\rrbracket(x, y)\)\\
\(\mathrm{coerce}\llbracket(\VAR{kwd}=A)\rrbracket(\VAR{kwd}=x)\)
\end{Fortress}
are rewritten to:
%(coerce_[\A\](x), coerce_[\B\](y))
%(coerce_[\(kwd=A)\](x))
\begin{Fortress}
\((\mathrm{coerce}\llbracket{}A\rrbracket(x), \mathrm{coerce}\llbracket{}B\rrbracket(y))\)\\
\((\mathrm{coerce}\llbracket(\VAR{kwd}=A)\rrbracket(x))\)
\end{Fortress}
However, if the type of \VAR{y} were a subtype of \VAR{B} then the
first coercion would instead be rewritten to:
%(coerce_[\A\](x), y)
\begin{Fortress}
\((\mathrm{coerce}\llbracket{}A\rrbracket(x), y)\)
\end{Fortress}

Coercions to varargs types such as the following:
%coerce_[\(A...)\](x, y)
\begin{Fortress}
\(\mathrm{coerce}\llbracket(A\ldots)\rrbracket(x, y)\)
\end{Fortress}
are invoked by applying the coercion to the type in the varargs type,
\VAR{A} in this example, to each of the elements of the tuple.
Additionally, there is a coercion from any type \VAR{T} to a tuple type
solely with varargs type \EXP{(T\ldots)}.



There is a coercion from arrow type
``\EXP{A \rightarrow B \KWD{throws} C}'' to arrow type
``\EXP{D \rightarrow E \KWD{throws} F}'' if all of the following
conditions hold:
\begin{enumerate}
\item ``\EXP{A \rightarrow B \KWD{throws} C}'' is not a subtype of
``\EXP{D \rightarrow E \KWD{throws} F}'';
\item
$\VAR{D}$ is substitutable for $\VAR{A}$;
\item
$\VAR{B}$ is substitutable for $\VAR{E}$; and
\item for all $X$ in $C$, there exists $Y$ in $F$ such that $X$ is
 substitutable for $Y$.
\end{enumerate}

Arrow type coercions are invoked by wrapping the argument function of the
coercion in a function expression that applies the appropriate
coercions to the argument and result of the function.  For example,
assuming that \VAR{f} is a function from type $\VAR{A}$ to type
$\VAR{B}$ the following coercion:
%coerce_[\C -> D\](f)
\begin{Fortress}
\(\mathrm{coerce}\llbracket{}C \rightarrow D\rrbracket(f)\)
\end{Fortress}
is rewritten to:
%fn x => coerce_[\D\](f(coerce_[\A\](x)))
\begin{Fortress}
\(\KWD{fn}\:x \Rightarrow \mathrm{coerce}\llbracket{}D\rrbracket(f(\mathrm{coerce}\llbracket{}A\rrbracket(x)))\)
\end{Fortress}


\section{Automatic Widening}
\seclabel{widening}

\begin{Grammar}
\emph{Coercion} &::=&
\KWD{coerce}\option{\emph{StaticParams}}\texttt(\emph{BindId} \emph{IsType}\texttt)\emph{CoercionClauses} \KWD{widens} \EXP{=} \emph{Expr}\\

\end{Grammar}


Fortress supports what is sometimes called ``widest-need'' evaluation
of numerical expressions.  To see the problem, suppose that \VAR{a}
and \VAR{b} are 32-bit floating-point numbers and \VAR{c} is a 64-bit
floating-point number.  It is easy, and tempting, to write an
expression such as
%% c = c + a \cdot b
\begin{Fortress}
\(c = c + a \cdot b\)
\end{Fortress}
but if you stop to think about it, if interpreted naively it will
compute the product \EXP{a \cdot b} as a 32-bit floating-point number,
and the impression that the sum may be accurate to the precision of a
64-bit floating-point number is only an illusion.  Widest-need
processing takes context into account and ``widens'' (or ``upgrades'')
the operands \VAR{a} and \VAR{b} to 64-bit floating-point numbers
before the product is computed, so that the product will be computed
as a 64-bit result.

Before widening is considered, a Fortress compiler,
in its normal course of operation, analyzes
an expression bottom-up.  For every functional invocation, it needs
to statically identify a specific declaration to be invoked.
Once this is done, then for %every argument position of
every functional invocation, one of two conditions holds:
(a) The type of the argument expression is a subtype of the
    type of the parameter in the identified declaration.
(b) The type of the argument expression can be coerced to the
    type of the parameter in the identified declaration.

This process has to be done bottom-up because in order to select a
specific declaration for a functional invocation, it is necessary to
know the types of the argument expressions, and if an argument
expression is itself a functional invocation, it is necessary to
select the specific declaration for that invocation in order to find
out the return type of the invocation.

Once an expression has been fully analyzed in this way, then widening
is performed as a top-down process.  For each functional invocation
that appears in a context where a coercion can occur, ask
whether the argument expression (which, remember, is a functional invocation)
falls under case (a) or case (b) above.  If case (b), coercion of the
functional result is required; if the relevant coercion declaration is
a ``widening'' coercion, then this functional invocation is a
candidate for widening.  Call the type of the functional invocation
\VAR{T}, and call the expected type of the coercion context, \VAR{U}.


When the functional call was considered during the bottom-up analysis,
in general many overloaded declarations may have been considered; of
all those that were applicable with coercion
to the static call, the most specific one was chosen.
When considering widening, we consider the same set of declarations, but
first discard all declarations whose return types are not subtypes of
\VAR{U}.  Then if any remain, and there is a unique most specific one
among them, then that declaration is attached to the functional call
instead.  In this way a coercion step is eliminated (coercion is no
longer needed to convert the result of the functional invocation from
type \VAR{T} to type \VAR{U}), possibly at the expense of requiring
a new coercion in a subexpression---but this is exactly the desired
effect.  Once this is done, then the subexpressions of the functional
call are processed recursively.


This process is general enough not only to provide for
widening floating-point precision, but to handle two other cases
of interest as well: widening point computations to interval computations,
and widening computations involving aggregate data structures whose
components are widenable.  For example, in \EXP{d (v \times w)} suppose that
\VAR{d} is of type $\mathbb{R}64$ and \VAR{v} and \VAR{w} are 3-vectors with components
of type $\mathbb{R}32$; this strategy is capable of promoting \VAR{v} and \VAR{w}
to 3-vectors with components of type $\mathbb{R}64$ and then performing all
computations with $\mathbb{R}64$ precision.

Let's go through the example of \EXP{c + a \cdot b} in detail.
The relevant declarations might look something like this:
%trait RR32
%  ...
%end
%
%trait RR64
%  coerce(x:RR32) widens = ...
%  ...
%end

%opr DOT(l: RR32, r: RR32): RR32 = ... (* 1 *)
%opr DOT(l: RR64, r: RR64): RR64 = ... (* 2 *)
%opr +(l: RR32, r: RR32): RR32 = ... (* 3 *)
%opr +(l: RR64, r: RR64): RR64 = ... (* 4 *)
\begin{Fortress}
\(\KWD{trait}\:\mathbb{R}32\)\\
{\tt~~}\pushtabs\=\+\(  \ldots\)\-\\\poptabs
\(\KWD{end}\)\\[4pt]
\(\KWD{trait}\:\mathbb{R}64\)\\
{\tt~~}\pushtabs\=\+\(  \KWD{coerce}(x\COLONOP\mathbb{R}32) \KWD{widens}\: = \ldots\)\\
\(  \ldots\)\-\\\poptabs
\(\KWD{end}\)\\[4pt]
\(\KWD{opr} \mathord{\cdot}(l\COLON \mathbb{R}32, r\COLON \mathbb{R}32)\COLON \mathbb{R}32 = \ldots \mathtt{(*}\;\hbox{\rm  1 \unskip}\;\mathtt{*)}\)\\
\(\KWD{opr} \mathord{\cdot}(l\COLON \mathbb{R}64, r\COLON \mathbb{R}64)\COLON \mathbb{R}64 = \ldots \mathtt{(*}\;\hbox{\rm  2 \unskip}\;\mathtt{*)}\)\\
\(\KWD{opr} +(l\COLON \mathbb{R}32, r\COLON \mathbb{R}32)\COLON \mathbb{R}32 = \ldots \mathtt{(*}\;\hbox{\rm  3 \unskip}\;\mathtt{*)}\)\\
\(\KWD{opr} +(l\COLON \mathbb{R}64, r\COLON \mathbb{R}64)\COLON \mathbb{R}64 = \ldots \mathtt{(*}\;\hbox{\rm  4 \unskip}\;\mathtt{*)}\)
\end{Fortress}

The bottom-up analysis observes that \VAR{a} and \VAR{b} are both of type
\EXP{\mathbb{R}32}, and discovers that declarations \EXP{1} and \EXP{2}
are both applicable to the product, but declaration \EXP{1}
would be chosen because it requires no coercion and declaration \EXP{2}
would require coercion of both arguments to type \EXP{\mathbb{R}64}.
The analysis then observes that \VAR{c} has type \EXP{\mathbb{R}64}
and the product expression has type \EXP{\mathbb{R}32}, so declaration
\EXP{3} is not applicable but declaration \EXP{4} is applicable
(though requiring a coercion from \EXP{\mathbb{R}32} to \EXP{\mathbb{R}64} for its second argument).

Now the top-down widening analysis is performed.  The product is
a functional call that is an argument to another functional call,
and its result requires coercion, and that coercion is a widening
coercion.  Therefore the compiler reconsiders the applicable
declarations, which were declarations \EXP{1} and \EXP{2}.
The result type of declaration \EXP{1} is not a subtype
of \EXP{\mathbb{R}64}, but the result type of declaration \EXP{2}
is a subtype of \EXP{\mathbb{R}64} (in fact, it \emph{is} \EXP{\mathbb{R}64}).
Declaration \EXP{2} is the most specific among the applicable declarations
with that property (in fact, in this example it's the only one),
so declaration \EXP{2} replaces declaration \EXP{1} for the
product invocation.  This in turn requires \VAR{a} and \VAR{b}
to be coerced from type \EXP{\mathbb{R}32} to \EXP{\mathbb{R}64}
before the product is performed.

Widening is a tricky business, best left to expert library designers.
When used judiciously, it can greatly improve the precision of
numerical computations without requiring a great deal of thought on
the part of the application programmer and without cluttering up code
with explicit conversions.


\emph{Future versions of this specification will include tables of
  coercions and widening coercions supported by \library.}
