%%%%%%%%%%%%%%%%%%%%%%%%%%%%%%%%%%%%%%%%%%%%%%%%%%%%%%%%%%%%%%%%%%%%%%%%%%%%%%%%
%   Copyright 2009, Oracle and/or its affiliates.
%   All rights reserved.
%
%
%   Use is subject to license terms.
%
%   This distribution may include materials developed by third parties.
%
%%%%%%%%%%%%%%%%%%%%%%%%%%%%%%%%%%%%%%%%%%%%%%%%%%%%%%%%%%%%%%%%%%%%%%%%%%%%%%%%

\chapter{Exceptions}
\chaplabel{exceptions}

\note{Methods and fields of exceptions, chained exceptions,
and static checking of checked exceptions are not yet supported.

Fortress examples in this chapter are not run by the interpreter.}

Exceptions are values that can be thrown and caught, via \KWD{throw}
expressions (described in \secref{throw-expr})
and \KWD{catch} clauses of \KWD{try} expressions (described in
\secref{try-expr}).
When a \KWD{throw} expression ``\EXP{ \KWD{throw} e}''
is evaluated, the subexpression \VAR{e} is evaluated to an exception.
The static type of \VAR{e} must be a subtype of \TYP{Exception}.
Then the \KWD{throw} expression tries to transfer control to
its \emph{dynamically containing block} (described in
\chapref{evaluation}), from the innermost outward, until
either (\emph{i}) an enclosing \KWD{try} expression is reached, with a
\KWD{catch} clause matching a type of the thrown exception, or (\emph{ii})
the outermost dynamically containing block is reached.

If a matching \KWD{catch} clause is reached,
the right-hand side of the first matching subclause is evaluated.  If no
matching \KWD{catch} clause is found before the outermost dynamically
containing block is reached, the outermost dynamically containing block
completes abruptly whose cause is the thrown exception.

If an enclosing \KWD{try} expression of a \KWD{throw} expression
includes a \KWD{finally} clause, and the \KWD{try} expression completes
abruptly, the \KWD{finally} clause is evaluated before control is
transferred to the dynamically containing block.


\section{Causes of Exceptions}

Every exception is thrown for one of the following reasons:
\begin{enumerate}
\item A \KWD{throw} expression is evaluated.
\item An implementation resource is exceeded (e.g., an attempt is made to
  allocate beyond the set of available locations).
\end{enumerate}


\section{Types of Exceptions}
\seclabel{exception-types}

All exceptions are subtypes of the type \TYP{Exception} declared as follows:
%% \input{\home/library/examples/ExceptionLibrary.tex}
%trait Exception comprises { CheckedException, UncheckedException }
%  settable message: Maybe[\String\]
%  settable chain: Maybe[\Exception\]
%  printStackTrace(): ()
%end
\begin{Fortress}
\(\KWD{trait}\:\TYP{Exception} \KWD{comprises} \{\,\TYP{CheckedException}, \TYP{UncheckedException}\,\}\)\\
{\tt~~}\pushtabs\=\+\(  \KWD{settable}\:\VAR{message}\COLON \TYP{Maybe}\llbracket\TYP{String}\rrbracket\)\\
\(  \KWD{settable}\:\VAR{chain}\COLON \TYP{Maybe}\llbracket\TYP{Exception}\rrbracket\)\\
\(  \VAR{printStackTrace}()\COLON ()\)\-\\\poptabs
\(\KWD{end}\)
\end{Fortress}
Every exception is a subtype of either type
\TYP{CheckedException} or \TYP{UncheckedException}:
\input{\home/library/examples/Exceptions.tex}

A functional declaration (described in \secref{methods}
and \secref{function-decls})
includes an optional \KWD{throws} clause in its
header listing the \TYP{CheckedException}s (also written
\emph{checked exceptions})
that can be thrown by invocation of the functional.  If a \KWD{throws}
clause is not explicitly provided, the \KWD{throws} clause of the
functional declaration is empty.  The body of a
functional is statically checked to ensure that no checked exceptions
are thrown by any subexpression of the functional
body other than those listed in the \KWD{throws} clause.
This static check is performed by examining each \KWD{throw} expression and
functional invocation $I$, determining the static
type of the functional $f$ invoked in $I$, and determining the
\KWD{throws} clause of
$f$. (If $f$ is polymorphic, or occurs in a polymorphic context,
instantiations of type variables free in the \KWD{throws} clause
of $f$  are substituted for formal type variables).
For each checked exception thrown in $I$, the enclosing expressions of
$I$ are checked for a matching \KWD{catch} clause.
The set $A$ of all
checked exceptions thrown by all invocations without a matching
\KWD{catch} clause in the functional body is accumulated and compared
against the \KWD{throws} clause of the enclosing functional declaration. If an
exception that is not a subtype of an exception
listed in the \KWD{throws} clause occurs in $A$, it is a static error.

A similar analysis is performed on top-level variable declarations.
If it is determined that their initialization expressions
can throw a checked exception, it is a static error.

\section{Information of Exceptions}
\seclabel{chained-exceptions}

Every exception has optional fields: a message and a chained exception.
These fields are default to \TYP{Nothing} as follows:
%trait Exception comprises { CheckedException, UncheckedException }
%  getter message(): Maybe[\String\] = Nothing
%  setter message(Maybe[\String\]):()
%  getter chain(): Maybe[\Exception\] = Nothing
%  setter chain(Maybe[\Exception)\]:()
%  printStackTrace(): ()
%end
\begin{Fortress}
\(\KWD{trait} \TYP{Exception} \KWD{comprises} \{\,\TYP{CheckedException}, \TYP{UncheckedException}\,\}\)\\
{\tt~~}\pushtabs\=\+\(  \KWD{getter} \VAR{message}()\COLON \TYP{Maybe}\llbracket\TYP{String}\rrbracket = \TYP{Nothing}\)\\
\(  \KWD{setter} \VAR{message}(\TYP{Maybe}\llbracket\TYP{String}\rrbracket)\COLONOP()\)\\
\(  \KWD{getter} \VAR{chain}()\COLON \TYP{Maybe}\llbracket\TYP{Exception}\rrbracket = \TYP{Nothing}\)\\
\(  \KWD{setter} \VAR{chain}(\TYP{Maybe}\llbracket\TYP{Exception}\rrbracket)\COLONOP()\)\\
\(  \VAR{printStackTrace}()\COLON ()\)\-\\\poptabs
\(\KWD{end}\)
\end{Fortress}
where an optional value \VAR{v} is either \TYP{Nothing} or
\EXP{\TYP{Just}(v)}.
The \VAR{chain} field can be set at most once.  If it is set more than
once, an \TYP{InvalidChain} exception is thrown.
It is generally set when the exception is created.


When an exception is created, the execution stack of its thread at the time
of the exception creation is captured in the exception.  The invocation of
\VAR{printStackTrace} prints the captured stack trace.
There is no way to update the captured stack trace.
If a programmer wants to catch a thrown exception and rethrow it, and
capture the stack trace at the time of the second throwing of the
exception, the programmer has to create a new exception (perhaps with the
original exception as its \VAR{chain} field).

When an exception is thrown, its \VAR{message} and \VAR{chain} fields may
be set.
For example, if a checked exception is caught in a \KWD{catch} clause, and
the \KWD{catch} clause in turn throws an unchecked exception, the unchecked
exception can be chained so that an examination of the unchecked exception
reveals information about the original exception.  For example:
%read(fileName) =
%  try
%    readFile(fileName)
%  catch e
%    IOFailure => throw Error("This code can't handle IOFailures", e)
%  end
\begin{Fortress}
\(\VAR{read}(\VAR{fileName}) = \)\\
{\tt~~}\pushtabs\=\+\(  \KWD{try}\)\\
{\tt~~}\pushtabs\=\+\(    \VAR{readFile}(\VAR{fileName})\)\-\\\poptabs
\(  \KWD{catch} e\)\\
{\tt~~}\pushtabs\=\+\(    \TYP{IOFailure} \Rightarrow\ \KWD{throw} \TYP{Error}(\hbox{\rm``\STR{This~code~can't~handle~IOFailures}''}, e)\)\-\\\poptabs
\(  \KWD{end}\)\-\\\poptabs
\end{Fortress}
where the \VAR{message} and \VAR{chain} fields of \TYP{Error} are set to
\EXP{\hbox{\rm``\STR{This~code~can't~handle~IOFailures}''}} and
\TYP{IOFailure} respectively.


By default, a \KWD{forbid} clause in a \KWD{try} expression
throws a new \TYP{ForbiddenException} by \emph{chaining}
the exception thrown by the \KWD{try} block in the \KWD{try}
expression that is a subtype of the exception type listed in the
\KWD{forbid} clause.
For example, the following \VAR{read} function:
%read(fileName) =
%  try
%    readFile(fileName)
%  forbid IOFailure
%  end
\begin{Fortress}
\(\VAR{read}(\VAR{fileName}) = \)\\
{\tt~~}\pushtabs\=\+\(  \KWD{try}\)\\
{\tt~~}\pushtabs\=\+\(    \VAR{readFile}(\VAR{fileName})\)\-\\\poptabs
\(  \KWD{forbid} \TYP{IOFailure}\)\\
\(  \KWD{end}\)\-\\\poptabs
\end{Fortress}
is equivalent to:
%read(fileName) =
%  try
%    readFile(fileName)
%  catch e
%    IOFailure => throw ForbiddenException(e)
%  end
\begin{Fortress}
\(\VAR{read}(\VAR{fileName}) = \)\\
{\tt~~}\pushtabs\=\+\(  \KWD{try}\)\\
{\tt~~}\pushtabs\=\+\(    \VAR{readFile}(\VAR{fileName})\)\-\\\poptabs
\(  \KWD{catch} e\)\\
{\tt~~}\pushtabs\=\+\(    \TYP{IOFailure} \Rightarrow \KWD{throw} \TYP{ForbiddenException}(e)\)\-\\\poptabs
\(  \KWD{end}\)\-\\\poptabs
\end{Fortress}
where the \VAR{chain} of \TYP{ForbiddenException} is set to \TYP{IOFailure}.
