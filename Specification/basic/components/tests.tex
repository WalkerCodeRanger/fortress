%%%%%%%%%%%%%%%%%%%%%%%%%%%%%%%%%%%%%%%%%%%%%%%%%%%%%%%%%%%%%%%%%%%%%%%%%%%%%%%%
%   Copyright 2009 Sun Microsystems, Inc.,
%   4150 Network Circle, Santa Clara, California 95054, U.S.A.
%   All rights reserved.
%
%   U.S. Government Rights - Commercial software.
%   Government users are subject to the Sun Microsystems, Inc. standard
%   license agreement and applicable provisions of the FAR and its supplements.
%
%   Use is subject to license terms.
%
%   This distribution may include materials developed by third parties.
%
%   Sun, Sun Microsystems, the Sun logo and Java are trademarks or registered
%   trademarks of Sun Microsystems, Inc. in the U.S. and other countries.
%%%%%%%%%%%%%%%%%%%%%%%%%%%%%%%%%%%%%%%%%%%%%%%%%%%%%%%%%%%%%%%%%%%%%%%%%%%%%%%%

\section{Tests in Components and \Apis}
\seclabel{component-tests}

A component may include definitions of tests,
as described in \chapref{tests}.
These definitions are allowed to refer to both test and non-test code defined in the same
component or declared in \apisN\ imported by the component.

An \apiN\ may also include definitions of tests.
These definitions may refer to all declarations in the \apiN\ as well as in
any \apisN\ it imports. Tests defined in \apisN\ should be thought of as ``executable documentation''
that partially specifies the required behavior of the declared entities.

See \secref{basicops} for an explanation of how tests
defined in components and \apisN\ are executed.
