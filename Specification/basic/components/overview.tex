%%%%%%%%%%%%%%%%%%%%%%%%%%%%%%%%%%%%%%%%%%%%%%%%%%%%%%%%%%%%%%%%%%%%%%%%%%%%%%%%
%   Copyright 2009, Oracle and/or its affiliates.
%   All rights reserved.
%
%
%   Use is subject to license terms.
%
%   This distribution may include materials developed by third parties.
%
%%%%%%%%%%%%%%%%%%%%%%%%%%%%%%%%%%%%%%%%%%%%%%%%%%%%%%%%%%%%%%%%%%%%%%%%%%%%%%%%

\section{Overview}
\begin{Grammar}
\emph{File} &::=& \emph{CompilationUnit}\\
&$|$& \option{\emph{Imports}} \emph{Exports} \option{\emph{Decls}}\\
&$|$& \option{\emph{Imports}} \emph{AbsDecls}\\
&$|$& \emph{Imports} \option{\emph{AbsDecls}}\\

\emph{CompilationUnit} &::=& \emph{Component}\\
&$|$&\emph{Api}\\
\end{Grammar}

Components are the fundamental structure of Fortress programs.
They export and import \apisN,
which serve as ``interfaces'' of the components.
A key design choice we make
is to require that components never refer
to other components directly;
all external references are to \apisN.
This requirement allows programmers
to extend and test existing components more easily,
swapping new implementations of libraries in and out of programs at will.
External references are resolved by linking components together:
the references of a component to an imported \apiN\
are resolved to a component that exports that \apiN.
Linking components produces new components,
whose \emph{constituents} are the components that were linked together.

Components are similar to modules in other programming languages,
such as those of ML and Scheme \cite{SML, OCaml, Scheme}.
But, unlike modules in those languages, components
are designed for use during both development and deployment of software.
In addition to compilation and linking,
components can be produced by upgrading one component
using another component that exports
some of the \apisN\ exported by the first component.

A key aspect of Fortress components is that they are encapsulated,
so that upgrading one component does not affect any other component,
even those produced by linking with the component that was upgraded.
Abstractly, each component has its own copy of its constituents.
However, implementations are expected to share common constituents
when possible.

Users do not manipulate components directly.
Instead,
every component is installed in a persistent database on the system.
We think of this database,
which we call a \emph{fortress},
as the agent that actually performs operations
such as compilation, linking, upgrading,
and execution of components:
a virtual machine, a compiler, and a library registry all rolled into one.
A fortress also maintains a list of \apisN\ that are installed on it.
A fortress also provides a shell
by which the user can issue commands to it.
Components and \apisN\ are immutable objects.
A fortress maps names to components installed on the system.
The fortress operations are modeled
as methods of the fortress that change the mapping.


The ways in which fortresses are actually realized on particular platforms
are beyond the scope of this specification.
An implementor might choose to instantiate a fortress as a process, or as
a persistent object database stored in a file system,
with fortress operations being implemented as scripts
that manipulate this database.
