%%%%%%%%%%%%%%%%%%%%%%%%%%%%%%%%%%%%%%%%%%%%%%%%%%%%%%%%%%%%%%%%%%%%%%%%%%%%%%%%
%   Copyright 2009, Oracle and/or its affiliates.
%   All rights reserved.
%
%
%   Use is subject to license terms.
%
%   This distribution may include materials developed by third parties.
%
%%%%%%%%%%%%%%%%%%%%%%%%%%%%%%%%%%%%%%%%%%%%%%%%%%%%%%%%%%%%%%%%%%%%%%%%%%%%%%%%

\section{Initialization Order for Components}
\seclabel{initialization}

To ensure that all objects and all variables are initialized before their
use, execution of program components
proceeds according to the procedure defined in this section.
This procedure assumes that the program's type hierarchy is already checked
to be acyclic.


If a component is a compound component,
all constituent components are initialized nondeterministically, but before
first use.
If a simple component has imports,
take the transitive closure of all imported \apisN.  Collect all
  declarations in this transitive closure, in any order, and prepend
  them to the component definition.
Finally, for a simple component without imports,
initialize all top-level variables and singleton object fields.
