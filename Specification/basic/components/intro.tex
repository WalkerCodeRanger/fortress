%%%%%%%%%%%%%%%%%%%%%%%%%%%%%%%%%%%%%%%%%%%%%%%%%%%%%%%%%%%%%%%%%%%%%%%%%%%%%%%%
%   Copyright 2009 Sun Microsystems, Inc.,
%   4150 Network Circle, Santa Clara, California 95054, U.S.A.
%   All rights reserved.
%
%   U.S. Government Rights - Commercial software.
%   Government users are subject to the Sun Microsystems, Inc. standard
%   license agreement and applicable provisions of the FAR and its supplements.
%
%   Use is subject to license terms.
%
%   This distribution may include materials developed by third parties.
%
%   Sun, Sun Microsystems, the Sun logo and Java are trademarks or registered
%   trademarks of Sun Microsystems, Inc. in the U.S. and other countries.
%%%%%%%%%%%%%%%%%%%%%%%%%%%%%%%%%%%%%%%%%%%%%%%%%%%%%%%%%%%%%%%%%%%%%%%%%%%%%%%%

\note{This chapter is out of date.}

Fortress programs
are developed, compiled, and deployed as
\emph{encapsulated upgradable components}
that exist not only as programming language features, but also as
self-contained run-time entities that are managed
throughout the life of the software.
The imported and exported references of a component
are described with explicit \emph{\apisN}.
With components and \apisN, Fortress
provides the stability benefits of
static linking with the sharing and upgrading benefits of
dynamic linking.
\footnote{The system described in this chapter is based on that
described in \cite{allen-05-components}.}
In addition to an informal description of the component system in this
chapter, we also formally specify key functionality of the system,
and illustrate how we can reason about the correctness of the system
in \appref{components}.
