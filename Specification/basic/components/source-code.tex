%%%%%%%%%%%%%%%%%%%%%%%%%%%%%%%%%%%%%%%%%%%%%%%%%%%%%%%%%%%%%%%%%%%%%%%%%%%%%%%%
%   Copyright 2009, Oracle and/or its affiliates.
%   All rights reserved.
%
%
%   Use is subject to license terms.
%
%   This distribution may include materials developed by third parties.
%
%%%%%%%%%%%%%%%%%%%%%%%%%%%%%%%%%%%%%%%%%%%%%%%%%%%%%%%%%%%%%%%%%%%%%%%%%%%%%%%%

\note{
Victor:

Define (in lexical structure?) when program constructs are ``the same''
 (i.e., identical except for ``syntax'' issues).  Or else, we may want o
 to include a generic statement in the lexical structure chapter saying
 that after that chapter, except for the appendices, unless otherwise
 stated,  two fragments of program text are considered the same if they
 are scanned into the same sequence of tokens (we need to be careful
 about the whitespace-sensitivity though).}

\seclabel{source-code}

We call the source code for a single software component a \emph{project}.
Typically, when a project written in other programming languages is compiled,
each file in the project is separately compiled.
To ship an application,
these files are linked together to form an application or library.
Fortress uses a different model:
a project is compiled directly into a single component,
which is installed in the compiling fortress.

From the point of view of the compiler, all the source code for a
project is contained in a single file.  This approach simplifies the design
but it is unworkable for
components of substantial size.  Therefore, the compiler can be
instructed to concatenate several source files together before
compiling, while maintaining the original source location
information.

After these components are compiled from source files, they can then
be linked together to form larger components.





\section{Components}
\seclabel{components}

\note{``\KWD{import} \KWD{api} \emph{AliasedAPINames}'' syntax and
qualified names are not yet supported.}

\begin{Grammar}
\emph{Component} &::=&
\option{\KWD{native}}
\KWD{component} \emph{APIName} \option{\emph{Imports}} \emph{Exports}
\option{\emph{Decls}} \KWD{end}\\
& &\options{\option{\KWD{component}} \emph{APIName}}\\

\emph{APIName} &::=& \emph{Id}(\EXP{.}\emph{Id})$^*$\\

\emph{Imports} &::=& \emph{Import}$^+$\\
\emph{Import}
&::=& \KWD{import} \emph{ImportedNames} \\
&$|$& \KWD{import} \KWD{api} \emph{AliasedAPINames} \\

\emph{ImportedNames}
&::=& \emph{APIName} \EXP{.} \{ \EXP{...} \}
\options{\KWD{except} \emph{SimpleNames}}
\\
&$|$& \emph{APIName} \EXP{.}
\{ \emph{SimpleNameList} \options{\EXP{,} \EXP{...}} \}\\
\{ \emph{AliasedSimpleNameList} \options{\EXP{,} \EXP{...}} \}\\
&$|$& \emph{QualifiedName} \options{\KWD{as} \emph{Id}}\\

\emph{SimpleNames} &::=& \emph{SimpleName} \\
&$|$& \{ \emph{SimpleNameList} \}\\

\emph{SimpleNameList} &::=& \emph{SimpleName}(\EXP{,} \emph{SimpleName})$^*$ \\

\end{Grammar}

\begin{GrammarTwo}
\emph{SimpleName} &::=& \emph{Id} \\
&$|$& \KWD{opr} \option{\KWD{BIG}} (\emph{Encloser} $|$ \emph{Op}) \\
&$|$& \KWD{opr} \option{\KWD{BIG}} \emph{EncloserPair} \\

\emph{AliasedSimpleNameList}
&::=& \emph{AliasedSimpleName}(\EXP{,} \emph{AliasedSimpleName})$^*$ \\

\emph{AliasedSimpleName} &::=& \emph{Id} \options{\KWD{as} \emph{Id}}\\
&$|$& \KWD{opr} \emph{Op} \options{\KWD{as} \emph{Op}} \\
&$|$& \KWD{opr} \emph{EncloserPair} \options{\KWD{as} \emph{EncloserPair}} \\

\emph{EncloserPair} &::=&
(\emph{LeftEncloser} $|$ \emph{Encloser}) \option{\EXP{\cdot}}
(\emph{RightEncloser} $|$ \emph{Encloser})\\

\emph{AliasedAPINames} &::=& \emph{AliasedAPIName} \\
&$|$& \{ \emph{AliasedAPINameList} \}\\

\emph{AliasedAPIName} &::=& \emph{APIName} \options{\KWD{as} \emph{Id}}\\

\emph{AliasedAPINameList} &::=& \emph{AliasedAPIName}(\EXP{,} \emph{AliasedAPIName})$^*$ \\

\emph{Exports} &::=& \emph{Export}$^+$\\

\emph{Export} &::=& \KWD{export} \emph{APINames} \\

\emph{APINames} &::=& \emph{APIName} \\
&$|$& \{ \emph{APINameList} \}\\

\emph{APINameList} &::=& \emph{APIName}(\EXP{,} \emph{APIName})$^*$ \\

\emph{Decls} &::=& \emph{Decl}$^+$\\
\end{GrammarTwo}

In this specification, we will refer to components created by
compiling a file as \emph{simple components}, while components created by
linking components together will be known as \emph{compound
components}.

The source code of a simple component definition
begins with an optional modifier \KWD{native} followed by
\KWD{component} followed by a
\emph{possibly qualified name} (an identifier or a sequence of
identifiers separated by periods with no intervening whitespace),
followed by a sequence of \emph{import} statements,
and a sequence of \emph{export} statements,
and finally a sequence of declarations,
where all sequences are separated by newlines or semicolons.
A component declaration may end with an optional \KWD{component} and its name again.
It is a static error if the identifier after \KWD{end} is not
the name of the component being declared.


\subsection{Import Statements}
An import statement imports declarations from one or more APIs.
There are two forms of import statements.
The first form imports all top-level and functional method declarations
(i.e., declarations whose reach is the entire API)
from the listed APIs,
and allows the names declared by these declarations
to be used \emph{qualified} by the name of the API
from which it is imported.
%import apiName, anotherApiName
\begin{Fortress}
\(\KWD{import} \VAR{apiName}, \VAR{anotherApiName}\)
\end{Fortress}
This form may also provide an \emph{alias}
for one or more of the APIs.
%import apiName as alias, anotherApiName
\begin{Fortress}
\(\KWD{import} \VAR{apiName} \KWD{as} \VAR{alias}, \VAR{anotherApiName}\)
\end{Fortress}
A name imported from an API with an alias
must be qualified by the alias
rather than the name of the API.
It is a static error for any alias
to be the name of any imported API.
Multiple APIs may be given the same alias,
but it is a static error
if declarations imported from two APIs with the same alias
declare the same name
unless the set of imported declarations with that name
from APIs with that alias
form a valid set of overloaded declarations.

There are two forms of import statements: \emph{explicit import statements} and
\emph{on-demand import statements}.

An explicit import statement specifies a single API and a set of names;
it imports the top-level and functional method declarations in the specified API.
It allows the specified names to be used \emph{unqualified}
in the importing component or API:
% import APIName.{ name (, name)^+ }
\begin{Fortress}
\( \KWD{import}\:\TYP{APIName}.\{\,\VAR{name} (, \VAR{name})^+\,\}\)
\end{Fortress}
We call an import statement of this form
an \emph{import-unqualified} statement.
We say an import statement of this form is an ``explicit import'' statement.
It is a static error if
a name specified by an import-unqualified
statement
is also declared by a top-level or functional method declaration
in the importing component or API,
or if it is also specified by another import-unqualified
statement,
unless all top-level or functional declarations for that name
in the importing component or any of APIs
imported by an import-unqualified
statement that specifies that name
form a valid set of overloaded declarations.
It is also a static error if any specified name
is not declared by a declaration imported from the API.


An ``ordinary'' operator (i.e., operators other than
 enclosing or vertical-line operators)
is a single token, and it is imported simply by putting \KWD{opr}
before it in the import statement.  This imports all of its prefix, postfix,
infix, multifix and nofix declarations.
A matched pair of enclosing operators is written \KWD{opr}
the left encloser, the right encloser, with \EXP{\cdot}
optionally between the two enclosers.
A vertical-line operator may be imported either as an infix operator or
a bracketing operator, according to the rules above for ordinary and
enclosing operators respectively.

An import-unqualified
statement may include an alias for each name it specifies.
%import APIName.{ name as name' (, name)^+ }
\begin{Fortress}
\(\KWD{import}\:\TYP{APIName}.\{\,\VAR{name} \KWD{as}\:\VAR{name}' (, \VAR{name})^+\,\}\)
\end{Fortress}
In this case, the alias (unqualified) must be used
instead of the name declared by the actual imported declaration,
and the restriction about declarations
in the importing component or API
and names specified in other import-unqualified
statements
applies to the alias rather than the specified name.

For convenience, an on-demand import statement
allows all names declared by imported declarations
of the specified \apiN\ to be referred to with unqualified names:
%import APIName.{...}
\begin{Fortress}
\(\KWD{import}\:\TYP{APIName}.\{\ldots\}\)
\end{Fortress}
This form permits an \KWD{except} clause,
which lists names that are \emph{not} permitted to be used unqualified
by that import-unqualified
statement
(it may be permitted by a declaration
in the importing component or API,
or by some other import-unqualified
statement).
%import APIName.{...} except { name (, name)^+ }
\begin{Fortress}
\(\KWD{import}\:\TYP{APIName}.\{\ldots\} \KWD{except} \{\,\VAR{name} (, \VAR{name})^+\,\}\)
\end{Fortress}
A name \VAR{name} is imported on demand in a component \VAR{C}
from an API \VAR{A} only if all the following conditions hold:
\begin{enumerate}
\item \VAR{C} contains an on-demand import statement.
\item \VAR{name} is neither listed in the \KWD{except} clause
nor explicitly imported by \VAR{C}.
\item Either 1) a top-level or functional method declaration of \VAR{name}
appears in \VAR{C} or 2) a reference to \VAR{name} appears in component \VAR{C}
and \VAR{C} does not provide an explicit declaration of \VAR{name}.
\end{enumerate}

The set of all declarations that are declared or imported
(either explicitly or on demand) by a component must satisfy the overloading
rules (and in particular, any nonfunctional declaration must not be overloaded).
In addition, it is a static error if in a set of overloaded declarations,
any of the following are true:
\begin{enumerate}
\item Any declaration imported on demand is more specific than any explicit declaration.
\item Any top-level function declaration is more specific than any functional method declaration.
(This rule actually applies to declarations in a single component as well.)
\item
Any top-level function declaration imported on demand is more specific than any other declaration not in the same API as the top-level function declaration.
\end{enumerate}
It is also a static error to explicitly import a name from an API
that does not declare it, or to have multiple import-on-demand statements
with the same API.

If there is no imported declaration matching a reference, it is a static error.
If there is more than one imported declaration that a reference may refer to,
it is a static error.
For example, it is a static error to have a reference \TYP{List}
in the context of the following import statements:
%import List.{...}
%import PureList.{...}
\begin{Fortress}
\(\KWD{import}\:\TYP{List}.\{\ldots\}\)\\
\(\KWD{import}\:\TYP{PureList}.\{\ldots\}\)
\end{Fortress}
A reference \TYP{List} may refer to
the type \TYP{List} declared in the API \TYP{List} or
the type \TYP{List} declared in the API \TYP{PureList}.

\subsection{Export Statements}
\note{
Victor's email on March 7, 2009 1:28:48 AM EST
titled "Re: Exposing a partial type hierarchy in an API"

Because the extends clause of a trait declaration in a component must be
the same as in an API exported by the component, the type hierarchy
above any trait can be determined simply by looking at the API declaring
that trait, and those APIs it imports. Fortress does not allow
components to contain private supertypes for traits in an exported API.
}

Export statements specify the APIs that a component exports.
One important restriction on components is that no \apiN\ may be both
imported and exported by the same component. This restriction helps to
avoid some (but not all) accidental cyclic dependencies.

Every component implicitly imports the Fortress core APIs;
every fortress has at least one component implementing all of these \apisN.
A \emph{preferred} component exporting these \apisN\ (configurable by the user)
is implicitly linked to
every component installed in the fortress.


A component must provide a declaration, or a set of declarations,
that \emph{satisfies} every top-level declaration in any API
that it exports, as described below.
A component may include declarations
that do not participate in satisfying any exported declaration
(i.e., a declaration of any exported API), and in particular,
no declaration modified by \KWD{private}
participates in satisfying any exported declaration.
Such declarations are not visible from outside the component.

Type aliases and dimension, unit, test and property declarations
are satisfied only by declarations that are the same.\footnote{The
declarations are compared after ASCII conversion and scanning,
and differences in the characters within whitespace elements
is permitted
provided they do not change whether the whitespace element
is line-breaking.
}
\note{Victor: what about trait/object/etc declarations with test modifier?}

A top-level variable declaration declaring a single variable
is satisfied by any top-level variable declaration
that declares the name with the same type
(in the component,
the type may be inferred).
A top-level variable declaration declaring multiple variables
is satisfied by a set of declarations (possibly just one)
that declare all the names with their respective types
(which again, may be inferred).
In either case,
the modifiers including the mutability of a variable must be the same in
the exported and satisfying declarations.
A top-level variable declaration
declaring a single immutable variable
with arrow type \EXP{P \rightarrow R}
is also satisfied
by a set of top-level function and functional method declarations
if the union of their parameter types is \VAR{P}
and each of their return types is a subtype of \VAR{R}.
Thus, an abstract function declaration
in the form of variable declarations in an API
need not be replicated in a component exporting that API.

A top-level function declaration is satisfied by
a set of top-level function and functional method declarations
the union of whose parameter types
is the parameter type of the exported declaration,
and each of whose return types
is a subtype of the return type of the exported declaration.
\note{Victor: also need type of thrown checked exceptions

Victor: it may also be satisfied by a declaration
 or set of declarations whose parameter type is a supertype
 of the exported declaration's parameter type
 provided that there is an overloaded exported declaration
 whose parameter type is a supertype of the satisfying declarations.
 (This comment also applied to the method declarations below.
 We need a term for this concept.)
}

A trait or object declaration is satisfied by
a declaration
that has the same header,\footnote{The order of
the modifiers,
the clauses,
and the types in the
\KWD{extends}, \KWD{excludes} and \KWD{comprises} clauses
may differ.}
and contains,
for each field declaration and non-abstract method declaration
in the exported declaration,
a satisfying declaration (or a set of declarations).
When a trait has an abstract method declared,
a satisfying trait declaration is allowed to provide a concrete declaration.
If a trait declaration (in an API)
has a \KWD{comprises} clause containing ``\EXP{\ldots}'',
then the \KWD{comprises} clause of its satisfying declaration
must contain all the types listed in the declaration in the API
but it cannot contain ``\EXP{\ldots}'';
instead, it may contain other types declared in the component
provided that those types are not declared by the API.
A field declaration is satisfied by
a (possibly implicit) getter and/or setter declaration,
depending on whether the field
is declared to be \KWD{hidden} and/or \KWD{settable}
by the exported declaration.

A method declaration is satisfied
by a set of method declarations
the union of whose parameter types
is the parameter type of the exported declaration,
and each of whose return types
is a subtype of the return type of the exported declaration.
\note{
Victor: need to be dotted iff exported decl is dotted.

Victor: also need type of thrown checked exceptions}
A satisfying trait or object declaration
may contain method and field declarations not exported by the API
but these might not be overloaded
with method or field declarations provided by
(contained in or inherited by)
any declarations exported by the API.
\note{
Victor: is this true even for getters and setters?
That is, can we not define both a getter and a setter
with the same name and export only one of them?
I think we should be allowed to do this.}

For functional declarations,
recall that several functional declarations
may define the same entity (i.e., they may be overloaded).
Given a set of overloaded declarations,
it is not permitted to export some of them and not others.


\note{
2. A trait with a comprises clause is considered to have a concrete method declaration with a particular signature even if its declaration is actually abstract provided that all its immediate subtraits are considered to have concrete method declaration with that signature.  Thus, such a trait can have a method declaration that is abstract in the component, but not so declared in the API.  (And we might also allow the analogous thing for top-level functions.)

3. A component may have a functional declaration that is not in the API provided that it is "covered" by some overloaded declaration.

4. A component exporting an API with an "abstract function declaration" need not have that declaration (or any matching one).  It does, however, need matching declarations for the concrete functional declarations in that API.
}

\subsection{Cross-Component Overloading}
When a component is compiled,
the \apisN\ it refers to must be present in the fortress.
The import statements in a component are not
a way to abbreviate unqualified names of objects or functions.
In our system,
an import statement merely allows references to the imported \apiN\
to appear in the component definition.
References to elements of an imported \apiN\ must be fully qualified
unless they are imported by an import statement with a \KWD{from} clause.
%
When a component imports a functional $f$ (either a top-level function or
a functional method) by an import statement,
with a \KWD{from} clause,
the imported $f$ may be overloaded with a functional $f$ declared by the
component.
When a component imports a top-level declaration $f$ from an API $A$,
all the relevant types to type check the uses of $f$ are implicitly
imported from $A$.  However, these implicitly imported types for type
checking are not expressible by programmers; programmers must import the
types by import statements to use them.  For example, the two
functional calls in the following component \VAR{C} are valid:
\note{This example is not tested nor run by the interpreter.}
% api A
%    trait T
%      m():()
%    end
% end
% api B
%    import A.{...}
%
%    f(): T
%    g(T): ()
% end
% component C
%    import B.{f, g}
%    export Executable
%
%    run(args) = do f().m(); g(f()) end
% end
\begin{Fortress}
{\tt~}\pushtabs\=\+\( \KWD{api}\:A\)\\
{\tt~~~}\pushtabs\=\+\(    \KWD{trait}\:T\)\\
{\tt~~}\pushtabs\=\+\(      m()\COLONOP()\)\-\\\poptabs
\(    \KWD{end}\)\-\\\poptabs
\( \KWD{end}\)\\
\( \KWD{api}\:B\)\\
{\tt~~~}\pushtabs\=\+\(    \KWD{import}\:A.\{\ldots\}\)\\[4pt]
\(    f()\COLON T\)\\
\(    g(T)\COLON ()\)\-\\\poptabs
\( \KWD{end}\)\\
\( \KWD{component}\:C\)\\
{\tt~~~}\pushtabs\=\+\(    \KWD{import}\:B.\{f, g\}\)\\
\(    \KWD{export}\:\TYP{Executable}\)\\[4pt]
\(    \VAR{run}(\VAR{args}) = \;\KWD{do}\:f().m(); g(f()) \KWD{end}\)\-\\\poptabs
\( \KWD{end}\)\-\\\poptabs
\end{Fortress}
because \EXP{T} is implicitly imported from \VAR{B} to type check
the functional calls.
However, the programmers cannot write \EXP{T} in \VAR{C} because \EXP{T}
is not imported by an import statement.


\subsection{Native Components}
A component may have the modifier \KWD{native} to declare an
``unsafe component''.  Within an unsafe component,
the syntax and semantics are implementation dependent.
However, an unsafe component can export an API,
which can be imported by safe components and APIs as usual.

%%%%%%%%%%%%%%%%%%%%%%%%%%%%%%%%%%%%%%%%%%%%%%%%%%%%%%%%%%%%%%%%%%%%%%%%%%%%%%%%
%   Copyright 2009 Sun Microsystems, Inc.,
%   4150 Network Circle, Santa Clara, California 95054, U.S.A.
%   All rights reserved.
%
%   U.S. Government Rights - Commercial software.
%   Government users are subject to the Sun Microsystems, Inc. standard
%   license agreement and applicable provisions of the FAR and its supplements.
%
%   Use is subject to license terms.
%
%   This distribution may include materials developed by third parties.
%
%   Sun, Sun Microsystems, the Sun logo and Java are trademarks or registered
%   trademarks of Sun Microsystems, Inc. in the U.S. and other countries.
%%%%%%%%%%%%%%%%%%%%%%%%%%%%%%%%%%%%%%%%%%%%%%%%%%%%%%%%%%%%%%%%%%%%%%%%%%%%%%%%

\section{\Apis}
\seclabel{apis}

\note{All APIs should provide all types explicitly in all declarations.}

\begin{Grammar}
\emph{Api} &::=&
\KWD{api} \emph{APIName} \option{\emph{Imports}} \option{\emph{AbsDecls}} \KWD{end}
\options{\option{\KWD{api}} \emph{APIName}}\\

\emph{AbsDecls} &::=& \emph{AbsDecl}$^+$\\

\emph{AbsDecl} &::=& \emph{AbsTraitDecl}\\
&$|$& \emph{AbsObjectDecl}\\
&$|$& \emph{AbsVarDecl}\\
&$|$& \emph{AbsFnDecl}\\
&$|$& \emph{DimUnitDecl}\\
&$|$& \emph{TypeAlias}\\
&$|$& \emph{TestDecl}\\
&$|$& \emph{PropertyDecl}\\
&$|$& \emph{AbsExternalSyntax}\\

\end{Grammar}

\begin{GrammarTwo}
\emph{AbsTraitDecl}  &::=& \option{\emph{AbsTraitMods}}
\emph{TraitHeaderFront} \emph{AbsTraitClauses}
\option{\emph{AbsGoInATrait}} \KWD{end} \\
&&\options{\option{\KWD{trait}} \emph{Id}}\\

\emph{AbsTraitMods} &::=& \emph{AbsTraitMod}$^+$\\

\emph{AbsTraitMod} &::=& \KWD{value} $|$ \KWD{test}\\

\emph{AbsTraitClauses} &::=& \emph{AbsTraitClause}$^*$\\

\emph{AbsTraitClause} &::=& \emph{Excludes} \\
&$|$& \emph{AbsComprises} \\
&$|$& \emph{Where} \\

\emph{AbsComprises} &::=& \KWD{comprises} \emph{ComprisingTypes} \\

\emph{ComprisingTypes}
&::=& \emph{TraitType} \\
&$|$& \texttt{\{} \emph{ComprisingTypeList} \texttt{\}} \\

\emph{ComprisingTypeList}
&::=& \EXP{\ldots} \\
&$|$& \emph{TraitType}(\EXP{,} \emph{TraitType})$^*$ \options{\EXP{,} \ldots}\\

\end{GrammarTwo}

\begin{GrammarTwo}
\emph{AbsGoInATrait}
&::=& \option{\emph{AbsCoercions}}
\emph{AbsGoFrontInATrait} \option{\emph{AbsGoBackInATrait}}\\
&$|$& \option{\emph{AbsCoercions}}
\emph{AbsGoBackInATrait} \\
&$|$& \emph{AbsCoercions}\\

\emph{AbsCoercions} &::=& \emph{AbsCoercion}$^+$\\

\emph{AbsGoFrontInATrait} &::=& \emph{AbsGoesFrontInATrait}$^+$\\

\emph{AbsGoesFrontInATrait}
&::=& \emph{ApiFldDecl} \\
&$|$& \emph{AbsGetterSetterDecl} \\
&$|$& \emph{PropertyDecl} \\

\emph{AbsGoBackInATrait} &::=& \emph{AbsGoesBackInATrait}$^+$\\

\emph{AbsGoesBackInATrait}
&::=& \emph{AbsMdDecl} \\
&$|$& \emph{PropertyDecl} \\

\emph{AbsObjectDecl} &::=& \option{\emph{AbsObjectMods}}
 \emph{ObjectHeader} \option{\emph{AbsGoInAnObject}} \KWD{end}
\options{\option{\KWD{object}} \emph{Id}}\\

\emph{AbsObjectMods} &::=& \emph{AbsTraitMods}\\

\emph{AbsGoInAnObject}
&::=& \option{\emph{AbsCoercions}}
\emph{AbsGoFrontInAnObject} \option{\emph{AbsGoBackInAnObject}}\\
&$|$& \option{\emph{AbsCoercions}}
\emph{AbsGoBackInAnObject} \\
&$|$& \emph{AbsCoercions}\\

\emph{AbsGoFrontInAnObject} &::=& \emph{AbsGoesFrontInAnObject}$^+$\\

\emph{AbsGoesFrontInAnObject}
&::=& \emph{ApiFldDecl} \\
&$|$& \emph{AbsGetterSetterDecl} \\
&$|$& \emph{PropertyDecl} \\

\emph{AbsGoBackInAnObject} &::=& \emph{AbsGoesBackInAnObject}$^+$\\

\emph{AbsGoesBackInAnObject}
&::=& \emph{AbsMdDecl} \\
&$|$& \emph{PropertyDecl} \\

\emph{AbsCoercion} &::=&
\KWD{coerce}\option{\emph{StaticParams}}\texttt(\emph{BindId} \emph{IsType}\texttt)\emph{CoercionClauses} \option{\KWD{widens}}\\

\emph{ApiFldDecl} &::=& \option{\emph{ApiFldMods}}
\emph{BindId} \emph{IsType}\\

\emph{ApiFldMods} &::=& \emph{ApiFldMod}$^+$\\

\emph{ApiFldMod} &::=& \KWD{hidden} $|$ \KWD{settable} $|$ \KWD{test}\\

\emph{AbsVarDecl} &::=& \option{\emph{AbsVarMods}} \emph{VarWTypes}\\
&$|$& \option{\emph{AbsVarMods}} \emph{BindIdOrBindIdTuple} \EXP{\mathrel{\mathtt{:}}} \emph{Type}\EXP{...}\\
&$|$& \option{\emph{AbsVarMods}} \emph{BindIdOrBindIdTuple} \EXP{\mathrel{\mathtt{:}}} \emph{TupleType}\\

\emph{AbsVarMods} &::=& \emph{AbsVarMod}$^+$\\

\emph{AbsVarMod} &::=& \KWD{var} $|$ \KWD{test}\\

\end{GrammarTwo}


\Apis\ are compiled from special \apiN\ definitions.  These are source
files which declare the entities declared by the \apiN, the names of all
\apisN\ referred to by those declarations, and prose documentation.  In
short, the source code of an \apiN\ must specify all the information
that is traditionally provided for the published \apisN\ of libraries in
other languages.

The syntax of an \apiN\ definition is
identical to the syntax of a component definition, except that:
\begin{enumerate}

\item
An \apiN\ definition begins with \KWD{api}
rather than \KWD{component}.
As with components,
the identifiers associated with \apisN\ that are not included in \library\
are prefixed with the reverse of the URL of the development team.

\item
An \apiN\ does not include \KWD{export} statements.
(However, it does include \KWD{import} statements,
which name the other \apisN\ used in the \apiN\ definition.)

\item
Only declarations (not definitions!) are included in an \apiN\
definition except test and property declarations.
A method or field declaration may include the modifier \KWD{abstract}.
(Whether a declaration includes the modifier \KWD{abstract}
has a significant effect on its meaning, as discussed below).

\end{enumerate}


Import statements in APIs permit names declared in imported APIs to be used
in the importing API either qualified or unqualified,
just as in components.  Those names are \emph{not}, however,
part of the importing API,
and thus cannot be imported from that API by a component or another API.

For the sake of simplicity, every identifier reference in an \apiN\
definition must refer either to a declaration in a used \apiN\
(i.e., an \apiN\ named in an import statement,
or the Fortress core APIs, which are implicitly imported),
or to a declaration in the \apiN\ itself.
In this way, \apisN\ differ from signatures in most module systems:
they are not parametric in their external dependencies.

\section{Component and API Identity}

Formally, we introduce two functions on components,
$\imports$ and $\exports$,
that return the imported and exported \apisN\ of the component, respectively.
For any component $\comp$, %% \in \components$,
$\imports(\comp) \intersect \exports(\comp) = \emptyset$.
This restriction is required throughout to ground the semantics of
operations on components, as discussed in \secref{basicops}.

Every component
has a unique name, used for the purposes of component linking.
This name includes a user-provided identifier.
In the case of a simple component, the identifier is determined by
a component name given at
the top of the source file from which it is compiled.
A build script may keep a tally on version numbers and append them to the
first line of a component,
incrementing its tally on each compilation.
The name of a compound component is specified as an argument
to the \shellcommand{link} operation (described in \secref{basicops})
that defines it.
Component equivalence is determined nominally
to allow mutually recursive linking of components.
By programmer convention, identifiers associated with components
that are not included in \library\
begin with the reverse of the URL of the development team.
%Two components with the same name must be identical, even if they are
%contained in different fortresses.
A fortress does not allow the installation of
distinct components with the same name.
Component names are used during
\shellcommand{link} and \shellcommand{upgrade}
operations to ensure that the restrictions on
upgrades to a component are respected,
as explained in \secref{basicops}.

Every component also includes
a vendor name,
the name of the fortress it is compiled on, and
a timestamp, denoting the time of compilation.
The time of compilation is measured by the compiling fortress,
and the name of the fortress is provided by the fortress automatically.
Every timestamp issued by a fortress must be unique.
The vendor name typically remains the same throughout a significant
portion of the life of a
user account, and is best provided as a user environment variable.

Every \apiN\ has a unique name that consists of a user-provided identifier.
As with components,
\apiN\ equivalence is determined nominally.
Every \apiN\ also includes a vendor name,
the name of the fortress it is compiled on,
and a timestamp.
\note{
Victor: Is it the \apiN\ name that includes the vendor, etc.?
It used to just say \apiN, and I changed it to \apiN\ name.

Jan: I changed it back, because we don't claim to distinguish APIs
  this way in the remainder of the text, and don't give a
  mechanism for specifying these details in an import.}
Component names must not conflict with \apiN\ names.
For convenience, a compiler can also
produce an \apiN\ directly from a project
with the same name as the component it is derived from.
Such an \apiN\ includes \emph{matching} declarations of the component.
All declarations in the component appear in the API.


A component must include, for every \apiN\ $A$ it exports,
matching definitions for all the declarations in $A$.
A matching definition of a declaration $d$
is a definition $d'$ with the same name as $d$
that includes definitions for all declarations
other than the methods or fields declared \KWD{abstract} in $d$.
The header and type of $d'$ must be the same as the header and type of $d$.
$d'$ may include additional definitions not declared in $d$.

Other than its identity,
the only relevant characteristic of an \apiN\ $\api$
is the set of \apisN\ that it uses,
denoted by $\uses(\api)$.
Because an \apiN\ $\api$ might expose types
defined in $\uses(\api)$,
we require that a component that exports $\api$
also exports all \apisN\ in $\uses(\api)$ that it does not import.
Formally, the following condition holds on the exported \apisN\
of a component $\comp$:
\[
\api \in \exports(\comp) \logicand \api' \in \uses(\api)
        \implies \api' \in \imports(\comp) \union \exports(\comp)
\]

