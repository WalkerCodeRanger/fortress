%%%%%%%%%%%%%%%%%%%%%%%%%%%%%%%%%%%%%%%%%%%%%%%%%%%%%%%%%%%%%%%%%%%%%%%%%%%%%%%%
%   Copyright 2009, Oracle and/or its affiliates.
%   All rights reserved.
%
%
%   Use is subject to license terms.
%
%   This distribution may include materials developed by third parties.
%
%%%%%%%%%%%%%%%%%%%%%%%%%%%%%%%%%%%%%%%%%%%%%%%%%%%%%%%%%%%%%%%%%%%%%%%%%%%%%%%%

\chapter{Static Parameters}
\chaplabel{trait-parameters}

\note{Non-type static parameters and static expressions are not yet supported.

The examples in this chapter are not tested nor run by the interpreter.}

Trait, object, and functional declarations may be
parameterized with \emph{static parameters}.
Static parameters are \emph{static variables} listed in white
square brackets \EXP{\llbracket} and \EXP{\rrbracket}
immediately after the name of a trait, object, or functional and
they are in scope of the entire body of the declaration.
Static parameters may be instantiated with static expressions discussed in
\secref{static-expr}.
In this chapter, we describe the forms that
these static parameters can take.

\begin{Grammar}
\emph{StaticParams} &::=&
\bTPl \emph{StaticParamList}\bTPr\\
\emph{StaticParamList} &::=& \emph{StaticParam}(\EXP{,} \emph{StaticParam})$^*$\\
\end{Grammar}

\section{Type Parameters}
\seclabel{type-param}

\begin{Grammar}
\emph{StaticParam} &::=&
\emph{Id} \option{\emph{Extends}} \options{\KWD{absorbs} \KWD{unit}} \\
\end{Grammar}

Static parameters may include one or more type parameters.
Syntactically, a type parameter consists of an identifier followed by an
optional \KWD{extends} clause, followed by an optional
``\EXP{\KWD{absorbs}\;\;\KWD{unit}}'' clause (described in
\secref{absorbing-units}).
If a type parameter does not have an \KWD{extends} clause, it has an
implicit ``\EXP{\KWD{extends} \TYP{Object}}'' clause.


Type parameters are instantiated with types such as trait types, tuple types,
and arrow types (See \chapref{types} for a discussion of Fortress
types).  We use the term \emph{naked type variable} to refer to an
occurrence of a type variable as a stand-alone type (rather than as
a parameter to another type).  Type parameters can appear in any context
that an ordinary type can appear, except that a naked type variable must not
appear in the \KWD{extends} clause of a trait or object declaration,
nor in the \KWD{throws} clause of a functional or object declaration,
nor as the type of a \KWD{wrapped} field (discussed in \secref{fields}).


Here is a parameterized trait \TYP{List}:
\input{\home/basic/examples/StatParam.Type.tex}


\section{Nat and Int Parameters}
\seclabel{natparams}

\begin{Grammar}
\emph{StaticParam} &::=& \KWD{nat} \emph{Id} \\
&$|$& \KWD{int} \emph{Id} \\
\end{Grammar}

Static parameters may include one or more
\KWD{nat} and \KWD{int} parameters.
Syntactically, a \KWD{nat} parameter consists of
\KWD{nat} followed by an identifier.
An \KWD{int} parameter consists of
\KWD{int} followed by an identifier.
These parameters are instantiated at runtime with numeric values.
A \KWD{nat} parameter may be used to instantiate other
\KWD{nat} parameters, or to appear in any context that a
variable of type \EXP{\mathbb{N}32} can appear,
except that it cannot be assigned to.
An \KWD{int} parameter may be used to instantiate other
\KWD{int} parameters, or to appear in any context that a
variable of type \EXP{\mathbb{Z}32} can appear,
except that it cannot be assigned to.



For example, the following function \VAR{makeVector}:
\input{\home/basic/examples/StatParam.Nat.tex}
declares a \KWD{nat} parameter \EXP{s_{0}}, which appears in both the parameter
type and return type of \VAR{makeVector}.



\section{Bool Parameters}
\seclabel{boolparams}

\begin{Grammar}
\emph{StaticParam} &::=& \KWD{bool} \emph{Id} \\
\end{Grammar}

Static parameters may include one or more \KWD{bool} parameters.
Syntactically, a \KWD{bool} parameter consists of
\KWD{bool} followed by an identifier.
These parameters are instantiated at runtime with boolean values. They may
be used to instantiate other \KWD{bool} parameters, or
to appear in any context that a variable of type \TYP{Boolean} can appear,
except that they cannot be assigned to.


For example, the following \KWD{coerce} declared in the trait \TYP{Boolean}:
\input{\home/basic/examples/StatParam.Bool.tex}
declares a \KWD{bool} parameter \VAR{b}, which appears in the parameter type.


\section{Dimension and Unit Parameters}
\seclabel{dimunitparams}

\begin{Grammar}
\emph{StaticParam} &::=& \KWD{dim} \emph{Id} \\
&$|$& \KWD{unit} \emph{Id} \options{\EXP{\mathrel{\mathtt{:}}} \emph{Type}}
 \options{\KWD{absorbs} \KWD{unit}}\\
\end{Grammar}


Static parameters may include one or more \KWD{dim} and \KWD{unit}
parameters.  Syntactically, a \KWD{dim} parameter begins with
\KWD{dim} followed by an identifier.
A \KWD{unit} parameter begins with \KWD{unit}
followed by an identifier, optionally followed by the token `\EXP{\COLONOP}'
and a dimension, and the unit is thereby restricted to be a unit of
the specified dimension.
A \KWD{unit} parameter may include the clause
``\EXP{\KWD{absorbs}\;\KWD{unit}}''; the meaning of this is described
in \secref{absorbing-units}.
A \KWD{dim} parameter is allowed to appear in any context that a
dimension can appear.
A \KWD{unit} parameter is allowed to appear in any context that a unit
can appear.


For example, here is a function (in this case actually a prefix operator definition) that is parameterized with a unit:
\note{I manually added a space before U -- Sukyoung}
%opr SQRT[\unit U\](x: RR64 U^2): RR64 U = numericalsqrt(x/U^2) U
\begin{Fortress}
\(\KWD{opr} \mathord{\surd}\llbracket\KWD{unit}\:U\rrbracket(x\COLON \mathbb{R}64\ U^{2})\COLON \mathbb{R}64\ U = \VAR{numericalsqrt}(x/U^{2})\:U\)
\end{Fortress}


\section{Operator Parameters}
\seclabel{opr-ident}
\begin{Grammar}
\emph{StaticParam} &::=& \KWD{opr} \emph{Op} \\
\end{Grammar}

Static parameters may include one or more operator names.
Syntactically, an operator parameter begins with
\KWD{opr} followed by an operator name.


Operator parameters may be freely intermixed with other
static parameters.  For example, the following trait \TYP{IdentityOp}:
\input{\home/basic/examples/StatParam.Opr.IdentityOp.tex}
is parameterized with a type parameter \VAR{T} and
an operator parameter \EXP{\odot}.
Unlike other static parameters,
operator parameters may be used in both type context and
value context.  The operator parameter \EXP{\odot}
is declared as the second static parameter of \TYP{IdentityOp},
instantiated as a static argument in
%IdentityOp[\T,ODOT\]
\EXP{\TYP{IdentityOp}\llbracket{}T,\odot\rrbracket}
which is the bound of \VAR{T},
and declared as an operator method in \TYP{IdentityOp}.
If a trait or object has an operator parameter \OPR{OP} and uses \OPR{OP}
in an expression, then the trait or object must declare or inherit
at least one operator method for \OPR{OP} of matching fixity.
(That is because it would be impossible to type-check \VAR{T}
with any global definition for \OPR{OP},
because the actual operator name is not known
yet, and a definition has to come from somewhere.)


Operator parameters are instantiated with operators.
They may be used to instantiate other operator parameters
and the names of operator declarations.
For example, the following trait \TYP{MyIdentity}:
\input{\home/basic/examples/StatParam.Opr.MyIdentity.tex}
instantiates the operator parameter of its supertrait \TYP{IdentityOp}
with the operator name \OPR{IDENTITY}.  It inherits the \OPR{IDENTITY}
operator from \TYP{IdentityOp}.
Two restrictions apply to instantiations of operator parameters:
1) If a trait or object \VAR{T} has operator parameters \EXP{\OPR{OP}_{1}} and
\EXP{\OPR{OP}_{2}}, and \VAR{T} declares or inherits operator methods of the same
fixity for both \EXP{\OPR{OP}_{1}} and \EXP{\OPR{OP}_{2}},
then the actual operators passed in any instantiation of \VAR{T} must
be different.
(That's because it would be impossible to type check any code
in trait \VAR{T} that uses either \EXP{\OPR{OP}_{1}} or
\EXP{\OPR{OP}_{2}} in an expression if such
aliasing were not forbidden.)
Note that the clause ``of the same fixity'' allows both \EXP{\OPR{OP}_{1}} and
\EXP{\OPR{OP}_{2}} to be given the same actual operator, say ``\EXP{-}'',
if \EXP{\OPR{OP}_{1}} has only a prefix definition and \EXP{\OPR{OP}_{2}} has
only an infix definition; this is desirable.
2) If a trait or object \VAR{T} has an operator parameter \OPR{OP} and
declares or inherits an operator method for \OPR{OP},
and also declares or inherits an operator method of the same fixity for
any actual operator \EXP{@} (that is, where \EXP{@} is not an operator
parameter), then the actual operator passed for the \OPR{OP} parameter
must not be \EXP{@}.
(That's because it would be impossible to type check any code
in trait \VAR{T} that uses the operator \EXP{@} in an expression if such
aliasing were not forbidden.)


Note that any declarations that may be associated with an actual operator name
that is passed for an operator parameter are \emph{irrelevant} to the behavior
of the operator parameter within the parameterized trait, object, or functional.
When a trait or object extends a trait parameterized by operator names,
the subtrait inherits methods whose names are the actual operator names instead
of the operator parameter names.


An operator method declaration whose name is one of the operator parameters
of its enclosing trait or object may be overloaded with other operator
declarations in the same component; the operator parameter may be instantiated
with any operator in the same component.  Therefore, such an operator method
declaration must satisfy the overloading rules
(described in \chapref{multiple-dispatch}) with every operator declaration
in the same component.
For example, the above \OPR{IDENTITY} operator declaration is overloaded
with the following \OPR{IDENTITY} operator declaration:
%opr IDENTITY(x: ZZ) = x
%opr +(x: ZZ) = x
\begin{Fortress}
\(\KWD{opr} \mathord{\OPR{IDENTITY}}(x\COLON \mathbb{Z}) = x\)\\
\(\KWD{opr} +(x\COLON \mathbb{Z}) = x\)
\end{Fortress}
where both \OPR{IDENTITY} and \EXP{+} are top-level operator declarations
in the same component.
Therefore, the declaration of the operator \EXP{\odot} in \TYP{IdentityOp}
must satisfy the overloading rules with \OPR{IDENTITY} and with \EXP{+}.


Here are the things programmers can do with an operator parameter:
\begin{enumerate}
\item[(1)] Use it as the defined name in an operator declaration.
    This is what trait
%BinaryOperator[\T,ODOT\]
\EXP{\TYP{BinaryOperator}\llbracket{}T,\odot\rrbracket}
(described in \secref{opr-properties}) does.

\item[(2)] Pass it as an actual operator to another trait that takes
     an operator parameter.  This is what
\EXP{\TYP{Associative}\llbracket{}T,\odot\rrbracket}
(described in \secref{opr-properties}) does when it extends
\EXP{\TYP{BinaryOperator}\llbracket{}T,\odot\rrbracket}.

\item[(3)] Use it to distinguish various objects.
    This is what
\EXP{\TYP{Identity}\llbracket\odot\rrbracket}
(described in \secref{opr-properties}) does.

\item[(4)] Use it as an operator in an expression.
    To do this, an operator declaration for that operator parameter
    must be accessible at the ponit of use, and for that to
    happen, the parameterized trait must either declare
    an operator method with that operator parameter as the name,
    as in case (1), or inherit such a declaration, and the
    only way to do the latter is to use case (2).
\end{enumerate}

Many interesting examples are described in \secref{opr-properties}.



\section{Where Clauses}
\seclabel{where-clauses}

\note{The \KWD{where} clause syntax in this section is out of date.}

\begin{Grammar}
\emph{Where} &::=& \KWD{where} \{ \emph{WhereClauseList} \}\\
\emph{WhereClauseList} &::=& \emph{WhereClause}(\EXP{,}
\emph{WhereClause})$^*$ \\
\emph{WhereClause} &::=& \emph{Id} \emph{Extends}\\
&$|$& \emph{TypeAlias} \\
&$|$& \emph{NatConstraint} \\
&$|$& \emph{IntConstraint} \\
&$|$& \emph{BoolConstraint} \\
&$|$& \emph{UnitConstraint} \\
&$|$& \emph{TypeRef} \KWD{coerces} \emph{TypeRef}\\
&$|$& \emph{TypeRef} \KWD{widens} \emph{TypeRef}\\
\end{Grammar}


Static parameters may have constraints placed on them
in a \KWD{where} clause.
A \KWD{where} clause begins with
\KWD{where}, followed by a sequence of static
parameter constraints enclosed in braces \EXP{\{} and \EXP{\}},
and separated by commas.


A \KWD{where} clause may introduce new static variables, i.e., identifiers
for types and other static entities that may not be static parameters.
We use the term \emph{\KWD{where}-clause variables} to refer to
static variables that are not also static parameters.
The \KWD{where}-clause variables must be bound in a \KWD{where} clause.


A static parameter constraint is one of the following forms:
\begin{itemize}
\item a trait constraint consisting of the identifier of
a naked type variable, followed by \KWD{extends}
followed by a set of trait references which may include naked type variables,
\item a type alias (described in \secref{type-alias}),
\item an arithmetic constraint,
\item a boolean constraint,
\item a unit equality constraint,
\item a \KWD{coerces} constraint (described in
  \secref{coercion-declarations}), or
\item a \KWD{widens} constraint (described in
  \secref{coercion-declarations}).
\end{itemize}
A \KWD{where} clause may include mutually recursive constraints.
All static variables in a trait, object, or functional declaration must
occur either as a static parameter or as a \KWD{where}-clause variable.
\appref{where-calculus} describes a Fortress core calculus with
\KWD{where} clauses.

Trait declarations are allowed to extend other instantiations of
themselves. For example, we can write:
%trait C[\S\] extends C[\T\]
%  where {S extends T, T extends Object}
%end
\begin{Fortress}
\(\KWD{trait} C\llbracket{}S\rrbracket \KWD{extends} C\llbracket{}T\rrbracket\)\\
{\tt~~}\pushtabs\=\+\(  \KWD{where} \{S \KWD{extends} T, T \KWD{extends} \TYP{Object}\}\)\-\\\poptabs
\(\KWD{end}\)
\end{Fortress}
In this declaration, for every subtype \VAR{S} of \VAR{T}, \EXP{C\llbracket{}S\rrbracket}
is a subtype of \EXP{C\llbracket{}T\rrbracket}. Effectively, we have expressed
  the fact that the static parameter \VAR{S} of \VAR{C} is covariant.


Trait declarations need not have any static parameters in order
to have a \KWD{where} clause. For example, the following trait
declaration is legal:
%trait C extends D[\T\]
%  where {T extends Object}
%end
\begin{Fortress}
\(\KWD{trait} C \KWD{extends} D\llbracket{}T\rrbracket\)\\
{\tt~~}\pushtabs\=\+\(  \KWD{where} \{T \KWD{extends} \TYP{Object}\}\)\-\\\poptabs
\(\KWD{end}\)
\end{Fortress}
In this declaration, trait \VAR{C} is a subtrait of \emph{every}
instantiation of parametric trait \VAR{D}. Thus, trait \VAR{C} has all
of the methods of every instantiation of \VAR{D}. By thinking of the
declaration this way, we can see what restrictions we need to impose on
the trait \VAR{C} in order for it to be sensible. If trait
\VAR{C} inherits a method declaration that refers to \VAR{T}, it really
contains infinitely many methods (one for each instantiation of \VAR{T}).
However, instantiations of the \KWD{where}-clause variables are not
explicit from the program text as static parameters are.
It must be possible to infer which method is referred to at the call site.
%%So every method invocation is annotated with static types by type inference.
If there is not enough information to infer which method is called,
type checking rejects the
program and requires more type information from the programmer.
Programmers always can provide more type information by using type
ascription as described in \secref{type-ascription}.


Object or functional declarations may include \KWD{where} clauses.
Here is an example declaration of an \TYP{Empty} list:
%object Empty extends List[\T\] where {T extends Object}
%  first() = throw Error
%  rest() = throw Error
%  cons(x) = Cons(x,self)
%  append(xs) = xs
%end
\begin{Fortress}
\(\KWD{object} \TYP{Empty} \KWD{extends} \TYP{List}\llbracket{}T\rrbracket \KWD{where} \{T \KWD{extends} \TYP{Object}\}\)\\
{\tt~~}\pushtabs\=\+\(  \VAR{first}() = \;\KWD{throw} \TYP{Error}\)\\
\(  \VAR{rest}() = \;\KWD{throw} \TYP{Error}\)\\
\(  \VAR{cons}(x) = \TYP{Cons}(x,\KWD{self})\)\\
\(  \VAR{append}(\VAR{xs}) = \VAR{xs}\)\-\\\poptabs
\(\KWD{end}\)
\end{Fortress}
where \TYP{Cons} is declared in \secref{object-decls}.
