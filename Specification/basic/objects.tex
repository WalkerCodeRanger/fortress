%%%%%%%%%%%%%%%%%%%%%%%%%%%%%%%%%%%%%%%%%%%%%%%%%%%%%%%%%%%%%%%%%%%%%%%%%%%%%%%%
%   Copyright 2009, Oracle and/or its affiliates.
%   All rights reserved.
%
%
%   Use is subject to license terms.
%
%   This distribution may include materials developed by third parties.
%
%%%%%%%%%%%%%%%%%%%%%%%%%%%%%%%%%%%%%%%%%%%%%%%%%%%%%%%%%%%%%%%%%%%%%%%%%%%%%%%%

\chapter{Objects}
\chaplabel{objects}

\note{Value objects and properties are not yet supported.}

\emph{Objects} are values that have \objecttypes\ described in
\secref{object-traits}.
Objects contain \emph{methods} (described in \secref{methods}) and
\emph{fields} (described in \secref{fields}).

An object is a \emph{value object}, a \emph{reference object},
or a \emph{function object}:
It is a function object if it has an arrow type,
a reference object if it has an object trait type
that is not declared with the \KWD{value} modifier
(see \secref{value-objects}),
and a value object otherwise
(i.e., if it has a tuple type, the type (),
or an object trait type declared
with the \KWD{value} modifier).

A value object cannot have mutable fields,
and it is completely determined by its type,
environment and its fields:
Value objects with the same type, environment and fields
are indistinguishable.
Thus, an implementation may freely copy value objects.
Most objects with simple standard types,
such as booleans,
numeric literals,
IEEE floating-point numbers,
and integers are value objects.
%
In contrast,
reference objects are thought to ``reside in memory'',
and are identified by an \emph{object reference}.
A new object reference is created
whenever a reference object is constructed,
so that reference objects constructed separately are always distinct.
What about reference objects that have no mutable fields?
Reference objects include
arbitrary-precision numbers
and aggregates such as arrays, lists and sets.
%
Function objects are immutable and have no fields.  Identity is not
well-defined for function objects, and attempting to check whether two
functions are equivalent returns an approximate result.
\secref{equivalence} describes object equivalence in further detail.



\section{Object Declarations}
\seclabel{object-decls}

\note{The \KWD{transient} modifier and varargs parameters are eliminated.}

\note{What does it mean to have an invariant on an object?

Christine: The easy answer is that on entrance and exit of every method the invariant is true. Unfortunately I could be in the middle of a method with the invariant violated and want to call another method.

Guy: One might argue that invariants should hold on entry and exit of methods that appear in the API. This can be approximated as: invariants must hold before and after every method that is not marked private.

Need a more concrete proposal.

Does the modifier \KWD{atomic} on a functional check its contract also?
}

\begin{Grammar}
\emph{ObjectDecl}  &::=& \option{\emph{ObjectMods}}
\emph{ObjectHeader} \option{\emph{GoInAnObject}} \KWD{end}
\options{\option{\KWD{object}} \emph{Id}} \\

\emph{ObjectHeader} &::=& \KWD{object} \emph{Id}
\option{\emph{StaticParams}} \option{\emph{ObjectValParam}}
\option{\emph{ExtendsWhere}} \emph{FnClauses} \\

\emph{ObjectMods} &::=& \emph{TraitMods}\\

\emph{ObjectValParam} &::=& \texttt( \option{\emph{ObjectParams}} \texttt) \\

\emph{ObjectParams}
&::=&
(\emph{ObjectParam} \EXP{,})$^*$ \options{\emph{ObjectVarargs} \EXP{,}}
      \emph{ObjectKeyword}(\EXP{,}\emph{ObjectKeyword})$^*$\\
&$|$&
(\emph{ObjectParam} \EXP{,})$^*$ \emph{ObjectVarargs}\\
&$|$& \emph{ObjectParam} (\EXP{,} \emph{ObjectParam})$^*$\\

\emph{ObjectVarargs} &::=& \KWD{transient} \emph{Varargs}\\

\emph{ObjectKeyword} &::=& \emph{ObjectParam}\EXP{=}\emph{Expr}\\

\emph{ObjectParam} &::=&
 \option{\emph{ParamFldMods}}
 \emph{Param} \\
&$|$& \KWD{var} \KWD{transient} \emph{ParamWType} \\
&$|$& \option{\KWD{transient}} \KWD{var} \emph{ParamWType} \\
&$|$& \KWD{transient} \emph{Param} \\

\emph{ParamWType} &::=& \emph{BindId} \option{\emph{IsType}} \\

\emph{ParamFldMods} &::=& \emph{ParamFldMod}$^+$\\

\emph{ParamFldMod} &::=& $|$ \KWD{hidden} $|$ \KWD{settable} $|$ \KWD{wrapped}\\

\emph{GoInAnObject}
&::=& \option{\emph{Coercions}}
\emph{GoFrontInAnObject} \option{\emph{GoBackInAnObject}}\\
&$|$& \option{\emph{Coercions}}
\emph{GoBackInAnObject} \\
&$|$& \emph{Coercions}\\

\emph{Coercions} &::=& \emph{Coercion}$^+$\\

\emph{GoFrontInAnObject} &::=& \emph{GoesFrontInAnObject}$^+$\\

\emph{GoesFrontInAnObject}
&::=& \emph{FldDecl} \\
&$|$& \emph{GetterSetterDef} \\
&$|$& \emph{PropertyDecl} \\

\emph{GoBackInAnObject} &::=& \emph{GoesBackInAnObject}$^+$\\

\emph{GoesBackInAnObject}
&::=& \emph{MdDef} \\
&$|$& \emph{PropertyDecl} \\

\emph{FnClauses} &::=& \option{\emph{Throws}} \option{\emph{Where}}
\emph{Contract} \\

\emph{Throws} &::=& \KWD{throws} \emph{MayTraitTypes}\\

\end{Grammar}

Object declarations declare both object values and \objecttypes.
Object declarations extend a
set of traits from which they inherit methods.
An object declaration inherits the
concrete methods of its supertraits and must include a definition for every
method declared but not defined by its supertraits.
Especially, an object declaration must not include abstract methods
(discussed in \secref{methods});
it must define all abstract methods inherited from its supertraits.
It is also allowed to define overloaded declarations of concrete methods
inherited from its supertraits.



Syntactically, an object declaration begins with
an optional sequence of modifiers followed by
\KWD{object}, followed by
the identifier of the object, optional static parameters
(described in \chapref{trait-parameters}), optional
value parameters, optional traits the object extends,
an optional declaration of thrown checked exceptions
(discussed in \chapref{exceptions}),
an optional \KWD{where} clause
(discussed in \secref{where-clauses}),
a contract for the object
(discussed in \secref{contracts}),
zero or more declarations of fields,
getter methods, and setter methods,
and zero or more declarations of methods, that are not getter or setter,
separated by newlines or semicolons,
and finally \KWD{end}.
An object declaration may end with an optional \KWD{object} and its name again.
It is a static error
if the identifier after \KWD{end} is not the name of the object being declared.
Property declarations (described in \secref{property-decls}) can be freely
commingled with the field and method declarations.
Method declarations in object declarations are syntactically identical to
method declarations in trait declarations.
If an object declaration has no \KWD{extends} clause,
the object implicitly extends only trait \TYP{Object}.
A \KWD{throws} clause does not include naked type variables.
Every element in a \KWD{throws} clause is a subtype of
\TYP{CheckedException}.
A contract of an object declaration is evaluated
when the object is created, as with a function contract (see \secref{contracts}).

There are two kinds of object declarations:
singleton declarations and constructor declarations.
A singleton declaration declares a sole, stand-alone, singleton object.
It may have static parameters but it does not have a list of value parameters;
every instantiation of such an object with the same static
arguments yields the same singleton object.
A constructor declaration declares an object constructor function.
It may have static parameters and it includes a list of value parameters;
every call to the constructor of such an object with the same argument
yields a new object.
Initialization of singleton objects is nondeterministic
as described in \chapref{execution}.
The type of an object constructor function is an arrow type
consisting of the object's value parameter type followed by
the token \EXP{\rightarrow}, followed by the object trait type.

A constructor declaration has a (possibly empty) parenthesized list of
value parameters.  Each value parameter of a constructor declaration
may be preceded by the special modifier \KWD{transient} or by a
sequence of field modifiers.  A value parameter preceded by the
modifier \KWD{transient} does not correspond to a field in an
instantiation of the object: \KWD{transient} parameters are not in
scope in the object's method declarations, but are in scope when
the initial-value expressions of the fields
are evaluated during object creation.

For example, the following leaf object extending trait \TYP{Tree}:
\input{\home/basic/examples/Object.Decl.Leaf.tex}
has no fields and two methods.



Here is an example of a \TYP{Cons} object constructor extending
trait \EXP{\TYP{List}\llbracket{}T\rrbracket}:
\input{\home/basic/examples/Object.Decl.Cons.tex}
Note that this declaration introduces the ``factory'' function
%Cons[\T\](first:T,rest:List[\T\]):Cons[\T\]
\EXP{\TYP{Cons}\llbracket{}T\rrbracket(\VAR{first}\COLONOP{}T,\VAR{rest}\COLONOP\TYP{List}\llbracket{}T\rrbracket)\COLONOP\TYP{Cons}\llbracket{}T\rrbracket}
which is used in the body of the object declaration to define the \VAR{cons}
and \VAR{append} methods.
Multiple factory functions can be defined by overloading an
object constructor with functions.  For example:
\input{\home/basic/examples/Object.Decl.ConsFn.tex}



\section{Field Declarations}
\seclabel{fields}

\begin{Grammar}
\emph{FldDecl}
&::=& \option{\emph{FldMods}} \emph{FldWTypes} \emph{InitVal} \\
&$|$& \emph{BindIdOrBindIdTuple} \EXP{=} \emph{Expr}\\
&$|$& \option{\emph{FldMods}} \emph{BindIdOrBindIdTuple} \EXP{\mathrel{\mathtt{:}}} \emph{Type}\EXP{...}
\emph{InitVal} \\
&$|$& \option{\emph{FldMods}} \emph{BindIdOrBindIdTuple} \EXP{\mathrel{\mathtt{:}}} \emph{TupleType}
\emph{InitVal} \\

\emph{FldMods} &::=& \emph{FldMod}$^+$\\

\emph{FldMod} &::=& \KWD{var} $|$ \emph{AbsFldMod}\\

\emph{AbsFldMod} &::=& \emph{ApiFldMod} $|$ \KWD{wrapped} $|$ \KWD{private}\\

\emph{ApiFldMod} &::=& \KWD{hidden} $|$ \KWD{settable} $|$ \KWD{test}\\

\emph{FldWTypes} &::=& \emph{FldWType} \\
&$|$& \texttt{(} \emph{FldWType}(\EXP{,} \emph{FldWType})$^+$ \texttt{)}\\

\emph{FldWType} &::=& \emph{BindId} \emph{IsType}\\

\end{Grammar}


\note{I think this should be phrased differently.
When I originally read that, I took it to mean
that fields were always ``private.'' -- Scott}
Fields are variables local to an object.
They must not be referred to outside their enclosing object declarations.
Each field is either a non-transient value parameter of a constructor
declaration, or it is explicitly defined by a field declaration within the body of a constructor
declaration, a singleton declaration, or an object expression
(described in \secref{object-expr}).
A field declaration in an object declaration is syntactically identical to
a top-level variable declaration (described in \secref{top-var-decls}),
with the same meanings attached to the form of variable declarations.
Additional modifiers apply to fields, as described in \secref{abstract-fields}.  Unlike with an abstract field in a trait declaration, an object declaration may declare a \KWD{hidden} field which is not \KWD{settable}.

Each field declaration includes an \emph{initial-value} expression,
which specifies the initial value of that field.
Field initialization occurs in textual program order:
evaluation of each initial-value expression
must complete before evaluation of the next initial-value expression,
and all previous field names
(and the parameters of the constructor,
in a constructor declaration)
are in scope when evaluating an initial-value expression
(see~\secref{scope}).
All fields of an object are initialized
before that object is made available for subsequent computation;
thus,
it is illegal to invoke methods on
an object being initialized:
\KWD{self} is \emph{not} implicitly declared by a field declaration.

\note{
Victor: Is the following code legal:

% trait T
%   x: ZZ
%   f() = object extends T
%           x = self.g()  (* self here is bound by f *)
%         end
%   g() = 4
% end
\begin{Fortress}
{\tt~}\pushtabs\=\+\( \KWD{trait}\:T\)\\
{\tt~~}\pushtabs\=\+\(   x\COLON \mathbb{Z}\)\\
\(   f() =  \null\)\pushtabs\=\+\(\KWD{object}\;\;\KWD{extends}\:T\)\\
{\tt~~~~~~~~}\pushtabs\=\+\(           x = \mathord{\KWD{self}}.g()  \mathtt{(*}\;\hbox{\rm  self here is bound by f \unskip}\;\mathtt{*)}\)\-\\\poptabs
\(         \KWD{end}\)\-\\\poptabs
\(   g() = 4\)\-\\\poptabs
\( \KWD{end}\)\-\\\poptabs
\end{Fortress}

What about:
% trait T
%   x: ZZ
%   f() = object extends T
%           x = g()
%         end
%   g() = 4
% end
\begin{Fortress}
{\tt~}\pushtabs\=\+\( \KWD{trait}\:T\)\\
{\tt~~}\pushtabs\=\+\(   x\COLON \mathbb{Z}\)\\
\(   f() =  \null\)\pushtabs\=\+\(\KWD{object}\;\;\KWD{extends}\:T\)\\
{\tt~~~~~~~~}\pushtabs\=\+\(           x = g()\)\-\\\poptabs
\(         \KWD{end}\)\-\\\poptabs
\(   g() = 4\)\-\\\poptabs
\( \KWD{end}\)\-\\\poptabs
\end{Fortress}
}

\note{Victor: This may not belong here, but not sure where!}
Within an object declaration or object expression,
a field can be accessed by a ``naked'' identifier reference
(see \secref{idn-ref}).
Unlike within trait declarations,
such a reference does \emph{not}
invoke the getter or setter method of that name.
To invoke a getter or setter of that name,
it is necessary to use dotted field access
(see \secref{field-access}).

\section{Value Objects}
\seclabel{value-objects}

An object declaration with the modifier \KWD{value} declares a value object
that is called in many languages a \emph{primitive} value.
The \objecttype\ declared by a value object implicitly has the modifier
\KWD{value}.



The fields of a \KWD{value} object are immutable;
they cannot be changed directly or it is a static error.
However, Fortress allows value objects to have settable fields
as an abbreviation for constructing a new value object with a different
value for one field.
If a value object has a setter method or a subscripted assignment operator
method (described in \secref{subscripted-assignment}),
then the return type of the method must be the value object trait type
instead of \EXP{()}.  When such a method is invoked, the receiver
must itself be assignable, and the value returned by the method
is assigned to the receiver.


For example, here is a value object \TYP{Complex} number:
%value object Complex(settable real:Double, settable imaginary:Double=0)
%  opr +(self,other:Complex) = Complex(real + other.real, imaginary + other.imaginary)
%end
\begin{Fortress}
\(\KWD{value}\;\;\KWD{object} \TYP{Complex}(\KWD{settable} \VAR{real}\COLONOP\TYP{Double}, \KWD{settable} \VAR{imaginary}\COLONOP\TYP{Double}=0)\)\\
{\tt~~}\pushtabs\=\+\(  \KWD{opr} +(\KWD{self},\VAR{other}\COLONOP\TYP{Complex}) = \TYP{Complex}(\VAR{real} + \VAR{other}.\VAR{real}, \VAR{imaginary} + \VAR{other}.\VAR{imaginary})\)\-\\\poptabs
\(\KWD{end}\)
\end{Fortress}
When a mutable variable \VAR{z}:
%var z : Complex = Complex(0)
\begin{Fortress}
\(\KWD{var} z \mathrel{\mathtt{:}} \TYP{Complex} = \TYP{Complex}(0)\)
\end{Fortress}
updates its \VAR{imaginary} field, the following syntax:
%z.imaginary := v
\begin{Fortress}
\(z.\VAR{imaginary} \ASSIGN v\)
\end{Fortress}
can be used as an abbreviation for:
%z := Complex(z.real, v)
\begin{Fortress}
\(z \ASSIGN \TYP{Complex}(z.\VAR{real}, v)\)
\end{Fortress}
So the setter for the field \VAR{imaginary} in \TYP{Complex} would do the
work of constructing and returning
%Complex(z.real, v)
\EXP{\TYP{Complex}(z.\VAR{real}, v)}, and the assignment:
%z.imaginary := v
\begin{Fortress}
\(z.\VAR{imaginary} \ASSIGN v\)
\end{Fortress}
would be construed to mean:
%z := z.imaginary(v)
\begin{Fortress}
\(z \ASSIGN z.\VAR{imaginary}(v)\)
\end{Fortress}


Note that modifying a settable field directly within the value object is
not allowed.  For example, the following:
%imaginary := 3
\begin{Fortress}
\(\VAR{imaginary} \ASSIGN 3\)
\end{Fortress}
within the \TYP{Complex} object means:
%self.imaginary := 3
\begin{Fortress}
\(\mathord{\KWD{self}}.\VAR{imaginary} \ASSIGN 3\)
\end{Fortress}
and because \EXP{\mathord{\KWD{self}}} is not mutable, the assignment is
disallowed.


\section{Object Equivalence}
\seclabel{equivalence}

The trait \TYP{Any} declares the object equivalence operator
\EXP{\sequiv}.  This operator is automatically defined for all
objects; it is a static error for the programmer to override it.  The
\EXP{\sequiv} operator is used to decide whether its two arguments
refer to ``the same object'' in the strictest sense possible.  If the
arguments have different dynamic types---including the instantiations
of all static parameters---the result is always \VAR{false}.  If both
arguments are value objects with the same type, then the result is
\VAR{true} if and only if corresponding fields of the objects are
themselves equivalent as defined by this operator; in particular, two
binary words are strictly equivalent if and only if they contain the
same bit pattern.  If both arguments are object references, then the
result is \VAR{true} if and only if the two object references refer to
the identical reference object (in implementation terms, occupying the
same memory locations in the heap).  If both arguments are functions,
the result is \VAR{true} only if the functions behave identically for
any choice of type-correct arguments.  Even if two functions behave
identically, the fortress implementation is free to return \VAR{false}
when they are compared for object equivalence.
