%%%%%%%%%%%%%%%%%%%%%%%%%%%%%%%%%%%%%%%%%%%%%%%%%%%%%%%%%%%%%%%%%%%%%%%%%%%%%%%%
%   Copyright 2009, Oracle and/or its affiliates.
%   All rights reserved.
%
%
%   Use is subject to license terms.
%
%   This distribution may include materials developed by third parties.
%
%%%%%%%%%%%%%%%%%%%%%%%%%%%%%%%%%%%%%%%%%%%%%%%%%%%%%%%%%%%%%%%%%%%%%%%%%%%%%%%%

\section{Operator Precedence and Associativity}
\seclabel{operator-precedence}

Fortress specifies that certain operators have higher precedence than
certain other operators and certain operators are associative,
so that one need not use parentheses in all
cases where operators are mixed in an expression.  (See
\appref{operator-precedence} for a detailed description of operator
precedence and associativity in Fortress.)  However, Fortress
does not follow the practice of other programming languages in
simply assigning an integer to each operator and then saying that
the precedence of any two operators can be compared by comparing
their assigned integers.  Instead, Fortress relies on defining traditional
groups of operators based on their meaning and shape, and specifies
specific precedence relationships between some of these groups.
If there is no specific precedence relationship between two
operators, then parentheses must be used.  For example, Fortress
does not accept the expression \EXP{a + b \cup c};
one must write either \EXP{(a + b) \cup c} or \EXP{a + (b \cup c)}.
(Whether or not the result then makes any sense depends on what definitions
have been made for the \EXP{+} and \EXP{\cup} operators---see
\chapref{operatordefs}.)

Here are the basic principles of operator precedence and associativity
in Fortress:

\begin{itemize}
\item
Member selection (\txt{.}) and method invocation
(\EXP{.\VAR{name}(\ldots)}) are not operators.  They have higher precedence
than any operator listed below.

\item
Subscripting (\txt{[ ]} and any kind of subscripting operators,
which can be any kind of enclosing operators),
superscripting (\txt{{\char'136}}),
and postfix operators have higher precedence than any operator
listed below; within this group, these operations are left-associative
(performed left-to-right).

\item
\emph{Tight juxtaposition}, that is, juxtaposition without intervening
whitespace, has higher precedence than any operator listed below.  The
associativity of tight juxtaposition is type-dependent;
see \secref{juxtameaning}.

\item
Next, \emph{tight fractions}, that is,
the use of the operator `\EXP{/}' with no whitespace on either side,
have higher precedence than any operator listed below.
The tight-fraction operator has no precedence compared with itself,
so it is not permitted to be used more than once in a tight fraction
without use of parentheses.

\item
\emph{Loose juxtaposition}, that is, juxtaposition with intervening
whitespace, has higher precedence than any operator listed below.  The
associativity of loose juxtaposition is type-dependent and is
different from that for tight juxtaposition; see \secref{juxtameaning}.
Note that \emph{lopsided juxtaposition}
(having whitespace on one side but not the other) is a static error as
described in \secref{operator-fixity}.


\item
Prefix operators have higher precedence than any operator listed below.
However, it is a static error for an operand of a loose prefix operator
to be an operand of a tight infix operator.

\item
The infix operators are partitioned into certain traditional groups, as
explained below.  They have higher precedence than any operator listed below.

\item
The equal symbol `\EXP{=}' in binding context,
the assignment operator `\EXP{\ASSIGN}', and
compound assignment operators (\EXP{\mathrel{+}=}, \EXP{-=}, \EXP{\wedge=},
\EXP{\vee=}, \EXP{\cap=}, \EXP{\cup=}, and so on as described in
\secref{operator-app-expr})
have lower precedence than any operator listed above.
Note that compound assignment operators themselves are not operator names.
\end{itemize}

The infix binary operators are divided into four general categories: arithmetic,
relational, boolean, and other.  The arithmetic operators are further categorized as
multiplication/division/intersection, addition/subtraction/union, and other.
The relational operators are further categorized as equivalence,
inequivalence, ordering, and other.
The boolean operators
are further categorized as conjunctive, disjunctive, and other.

The arithmetic and relational operators are further divided into groups based on shape:
\begin{itemize}
\item ``ordinary'' operators:
\EXP{+ - \cdot \times / \pm \mp \oplus \ominus
\odot \otimes \oslash
\boxplus \boxminus \boxdot \boxtimes < \leq \geq > \ll \lll \ggg \gg
\nless \nleq \ngeq \ngtr} etc.

The arithmetic operations in this group are further subdivided into
``plain'' (\EXP{+ - \cdot \times / \pm \mp} etc.),
``circled'' (\EXP{\oplus \ominus \odot \otimes \oslash} etc.),
``boxed'' (\EXP{\boxplus \boxminus \boxdot \boxtimes} etc.), and so on;
 any of these groups may be used with
the plain relational operators
(\EXP{< \leq \geq > \ll \lll \ggg \gg \nless \nleq \ngeq \ngtr} etc.),
but the groups might not be mixed.

\item ``rounded horseshoe'' or ``set'' operators:
\EXP{\cap \Cap \cup \Cup \uplus \subset \subseteq \supseteq \supset
\Subset \Supset \not\subset \nsubseteq \nsupseteq \not\supset}  etc.

\item ``square horseshoe'' operators:
\EXP{\sqcap \sqcup \sqsubset \sqsubseteq \sqsupseteq \sqsupset
\not\sqsubseteq \not\sqsupseteq} etc.

\item ``curly'' operators:
\EXP{\curlywedge \curlyvee \prec \preceq \succeq \succ
\nprec \not\preccurlyeq \not\succcurlyeq \nsucc} etc.

\item ``triangular'' relations:
\EXP{\lhd \unlhd \unrhd \rhd
\ntriangleleft \ntrianglelefteq \ntrianglerighteq \ntriangleright} etc.

\item ``chickenfoot'' relations: \EXP{<\!\llap{-} \rlap{-}\!>} etc.
\end{itemize}
The principles of precedence for binary operators are then as follows:
\begin{itemize}
\item
A multiplication or division or intersection operator has higher precedence than any addition or subtraction or union
operator that is in the same shape group.

\item Certain addition and subtraction operators come in pairs,
such as \EXP{+} and \EXP{-}, or \EXP{\oplus} and \EXP{\ominus},
 which are considered to have the
same precedence and so may be mixed within an expression and are grouped
left-associatively.  These addition-subtraction pairs are the \emph{only}
cases where two different operators are considered to have the same precedence.

\item An arithmetic operator has higher precedence than any equivalence or inequivalence operator.

\item An arithmetic operator has higher precedence than any relational operator that is in the same shape group.

\item A relational operator has higher precedence than any boolean operator.

\item A conjunctive boolean operator has higher precedence than any disjunctive boolean operator.
\end{itemize}

While the rules of precedence are complicated, they
are intended to be both unsurprising and conservative.
Note that operator precedence in Fortress is not always transitive;
for example,  while \EXP{+} has higher precedence than \EXP{<}
(so you can write \EXP{a~+~b~<~c} without parentheses), and \EXP{<} has higher precedence than
\EXP{\vee} (so you can write \EXP{a~<~b~\vee~c~<~d} without
parentheses), it is \emph{not} true
that \EXP{+} has higher precedence than \EXP{\vee}---the expression
\EXP{a~\vee~b~+~c} is not permitted,
and one must instead write \EXP{(a~\vee~b)~+~c} or \EXP{a~\vee~(b~+~c)}.

Another point is that the various multiplication and division operators
do \emph{not} have ``the same precedence''; they may not be mixed freely with each other.
For example, one cannot write
%% \EXP{u~\cdot~v~\times~w};
\EXP{u~\cdot~v~\times~w};
one must write \EXP{(u~\cdot~v)~\times~w} or (more likely)
\EXP{u~\cdot~(v~\times~w)}.
Similarly, one cannot write
%% \EXP{a~\cdot~b~/~c~\cdot~d};
\EXP{a~\odot~b~/~c~\odot~d};
but juxtaposition
does bind more tightly than a loose (whitespace-surrounded) division
slash, so one is allowed to write \EXP{a~b~/~c~d}, and this means the
same as \EXP{(a~b)/(c~d)}.
On the other hand, loose juxtaposition binds less tightly than a tight
division slash,
so that \EXP{a~b/c~d} means the same as \EXP{a~(b/c)~d}.  On the other
other hand,
tight juxtaposition binds more tightly than tight division, so that
\EXP{(n+1)/(n+2)(n+3)}
means the same as \EXP{(n+1)/((n+2)(n+3))}.

There are two additional rules intended to catch misleading code:
it is a static error for an operand of a tight infix or tight prefix operator
to be a loose juxtaposition, and
it is a static error if the rules of precedence determine that a use
of infix operator \VAR{a} has higher or equal precedence
than a use of infix operator \VAR{b}, but that particular use of
\VAR{a} is loose  and that particular
use of \VAR{b} is tight.  Thus, for example, the expression
\EXP{\sin~x~+~y} is permitted, but \EXP{\sin~x\!+\!y} is not permitted.
Similarly, the expression \EXP{a~\cdot~b~+~c} is permitted,
as are \EXP{a{\cdot}b~+~c} and \EXP{a{\cdot}b\!+\!c}, but
\EXP{a~\cdot~b\!+\!c} is not permitted.
(The rule detects only the presence or absence of whitespace, not the
amount of whitespace,
so \EXP{a~~~\cdot~b~+~c} is permitted.  You have to draw the line somewhere.)

When in doubt, just use parentheses.  If there's a problem, the compiler will (probably) let you know.
