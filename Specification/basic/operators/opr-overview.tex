%%%%%%%%%%%%%%%%%%%%%%%%%%%%%%%%%%%%%%%%%%%%%%%%%%%%%%%%%%%%%%%%%%%%%%%%%%%%%%%%
%   Copyright 2009, Oracle and/or its affiliates.
%   All rights reserved.
%
%
%   Use is subject to license terms.
%
%   This distribution may include materials developed by third parties.
%
%%%%%%%%%%%%%%%%%%%%%%%%%%%%%%%%%%%%%%%%%%%%%%%%%%%%%%%%%%%%%%%%%%%%%%%%%%%%%%%%

\section{Overview of Operators in the Fortress Standard Libraries}

\note{The operators in this section are not tested nor run by the interpreter yet.}

This section provides a high-level overview of the operators in
\library.
See \appref{operator-precedence} for the detailed rules for the operators
provided by \library.

\subsection{Prefix Operators}
\seclabel{prefixops}

For all standard numeric types, the prefix operator \EXP{+} simply returns its
argument and the prefix operator \EXP{-} returns the negative of its argument.

The operator \EXP{\neg} is the logical \scap{not} operator on boolean values
and boolean intervals.

The operator \EXP{\twointersectnot} computes the bitwise \scap{not} of
an integer.

Big operators such as \EXP{\sum} begin a \emph{reduction expression}
(\secref{reduction-expr}).  The big operators include \EXP{\sum}
(summation) and \EXP{\prod} (product), along with \EXP{\bigcap},
\EXP{\bigcup}, \EXP{\bigwedge}, \EXP{\bigvee},
\EXP{\underline{\bigvee}}, \EXP{\bigoplus},
\EXP{\bigotimes}, \EXP{\biguplus}, \EXP{\displaystyle\boxplus},
\EXP{\displaystyle\boxtimes}, \OPR{MAX}, \OPR{MIN}, and so on.

\subsection{Postfix Operators}

The operator \EXP{!} computes factorial; the operator \EXP{!!}
computes double factorials.
They may be applied to a value of any integral type and produces a
result of the same type.

When applied to a floating-point value \VAR{x}, \EXP{x!} computes
\EXP{\Gamma(1+x)}, where \EXP{\Gamma} is the Euler gamma function.

\subsection{Enclosing Operators}

When used as left and right enclosing operators, $|\quad|$ computes the
absolute value or magnitude of any number, and is also used to compute
the number of elements in an aggregate, for example the cardinality
of a set or the length of a list.  Similarly, $\|\quad\|$ is used
to compute the norm of a vector or matrix.

The floor operator $\lfloor\quad\rfloor$ and ceiling operator
$\lceil\quad\rceil$
may be applied to any standard integer, rational, or real value (their behavior
is trivial when applied to integers, of course).  The operators
hyperfloor $\lhfloor x \rhfloor$ = $2^{\lfloor \log_2 x \rfloor}$,
hyperceiling $\lhceil x \rhceil$ = $2^{\lceil \log_2 x \rceil}$,
hyperhyperfloor $\lhhfloor x \rhhfloor$ = $2^{\lhfloor \log_2 x \rhfloor}$,
and hyperhyperceiling $\lhhceil x \rhhceil$ = $2^{\lhceil \log_2 x
  \rhceil}$ are also available.

\subsection{Exponentiation}
Given two expressions \VAR{e} and \EXP{e'} denoting numeric quantities \VAR{v} and \EXP{v'} that
are not vectors or matrices,
the expression \EXP{e^{e'}} denotes the quantity obtained by raising \VAR{v} to the power \EXP{v'}.
This operation is defined in the usual way on numerals.

Given an expression \VAR{e} denoting a vector and an expression \EXP{e'} denoting a value
of type \EXP{\mathbb{Z}}, the expression \EXP{e^{e'}} denotes repeated vector multiplication
of \VAR{e} by itself \EXP{e'} times.

Given an expression \VAR{e} denoting a square matrix and an expression \EXP{e'} denoting a value
of type \EXP{\mathbb{Z}}, the expression \EXP{e^{e'}} denotes repeated matrix multiplication
of \VAR{e} by itself \EXP{e'} times.

\note{
Eric: Do we also want to support the matrix exponential via this
operator? Doing so requires having a special constant value for e.}

\subsection{Superscript Operators}

The superscript operator \EXP{\txt{\char'136}\VAR{T}} transposes a
matrix. It also converts a column vector to a row vector or a row
vector to a column vector.

\subsection{Subscript Operators}

Subscripting of arrays and other aggregates is written using square brackets:

\begin{tabular}{llll}
\txt{a[i]}        & \emph{is displayed as} & \EXP{a_i}
& $i$th element of one-dimensional array \VAR{a} \\
\txt{m[i,j]}      & \emph{is displayed as} &    \EXP{m_{ij}}
& ${i,j}$th element  of two-dimensional matrix \VAR{m} \\
\txt{space[i,j,k]}   & \emph{is displayed as} & \EXP{\VAR{space}_{ijk}}
& $i,j,k$th element of three-dimensional array \VAR{space} \\
\txt{a[3] := 4}      & \emph{is displayed as}
& \EXP{a_{3} \mathrel{\mathtt{:}}= 4}
& assign \EXP{4} to the third element of mutable array \VAR{a} \\
\txt{m["foo"]}       & \emph{is displayed as}
& \EXP{m_{\hbox{\rm``{\tt foo}''}}}
& fetch the entry associated with string \EXP{\hbox{\rm``{\tt foo}''}}
from map \VAR{m}
\end{tabular}

\subsection{Multiplication, Division, Modulo, and Remainder Operators}

For most integer, rational, floating-point, complex, and interval expressions,
multiplication can be expressed using any of `\EXP{\cdot}' or
`\EXP{\times}' or simply juxtaposition.
However,
the \EXP{\times} operator is used to
express the shape of matrices, so if expressions using multiplication are used
in expressing the shape of a matrix, it may be necessary to avoid the
use of '\EXP{\times}' to
express multiplication, or to use parentheses.

For integer, rational, floating-point, complex, and interval expressions, division is expressed by \EXP{/}.
When the operator \EXP{/} is used to divide one integer by another,
the result is rational.
The operator \EXP{\div} performs truncating integer division:
\EXP{m \div n = \mathit{signum}\left({m \over n}\right) \left\lfloor \left| {m \over n} \right| \right\rfloor}.
The operator \OPR{REM} gives the remainder from such a truncating division:
\EXP{m \OPR{REM} n = m - n (m \div n)}.
The operator \OPR{MOD} gives the remainder from a floor division:
\EXP{m \OPR{MOD} n = m - n \left\lfloor {m \over n} \right\rfloor};
when \EXP{n>0} this is the usual modulus computation that evaluates integer \VAR{m}
to an integer \VAR{k} such that \EXP{0\leq{}k<n} and \VAR{n} evenly divides \EXP{m-k}.

The special operators \OPR{DIVREM} and \OPR{DIVMOD} each return a pair of values,
the quotient and the remainder; \EXP{m \OPR{DIVREM} n} returns
\EXP{(m \div n, m \OPR{REM} n)} while
\EXP{m \OPR{DIVMOD} n} returns
\EXP{(\left\lfloor {m \over n} \right\rfloor, m \OPR{MOD} n)}.

Multiplication of a vector or matrix by a scalar is done with juxtaposition,
as is multiplication of a vector by a matrix (on either side).  Vector dot product
is expressed by `\EXP{\cdot}' and vector cross product by
`\EXP{\times}'.  Division of
a matrix or vector by a scalar may be expressed using `\EXP{/}'.

The syntactic interaction of juxtaposition, \EXP{\cdot}, \EXP{\times},
and \EXP{/} is subtle.
\See{operator-precedence} for a discussion of the relative precedence of
these operations and how precedence may depend on the use of whitespace.

The handling of overflow depends on the type of the number produced.
For integer results, overflow throws an \TYP{IntegerOverflow}.
Rational computations do not overflow.
For floating-point results, overflow produces \EXP{+\infty} or \EXP{-\infty}
according to the rules of IEEE 754.
For intervals, overflow produces an appropriate containing interval.

Underflow is relevant only to floating-point computations and is handled
according to the rules of IEEE 754.

The handling of division by zero depends on the type of the number produced.
For integer results, division by zero throws a \TYP{DivisionByZero}.
For rational results, division by zero produces \EXP{1/0}.
For floating-point results, division by zero produces a \TYP{NaN} value
according to the rules of IEEE 754.
For intervals, division by zero produces an appropriate containing interval
(which under many circumstances will be the interval of all possible real values and infinities).

Wraparound multiplication on fixed-size integers is expressed by
\EXP{\dottimes}.
Saturating multiplication on fixed-size integers is expressed by
\EXP{\boxdot} or \EXP{\boxtimes}.
These operations do not overflow.

Ordinary multiplication and division of floating-point numbers always
use the IEEE 754 ``round to nearest'' rounding mode.  This rounding
mode may be emphasized by using the operators \EXP{\otimes} (or \EXP{\odot})
and \EXP{\oslash}.  Multiplication and division in ``round toward zero''
mode may be expressed with \EXP{\boxtimes} (or \EXP{\boxdot}) and
\EXP{\boxslash}.
Multiplication and division in ``round toward positive infinity''
mode may be expressed with \EXP{\uptimes} (or \EXP{\updot}) and \EXP{\upslash}.
Multiplication and division in ``round toward negative infinity''
mode may be expressed with \EXP{\downtimes} (or \EXP{\downdot}) and
\EXP{\downslash}.

\subsection{Addition and Subtraction Operators}

Addition and subtraction are expressed with \EXP{+} and \EXP{-} on all numeric quantities,
including intervals, as well as vectors and matrices.

The handling of overflow depends on the type of the number produced.
For integer results, overflow throws an \TYP{IntegerOverflow}.
Rational computations do not overflow.
For floating-point results, overflow produces \EXP{+\infty} or \EXP{-\infty}
according to the rules of IEEE 754.
For intervals, overflow produces an appropriate containing interval.

Underflow is relevant only to floating-point computations and is handled
according to the rules of IEEE 754.

Wraparound addition and subtraction on fixed-size integers are
expressed by \EXP{\dotplus} and \EXP{\dotminus}.
Saturating addition and subtraction on fixed-size integers are
expressed by \EXP{\boxplus} and \EXP{\boxminus}.
These operations do not overflow.

Ordinary addition and subtraction of floating-point numbers always
use the IEEE 754 ``round to nearest'' rounding mode.  This rounding
mode may be emphasized by using the operators \EXP{\oplus}
and \EXP{\ominus}.  Addition and subtraction in ``round toward zero''
mode may be expressed with \EXP{\boxplus} and \EXP{\boxminus}.
Addition and subtraction in ``round toward positive infinity''
mode may be expressed with \EXP{\upplus} and \EXP{\upminus}.
Addition and subtraction in ``round toward negative infinity''
mode may be expressed with \EXP{\downplus} and \EXP{\downminus}.

The construction \EXP{x \pm y} produces the interval \EXP{\bsINT{x-y,x+y}}.

\subsection{Intersection, Union, and Set Difference Operators}

Sets support the operations of intersection \EXP{\cap}, union
\EXP{\cup}, disjoint union \EXP{\uplus},
set difference \EXP{\setminus}, and symmetric set difference
\EXP{\ominus}.  Disjoint union
throws \TYP{DisjointUnionError} if the arguments are in fact not disjoint.

Intervals support the operations of intersection \EXP{\cap}, union
\EXP{\cup}, and interior hull \EXP{\underline{\union}}.
The operation \EXP{\Cap} returns a pair of intervals; if the intersection of
the arguments is a single contiguous span of real numbers, then the first result is an
interval representing that span and the second result is an empty interval,
but if the intersection is two spans, then two (disjoint) intervals are returned.
The operation \EXP{\Cup} returns a pair of intervals; if the arguments overlap, then
the first result is the union of the two intervals and the second result is an empty interval,
but if the arguments are disjoint, they are simply returned as is.

\subsection{Minimum and Maximum Operators}

The operator \OPR{MAX} returns the larger of its two operands,
and \OPR{MIN} returns the smaller of its two operands.

For floating-point numbers, if either argument is a \TYP{NaN} then
\TYP{NaN} is returned.  The floating-point operations \OPR{MAXNUM} and
\OPR{MINNUM}
behave similarly except that if one argument is \TYP{NaN} and the
other is a number, the number is returned.  For all four of these
operators, when applied to floating-point values,
\EXP{-0} is considered to be smaller than \EXP{+0}.

\subsection{GCD, LCM, and CHOOSE Operators}

The infix operator \OPR{GCD} computes the greatest common
divisor of its integer operands, and \OPR{LCM} computes the least
common multiple.  The operator \OPR{CHOOSE} computes binomial
coefficients:
\EXP{n \OPR{CHOOSE} k = \binom{n}{k} = { n! \over k!(n-k)! }}.

The expression \EXP{e \neq e'} is semantically equivalent to the expression
\EXP{\neg(e = e')}.


\subsection{Comparisons Operators}

Unless otherwise noted, the operators described in this section
produce boolean (\VAR{true}/\VAR{false}) results.

The operators $<$, \EXP{\leq}, \EXP{\geq}, and $>$ are
  used for numerical comparisons
and are supported by integer, rational, and floating-point types.
Comparison of rational values throws \TYP{RationalComparisonError}
if either argument is the rational infinity \EXP{1/0} or the
rational indefinite \EXP{0/0}.
Comparison of floating-point values throws \TYP{FloatingComparisonError}
if either argument is a \TYP{NaN}.

The operators $<$, \EXP{\leq}, \EXP{\geq}, and \EXP{>} may also be
used to compare characters
(according to the numerical value of their Unicode codepoint values)
and strings (lexicographic order based on the character ordering).
They also use lexicographic order when used to compare lists
whose elements support these same comparison operators.

When $<$, \EXP{\leq}, \EXP{\geq}, and \EXP{>} are used to compare numerical
intervals, the result is a boolean interval.  The functions
\VAR{possibly} and \VAR{certainly} are useful for converting
boolean intervals to boolean values for testing.  Thus
\EXP{\VAR{possibly}(x>y)} is true if and only if there is some value in the interval
\VAR{x} that is greater than some value in the interval \VAR{y}, while
\EXP{\VAR{certainly}(x>y)} is true if and only if \VAR{x} and \VAR{y} are nonempty
and every value in \VAR{x} is greater than every value in \VAR{y}.

The operators \EXP{\subset}, \EXP{\subseteq}, \EXP{\supseteq}, and
\EXP{\supset} may be used to compare sets or intervals regarded as sets.

The operator \EXP{\in} may be used to test whether a value is a member of
a set, list, array, interval, or range.

\subsection{Logical Operators}

The following binary operators may be used on boolean values:
\begin{center}
\begin{tabular}{rl}
$\wedge$ & \scap{and} \\
$\vee$ & inclusive \scap{or} \\
$\underline{\vee}$ or $\oplus$ or $\neq$ & exclusive \scap{or} \\
$\equiv$ or $\leftrightarrow$ or $=$ & equivalence (if and only if) \\
$\rightarrow$ & \scap{implies} \\
$\overline{\wedge}$ & \scap{nand} \\
$\overline{\vee}$ & \scap{nor}
\end{tabular}
\end{center}

These same operators may also be applied to boolean intervals
to produce boolean interval results.

The following operators may be used on integers to perform
``bitwise'' operations:
\begin{center}
\begin{tabular}{rl}
$\twointersectand$ & bitwise \scap{and} \\
$\twointersector$ & bitwise inclusive \scap{or} \\
$\underline{\twointersector}$ & bitwise exclusive \scap{or}
\end{tabular}
\end{center}
The prefix operator \EXP{\twointersectnot} computes the bitwise
\scap{not} of an integer.


\subsection{Conditional Operators}
\seclabel{standard-conditional-oprs}

If \EXP{p(x)} and \EXP{q(x)} are expressions that produce
boolean results, the expression \EXP{p(x) \wedge\mathpunct{\mathtt{:}}
  q(x)} computes the
logical \scap{and} of those two results by first evaluating \EXP{p(x)}.
If the result of \EXP{p(x)} is \VAR{true}, then \EXP{q(x)} is also
evaluated, and its
result becomes the result of the entire expression; but if the result
of \EXP{p(x)} is \VAR{false}, then \EXP{q(x)} is not evaluated,
and the result of the entire expression is \VAR{false} without further ado.
(Similarly, evaluating the expression \EXP{p(x)
  \vee\mathpunct{\mathtt{:}} q(x)} does not
evaluate \EXP{q(x)} if the result of \EXP{p(x)} is \VAR{true}.)
Contrast this with the expression \EXP{p(x) \wedge q(x)} (with no colon),
which evaluates both \EXP{p(x)} and \EXP{q(x)}, in no specified order and
possibly in parallel.
