%%%%%%%%%%%%%%%%%%%%%%%%%%%%%%%%%%%%%%%%%%%%%%%%%%%%%%%%%%%%%%%%%%%%%%%%%%%%%%%%
%   Copyright 2009, Oracle and/or its affiliates.
%   All rights reserved.
%
%
%   Use is subject to license terms.
%
%   This distribution may include materials developed by third parties.
%
%%%%%%%%%%%%%%%%%%%%%%%%%%%%%%%%%%%%%%%%%%%%%%%%%%%%%%%%%%%%%%%%%%%%%%%%%%%%%%%%

\section{Chained and Multifix Operators}
\seclabel{chained-multifix}

Certain infix mathematical operators that are traditionally regarded as \emph{relational}
operators, delivering boolean results, may be \emph{chained}.  For example,
an expression such as \EXP{A \subseteq B \subset C \subseteq D}
is treated as being equivalent to
\EXP{(A \subseteq B) \wedge (B \subset C) \wedge (C \subseteq D)}
except that the expressions \VAR{B} and \VAR{C} are evaluated only once
(which matters only if they have side effects such as writes or
input/output actions).
Similarly, the expression \EXP{A \subseteq B = C \subset D}
  is treated as being equivalent to
  \EXP{(A \subseteq B) \wedge (B = C) \wedge (C \subset D)},
  except that \VAR{B} and \VAR{C} are evaluated only once.
Fortress restricts such chaining to a mixture of equivalence
  operators and ordering operators; if a chain contains two or
  more ordering operators, then they must be of the same
  kind and have the same sense of monotonicity;
for example, neither \EXP{A \subseteq B \leq C} nor \EXP{A \subseteq B
  \supset C}
is permitted.
This transformation is done before type checking.
In particular, it is done even if these operators do not return
boolean values, and the resulting expression is checked for type
correctness.
(See \secref{precedence:relops} for a detailed description of which
operators may be chained.)


Any infix operator that does not chain may be treated as \emph{multifix}.
If $n-1$ occurrences of the same operator separate $n$ operands where
$n \geq 3$,
then the compiler first checks to see whether there is a definition
for that operator that will accept $n$ arguments.  If so, that definition is used;
if not, then the operator is treated as left-associative and the compiler
looks for a two-argument definition for the operator to use for each occurrence.
As an example, the cartesian product
\EXP{S_1 \times S_2 \times \cdots \times S_n}
of $n$ sets may usefully be defined as a multifix operator, but
ordinary addition
\EXP{p + q + r + s} is normally treated as \EXP{((p + q) + r) + s}.
