%%%%%%%%%%%%%%%%%%%%%%%%%%%%%%%%%%%%%%%%%%%%%%%%%%%%%%%%%%%%%%%%%%%%%%%%%%%%%%%%
%   Copyright 2009, Oracle and/or its affiliates.
%   All rights reserved.
%
%
%   Use is subject to license terms.
%
%   This distribution may include materials developed by third parties.
%
%%%%%%%%%%%%%%%%%%%%%%%%%%%%%%%%%%%%%%%%%%%%%%%%%%%%%%%%%%%%%%%%%%%%%%%%%%%%%%%%

\section{Operator Fixity}
\seclabel{operator-fixity}

\note{Multifix operators are not yet supported.}

Most operators in Fortress can be used variously as prefix, postfix,
infix, nofix, or multifix operators.
(See \secref{chained-multifix} for a discussion of how
infix operators may be chained
or treated as multifix operators.)
\note{
Victor: I think that we shouldn't list ``multifix'' separately here.
 (Note that multifix does not appear in any of the tables.)
 Rather multifix is a way to interpret infix operators.}
Some operators can be used in pairs
as enclosing (bracketing) operators---see \secref{enclosing-ops}.
The Fortress language
dictates only the rules of syntax; whether an operator has
a meaning when used in a particular way depends only on whether
there is a definition in the program for that operator when
used in that particular way (see \chapref{operatordefs}).

The fixity of a non-enclosing operator is determined by context.
To the left of such an operator we may find
(1) a \emph{primary tail} (described below),
(2) another operator, or
(3) a comma, semicolon, or left encloser.  To the right we may find
(1) a \emph{primary front} (described below),
(2) another operator, (3) a comma, semicolon, or right
encloser, or (4) a line break.
%
A primary tail is an identifier, a literal,
a right encloser, or a superscripted postfix operator (exponent operator).
A primary front is an identifier, a literal, or a left encloser.
%
\note{
A primary expression is an identifier, a literal,
an expression enclosed by matching enclosers, a field selection,
or an expression followed by a postfix operator.}
Considered in all combinations, this makes twelve possibilities.
In some cases one must also consider whether or not whitespace separates the operator
from what lies on either side.  The rules of operator fixity are specified
by \figref{opr-fixity1}, where the center column indicates the fixity that results
from the left and right context specified by the other columns.

\begin{figure}[htbp]
\begin{center}
{ %start of local bindings
  \def\two#1#2{\begin{tabular}{c}#1\\#2\end{tabular}}
  \def\twoone#1{\two{\strut}{\raisebox{1.5ex}[0in][0in]{#1}}}
  \def\four#1#2#3#4{\begin{tabular}{c}#1\\#2\\#3\\#4\end{tabular}}
  \def\fourone#1{\four{\strut}{\strut}{\raisebox{1.5ex}[0in][0in]{#1}}{\strut}}

\begin{tabular}{|c|c|c|c|c|}
\hline
{\bf left context} & {\bf whitespace} & {\bf operator fixity} & {\bf whitespace} & {\bf right context} \\ \hline
\fourone{primary tail} & \four{yes}{yes}{no}{no} & \four{{\bf infix}}{{\bf error} (infix)}{{\bf postfix}}{{\bf infix}}
    & \four{yes}{no}{yes}{no} & \fourone{primary front} \\ \hline
\fourone{primary tail} & \four{yes}{yes}{no}{no} & \four{{\bf infix}}{{\bf error} (infix)}{{\bf postfix}}{{\bf infix}}
    & \four{yes}{no}{yes}{no} & \fourone{operator} \\ \hline
\twoone{primary tail} & \two{yes}{no} & \two{{\bf error} (postfix)}{{\bf
    postfix}}  &  & \twoone{\txt{, ; }right encloser} \\ \hline
\twoone{primary tail} & \two{yes}{no} & \two{{\bf infix}}{{\bf postfix}}  &  & \twoone{line break} \\ \hline
operator         & & {\bf prefix}        & & primary front           \\ \hline
operator         & & {\bf prefix}        & & operator          \\ \hline
operator         & & {\bf error} (nofix) & & \txt{, ; }right encloser \\ \hline
operator         & & {\bf error} (nofix) & & line break        \\ \hline
\txt{, ; }left encloser& & {\bf prefix}  & & primary front           \\ \hline
\txt{, ; }left encloser& & {\bf prefix}  & & operator          \\ \hline
\txt{, ; }left encloser& & {\bf nofix}   & & \txt{, ; }right encloser \\ \hline
\txt{, ; }left encloser& & {\bf error} (prefix) & & line break        \\ \hline
\end{tabular}

} %end of local bindings
\end{center}
\caption{Operator Fixity (I)}
\figlabel{opr-fixity1}
\end{figure}

A case described in the center column of the table as an {\bf error}
is a static error; for such cases, the fixity mentioned in parentheses
is the recommended treatment of the operator for the purpose of
attempting to continuing the parse in search of other errors.

The table may seem complicated, but it all boils down to a couple of practical rules of thumb:
\begin{enumerate}
\item {\it Any} operator can be prefix, postfix, infix, or nofix.

\item An infix operator can be \emph{loose} (having whitespace on both sides)
or \emph{tight} (having whitespace on neither side), but it mustn't be \emph{lopsided}
(having whitespace on one side but not the other).

\item A postfix operator must have no whitespace before it and must
be followed (possibly after some whitespace) by a comma, semicolon, right encloser, or line break.
\end{enumerate}
