%%%%%%%%%%%%%%%%%%%%%%%%%%%%%%%%%%%%%%%%%%%%%%%%%%%%%%%%%%%%%%%%%%%%%%%%%%%%%%%%
%   Copyright 2009, Oracle and/or its affiliates.
%   All rights reserved.
%
%
%   Use is subject to license terms.
%
%   This distribution may include materials developed by third parties.
%
%%%%%%%%%%%%%%%%%%%%%%%%%%%%%%%%%%%%%%%%%%%%%%%%%%%%%%%%%%%%%%%%%%%%%%%%%%%%%%%%

\section{Juxtaposition}
\seclabel{juxtameaning}

\note{Qualified names are not yet supported.}

Juxtaposition in Fortress may be a function call or
a special infix operator.  \Library\ include several declarations of
a \KWD{juxtaposition} operator.

When two expressions are juxtaposed, the juxtaposition is interpreted as
follows:
if the left-hand-side expression is a function, juxtaposition performs
function application;
otherwise, juxtaposition performs the
\KWD{juxtaposition} operator application.

The manner in which a juxtaposition of three or more items must be associated
requires type information and awareness of whitespace.  (This is an inherent property
of customary mathematical notation, which Fortress is designed to emulate where feasible.)
Therefore a Fortress compiler must produce a provisional parse in which such
multi-element juxtapositions are held in abeyance, then perform a type analysis
on each element and use that information to rewrite the n-ary juxtaposition
into a tree of binary juxtapositions.

All we need to know is whether each element of a
juxtaposition has an arrow type.
There are actually three legitimate possibilities for each element of a juxtaposition:
(a) it has an arrow type, in which case it is considered to be a function
element;
(b) it has a type that is not an arrow type, in which case it is considered
to be a non-function element.
(c) it is an identifier that has no visible declaration,
in which case it is considered to be a function element (and everything will work
out okay if it turns out to be the name of an appropriate functional method).

The rules below are designed to forbid certain forms of notational ambiguity
that can arise if the name of a functional method happens to be used also
as the name of a variable.  For example, suppose that trait \VAR{T} has a functional
method of one parameter named \VAR{n}; then in the code
%% do
%%   a: T = t
%%   n: ZZ = 14
%%   z = n a
%% end
\begin{Fortress}
\(\KWD{do}\)\\
{\tt~~}\pushtabs\=\+\(  a\COLON T = t\)\\
\(  n\COLON \mathbb{Z} = 14\)\\
\(  z = n\; a\)\-\\\poptabs
\(\KWD{end}\)
\end{Fortress}
it might not be clear whether the intended meaning was to invoke
the functional method \VAR{n} on \VAR{a} or to multiply \VAR{a} by \EXP{14}.
The rules specify that such a situation is a static error.

The rules for reassociating a loose juxtaposition are as follows:
\begin{itemize}
\item
First the loose juxtaposition is broken
into nonempty chunks; wherever there is a non-function element followed by a
function element, the latter begins a new chunk.  Thus a chunk
consists of some number (possibly zero) of functions followed by some
number (possibly zero) of non-functions.

\item
It is a static error if any non-function element in a chunk
is an unparenthesized identifier \VAR{f} and is followed by another non-function
element whose type is such that \VAR{f} can be applied to that latter element
as a functional method.

\item
The non-functions in each
chunk, if any, are replaced by a single element consisting of the
non-functions grouped left-associatively into binary juxtapositions.

\item
What remains in each chunk is then grouped right-associatively.

(Notice that, up to this point, no analysis of the types of newly constructed chunks is needed
during this process.)

\item
It is a static error if an element of the original juxtaposition
was the last element in its chunk before reassociation, the chunk was not the last chunk
(and therefore the element in question is a non-function element),
the element was an unparenthesized identifier \VAR{f},
and the type of the following chunk after reassociation
is such that \VAR{f} can be applied to that following chunk
as a functional method.

\item
If any element that remains has type String, then it is a static error
if there is any pair of adjacent elements within the juxtaposition
such that neither element is of type String.

\item
What remains is considered to be a binary or multifix application of the juxtaposition
operator; if more than two elements remain, the application is handled in the same
way as for any other multifix operator (see \secref{chained-multifix}).
\end{itemize}


Here is an example: \EXP{n~(n+1)~\sin~3~n~x~\log~\log~x}.  Assuming that
\EXP{\sin} and \EXP{\log} name functions in the usual manner and that
\VAR{n}, \EXP{(n+1)}, and \VAR{x} are not functions, this loose juxtaposition
splits into three chunks: \EXP{n~(n+1)}, 
\EXP{\sin~3~n~x}, and \EXP{\log~\log~x}.
The first chunk has only two elements and needs no further reassociation.
In the second chunk, the non-functions \EXP{3~n~x} are replaced by
\EXP{((3~n)~x)}.
In the third chunk, there is only one non-function, so that remains unchanged;
the chunk is the right-associated to form \EXP{(\log~(\log~x))}.  Assuming
that no multifix definition of juxtaposition has been provided,
the three chunks are left-associated, to produce the final interpretation
\EXP{((n~(n+1))~(\sin~((3~n)~x)))~(\log~(\log~x))}.  Now the original
juxtaposition has been reduced to binary juxtaposition expressions.


A tight juxtaposition is always left-associated if it contains any dot
(i.e., ``.'') not within parentheses or some pair of enclosing operators.
If such a tight juxtaposition begins with an identifier immediately followed by
a dot then the maximal prefix of identifiers separated by dots (whitespace may
follow but not precede the dots) is collected into a ``dotted id chain'',
which is subsequently partitioned into
the longest prefix, if any, that is a qualified name, and the remaining
the dots and identifiers, which are interpreted as selectors.  If the last
identifier in the dotted id chain
is part of a selector (i.e., the entire dotted id chain is not a
qualified name), and it
is immediately followed by a left parenthesis, then the last selector together
with the subsequent parenthesis-delimited expression is a method invocation.


A tight juxtaposition without any dots might not be entirely left-associated.
Rather, it is considered as a nonempty sequence of elements: the front
expression, any ``math items'', and any postfix operator, and subject to
reassociation as described below.  A math item may be a subscripting,
an exponentiation, or one of a few kinds of expressions.
It is a static error if an exponentiation is immediately followed by
a subcripting or an exponentiation.

The procedure for reassociation is as follows:
\begin{itemize}
\item For each expression element (i.e., not a subscripting, exponentiation or
postfix operator), determine whether it is a function.
\item If some function element is immediately followed by an expression element
then, find the first such function element, and call the next element the
argument.  It is a static error if either the argument is not parenthesized,
or the argument is immediately followed by a non-expression element.  Otherwise,
replace the function and argument with a single element that is the application
of the function to the argument.  This new element is an expression.
Reassociate the resulting sequence (which is one element shorter).
\item If there is any non-expression element (it cannot be the first element)
then replace the first such element and the element immediately preceding it
(which must be an expression) with a single element that does the appropriate
operator application.  This new element is an expression.  Reassociate the
resulting sequence (which is one element shorter).
\item Otherwise, the sequence necessarily has only expression elements,
only the last of which may be a function.
If any element that remains has type String, then it is a static error
if there is any pair of adjacent elements within the juxtaposition
such that neither element is of type String.
\item
What remains is considered to be a binary or multifix application of the juxtaposition
operator; if more than two elements remain, the application is handled in the same
way as for any other multifix operator (see \secref{chained-multifix}).
\end{itemize}
(Note that this process requires type analysis of newly created chunks
all along the way.)

Here is an (admittedly contrived) example:
\EXP{\VAR{reduce}(f)(a)(x+1)\VAR{atan}(x+2)}.
Suppose that \VAR{reduce} is a curried function
that accepts a function \VAR{f} and returns a function that can
be applied to an array \VAR{a} (the idea is to use the function \VAR{f},
which ought to take two arguments, to combine the elements of the
array to produce an accumulated result).

The leftmost function is \VAR{reduce}, and the following element \VAR{(f)}
is parenthesized, so the two elements are replaced with one:
\EXP{(\VAR{reduce}(f))(a)(x+1)\VAR{atan}(x+2)}.
Now type analysis determines that the element \EXP{(\VAR{reduce}(f))}
is a function.

The leftmost function is \EXP{(\VAR{reduce}(f))}, and the following element
\EXP{(a)}
is parenthesized, so the two elements are replaced with one:
\EXP{((\VAR{reduce}(f))(a))(x+1)\VAR{atan}(x+2)}.
Now type analysis determines that the element
\EXP{((\VAR{reduce}(f))(a))} is not a function.

The leftmost function is \EXP{\VAR{atan}},
and the following element \EXP{(x+2)}
is parenthesized, so the two elements are replaced with one:
\EXP{((\VAR{reduce}(f))(a))(x+1)(\VAR{atan}(x+2))}.
Now type analysis determines that the element \EXP{(\VAR{atan}(x+2))}
is not a function.

There are no functions remaining in the juxtaposition.  Assuming
that no multifix definition of juxtaposition has been provided,
the remaining elements are left-associated:
\begin{center}
\EXP{(((\VAR{reduce}(f))(a))(x+1))(\VAR{atan}(x+2))}
\end{center}
Now the original juxtaposition has been reduced to binary juxtaposition
expressions.
