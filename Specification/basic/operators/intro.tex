%%%%%%%%%%%%%%%%%%%%%%%%%%%%%%%%%%%%%%%%%%%%%%%%%%%%%%%%%%%%%%%%%%%%%%%%%%%%%%%%
%   Copyright 2009, Oracle and/or its affiliates.
%   All rights reserved.
%
%
%   Use is subject to license terms.
%
%   This distribution may include materials developed by third parties.
%
%%%%%%%%%%%%%%%%%%%%%%%%%%%%%%%%%%%%%%%%%%%%%%%%%%%%%%%%%%%%%%%%%%%%%%%%%%%%%%%%

\note{Multifix operators and dimensions and units
are not yet supported.}

\note{Synonym Operators: Victor's email titled ``Synonyms [Fwd: Re: What the heck are 0-width spaces for?]'' on 09/12/07}

Operators are like functions or methods; operator declarations are
described in \chapref{operatordefs} and operator applications are
described in \secref{operator-app-expr}, \secref{reduction-expr}, and
\secref{comprehensions}.
Operators are not values; when an operator is passed as an argument to a
function, it should be eta expanded.
Just as functions or methods may be overloaded (see
\chapref{multiple-dispatch} for a discussion of overloading),
so operators may have overloaded declarations of the same fixity.
Operator declarations with the same operator name but with different fixities
are valid declarations because it is always unambiguous which declaration
shall be applied to an application of the operator.
Calls to overloaded operators are resolved first via the
fixity of the operators based on the context of the calls.
Then, among the applicable declarations with that fixity,
the most specific declaration is chosen.

Most operators can be used as prefix, infix, postfix, or
nofix operators as described in \secref{operator-fixity}
(nofix operators take no arguments);
the fixity of an operator is determined syntactically,
and the same operator may have declarations for multiple fixities.
A simple example is that `\EXP{-}' may be either infix or prefix,
as is conventional.
As another example, `\EXP{!}' may be a
postfix operator that computes factorial when applied to integers.
These operators might not be used as enclosing operators.


Several pairs of operators can be used as enclosing operators.  Any number
of `\txt{|}' (vertical line) can be used as both infix operators and
enclosing operators.


Some operators are always postfix: a `\txt{\char'136}' followed by
any ordinary operator (with no intervening whitespace) is considered to be a
superscripted postfix operator.  For example, `\EXP{\txt{\char'136}*}' and
`\EXP{\txt{\char'136}{+}}' and `\EXP{\txt{\char'136}?}'
are available for use as part
of the syntax of extended regular expressions.  As a very special case,
`\EXP{\txt{\char'136}T}' is also considered to be a superscripted
postfix operator,
typically used to signify matrix transposition.


Finally, there are special operators such as juxtaposition
and operators on dimensions and units.
Juxtaposition may be a function application or
an infix operator in Fortress.
When the left-hand-side expression is a function, juxtaposition performs
function application; when the left-hand-side expression is a number,
juxtaposition conventionally performs multiplication; when the
left-hand-side expression is a string,
juxtaposition conventionally performs string concatenation.
Fortress provides several operators on dimensions and units
as described in \chapref{dimunits}.
