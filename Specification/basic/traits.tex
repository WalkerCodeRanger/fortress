%%%%%%%%%%%%%%%%%%%%%%%%%%%%%%%%%%%%%%%%%%%%%%%%%%%%%%%%%%%%%%%%%%%%%%%%%%%%%%%%
%   Copyright 2009, Oracle and/or its affiliates.
%   All rights reserved.
%
%
%   Use is subject to license terms.
%
%   This distribution may include materials developed by third parties.
%
%%%%%%%%%%%%%%%%%%%%%%%%%%%%%%%%%%%%%%%%%%%%%%%%%%%%%%%%%%%%%%%%%%%%%%%%%%%%%%%%

\chapter{Traits}
\chaplabel{traits}

\note{Where clauses, the \KWD{override} modifier, value traits,
method contracts, and abstract functional declarations are not yet supported.}

\note{Covariant list with \KWD{comprises} clause (Victor's email titled
``Problem with covariant List?'' on 10/27/07)}

\emph{Traits} are declared by trait declarations.
Traits define new named types.
A trait specifies a collection of \emph{methods}
(described in \secref{methods}).
One trait can extend others,
which means that it inherits the methods from those traits,
and that the type defined by that trait
is a subtype of the types of traits it extends.


\section{Trait Declarations}
\seclabel{trait-decls}

\begin{Grammar}
\emph{TraitDecl}  &::=& \option{\emph{TraitMods}}
\emph{TraitHeaderFront} \emph{TraitClauses} \option{\emph{GoInATrait}} \KWD{end}
\options{\option{\KWD{trait}} \emph{Id}} \\

\emph{TraitHeaderFront} &::=&
\KWD{trait} \emph{Id} \option{\emph{StaticParams}} \option{\emph{ExtendsWhere}} \\

\emph{TraitClauses} &::=& \emph{TraitClause}$^*$\\

\emph{TraitClause} &::=& \emph{Excludes} \\
&$|$& \emph{Comprises} \\
&$|$& \emph{Where} \\

\emph{GoInATrait}
&::=& \option{\emph{Coercions}}
\emph{GoFrontInATrait} \option{\emph{GoBackInATrait}}\\
&$|$& \option{\emph{Coercions}}
\emph{GoBackInATrait} \\
&$|$& \emph{Coercions}\\

\emph{Coercions} &::=& \emph{Coercion}$^+$\\

\emph{GoFrontInATrait} &::=& \emph{GoesFrontInATrait}$^+$\\

\emph{GoesFrontInATrait}
&::=& \emph{AbsFldDecl} \\
&$|$& \emph{GetterSetterDecl} \\
&$|$& \emph{PropertyDecl} \\

\emph{GoBackInATrait} &::=& \emph{GoesBackInATrait}$^+$\\

\emph{GoesBackInATrait}
&::=& \emph{MdDecl} \\
&$|$& \emph{PropertyDecl} \\

\emph{TraitMods} &::=& \emph{TraitMod}$^+$\\

\emph{TraitMod} &::=& \emph{AbsTraitMod} $|$ \KWD{private}\\

\emph{AbsTraitMod} &::=& \KWD{value} $|$ \KWD{test}\\

\emph{ExtendsWhere} &::=& \KWD{extends} \emph{TraitTypeWheres} \\

\emph{TraitTypeWheres} &::=& \emph{TraitTypeWhere} \\
&$|$& \{ \emph{TraitTypeWhereList} \} \\

\emph{TraitTypeWhereList} &::=& \emph{TraitTypeWhere}(\EXP{,} \emph{TraitTypeWhere})$^*$ \\

\emph{TraitTypeWhere} &::=& \emph{TraitType} \option{\emph{Where}}\\
\end{Grammar}

\begin{GrammarTwo}
\emph{Excludes} &::=& \KWD{excludes} \emph{TraitTypes} \\
\emph{Comprises} &::=& \KWD{comprises} \emph{TraitTypes} \\

\emph{TraitTypes} &::=& \emph{TraitType} \\
&$|$& \{ \emph{TraitTypeList} \} \\
\emph{TraitTypeList} &::=& \emph{TraitType}(\EXP{,} \emph{TraitType})$^*$ \\

\end{GrammarTwo}

\begin{GrammarTwo}
\emph{TraitType}
&::=& \emph{TypeRef}\\
&$|$& \emph{Type} \texttt{[} \option{\emph{ArraySize}} \texttt{]} \\
&$|$& \emph{Type} \verb+^+ \emph{IntExpr}\\
&$|$& \emph{Type} \verb+^+ \texttt{(} \emph{ExtentRange}
(\EXP{\times} \emph{ExtentRange})$^*$ \texttt{)}\\

\emph{ArraySize} &::=& \emph{ExtentRange}(\EXP{,} \emph{ExtentRange})$^*$ \\

\emph{ExtentRange}
&::=& \option{\emph{StaticArg}}\EXP{\mathinner{\hbox{\tt\char'43}}}
\option{\emph{StaticArg}}\\
&$|$& \option{\emph{StaticArg}}\KWD:\option{\emph{StaticArg}}\\
&$|$& \emph{StaticArg} \\

\emph{StaticArgs}    &::=& \bTPl \emph{StaticArgList} \bTPr \\

\emph{StaticArgList} &::=& \emph{StaticArg}(\EXP{,} \emph{StaticArg})$^*$ \\

\emph{StaticArg} &::=&
\emph{Op} \\
&$|$& \TYP{Unity} \\
&$|$& \TYP{dimensionless}\\
&$|$& \EXP{1}\EXP{/}\emph{Type}\\
&$|$& \emph{DimPrefixOp} \emph{Type}\\
&$|$& \emph{IntExpr}\\
&$|$& \emph{BoolExpr}\\
&$|$& \emph{Type}\\
&$|$& \emph{UnitExpr}\\

\end{GrammarTwo}

Syntactically, a trait declaration starts with
an optional sequence of modifiers followed by
\KWD{trait}, followed by the name of the trait,
an optional sequence of static parameters (described in
\chapref{trait-parameters}),
an optional set of \emph{extended} traits,
an optional set of \emph{excluded} traits,
an optional set of \emph{comprises} on the trait,
an optional \KWD{where} clause (described in \secref{where-clauses}),
zero or more \emph{coercion} declarations
(discussed in \chapref{conversions-coercions}),
zero or more declarations of abstract fields,
getter and setter methods,
and zero or more declarations of methods
separated by newlines or semicolons,
that are not getters or setters,
and finally \KWD{end}.
A trait declaration may end with an optional \KWD{trait} and its name again.
It is a static error if the identifier after \KWD{end} is not the name of the trait being declared.
Property declarations (described in \secref{property-decls}) can be freely
commingled with the declarations of methods and abstract fields.
An \KWD{extends} clause comes first, if any, and
\KWD{excludes},
\KWD{comprises},
and \KWD{where}
clauses come in any order.

Each of \KWD{extends}, \KWD{excludes}, and \KWD{comprises} clauses consists
of \KWD{extends}, \KWD{excludes}, and \KWD{comprises}
respectively followed by a set of trait references
separated by commas and enclosed in braces `\txt{\{}' and `\txt{\}}'.
If such a clause contains only one trait, the enclosing braces may be elided.
In an API (but not a component),
a \KWD{comprises} clause may include ``\EXP{\ldots}''.
A trait reference listed in the \KWD{comprises} clause is a declared
trait identifier (within the same component or an API imported by the
component).
\note{
Victor: Why is this property stated only for a comprises clause?
Doesn't it apply to extends and excludes clauses as well?
Also, do we really mean ``trait identifier'' above?  Where is that
term defined?  Does it include instantiations of parametric traits?}

A trait with an \KWD{extends} clause
\emph{explicitly extends} those traits listed in its \KWD{extends} clause.
\note{
Victor: It is possible for a trait to explicitly extend itself
 using hidden type variables.  In particular, the straightforward
 interpretation of covariant/contravariant declarations has this property.
 (But we could just rule this a special case, since it is special in other
 ways already.)
}
In addition, every trait except \TYP{Any} and \TYP{Object}
implicitly extends the trait \TYP{Object}
if it does not do so explicitly.
We define the extension relation to be the transitive closure
of implicit and
explicit extension.
That is, trait \VAR{T} \emph{extends} trait \VAR{U}
if and only if \VAR{T} explicitly or implicitly
extends \VAR{U} or if there is some trait \VAR{S}
that \VAR{T} explicitly extends and that extends \VAR{U}.
The extension relation induced by a program is the smallest
relation satisfying these conditions.
This relation must form an acyclic hierarchy
rooted at trait \TYP{Object}.
\note{
Victor: Technically, there may be self cycles in explicit extension due to
 hidden type variables, as described above.}
If a trait \VAR{T} extends trait \VAR{U},
or if \VAR{T} and \VAR{U} are the same trait,
we say that \VAR{T} is a subtrait of \VAR{U}
and that \VAR{U} is a supertrait of \VAR{T}.

We say that trait \VAR{T} \emph{strictly extends} trait \VAR{U}
if and only if
(\emph{i}) \VAR{T} extends \VAR{U} and (\emph{ii}) \VAR{T} is not \VAR{U}.
We say that trait \VAR{T} \emph{immediately extends} trait \VAR{U}
\note{
Victor: I think we don't use ``immediately extends'' anywhere in the spec,
  only ``immediate supertype/subtype''.}
if and only if (\emph{i}) \VAR{T} strictly extends \VAR{U} and
(\emph{ii}) there is no trait \VAR{V}
such that \VAR{T} strictly extends \VAR{V}
and \VAR{V} strictly extends \VAR{U}.
We call \VAR{U} an \emph{immediate supertrait} of \VAR{T}
and \VAR{T} an \emph{immediate subtrait} of \VAR{U}.
\note{Victor: this belongs in the type chapter.}

A trait with an \KWD{excludes} clause excludes
every trait listed in its \KWD{excludes} clause.
If a trait \VAR{T} excludes a trait \VAR{U},
the two traits are mutually exclusive:
neither can extend the other, and no trait can extend them both.
As discussed in \secref{type-relations},
the exclusion relation is symmetric.
Thus, if \VAR{T} excludes \VAR{U}
then \VAR{U} excludes \VAR{T}
whether or not \VAR{T} is listed in an \KWD{excludes} clause
in the declaration of \VAR{U}.

\note{
If a trait declaration of \VAR{T} includes a \KWD{comprises} clause
then the traits listed in its \KWD{comprises} clause
are exactly the traits that immediately extend \VAR{T}
and they must explicitly extend \VAR{T}
(i.e., list \VAR{T} in their \KWD{extends} clause).
If the \KWD{comprises} clause of a trait \VAR{T} includes ``\EXP{\ldots}''
(and must therefore be in an API),
then in a component exporting the enclosing API,
other traits may explicitly extend \VAR{T},
but these traits may not be declared or imported by the API
(see \secref{apis}).}


For example, the following trait declaration:
\input{\home/basic/examples/Trait.Decl.tex}
declares a trait \TYP{Catalyst} with
no modifiers,
no static parameters,
and
no \KWD{excludes} clauses,
no \KWD{comprises} clauses,
and no \KWD{where} clauses.
Trait \TYP{Catalyst} extends a trait named \TYP{Object}.
A single method (named \VAR{catalyze}) is declared, which
has a parameter of type
\TYP{Reaction} and the return type \TYP{()}.


The following example trait:
%trait Molecule comprises { OrganicMolecule, InorganicMolecule }
%  mass(): Mass
%end
\begin{Fortress}
\(\KWD{trait} \TYP{Molecule} \KWD{comprises} \{\,\TYP{OrganicMolecule}, \TYP{InorganicMolecule}\,\}\)\\
{\tt~~}\pushtabs\=\+\(  \VAR{mass}()\COLON \TYP{Mass}\)\-\\\poptabs
\(\KWD{end}\)
\end{Fortress}
comprises of two traits: \TYP{OrganicMolecule} and \TYP{InorganicMolecule}.
Therefore, the following trait declaration is not allowed:
%(* Not allowed! *)
%trait ExclusiveMolecule extends Molecule end
\begin{Fortress}
\(\mathtt{(*}\;\hbox{\rm  Not allowed! \unskip}\;\mathtt{*)}\)\\
\(\KWD{trait} \TYP{ExclusiveMolecule} \KWD{extends} \TYP{Molecule} \KWD{end}\)
\end{Fortress}
Traits \TYP{OrganicMolecule} and \TYP{InorganicMolecule} may be exclusive:
%trait OrganicMolecule extends Molecule excludes InorganicMolecule end
%
%trait InorganicMolecule extends Molecule end
\begin{Fortress}
\(\KWD{trait} \TYP{OrganicMolecule} \KWD{extends} \TYP{Molecule} \KWD{excludes} \TYP{InorganicMolecule} \KWD{end}\)\\[4pt]
\(\KWD{trait} \TYP{InorganicMolecule} \KWD{extends} \TYP{Molecule} \KWD{end}\)
\end{Fortress}
\TYP{OrganicMolecule} and \TYP{InorganicMolecule} exclude each other, even
though only \TYP{OrganicMolecule} has an \KWD{excludes} clause.
For example, the following trait declaration is not allowed:
%(* Not allowed! *)
%trait InclusiveMolecule extends { InorganicMolecule, OrganicMolecule } end
\begin{Fortress}
\(\mathtt{(*}\;\hbox{\rm  Not allowed! \unskip}\;\mathtt{*)}\)\\
\(\KWD{trait} \TYP{InclusiveMolecule} \KWD{extends} \{\,\TYP{InorganicMolecule}, \TYP{OrganicMolecule}\,\} \KWD{end}\)
\end{Fortress}

A trait is allowed to have multiple immediate supertraits.
The following trait has two immediate supertraits:
%trait Enzyme extends { OrganicMolecule, Catalyst } end
\begin{Fortress}
\(\KWD{trait} \TYP{Enzyme} \KWD{extends} \{\,\TYP{OrganicMolecule}, \TYP{Catalyst}\,\} \KWD{end}\)
\end{Fortress}

\section{Method Declarations}
\seclabel{methods}

\note{
hidden without settable is a static error
only within traits
not  within objects !!!
-- Sukyoung}

\note{Accessing a field by a ``naked'' identifier reference
within trait declarations are not allowed.
We require a dotted field access by a ``self.'' prefix. (08/11/08)
-- Sukyoung}

\note{Getters must always be called with the special syntax:
\EXP{\VAR{receiver}.\VAR{getterName}
}

A getter is required to have a type consistent with a corresponding setter.}

\begin{Grammar}
\emph{MdDecl} &::=& \emph{MdDef} \\
&$|$&
\option{\KWD{abstract}} \option{\emph{MdMods}} \option{\KWD{getter}}
\emph{MdHeaderFront} \emph{FnHeaderClause} \\

\emph{MdDef}  &::=& \option{\emph{MdMods}} \option{\KWD{getter}}
\emph{MdHeaderFront} \emph{FnHeaderClause}
\EXP{=} \emph{Expr} \\

\emph{AbsMdDecl} &::=&
\option{\KWD{abstract}} \option{\emph{AbsMdMods}} \option{\KWD{getter}}
\emph{MdHeaderFront} \emph{FnHeaderClause} \\

\emph{MdMods} &::=& \emph{MdMod}$^+$\\

\emph{MdMod} &::=& \emph{FnMod} $|$ \KWD{override}\\

\emph{FnMod} &::=& \emph{AbsFnMod} $|$ \KWD{private}\\

\emph{AbsFnMod} &::=& \emph{LocalFnMod} $|$
\KWD{test}\\

\emph{LocalFnMod} &::=& \KWD{atomic} $|$ \KWD{io}\\

\emph{AbsMdMods} &::=& \emph{AbsMdMod}$^+$\\

\emph{AbsMdMod} &::=& \emph{AbsFnMod} $|$ \KWD{override}\\

\emph{GetterSetterDecl} &::=& \emph{GetterSetterDef} \\
&$|$& \option{\KWD{abstract}} \option{\emph{FnMods}}
\emph{GetterSetterMod} \emph{MdHeaderFront} \emph{FnHeaderClause} \\

\emph{GetterSetterDef}  &::=& \option{\emph{FnMods}} \emph{GetterSetterMod}
\emph{MdHeaderFront} \emph{FnHeaderClause}
\EXP{=} \emph{Expr} \\

\emph{GetterSetterMod} &::=& \KWD{getter} $|$ \KWD{setter}\\

\emph{AbsGetterSetterDecl} &::=& \option{\KWD{abstract}} \option{\emph{AbsFnMods}}
 \emph{GetterSetterMod} \emph{MdHeaderFront} \emph{FnHeaderClause} \\

\emph{MdHeaderFront}  &::=& \emph{NamedMdHeaderFront}\\
&$|$& \emph{OpMdHeaderFront} \\

\emph{NamedMdHeaderFront} &::=&
\emph{Id} \option{\emph{StaticParams}} \emph{MdValParam} \\

\emph{MdValParam} &::=& \texttt( \option{\emph{MdParams}} \texttt) \\

\emph{MdParams}
&::=&
(\emph{MdParam}\EXP{,})$^*$ \options{\emph{Varargs}\EXP{,}} \emph{MdKeyword}(\EXP{,}\emph{MdKeyword})$^*$\\
&$|$&
(\emph{MdParam}\EXP{,})$^*$  \emph{Varargs}\\
&$|$& \emph{MdParam}(\EXP{,} \emph{MdParam})$^*$\\

\emph{MdKeyword} &::=& \emph{MdParam}\EXP{=}\emph{Expr} \\

\emph{MdParam} &::=& \emph{Param} \\
&$|$& \KWD{self} \\

\emph{Param} &::=& \emph{PlainParam}\\

\emph{PlainParam} &::=& \emph{BindId} \option{\emph{IsType}} \\
&$|$& \emph{Type} \\

\emph{FnHeaderClause} &::=& \option{\emph{IsType}} \emph{FnClauses} \\

\emph{IsType} &::=& \EXP{\mathrel{\mathtt{:}}} \emph{Type}\\

\emph{FnClauses} &::=& \option{\emph{Throws}} \option{\emph{Where}} \emph{Contract} \\

\emph{Throws} &::=& \KWD{throws} \emph{MayTraitTypes}\\

\emph{MayTraitTypes} &::=& \texttt{\{\}} \\
&$|$& \emph{TraitTypes} \\

\end{Grammar}

A trait declaration contains a set of method declarations.
Syntactically, a method declaration begins with
an optional sequence of modifiers followed by
the method's name,
optional static parameters
(described in \chapref{trait-parameters}),
the value parameter with its (optionally) declared type,
an optional type of a return value,
an optional declaration of thrown checked exceptions
(discussed in \chapref{exceptions}),
an optional \KWD{where} clause
(discussed in \secref{where-clauses}),
a contract for the method (discussed in \secref{method-contracts}),
and finally an optional body expression preceded by the token \EXP{=}.
If a method declaration has the \KWD{getter} modifier,
it does not affect the behavior of a program; rather,
it documents that the method is intended to serve as a getter.
A \KWD{throws} clause does not include naked type variables.
Every element in a \KWD{throws} clause is a subtype of
\TYP{CheckedException}.


Method declarations can include the following special modifiers:
\paragraph{\KWD{getter}:}
A method declaration with the modifier \KWD{getter} explicitly declares a
getter method for a field, even in the absence of an actual field declaration.
\note{The following description is too confusing.  Reword it!!!}
If such a field exists, there is no implicit getter for the field.
An explicitly declared getter method must take no arguments and return a
result of the field's type.
A getter method must not throw any checked exception.
Getter names may not overlap ordinary method names.
A getter method must be invoked with the field access syntax:
%expr.id
\begin{Fortress}
\(\VAR{expr}.\VAR{id}\)
\end{Fortress}
where \VAR{id} is the name of the getter method.



\paragraph{\KWD{setter}:}
A method declaration with the modifier \KWD{setter} explicitly declares a
setter method for a field, even in the absence of an actual field.
\note{The following description is too confusing.  Reword it!!!}
If such a field exists, there is no implicit setter for the field.
An explicitly declared setter method must take a single argument---the
value being set---and return \EXP{()}.
A setter method must not throw any checked exception.
Setter names may not overlap ordinary method names.
A setter method must be invoked with the \emph{assignment syntax}:
%expr1.id := expr2
\begin{Fortress}
\({expr}_{1}.\VAR{id} \ASSIGN {expr}_{2}\)
\end{Fortress}


We say that a method declaration \emph{occurs} in a trait declaration.
A trait declaration \emph{declares} a method declaration that occurs
in that trait declaration.
A trait declaration \emph{inherits} method declarations from the
declarations of its immediate supertraits
except those method declarations
that are \emph{overridden} (as described later in this section) or
whose parameter type (not including the type of the self parameter)
is equal to the parameter type of a method declaration
that occurs in the trait declaration.
A trait declaration \emph{provides} the method declarations that it
declares or inherits.




There are two sorts of method declarations:
\emph{dotted method} declarations and
\emph{functional method} declarations.
Syntactically, a dotted method declaration is identical to a function
declaration.
When a method is invoked, a special self parameter is bound to the
object on which it is invoked.



A functional method declaration has a parameter named \KWD{self} at an
arbitrary position in its parameter list.  This parameter is not given a type
and implicitly has the type of the enclosing trait or object.
Semantically, functional method declarations can be viewed as top-level
function declarations.  For example, the following overloaded functional method
\VAR{f} declared within a trait declaration \VAR{A}:
\input{\home/basic/examples/Trait.Method.a.tex}
\input{\home/basic/examples/Trait.Method.b.tex}
may be rewritten as top-level functions as follows:
\input{\home/basic/examples/Trait.Method.c.tex}
\input{\home/basic/examples/Trait.Method.d.tex}
where \VAR{internalF} is a freshly generated name.
Functional method declarations may be overloaded with top-level function
declarations.
An abstract function declaration
(described in \secref{abstractFunctionDeclarations})
can be provided also for overloaded functional method declarations.
See \chapref{multiple-dispatch} for a discussion of overloaded
functionals in Fortress.



When a method declaration includes a body expression, it is called a
\emph{method definition}.
A method declaration that does not have its body expression is referred to
as an \emph{abstract method declaration}.
An abstract method declaration declares an \emph{abstract method};
any object inheriting an abstract method must define a body expression for the
method.  An abstract method declaration may include the modifier \KWD{abstract}.
It may elide parameter names but parameter
types cannot be omitted except for the self parameter.


\note{
Method declarations that are overridden by some declaration are not
accessible.  They are not considered for overloading checks.

Partial overriding of an overloaded functional is allowed
but partial overriding of a functional declaration is not allowed.
}

A method declaration may have the modifier \KWD{override}.
Such a declaration \emph{overrides} any inherited declaration with the same name
and the self parameter (if any) at the same position,
whose parameter type, not counting the type of the self parameter,
is a strict subtype of the overriding declaration's parameter type.
The return type of the overriding declaration must be subtype of
that of the overridden declaration to preserve type safety.
A declaration with the modifier \KWD{override} must override
some inherited declaration.
It is a static error if a declaration with the modifier \KWD{override}
does not override any inherited declaration.

Here is an example trait \TYP{Enzyme} which provides methods
\VAR{reactionSpeed} and \VAR{catalyze}:
\input{\home/basic/examples/Trait.Method.e.tex}
\TYP{Enzyme}
inherits the abstract method \VAR{mass} from \TYP{OrganicMolecule},
declares the abstract method \VAR{reactionSpeed}, and
declares the concrete method \VAR{catalyze}
which is inherited as an abstract method
from its supertrait \TYP{Catalyst}.





\section{Abstract Field Declarations}
\seclabel{abstract-fields}

\begin{Grammar}
\emph{AbsFldDecl}
&::=& \option{\emph{AbsFldMods}} \emph{AbsFldWTypes}\\
&$|$& \option{\emph{AbsFldMods}} \emph{BindIdOrBindIdTuple} \EXP{\mathrel{\mathtt{:}}} \emph{Type}\EXP{...}\\
&$|$& \option{\emph{AbsFldMods}} \emph{BindIdOrBindIdTuple} \EXP{\mathrel{\mathtt{:}}} \emph{TupleType}\\

\emph{AbsFldWTypes} &::=& \emph{AbsFldWType} \\
&$|$& \texttt{(} \emph{AbsFldWType}(\EXP{,} \emph{AbsFldWType})$^+$ \texttt{)}\\

\emph{AbsFldWType} &::=& \emph{BindId} \emph{IsType}\\

\emph{AbsFldMods} &::=& \emph{AbsFldMod}$^+$\\

\emph{AbsFldMod} &::=& \emph{ApiFldMod} $|$ \KWD{wrapped} $|$ \KWD{private}\\

\emph{ApiFldMod} &::=& \KWD{hidden} $|$ \KWD{settable} $|$ \KWD{test}\\

\end{Grammar}


Traits may also include abstract field declarations that are implicit
declarations of abstract getter methods.
Syntactically, an abstract field declaration consists of an optional
sequence of modifiers followed by the field name, followed by the token
`\EXP{\COLONOP}', and the type of the field.


By default, a field declaration implicitly
declares a getter method for the field unless there is an explicit
getter declared in the enclosing trait.
An implicit getter method takes no arguments, has the same name as the
field, and has a return type equal to the field type.  When called, the
implicit getter returns the value of the field when called.




Abstract field declarations can include the following special modifiers:
\paragraph{\KWD{hidden}:}
A field declaration with the
modifier \KWD{hidden} has no implicit getter method.


\paragraph{\KWD{settable}:}
A field declaration with the modifier \KWD{settable} has an implicit
setter method unless there is an explicit setter declared in the enclosing
trait.  An implicit setter method takes a parameter
(with no default expression) whose type is the type of the field, and
returns \EXP{()}. When called, the implicit setter rebinds the
corresponding field to its argument.
If a field declaration includes the modifier  \KWD{settable} and
\KWD{hidden}, only an abstract setter is declared.
%
If a field declaration includes the modifier \KWD{hidden} without
\KWD{settable}, it is a static error.
If an abstract field declaration includes the modifier \KWD{settable},
it is mutable.
Because the \KWD{settable} modifier implies \KWD{var}, and
the \KWD{var} modifier without \KWD{settable} only affects implementations,
abstract field declarations do not include the \KWD{var} modifier.




\paragraph{\KWD{wrapped}:}
If a field declaration of $f$ has the modifier \KWD{wrapped}
and the type of $f$ is trait type $T$, and $T$ is not a naked type variable,
then the enclosing trait $S$ implicitly includes ``forwarding methods'' for
all methods in $T$ that are also inherited from any supertrait of $S$.
Each of these methods simply calls
the corresponding method on the trait referred to by field $f$.
%
If the trait declaration enclosing $f$ explicitly declares a method $m$
that conflicts with an implicitly declared forwarding method $m'$, then the
enclosing trait contains only method $m$, not $m'$.
%
If the trait declaration enclosing $f$ inherits a concrete method $m$
that conflicts with an implicitly declared forwarding method $m'$, then the
enclosing trait contains only method $m$, not $m'$.
%
Because wrapped fields do not change declarations of methods but change
definitions of methods, they only affect implementations;
APIs do not include wrapped fields.


For example, in the following declarations:
%trait Dictionary[\T\]
%  put(T): ()
%  get(): T
%end
%
%trait WrappedDictionary[\T\] extends Dictionary[\T\]
%  wrapped val: Dictionary[\T\]
%  get(): T
%end
\begin{Fortress}
\(\KWD{trait} \TYP{Dictionary}\llbracket{}T\rrbracket\)\\
{\tt~~}\pushtabs\=\+\(  \VAR{put}(T)\COLON ()\)\\
\(  \VAR{get}()\COLON T\)\-\\\poptabs
\(\KWD{end}\)\\[4pt]
\(\KWD{trait} \TYP{WrappedDictionary}\llbracket{}T\rrbracket \KWD{extends} \TYP{Dictionary}\llbracket{}T\rrbracket\)\\
{\tt~~}\pushtabs\=\+\(  \KWD{wrapped} \VAR{val}\COLON \TYP{Dictionary}\llbracket{}T\rrbracket\)\\
\(  \VAR{get}()\COLON T\)\-\\\poptabs
\(\KWD{end}\)
\end{Fortress}
the parametric trait \TYP{WrappedDictionary} implicitly includes the
following forwarding method:
%put(x) = val.put(x)
\begin{Fortress}
\(\VAR{put}(x) = \VAR{val}.\VAR{put}(x)\)
\end{Fortress}
If \VAR{get} were not explicitly declared in \TYP{WrappedDictionary}, then
\TYP{WrappedDictionary} would also include the forwarding method:
%get() = val.get()
\begin{Fortress}
\(\VAR{get}() = \VAR{val}.\VAR{get}()\)
\end{Fortress}

\note{
Wrapped fields imply subtyping.

what methods are implicitly defined?

% trait T
%    f(x) = 17
%    abstract h(x)
% end
% object A extends T
%    f(x) = 23
%    g(x) = 5
%    h(x) = 3
% end
% object B extends T
%    wrapped a: A
% end

% f(x) = (self asif T).f(x)
% h(x) = a.h(x)

% trait U
%    abstract g(x)
% end
% object B extends {T, U}
%    wrapped a: A                  ERROR!
% end
\begin{Fortress}
{\tt~}\pushtabs\=\+\( \KWD{trait}\:T\)\\
{\tt~~~}\pushtabs\=\+\(    f(x) = 17\)\\
\(    \KWD{abstract}\:h(x)\)\-\\\poptabs
\( \KWD{end}\)\\
\( \KWD{object}\:A \KWD{extends}\:T\)\\
{\tt~~~}\pushtabs\=\+\(    f(x) = 23\)\\
\(    g(x) = 5\)\\
\(    h(x) = 3\)\-\\\poptabs
\( \KWD{end}\)\\
\( \KWD{object}\:B \KWD{extends}\:T\)\\
{\tt~~~}\pushtabs\=\+\(    \KWD{wrapped}\:a\COLON A\)\-\\\poptabs
\( \KWD{end}\)\\[4pt]
\( f(x) = (\mathord{\KWD{self}} \KWD{asif}\:T).f(x)\)\\
\( h(x) = a.h(x)\)\\[4pt]
\( \KWD{trait}\:U\)\\
{\tt~~~}\pushtabs\=\+\(    \KWD{abstract}\:g(x)\)\-\\\poptabs
\( \KWD{end}\)\\
\( \KWD{object}\:B \KWD{extends} \{T, U\}\)\\
{\tt~~~}\pushtabs\=\+\(    \KWD{wrapped}\:a\COLON A                  \OPR{ERROR}!\)\-\\\poptabs
\( \KWD{end}\)\-\\\poptabs
\end{Fortress}
}

\section{Method Contracts}
\seclabel{method-contracts}

\begin{Grammar}
\emph{Contract} &::=& \option{\emph{Requires}} \option{\emph{Ensures}}
\option{\emph{Invariant}}\\

\emph{Requires} &::=&
\KWD{requires} \{ \option{\emph{ExprList}} \} \\

\emph{Ensures}  &::=&
\KWD{ensures} \{ \option{\emph{EnsuresClauseList}} \} \\

\emph{EnsuresClauseList} &::=&
\emph{EnsuresClause}(\EXP{,} \emph{EnsuresClause})$^*$ \\

\emph{EnsuresClause} &::=& \emph{Expr} \options{\KWD{provided} \emph{Expr}} \\

\emph{Invariant}&::=&
\KWD{invariant} \{ \option{\emph{ExprList}} \} \\

\end{Grammar}

Method contracts consist of three optional clauses: a \KWD{requires}
clause, an \KWD{ensures} clause, and an \KWD{invariant} clause.
All three clauses are evaluated in the scope of the method body.
See \secref{contracts} for a discussion of each clause.


Method contracts are handled similarly to the manner described in
\cite{MethodContracts}. In particular, substitutability under subtyping is
preserved.  For a call to a method \VAR{m} with receiver \VAR{e},
we use the term \emph{static contract} of $m$ to refer to
a contract declared in the statically most
applicable method declaration provided by the static type of $e$
and the term \emph{dynamic contract} of $m$ to refer to a contract declared
in the dynamically most applicable method declaration provided by the
runtime type of $e$.
Three exceptions may be thrown due to a method contract
violation: \TYP{CallerViolation} is thrown when the \KWD{requires} clause
of the static contract fails, \TYP{CalleeViolation} is thrown when the
\KWD{ensures} or \KWD{invariant} clause of the dynamic contract fails,
and \TYP{ContractHierarchyViolation} is thrown
when the \KWD{requires} clause of the dynamic contract
or the \KWD{ensures} or \KWD{invariant} clause of the static contract fails.


Evaluation of a call to a method \VAR{m} with receiver \VAR{e}
proceeds as follows.
First, $e$ is evaluated to a value $v$ with runtime type \VAR{U}.
Let $C$ and $C'$ be the static and dynamic contracts of $m$, respectively.
If the \KWD{requires} clause of $C$ fails,
a \TYP{CallerViolation} exception is thrown.
%
Otherwise, if the \KWD{requires} clause of $C'$ fails,
a \TYP{ContractHierarchyViolation} exception is thrown.
%
Otherwise, the \KWD{provided} subclauses of $C$ and $C'$ are evaluated. For
every \KWD{provided} subclause that evaluates to \VAR{true}, the
corresponding \KWD{ensures} subclause is recorded in a table
$E$ for later comparison.  Similarly, the \KWD{invariant} clauses of $C$
and $C'$ are evaluated and the results are stored in $E$ for later
comparison.
%
Then the body of $m$ provided by \VAR{U}
is evaluated.  After evaluation of the body,
all \KWD{ensures} subclauses of the dynamic contract recorded in $E$
are checked to ensure that they evaluate to \VAR{true}, and
all \KWD{invariant} clauses of the dynamic contract recorded in $E$
are checked to ensure that they evaluate to values equal to the
values they evaluated to before evaluation of the body.
If any such check fails, a \TYP{CalleeViolation}
exception is thrown. Otherwise, all \KWD{ensures} subclauses and
\KWD{invariant} clauses of the static contract in $E$ are checked.
If any of these checks fails, a \TYP{ContractHierarchyViolation} exception
is thrown.


\section{Value Traits}
\seclabel{value-traits}

\begin{Grammar}
\emph{TraitMod} &::=& \KWD{value}\\
\end{Grammar}


If a trait declaration has the modifier \KWD{value}, all subtraits
of that trait must also have the modifier \KWD{value}, and all objects
extending that trait are required to be value objects (described in
\secref{value-objects}).
If a field declaration of a value trait has the modifier \KWD{settable},
the return type of its implicit setter method is the value trait type.
If a value trait has an explicit setter method, the setter must be an
abstract method and its return type must be the value trait type.
See \secref{value-objects} for a discussion of updating fields of value
objects.
