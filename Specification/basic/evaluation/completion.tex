%%%%%%%%%%%%%%%%%%%%%%%%%%%%%%%%%%%%%%%%%%%%%%%%%%%%%%%%%%%%%%%%%%%%%%%%%%%%%%%%
%   Copyright 2009, Oracle and/or its affiliates.
%   All rights reserved.
%
%
%   Use is subject to license terms.
%
%   This distribution may include materials developed by third parties.
%
%%%%%%%%%%%%%%%%%%%%%%%%%%%%%%%%%%%%%%%%%%%%%%%%%%%%%%%%%%%%%%%%%%%%%%%%%%%%%%%%

\section{Normal and Abrupt Completion of Evaluation}
\seclabel{eval-completion}

\note{Victor:

What happens when a thread completes abruptly?
What happens if the program completes abruptly?}

Conceptually, an expression is evaluated until it \emph{completes}.
Evaluation of an expression may \emph{complete normally}, resulting in
a value, or it may \emph{complete abruptly}.  Each abrupt completion
has an associated value, either
an exception value that is thrown and uncaught
or the \KWD{with} clause value of an \KWD{exit} expression (described in
\secref{label-expr}).
  In addition to programmer-defined exceptions
thrown explicitly by a \KWD{throw} expression (described in
\secref{throw-expr}), there are predefined exceptions thrown by
\library.  For example, dividing an integer by zero (using the \EXP{/}
operator) causes a \TYP{DivisionByZero} to be thrown.

When an expression completes abruptly, control passes to the
dynamically immediately enclosing expression.
This continues until the abrupt completion is handled
either
by a \KWD{try} expression (described in \secref{try-expr})
if an exception is being thrown
or by an appropriately tagged \KWD{label} expression
(described in \secref{label-expr})
if an \KWD{exit} expression was evaluated.
If abrupt completion is not handled within a thread
and its outermost expression completes abruptly,
the thread itself completes abruptly.
If the main thread of a program completes abruptly,
the program as a whole also completes abruptly.

\note{Jan: abrupt completion only causes the thread to complete abruptly.
A program completes abruptly if its initial thread completes
abruptly.
 Pointer to discussion of abrupt completion of threads, or pull that
 discussion here.}
