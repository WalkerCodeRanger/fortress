%%%%%%%%%%%%%%%%%%%%%%%%%%%%%%%%%%%%%%%%%%%%%%%%%%%%%%%%%%%%%%%%%%%%%%%%%%%%%%%%
%   Copyright 2009, Oracle and/or its affiliates.
%   All rights reserved.
%
%
%   Use is subject to license terms.
%
%   This distribution may include materials developed by third parties.
%
%%%%%%%%%%%%%%%%%%%%%%%%%%%%%%%%%%%%%%%%%%%%%%%%%%%%%%%%%%%%%%%%%%%%%%%%%%%%%%%%

\section{Threads and Parallelism}
\seclabel{threads-parallelism}

\note{Reduction variables and deferred exceptions are not yet supported.}

\note{Thread fairness guarantee (Jan's email titled ``Re: Clarifying fairness guarantees'' sent on 14 Mar 2007)}

\note{Fix citations:

Victor's email titled ``Re: Fortress 1.0 spec (revision 4793)'' sent on
31 Mar 2008}

There are two kinds of threads in Fortress:
\emph{implicit} threads and \emph{spawned}  (or \emph{explicit}) threads.
Spawned threads are objects created by the \KWD{spawn} construct,
described in \secref{spawn}.

A thread may be in one of five states: \emph{not started},
\emph{executing}, \emph{suspended}, \emph{normally completed} or
\emph{abruptly completed}.  We say a thread is \emph{not started}
after it is created but before it has taken a step.  This is only
important for purposes of determining whether two threads are
executing simultaneously; see \secref{threadlocal}.  A thread is
\emph{executing} or \emph{suspended} if it has taken a step, but has
not completed; see below for the distinction between executing and
suspended threads.  A thread is \emph{complete} if its expression has
completed evaluation.  If the expression completes normally, its value
is the result of the thread.

Every thread has a \emph{body} and an \emph{execution environment}.
The body is an intermediate expression, which the thread evaluates in
the context of the execution environment; both the body and the
execution environment may change when the thread takes a step.  This
environment is used to look up names in scope but bound in a block
that encloses the construct that created the thread.  The execution
environment of a newly created thread is the environment of the thread
that created the new thread.

In Fortress, a number of constructs are \emph{implicitly parallel}.
An implicitly parallel construct creates a \emph{group} of one or
more implicit threads.  The implicitly parallel constructs are:
\begin{itemize}
\item {\bf Tuple expressions:} Each element expression of the tuple
  expression is evaluated in a separate implicit thread
  (see \secref{tuple-expr}).
\item {\bf \EXP{\KWD{also}\;\KWD{do}} blocks:} each
  sub-block separated by an \KWD{also} clause is evaluated in a
  separate implicit thread (see \secref{also-block}).
\item {\bf Method invocations and function calls:}
The receiver or the function and each of the arguments is
  evaluated in a separate implicit thread
  (see \secref{dotted-method-calls}, \secref{naked-method-calls},
 and \secref{function-calls}).
\item {\bf \KWD{for} loops, comprehensions, sums, generated
    expressions, and big operators:} Parallelism in looping constructs
  is specified by the generators used (see \secref{generators}).
  Programmers must assume that generators that do not extend
  \TYP{SequentialGenerator} can execute each iteration of the body
  expression in a separate implicit thread.  The functional method
  \VAR{seq} and the equivalent function \VAR{sequential} can be used
  to turn any \TYP{Generator} into a \TYP{SequentialGenerator}.
\item {\bf Extremum expressions:} Each guarding expression of the extremum
  expression is evaluated in a separate implicit thread
  (see \secref{extremum-expr}).
\item {\bf Tests:}
Each test is evaluated in a separate implicit thread
(see \chapref{tests}).

\end{itemize}

Implicit threads run fork-join style:
all threads in a group are created together,
and they all must complete before the group as a whole completes.
There is no way for a programmer to single out an implicit
thread and operate upon it in any way; they are managed purely by the
compiler, runtime, and libraries of the Fortress implementation.
Implicit threads need not be scheduled fairly; indeed, a Fortress
implementation may choose to serialize portions of any group of
implicit threads, interleaving their operations in any way it likes.
For example, the following code fragment may loop forever:
\note{The TYP{Thread} example is commented out.}
%% \input{\home/basic/examples/Eval.Thread.tex}

If any implicit thread in a group completes abruptly, the group as a
whole will complete abruptly as well.  Every thread in the group will
either not run at all, complete (normally or abruptly), or may be
terminated early as described in \secref{early-termination}.
The result of the abrupt completion of the group is
the result of one of the constituent threads that completes abruptly.
After abrupt completion of a group of implicit threads, each reduction
variable (see \secref{reduction-vars})
may reflect an arbitrary subset of updates performed by the
threads in the group.  This means in general that reduction variables
should not be accessed after abrupt completion.  The exact behavior of reduction variables depends on the supporting generator structure and is described in \secref{defining-generators}.

Spawned thread objects are reference objects of type
\EXP{\TYP{Thread}\llbracket{}T\rrbracket}, where
\VAR{T} is the static type of the expression spawned;
this trait has the following methods, each taking no arguments:
\begin{itemize}
\item The \VAR{val} method returns the value computed by the
  subexpression of the \KWD{spawn} expression.  If the thread has not
  yet completed execution, the invocation of \VAR{val} blocks until it
 has done so.
 A spawned thread is not permitted to exit (discussed
 in \secref{label-expr}) to a surrounding label (the label is
 considered to be out of scope), but may fail to catch an exception
 and thus complete abruptly.
 When this happens, the exception is
 \emph{deferred}.  Any invocation of the \VAR{val} method on the
 spawned thread throws the deferred exception.  If the spawned thread
 object is discarded, the exception will be silently ignored.  In
 particular, if the result of a \KWD{spawn} expression is dropped, it
 is impossible to detect completion of the spawned thread or to
 recover from any deferred exceptions.
\item The \VAR{wait} method waits for a thread to complete,
but does not return a value and will not throw a deferred exception.
\item The \VAR{ready} method returns \VAR{true} if a thread
has completed, and returns \VAR{false} otherwise.
\item The \VAR{stop} method attempts to terminate a thread
as described in \secref{early-termination}.
\end{itemize}

We say a spawned thread has been \emph{observed to complete} after
invoking the \VAR{val} or \VAR{wait} methods, or when an invocation of
the \VAR{ready} method returns \VAR{true}.

In the absence of sufficient parallel resources, an attempt is made to
run the subexpression of the \KWD{spawn} expression before continuing
succeeding evaluation.  We can imagine that
it is actually the \emph{rest} of
the evaluation \emph{after} the parallel block which is spawned off
in parallel.  This is a subtle technical point, but makes the
sequential execution of parallel code simpler to understand, and
avoids subtle problems with the asymptotic stack usage of parallel
code~\cite{lazyTask,LazyThreads}.

\label{abort-thread}
There are three
ways in which a thread can be suspended.  First, a
thread that begins evaluating an implicitly parallel construct,
creating a thread group, is suspended until that thread group has
completed.  Second, a thread that invokes \VAR{val} or \VAR{wait} is
suspended until the spawned thread completes.
Finally, invoking the
\VAR{abort} function from within an \KWD{atomic} expression may cause
a thread to suspend; see \secref{transactions}.

Threads in a Fortress program can perform operations simultaneously on
shared objects.  In order to synchronize data accesses, Fortress
provides \KWD{atomic} expressions (see \secref{atomic}).
\chapref{memory-model} describes the memory model which is obeyed by
Fortress programs.

%%%%%%%%%%%%%%%%%%%%%%%%%%%%%%%%%%%%%%%%%%%%%%%%%%%%%%%%%%%%%%%%%%%%%%%%%%%%%%%%
%   Copyright 2009, Oracle and/or its affiliates.
%   All rights reserved.
%
%
%   Use is subject to license terms.
%
%   This distribution may include materials developed by third parties.
%
%%%%%%%%%%%%%%%%%%%%%%%%%%%%%%%%%%%%%%%%%%%%%%%%%%%%%%%%%%%%%%%%%%%%%%%%%%%%%%%%

\subsection{Reduction Variables}
\seclabel{reduction-vars}

\note{Reduction variables are not yet supported.}

To perform computations as locally as possible, and avoid the need to
synchronize in the middle of relatively simple \KWD{for} loops,
Fortress gives special treatment to \emph{reductions}.  We say that an
operator \EXP{\odot} is a \emph{reduction operator} for type $T$ if
$T$ is a subtype of
%Monoid[\T, ODOT\]
\(\TYP{Monoid}\llbracket{}T, \odot\rrbracket\), which implies that
\EXP{\odot} is an associative binary infix operator on $T$ (see
\secref{monoids-groups-rings-fields} for details about the
\TYP{Monoid} trait).

We say that a variable \VAR{l} is a \emph{reduction variable} reduced using
the reduction operator $\odot$ for a particular thread group if it
satisfies the following conditions:
\begin{itemize}
\item Every assignment to \VAR{l} within the thread group is of the
  form \EXP{l \mathrel{\oplus}= e}, where exactly one operator $\oplus$ or its
  group inverse $\ominus$ (see below) is used in these assignments.
\item The value of \VAR{l} is not otherwise read within the thread group.
\item The variable \VAR{l} is not a free variable of a functional.
  This includes the fields of the receiver object in a method
  definition.
\end{itemize}
Other threads which simultaneously reference a reduction variable
while a loop is running may see an arbitrary value for that variable.
Any updates performed by those threads may be lost.  The association
of terms in the reduction is arbitrary and guided by the loop
generators (\secref{defining-generators}).  The order of terms in the
reduction is the natural order of elements in the generator.  When the
operator is commutative, the order of terms is also arbitrary.

Several common mathematical operators are declared to be monoids in
\library.  These include \EXP{+}, $\cdot$, \EXP{\wedge}, \EXP{\vee},
and \EXP{\xor}.  If a type $T$ extends
%Group[\T, OPLUS, OMINUS\]
\EXP{\TYP{Group}\llbracket{}T, \oplus, \ominus\rrbracket}
(see \secref{monoids-groups-rings-fields} for details about the
\TYP{Group} trait)
then reduction expressions of the following form are also permitted:
%x OMINUS= y
\begin{Fortress}
\(x \mathrel{\ominus}= y\)
\end{Fortress}
Such expressions may be transformed into an algebraic equivalent such
as the following:
%x OPLUS= Identity[\OPLUS\] OMINUS y
\begin{Fortress}
\(x \mathrel{\oplus}= \TYP{Identity}\llbracket\oplus\rrbracket \ominus y\)
\end{Fortress}

The semantics of reductions enables implementation strategies such as
the one used in OpenMP~\cite{OpenMP}: A reduction variable \VAR{l} is
assigned \EXP{\TYP{Identity}\llbracket\oplus\rrbracket}
at the beginning of each iteration.  The
original variable value may be read ahead of time, resulting in the
loss of parallel updates to the variable which occur in other threads
while the loop is running.  When all iterations are complete, the
initial value of the reduction variable and values of the variable at
the end of each implicit thread are reduced and the result is assigned
to the reduction variable as the group completes.

In the following example, \VAR{sum} is a reduction variable:
\note{I manually rendered the array type.}
%arraySum[\nat x\](a:ZZ64[x]):ZZ64 = do
%    sum:ZZ64 := 0
%    for i<-a.indices do
%        sum+=a[i]
%    end
%    sum
%end
\begin{Fortress}
\(\VAR{arraySum}\llbracket\KWD{nat}\mskip 4mu plus 4mu{x}\rrbracket(a\COLONOP\mathbb{Z}64[x])\COLONOP\mathbb{Z}64 = \;\KWD{do}\)\\
{\tt~~~~}\pushtabs\=\+\(    \VAR{sum}\COLONOP\mathbb{Z}64 \ASSIGN 0\)\\
\(    \KWD{for}  \null\)\pushtabs\=\+\(i\leftarrow{}a.\VAR{indices} \KWD{do}\)\\
\(        \VAR{sum}\mathrel{+}=a[i]\)\-\\\poptabs
\(    \KWD{end}\)\\
\(    \VAR{sum}\)\-\\\poptabs
\(\KWD{end}\)
\end{Fortress}

The following demonstrates the use of a reduction variable \VAR{prod}
in a parallel \KWD{do}:
\note{The $\mathrel{\cdot}$ is manually edited.}
%prod : ZZ64 := 1
%do
%    prod DOT= g(y)
%also do
%    prod DOT= f(x)
%    prod DOT= g(x)
%also do
%    h(x,y)
%end
\begin{Fortress}
\(\VAR{prod} \mathrel{\mathtt{:}}\mskip 4mu plus 4mu\mathbb{Z}64 \ASSIGN 1\)\\
\(\KWD{do}\)\\
{\tt~~~~}\pushtabs\=\+\(    \VAR{prod} \mathrel{\cdot}= g(y)\)\-\\\poptabs
\(\KWD{also}\;\;\KWD{do}\)\\
{\tt~~~~}\pushtabs\=\+\(    \VAR{prod} \mathrel{\cdot}= f(x)\)\\
\(    \VAR{prod} \mathrel{\cdot}= g(x)\)\-\\\poptabs
\(\KWD{also}\;\;\KWD{do}\)\\
{\tt~~~~}\pushtabs\=\+\(    h(x,y)\)\-\\\poptabs
\(\KWD{end}\)
\end{Fortress}

The operator associated with a reduction variable can be invoked in
several places during evaluation of a thread group.  Most obviously,
it can be invoked anywhere the reduction operation occurs textually in
the code for that group.  The operator can also be invoked multiple
times implicitly and in parallel after the completion of some or all
of the threads in the group.  The group as a whole does not complete
normally until all these operator invocations have completed normally
and the reduced result has been assigned to the reduction variable for
use in subsequent computation.  If any operator invocation completes
abruptly, it is treated just as if an ordinary implicit thread in the
group completed abruptly.


%%%%%%%%%%%%%%%%%%%%%%%%%%%%%%%%%%%%%%%%%%%%%%%%%%%%%%%%%%%%%%%%%%%%%%%%%%%%%%%%
%   Copyright 2009, Oracle and/or its affiliates.
%   All rights reserved.
%
%
%   Use is subject to license terms.
%
%   This distribution may include materials developed by third parties.
%
%%%%%%%%%%%%%%%%%%%%%%%%%%%%%%%%%%%%%%%%%%%%%%%%%%%%%%%%%%%%%%%%%%%%%%%%%%%%%%%%

\note{This sticks out like a sore thumb in this chapter.  Can we find a
 happier home for it?  It seems like it really belongs in the
 introductory material.}

Different implicit threads (even different iterations of the same loop
body) may execute in very different amounts of time.  A naively
parallelized loop will cause processors to idle until every iteration
finishes.  The simplest way to mitigate this delay is to expose
substantially more parallel work than the number of underlying
processors available to run them.  Load balancing can move the
resulting (smaller) units of work onto idle processors to balance
load.

The ratio between available work and number of threads is called
\emph{parallel slack}~\cite{blumofe,CilkSched}.
With support for very lightweight
threading and load balancing, 
slack in hundreds or thousands (or more) proves beneficial; 
very slack computations easily adapt to differences in the
number of available processors.  Slack is a desirable property, and
the Fortress programmer should endeavor to expose parallelism where
possible.

