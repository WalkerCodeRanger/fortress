%%%%%%%%%%%%%%%%%%%%%%%%%%%%%%%%%%%%%%%%%%%%%%%%%%%%%%%%%%%%%%%%%%%%%%%%%%%%%%%%
%   Copyright 2009, Oracle and/or its affiliates.
%   All rights reserved.
%
%
%   Use is subject to license terms.
%
%   This distribution may include materials developed by third parties.
%
%%%%%%%%%%%%%%%%%%%%%%%%%%%%%%%%%%%%%%%%%%%%%%%%%%%%%%%%%%%%%%%%%%%%%%%%%%%%%%%%

\chapter{Variables}
\chaplabel{variables}

\note{Functionals with the modifier \KWD{io} and
matrix unpasting are not yet supported.}

\section{Top-Level Variable Declarations}
\seclabel{top-var-decls}
\begin{Grammar}
\emph{VarDecl}
&::=& \option{\emph{VarMods}} \emph{VarWTypes} \emph{InitVal} \\
&$|$& \option{\emph{VarImmutableMods}} \emph{BindIdOrBindIdTuple} \EXP{=} \emph{Expr}\\
&$|$& \option{\emph{VarMods}} \emph{BindIdOrBindIdTuple} \EXP{\mathrel{\mathtt{:}}} \emph{Type}\EXP{...}
\emph{InitVal} \\
&$|$& \option{\emph{VarMods}} \emph{BindIdOrBindIdTuple} \EXP{\mathrel{\mathtt{:}}} \emph{TupleType}
\emph{InitVal} \\

\emph{VarMods} &::=& \emph{VarMod}$^+$\\

\emph{VarImmutableMods} &::=& \emph{VarImmutableMod}$^+$\\

\emph{VarMod} &::=& \emph{AbsVarMod} $|$ \KWD{private}\\

\emph{VarImmutableMod} &::=& \emph{AbsVarImmutableMod} $|$ \KWD{private}\\

\emph{AbsVarMod} &::=& \KWD{var} $|$ \KWD{test}\\

\emph{AbsVarImmutableMod} &::=& \KWD{test}\\

\emph{VarWTypes} &::=& \emph{VarWType} \\
&$|$& \texttt{(} \emph{VarWType}(\EXP{,} \emph{VarWType})$^+$ \texttt{)}\\

\emph{VarWType} &::=& \emph{BindId} \emph{IsType}\\

\emph{InitVal} &::=& (\EXP{=}$|$\EXP{\ASSIGN}) \emph{Expr} \\

\emph{TupleType} &::=&
\texttt{(} \emph{Type}\EXP{,} \emph{TypeList} \texttt{)}\\

\emph{TypeList} &::=& \emph{Type}(\EXP{,} \emph{Type})$^*$ \\

\emph{IsType} &::=& \EXP{\mathrel{\mathtt{:}}} \emph{Type}\\
\end{Grammar}

A \emph{variable}'s name can be any valid Fortress identifier.
There are three forms of variable declarations.  The first form:
%id : Type = expr
\begin{Fortress}
\(\VAR{id} \mathrel{\mathtt{:}} \TYP{Type} = \VAR{expr}\)
\end{Fortress}
declares \VAR{id} to be an immutable variable with static type \TYP{Type}
whose value is computed to be the value of the \emph{initializer} expression
\VAR{expr}.
The static type of \VAR{expr} must be a subtype of \TYP{Type}.


The second (and most convenient) form:
%id = expr
\begin{Fortress}
\(\VAR{id} = \VAR{expr}\)
\end{Fortress}
declares \VAR{id} to be an immutable variable
whose value is computed to be the value of the expression \VAR{expr};
the static type of the variable is the static type of \VAR{expr}.

The third form:
%var id : Type = expr
\begin{Fortress}
\(\KWD{var} \VAR{id} \mathrel{\mathtt{:}} \TYP{Type} = \VAR{expr}\)
\end{Fortress}
declares \VAR{id} to be a mutable variable of type \TYP{Type}
whose initial value is computed to be the value of the expression \VAR{expr}.
As before, the static type of \VAR{expr} must be a subtype of \TYP{Type}.
The modifier \KWD{var} is optional
when `\EXP{\ASSIGN}' is used instead of `\EXP{=}' as follows:

%var? id : Type := expr
\begin{Fortress}
\(\option{\KWD{var}}\ \VAR{id} \mathrel{\mathtt{:}} \TYP{Type} \ASSIGN \VAR{expr}\)
\end{Fortress}

In short, immutable variables are declared and initialized by `\EXP{=}'
and mutable variables are declared and initialized either by `\EXP{\ASSIGN}'
or by declaring them according to the third form above with the modifier
\KWD{var}.


All forms can be used with \emph{tuple notation} to declare multiple
variables together.  Variables to declare
are enclosed in parentheses and separated by commas, as are the
types declared for them:
%(id(, id)^+) : (Type(, Type)^+)
\begin{Fortress}
\((\VAR{id}(, \VAR{id})^+) \mathrel{\mathtt{:}} (\TYP{Type}(, \TYP{Type})^+)\)
\end{Fortress}
Alternatively, the types can be included alongside the respective variables,
optionally eliding types that can be inferred from context:
%(id(: Type)?(, id(: Type)?)^+)
\begin{Fortress}
\((\VAR{id}\options{\COLON \TYP{Type}}(, \VAR{id}\options{\COLON \TYP{Type}})^+)\)
\end{Fortress}

Alternatively, a single type followed by `\EXP{\ldots}'
can be declared for all of the variables:
%(id(, id)^+): Type...
\begin{Fortress}
\((\VAR{id}(, \VAR{id})^+)\COLON \TYP{Type}\ldots\)
\end{Fortress}
This notation is especially helpful when a function application returns a tuple of values.

The initializer expressions of top-level variable declarations can refer to
variables declared later in textual order.
\label{decl-io}
Evaluation of top-level initializer expressions cannot call
functionals with the modifier \KWD{io}.


Here are some simple examples of variable declarations:
\input{\home/basic/examples/Var.Top.a.tex}
declares the variable \EXP{\pi}
to be an approximate representation of the mathematical object \EXP{\pi}.
It is also legal to write:
\input{\home/basic/examples/Var.Top.b.tex}
This definition enforces that \EXP{\pi} has static type \EXP{\mathbb{R}64}.

The following example declares multiple variables using tuple notation:
\input{\home/basic/examples/Var.Top.c.tex}

The following three declarations are equivalent:
\input{\home/basic/examples/Var.Top.d.tex}
\vspace*{-1em}
\input{\home/basic/examples/Var.Top.e.tex}
\vspace*{-1.4em}
\input{\home/basic/examples/Var.Top.f.tex}


\section{Local Variable Declarations}
\seclabel{local-var-decls}

\note{Delayed initialization of local variable declarations:

We want to disallow type annotation for delayed initialization of local variable declarations:
% do
%   i: ZZ32
%   _ = if b
%       then i: ZZ32 = 1
%       else i: ZZ32 = 0
%       end
%   println i
% end
\begin{Fortress}
{\tt~}\pushtabs\=\+\( \KWD{do}\)\\
{\tt~~}\pushtabs\=\+\(   i\COLON \mathbb{Z}32\)\\
\(   {\tt\_} =  \null\)\pushtabs\=\+\(\KWD{if}\:b\)\\
\(       \KWD{then}\:i\COLON \mathbb{Z}32 = 1\)\\
\(       \KWD{else}\:i\COLON \mathbb{Z}32 = 0\)\\
\(       \KWD{end}\)\-\\\poptabs
\(   \VAR{println}\:i\)\-\\\poptabs
\( \KWD{end}\)\-\\\poptabs
\end{Fortress}
}

\begin{Grammar}
\emph{LocalVarDecl}
&::=& \option{\KWD{var}} \emph{LocalVarWTypes} \emph{InitVal} \\
&$|$& \option{\KWD{var}} \emph{LocalVarWTypes}\\
&$|$& \emph{LocalVarWoTypes} \EXP{=} \emph{Expr} \\
&$|$& \option{\KWD{var}} \emph{LocalVarWoTypes} \EXP{\mathrel{\mathtt{:}}} \emph{Type}\EXP{...}
\option{\emph{InitVal}} \\
&$|$& \option{\KWD{var}} \emph{LocalVarWoTypes} \EXP{\mathrel{\mathtt{:}}} \emph{TupleType}
\option{\emph{InitVal}} \\

\emph{LocalVarWTypes} &::=& \emph{LocalVarWType} \\
&$|$& \texttt{(} \emph{LocalVarWType}(\EXP{,} \emph{LocalVarWType})$^+$ \texttt{)}\\
\emph{LocalVarWType} &::=& \emph{BindId} \emph{IsType}\\

\emph{LocalVarWoTypes} &::=& \emph{LocalVarWoType} \\
&$|$& \texttt{(} \emph{LocalVarWoType}(\EXP{,} \emph{LocalVarWoType})$^+$ \texttt{)}\\
\emph{LocalVarWoType} &::=& \emph{BindId}\\
&$|$& \emph{Unpasting} \\

\end{Grammar}

Variables can be declared within expression blocks (described in
\secref{block-expr}) via the same syntax as is used for top-level
variable declarations (described in \secref{top-var-decls}) except that
local variables must not include modifiers besides \KWD{var} and
additional syntax is allowed as follows:
\begin{itemize}
\item
The form:
%var? id : Type
\begin{Fortress}
\(\option{\KWD{var}}\ \VAR{id} \mathrel{\mathtt{:}} \TYP{Type}\)
\end{Fortress}
declares a variable without giving it an initial value
(where mutability is determined by the presence of the \KWD{var} modifier).
It is a static error if such a variable is referred to before it has been
given a value; an immutable variable is initialized by another variable
declaration and a mutable variable is initialized by assignment.
It is also a static error if an immutable variable is initialized more than once.
Whenever a variable bound in this manner is assigned a value, the type of
that value must be a subtype of its declared type.
This form allows declaration of the types of variables to be separated
from definitions, and it allows programmers to delay assigning to a variable
before a sensible value is known.
In the following example, the declaration of the type of a variable and its
definition are separated:
\input{\home/basic/examples/Var.Local.tex}

\item Special syntax for declaring local variables as parts of a
matrix is provided as described in \secref{unpasting}.
\end{itemize}

%%%%%%%%%%%%%%%%%%%%%%%%%%%%%%%%%%%%%%%%%%%%%%%%%%%%%%%%%%%%%%%%%%%%%%%%%%%%%%%%
%   Copyright 2009 Sun Microsystems, Inc.,
%   4150 Network Circle, Santa Clara, California 95054, U.S.A.
%   All rights reserved.
%
%   U.S. Government Rights - Commercial software.
%   Government users are subject to the Sun Microsystems, Inc. standard
%   license agreement and applicable provisions of the FAR and its supplements.
%
%   Use is subject to license terms.
%
%   This distribution may include materials developed by third parties.
%
%   Sun, Sun Microsystems, the Sun logo and Java are trademarks or registered
%   trademarks of Sun Microsystems, Inc. in the U.S. and other countries.
%%%%%%%%%%%%%%%%%%%%%%%%%%%%%%%%%%%%%%%%%%%%%%%%%%%%%%%%%%%%%%%%%%%%%%%%%%%%%%%%

\section{Matrix Unpasting}
\seclabel{unpasting}

\note{Matrix unpasting is not yet supported.}

\note{We should add a static check for an unpasting to avoid runtime unpasting exceptions.}

\begin{Grammar}
\emph{Unpasting} &::=& \texttt{[} \emph{UnpastingElems} \texttt{]} \\

\emph{UnpastingElems}
&::=& \emph{UnpastingElem} \emph{RectSeparator} \emph{UnpastingElems} \\
&$|$& \emph{UnpastingElem} \\

\emph{UnpastingElem}
&::=& \emph{BindId} \options{\texttt{[} \emph{UnpastingDim} \texttt{]}} \\
&$|$& \emph{Unpasting} \\

\emph{UnpastingDim} &::=& \emph{ExtentRange} (\BY\ \emph{ExtentRange})$^+$ \\

\emph{ExtentRange}
&::=& \option{\emph{StaticArg}}\EXP{\mathinner{\hbox{\tt\char'43}}}
\option{\emph{StaticArg}}\\
&$|$& \option{\emph{StaticArg}}\KWD:\option{\emph{StaticArg}}\\
&$|$& \emph{StaticArg} \\

\emph{RectSeparator} &::=& \EXP{;}+\\
&$|$& \emph{Whitespace}\\
\end{Grammar}

Matrix unpasting is an extension of local variable declaration syntax as a
shorthand for breaking a matrix into parts.  On the left-hand side of
a declaration, what looks like a matrix pasting of unbound variables
is actually a declaration of several new variables.
This syntax
serves to break the right-hand side into pieces and bind the pieces to
the variables.  Matrix unpastings are concise, eliminate several
opportunities for fencepost errors, guarantee unaliased parts, and
avoid overspecification of how the matrix should be taken apart.

The motivating example for matrix unpasting is cache-oblivious
matrix multiplication.  The general plan in a cache oblivious
algorithm is to break the input apart on its largest dimension,
and recursively attack the resulting smaller and more compact
problems.

\note{
Several bugs in the emacs tool for the code in this section.
 - superscripts were too complicated for the tool
 - left-hand bracket for the array was wrong
}
%% mm[\nat m, nat n, nat p\](left  :RR^{m \times n}, right :RR^{n \times p}, result:RR^{m \times p}):() =
%%   case largest of
%%     1 \Rightarrow result[0,0] += (left[0,0] right[0,0])
%%     m \Rightarrow [ lefttop
%%                     leftbottom   ] = left
%%                   [ resulttop
%%                     resultbottom ] = result
%%         t1 = spawn mm(lefttop, right, resulttop)
%%         mm(leftbottom, right, resultbottom)
%%         t1.wait()
%%     p \Rightarrow [ rightleft  rightright  ] = right
%%                   [ resultleft resultright ] = result
%%         t1 = spawn mm(left, rightleft, resultleft)
%%         mm(left, rightright, resultright)
%%         t1.wait()
%%     n \Rightarrow [ leftleft leftright ] = left
%%          [ righttop
%%            rightbottom ] = right
%%          mm(leftleft , righttop   , result)
%%          mm(leftright, rightbottom, result)
%%   end
\begin{Fortress}
\(\VAR{mm}\llbracket\KWD{nat} m, \KWD{nat} n, \KWD{nat} p\rrbracket(\VAR{left}  \mathrel{\mathtt{:}}\mathbb{R}^{m \times n}, \VAR{right} \mathrel{\mathtt{:}}\mathbb{R}^{n \times p}, \VAR{result}\COLONOP\mathbb{R}^{m \times p})\COLONOP() =\)\\
{\tt~~}\pushtabs\=\+\(  \KWD{case}\;\;\KWD{largest}\;\;\KWD{of}\)\\
{\tt~~}\pushtabs\=\+\(    1 \Rightarrow \VAR{result}_{0,0} \mathrel{+}= (\VAR{left}_{0,0} \VAR{right}_{0,0})\)\\
\(    m \Rightarrow \null\)\pushtabs\=\+\([\,\null\)\pushtabs\=\+\(\VAR{lefttop}\)\\
\(                    \VAR{leftbottom}  \,] = \VAR{left}\)\-\\\poptabs
\(                  [\,\null\)\pushtabs\=\+\(\VAR{resulttop} \)\\
\(                    \VAR{resultbottom}\,] = \VAR{result}\)\-\-\\\poptabs\poptabs
{\tt~~~~}\pushtabs\=\+\(        t_{1} = \;\KWD{spawn} \VAR{mm}(\VAR{lefttop}, \VAR{right}, \VAR{resulttop})\)\\
\(        \VAR{mm}(\VAR{leftbottom}, \VAR{right}, \VAR{resultbottom})\)\\
\(        t_{1}.\VAR{wait}()\)\-\\\poptabs
\(    p \Rightarrow \null\)\pushtabs\=\+\([\,\VAR{rightleft}\mskip 4mu plus 4mu\VAR{rightright} \,] = \VAR{right}\)\\
\(                  [\,\VAR{resultleft}\mskip 4mu plus 4mu\VAR{resultright}\,] = \VAR{result}\)\-\\\poptabs
{\tt~~~}\pushtabs\=\+\(        t_{1} = \;\KWD{spawn} \VAR{mm}(\VAR{left}, \VAR{rightleft}, \VAR{resultleft})\)\\
\(        \VAR{mm}(\VAR{left}, \VAR{rightright}, \VAR{resultright})\)\\
\(        t_{1}.\VAR{wait}()\)\-\\\poptabs
\(    n \Rightarrow [\,\VAR{leftleft}\mskip 4mu plus 4mu\VAR{leftright}\,] = \VAR{left}\)\\
{\tt\;~~~}\pushtabs\=\+\(         [\,\null\)\pushtabs\=\+\(\VAR{righttop}\)\\
\(           \VAR{rightbottom}\,] = \VAR{right}\)\-\\\poptabs
\(         \VAR{mm}(\VAR{leftleft} , \VAR{righttop}   , \VAR{result})\)\\
\(         \VAR{mm}(\VAR{leftright}, \VAR{rightbottom}, \VAR{result})\)\-\-\\\poptabs\poptabs
\(  \KWD{end}\)\-\\\poptabs
\end{Fortress}

In unpasting, the element syntax is slightly enhanced both to
permit some specification of the split location and to
receive information about the split that was performed.  For
example, perhaps only the upper left square of a matrix is
interesting.  The programmer can annotate bounds to the
square unpasted element:
\note{
I manually edited superscripts, subscripts, and the left square bracket.}
%% foo[\nat m, nat n\](A:RR^{m \times n}):() =
%%   if   m < n then
%%        [ squareShape_{m \times m} rest ] = A
%%        ...
%%   elif m > n then
%%        [ squareShape_{n \times n}
%%          rest                     ] = A
%%        ...
%%   else (* A already square *)
%%        squareShape = A
%%        ...
%%   end
\begin{Fortress}
\(\VAR{foo}\llbracket\KWD{nat} m, \KWD{nat} n\rrbracket(A\COLONOP\mathbb{R}^{m \times n})\COLONOP() =\)\\
{\tt~~}\pushtabs\=\+\(  \KWD{if}   \null\)\pushtabs\=\+\(m < n \KWD{then}\)\\
\(       [\,\VAR{squareShape}_{m \times m} \VAR{rest}\,] = A\)\\
\(       \ldots\)\-\\\poptabs
\(  \KWD{elif} \null\)\pushtabs\=\+\(m > n \KWD{then}\)\\
\(       [\,\null\)\pushtabs\=\+\(\VAR{squareShape}_{n \times n}\)\\
\(         \VAR{rest}                    \,] = A\)\-\\\poptabs
\(       \ldots\)\-\\\poptabs
\(  \KWD{else} \null\)\pushtabs\=\+\(\mathtt{(*}\;\hbox{\rm  A already square \unskip}\;\mathtt{*)}\)\\
\(       \VAR{squareShape} = A\)\\
\(       \ldots\)\-\\\poptabs
\(  \KWD{end}\)\-\\\poptabs
\end{Fortress}
The types of the elements of the newly declared matrix variables on the left-hand side
of an unpasting are inferred (trivially) to be the type of the elements on the right-hand side.

If an unpasting into explicitly sized pieces does not
exactly cover the right-hand-side matrix, an \TYP{UnpastingError}
is thrown.

Each element of the left-hand-side of unpasting
includes an optional \emph{extent specification}.
An extent specification
\EXP{\VAR{low}\mathinner{\hbox{\tt\char'43}}\VAR{num}}
describes the indexing and the size of the given part of the matrix.
The lower extent must be bound, either before
the unpasting, or earlier (left-or-above) in the unpasting.
For example, suppose that an algorithm
chooses to break a matrix into 4 pieces,
but retain the original indices for each piece:
%% bar[\nat p, nat q\](X:RR^{r0#p \times c0#q}):() = do
%%    [ A_{r0#m~~~~~\times c0#n} B_{r0#m~~~~~\times c0+n#q-n}
%%      C_{r0+m#p-m \times c0#n} D_{r0+m#p-m \times c0+n#q-n} ] = X
%%    ...
%% end
\begin{Fortress}
\(\VAR{bar}\llbracket\KWD{nat} p, \KWD{nat} q\rrbracket(X\COLONOP\mathbb{R}^{r_0\mathinner{\hbox{\tt\char'43}}p \times c_{0}\mathinner{\hbox{\tt\char'43}}q})\COLONOP() = \;\KWD{do}\)\\
{\tt~~~}\pushtabs\=\+\(
[\,\null\)\pushtabs\=\+\(\mathrm{A}_{r_0\mathinner{\hbox{\tt\char'43}}m~~~~~~~~~\times
         c_{0}\mathinner{\hbox{\tt\char'43}}n}~~\mathrm{B}_{r_0\mathinner{\hbox{\tt\char'43}}m~~~~~~~~~~\times c_{0}+n\mathinner{\hbox{\tt\char'43}}q-n}\)\\
\(     \mathrm{C}_{r_0+m\mathinner{\hbox{\tt\char'43}}p-m \times c_{0}\mathinner{\hbox{\tt\char'43}}n}~~\mathrm{D}_{r_0+m\mathinner{\hbox{\tt\char'43}}p-m \times c_{0}+n\mathinner{\hbox{\tt\char'43}}q-n}\,] = X\)\-\\\poptabs
\(   \ldots\)\-\\\poptabs
\(\KWD{end}\)
\end{Fortress}

Unpasting does not directly support non-uniform
decomposition, and does not provide any sort of constraint
satisfaction between the extents of the parts.  For example, the following
decomposition is not legal because it constrains the split sizes to be
equal with respect to unbound \KWD{nat} parameters:

%% (* Not allowed! *)
%% fubar[\nat m, nat n\](X:RR^{m \times n}):() = do
%%   (* p and q unbound *)
%%   [ A_{p \times q} B_{p \times q}
%%     C_{p \times q} D_{p \times q} ] = X
%%   ...
%% end
\begin{Fortress}
\(\mathtt{(*}\;\hbox{\rm  Not allowed! \unskip}\;\mathtt{*)}\)\\
\(\VAR{fubar}\llbracket\KWD{nat} m, \KWD{nat} n\rrbracket(X\COLONOP\mathbb{R}^{m \times n})\COLONOP() = \;\KWD{do}\)\\
{\tt~~}\pushtabs\=\+\(  \mathtt{(*}\;\hbox{\rm  p and q unbound \unskip}\;\mathtt{*)}\)\\
\(  [\,\null\)\pushtabs\=\+\(\mathrm{A}_{p \times q}~~ \mathrm{B}_{p \times q}\)\\
\(    \mathrm{C}_{p \times q}~~ \mathrm{D}_{p \times q}\,] = X\)\-\\\poptabs
\(  \ldots\)\-\\\poptabs
\(\KWD{end}\)
\end{Fortress}
To get this effect, the programmer should compute the constrained values:
%% fubar[\nat m, nat n\](X:RR^{2m \times 2n}):() = do
%%   [ A_{m \times n} B_{m \times n}
%%     C_{m \times n} D_{m \times n} ] = X
%%   ...
%% end
\begin{Fortress}
\(\VAR{fubar}\llbracket\KWD{nat} m, \KWD{nat} n\rrbracket(X\COLONOP\mathbb{R}^{2m \times 2n})\COLONOP() = \;\KWD{do}\)\\
{\tt~~}\pushtabs\=\+\(  [\,\null\)\pushtabs\=\+\(\mathrm{A}_{m \times n}~~ \mathrm{B}_{m \times n}\)\\
\(    \mathrm{C}_{m \times n}~~ \mathrm{D}_{m \times n}\,] = X\)\-\\\poptabs
\(  \ldots\)\-\\\poptabs
\(\KWD{end}\)
\end{Fortress}

Some non-uniform unpastings can be obtained with composition,
which can be expressed either by repeated unpasting:
%% unequalRows[\nat m, nat n\](X:R^{4m \times 2n}) = do
%%     [ c1_{4m \times n} \  c2_{4m \times n} ] = X
%%     [ A_{m \times n}
%%       C_{3m \times n} ] = c1
%%     [ B_{3m \times n}
%%       D_{m \times n} ] = c2
%%     ...
%% end
\begin{Fortress}
\(\VAR{unequalRows}\llbracket\KWD{nat} m, \KWD{nat} n\rrbracket(X\COLONOP{}R^{4m \times 2n}) = \;\KWD{do}\)\\
{\tt~~~~}\pushtabs\=\+\(    [\,c1_{4m \times n} \  c2_{4m \times n}\,] = X\)\\
\(    [\,\null\)\pushtabs\=\+\(\mathrm{A}_{m \times n}\)\\
\(      \mathrm{C}_{3m \times n}\,] = c1\)\-\\\poptabs
\(    [\,\null\)\pushtabs\=\+\(\mathrm{B}_{3m \times n}\)\\
\(      \mathrm{D}_{m \times n}\,] = c2\)\-\\\poptabs
\(    \ldots\)\-\\\poptabs
\(\KWD{end}\)
\end{Fortress}
or simply by nesting matrices in the unpasting:
%% unequalRows[\nat m, nat n\](X:R^{2m \times 4n}) = do
%%   [ [ A_{m \times n}  B_{m \times 3n}]
%%     [ C_{m \times 3n} D_{m \times n} ]  ] = X
%%   ...
%% end
\begin{Fortress}
\(\VAR{unequalRows}\llbracket\KWD{nat} m, \KWD{nat} n\rrbracket(X\COLONOP{}R^{2m \times 4n}) = \;\KWD{do}\)\\
{\tt~~}\pushtabs\=\+\(  [\,\null\)\pushtabs\=\+\([\,\mathrm{A}_{m \times n}  \mathrm{B}_{m \times 3n}]\)\\
\(    [\,\mathrm{C}_{m \times 3n} \mathrm{D}_{m \times n}\,] \,] = X\)\-\\\poptabs
\(  \ldots\)\-\\\poptabs
\(\KWD{end}\)
\end{Fortress}

