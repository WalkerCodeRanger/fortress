%%%%%%%%%%%%%%%%%%%%%%%%%%%%%%%%%%%%%%%%%%%%%%%%%%%%%%%%%%%%%%%%%%%%%%%%%%%%%%%%
%   Copyright 2009, Oracle and/or its affiliates.
%   All rights reserved.
%
%
%   Use is subject to license terms.
%
%   This distribution may include materials developed by third parties.
%
%%%%%%%%%%%%%%%%%%%%%%%%%%%%%%%%%%%%%%%%%%%%%%%%%%%%%%%%%%%%%%%%%%%%%%%%%%%%%%%%

\section{Matrix Unpasting}
\seclabel{unpasting}

\note{Matrix unpasting is not yet supported.}

\note{We should add a static check for an unpasting to avoid runtime unpasting exceptions.}

\begin{Grammar}
\emph{Unpasting} &::=& \texttt{[} \emph{UnpastingElems} \texttt{]} \\

\emph{UnpastingElems}
&::=& \emph{UnpastingElem} \emph{RectSeparator} \emph{UnpastingElems} \\
&$|$& \emph{UnpastingElem} \\

\emph{UnpastingElem}
&::=& \emph{BindId} \options{\texttt{[} \emph{UnpastingDim} \texttt{]}} \\
&$|$& \emph{Unpasting} \\

\emph{UnpastingDim} &::=& \emph{ExtentRange} (\BY\ \emph{ExtentRange})$^+$ \\

\emph{ExtentRange}
&::=& \option{\emph{StaticArg}}\EXP{\mathinner{\hbox{\tt\char'43}}}
\option{\emph{StaticArg}}\\
&$|$& \option{\emph{StaticArg}}\KWD:\option{\emph{StaticArg}}\\
&$|$& \emph{StaticArg} \\

\emph{RectSeparator} &::=& \EXP{;}+\\
&$|$& \emph{Whitespace}\\
\end{Grammar}

Matrix unpasting is an extension of local variable declaration syntax as a
shorthand for breaking a matrix into parts.  On the left-hand side of
a declaration, what looks like a matrix pasting of unbound variables
is actually a declaration of several new variables.
This syntax
serves to break the right-hand side into pieces and bind the pieces to
the variables.  Matrix unpastings are concise, eliminate several
opportunities for fencepost errors, guarantee unaliased parts, and
avoid overspecification of how the matrix should be taken apart.

The motivating example for matrix unpasting is cache-oblivious
matrix multiplication.  The general plan in a cache oblivious
algorithm is to break the input apart on its largest dimension,
and recursively attack the resulting smaller and more compact
problems.

\note{
Several bugs in the emacs tool for the code in this section.
 - superscripts were too complicated for the tool
 - left-hand bracket for the array was wrong
}
%% mm[\nat m, nat n, nat p\](left  :RR^{m \times n}, right :RR^{n \times p}, result:RR^{m \times p}):() =
%%   case largest of
%%     1 \Rightarrow result[0,0] += (left[0,0] right[0,0])
%%     m \Rightarrow [ lefttop
%%                     leftbottom   ] = left
%%                   [ resulttop
%%                     resultbottom ] = result
%%         t1 = spawn mm(lefttop, right, resulttop)
%%         mm(leftbottom, right, resultbottom)
%%         t1.wait()
%%     p \Rightarrow [ rightleft  rightright  ] = right
%%                   [ resultleft resultright ] = result
%%         t1 = spawn mm(left, rightleft, resultleft)
%%         mm(left, rightright, resultright)
%%         t1.wait()
%%     n \Rightarrow [ leftleft leftright ] = left
%%          [ righttop
%%            rightbottom ] = right
%%          mm(leftleft , righttop   , result)
%%          mm(leftright, rightbottom, result)
%%   end
\begin{Fortress}
\(\VAR{mm}\llbracket\KWD{nat} m, \KWD{nat} n, \KWD{nat} p\rrbracket(\VAR{left}  \mathrel{\mathtt{:}}\mathbb{R}^{m \times n}, \VAR{right} \mathrel{\mathtt{:}}\mathbb{R}^{n \times p}, \VAR{result}\COLONOP\mathbb{R}^{m \times p})\COLONOP() =\)\\
{\tt~~}\pushtabs\=\+\(  \KWD{case}\;\;\KWD{largest}\;\;\KWD{of}\)\\
{\tt~~}\pushtabs\=\+\(    1 \Rightarrow \VAR{result}_{0,0} \mathrel{+}= (\VAR{left}_{0,0} \VAR{right}_{0,0})\)\\
\(    m \Rightarrow \null\)\pushtabs\=\+\([\,\null\)\pushtabs\=\+\(\VAR{lefttop}\)\\
\(                    \VAR{leftbottom}  \,] = \VAR{left}\)\-\\\poptabs
\(                  [\,\null\)\pushtabs\=\+\(\VAR{resulttop} \)\\
\(                    \VAR{resultbottom}\,] = \VAR{result}\)\-\-\\\poptabs\poptabs
{\tt~~~~}\pushtabs\=\+\(        t_{1} = \;\KWD{spawn} \VAR{mm}(\VAR{lefttop}, \VAR{right}, \VAR{resulttop})\)\\
\(        \VAR{mm}(\VAR{leftbottom}, \VAR{right}, \VAR{resultbottom})\)\\
\(        t_{1}.\VAR{wait}()\)\-\\\poptabs
\(    p \Rightarrow \null\)\pushtabs\=\+\([\,\VAR{rightleft}\mskip 4mu plus 4mu\VAR{rightright} \,] = \VAR{right}\)\\
\(                  [\,\VAR{resultleft}\mskip 4mu plus 4mu\VAR{resultright}\,] = \VAR{result}\)\-\\\poptabs
{\tt~~~}\pushtabs\=\+\(        t_{1} = \;\KWD{spawn} \VAR{mm}(\VAR{left}, \VAR{rightleft}, \VAR{resultleft})\)\\
\(        \VAR{mm}(\VAR{left}, \VAR{rightright}, \VAR{resultright})\)\\
\(        t_{1}.\VAR{wait}()\)\-\\\poptabs
\(    n \Rightarrow [\,\VAR{leftleft}\mskip 4mu plus 4mu\VAR{leftright}\,] = \VAR{left}\)\\
{\tt\;~~~}\pushtabs\=\+\(         [\,\null\)\pushtabs\=\+\(\VAR{righttop}\)\\
\(           \VAR{rightbottom}\,] = \VAR{right}\)\-\\\poptabs
\(         \VAR{mm}(\VAR{leftleft} , \VAR{righttop}   , \VAR{result})\)\\
\(         \VAR{mm}(\VAR{leftright}, \VAR{rightbottom}, \VAR{result})\)\-\-\\\poptabs\poptabs
\(  \KWD{end}\)\-\\\poptabs
\end{Fortress}

In unpasting, the element syntax is slightly enhanced both to
permit some specification of the split location and to
receive information about the split that was performed.  For
example, perhaps only the upper left square of a matrix is
interesting.  The programmer can annotate bounds to the
square unpasted element:
\note{
I manually edited superscripts, subscripts, and the left square bracket.}
%% foo[\nat m, nat n\](A:RR^{m \times n}):() =
%%   if   m < n then
%%        [ squareShape_{m \times m} rest ] = A
%%        ...
%%   elif m > n then
%%        [ squareShape_{n \times n}
%%          rest                     ] = A
%%        ...
%%   else (* A already square *)
%%        squareShape = A
%%        ...
%%   end
\begin{Fortress}
\(\VAR{foo}\llbracket\KWD{nat} m, \KWD{nat} n\rrbracket(A\COLONOP\mathbb{R}^{m \times n})\COLONOP() =\)\\
{\tt~~}\pushtabs\=\+\(  \KWD{if}   \null\)\pushtabs\=\+\(m < n \KWD{then}\)\\
\(       [\,\VAR{squareShape}_{m \times m} \VAR{rest}\,] = A\)\\
\(       \ldots\)\-\\\poptabs
\(  \KWD{elif} \null\)\pushtabs\=\+\(m > n \KWD{then}\)\\
\(       [\,\null\)\pushtabs\=\+\(\VAR{squareShape}_{n \times n}\)\\
\(         \VAR{rest}                    \,] = A\)\-\\\poptabs
\(       \ldots\)\-\\\poptabs
\(  \KWD{else} \null\)\pushtabs\=\+\(\mathtt{(*}\;\hbox{\rm  A already square \unskip}\;\mathtt{*)}\)\\
\(       \VAR{squareShape} = A\)\\
\(       \ldots\)\-\\\poptabs
\(  \KWD{end}\)\-\\\poptabs
\end{Fortress}
The types of the elements of the newly declared matrix variables on the left-hand side
of an unpasting are inferred (trivially) to be the type of the elements on the right-hand side.

If an unpasting into explicitly sized pieces does not
exactly cover the right-hand-side matrix, an \TYP{UnpastingError}
is thrown.

Each element of the left-hand-side of unpasting
includes an optional \emph{extent specification}.
An extent specification
\EXP{\VAR{low}\mathinner{\hbox{\tt\char'43}}\VAR{num}}
describes the indexing and the size of the given part of the matrix.
The lower extent must be bound, either before
the unpasting, or earlier (left-or-above) in the unpasting.
For example, suppose that an algorithm
chooses to break a matrix into 4 pieces,
but retain the original indices for each piece:
%% bar[\nat p, nat q\](X:RR^{r0#p \times c0#q}):() = do
%%    [ A_{r0#m~~~~~\times c0#n} B_{r0#m~~~~~\times c0+n#q-n}
%%      C_{r0+m#p-m \times c0#n} D_{r0+m#p-m \times c0+n#q-n} ] = X
%%    ...
%% end
\begin{Fortress}
\(\VAR{bar}\llbracket\KWD{nat} p, \KWD{nat} q\rrbracket(X\COLONOP\mathbb{R}^{r_0\mathinner{\hbox{\tt\char'43}}p \times c_{0}\mathinner{\hbox{\tt\char'43}}q})\COLONOP() = \;\KWD{do}\)\\
{\tt~~~}\pushtabs\=\+\(
[\,\null\)\pushtabs\=\+\(\mathrm{A}_{r_0\mathinner{\hbox{\tt\char'43}}m~~~~~~~~~\times
         c_{0}\mathinner{\hbox{\tt\char'43}}n}~~\mathrm{B}_{r_0\mathinner{\hbox{\tt\char'43}}m~~~~~~~~~~\times c_{0}+n\mathinner{\hbox{\tt\char'43}}q-n}\)\\
\(     \mathrm{C}_{r_0+m\mathinner{\hbox{\tt\char'43}}p-m \times c_{0}\mathinner{\hbox{\tt\char'43}}n}~~\mathrm{D}_{r_0+m\mathinner{\hbox{\tt\char'43}}p-m \times c_{0}+n\mathinner{\hbox{\tt\char'43}}q-n}\,] = X\)\-\\\poptabs
\(   \ldots\)\-\\\poptabs
\(\KWD{end}\)
\end{Fortress}

Unpasting does not directly support non-uniform
decomposition, and does not provide any sort of constraint
satisfaction between the extents of the parts.  For example, the following
decomposition is not legal because it constrains the split sizes to be
equal with respect to unbound \KWD{nat} parameters:

%% (* Not allowed! *)
%% fubar[\nat m, nat n\](X:RR^{m \times n}):() = do
%%   (* p and q unbound *)
%%   [ A_{p \times q} B_{p \times q}
%%     C_{p \times q} D_{p \times q} ] = X
%%   ...
%% end
\begin{Fortress}
\(\mathtt{(*}\;\hbox{\rm  Not allowed! \unskip}\;\mathtt{*)}\)\\
\(\VAR{fubar}\llbracket\KWD{nat} m, \KWD{nat} n\rrbracket(X\COLONOP\mathbb{R}^{m \times n})\COLONOP() = \;\KWD{do}\)\\
{\tt~~}\pushtabs\=\+\(  \mathtt{(*}\;\hbox{\rm  p and q unbound \unskip}\;\mathtt{*)}\)\\
\(  [\,\null\)\pushtabs\=\+\(\mathrm{A}_{p \times q}~~ \mathrm{B}_{p \times q}\)\\
\(    \mathrm{C}_{p \times q}~~ \mathrm{D}_{p \times q}\,] = X\)\-\\\poptabs
\(  \ldots\)\-\\\poptabs
\(\KWD{end}\)
\end{Fortress}
To get this effect, the programmer should compute the constrained values:
%% fubar[\nat m, nat n\](X:RR^{2m \times 2n}):() = do
%%   [ A_{m \times n} B_{m \times n}
%%     C_{m \times n} D_{m \times n} ] = X
%%   ...
%% end
\begin{Fortress}
\(\VAR{fubar}\llbracket\KWD{nat} m, \KWD{nat} n\rrbracket(X\COLONOP\mathbb{R}^{2m \times 2n})\COLONOP() = \;\KWD{do}\)\\
{\tt~~}\pushtabs\=\+\(  [\,\null\)\pushtabs\=\+\(\mathrm{A}_{m \times n}~~ \mathrm{B}_{m \times n}\)\\
\(    \mathrm{C}_{m \times n}~~ \mathrm{D}_{m \times n}\,] = X\)\-\\\poptabs
\(  \ldots\)\-\\\poptabs
\(\KWD{end}\)
\end{Fortress}

Some non-uniform unpastings can be obtained with composition,
which can be expressed either by repeated unpasting:
%% unequalRows[\nat m, nat n\](X:R^{4m \times 2n}) = do
%%     [ c1_{4m \times n} \  c2_{4m \times n} ] = X
%%     [ A_{m \times n}
%%       C_{3m \times n} ] = c1
%%     [ B_{3m \times n}
%%       D_{m \times n} ] = c2
%%     ...
%% end
\begin{Fortress}
\(\VAR{unequalRows}\llbracket\KWD{nat} m, \KWD{nat} n\rrbracket(X\COLONOP{}R^{4m \times 2n}) = \;\KWD{do}\)\\
{\tt~~~~}\pushtabs\=\+\(    [\,c1_{4m \times n} \  c2_{4m \times n}\,] = X\)\\
\(    [\,\null\)\pushtabs\=\+\(\mathrm{A}_{m \times n}\)\\
\(      \mathrm{C}_{3m \times n}\,] = c1\)\-\\\poptabs
\(    [\,\null\)\pushtabs\=\+\(\mathrm{B}_{3m \times n}\)\\
\(      \mathrm{D}_{m \times n}\,] = c2\)\-\\\poptabs
\(    \ldots\)\-\\\poptabs
\(\KWD{end}\)
\end{Fortress}
or simply by nesting matrices in the unpasting:
%% unequalRows[\nat m, nat n\](X:R^{2m \times 4n}) = do
%%   [ [ A_{m \times n}  B_{m \times 3n}]
%%     [ C_{m \times 3n} D_{m \times n} ]  ] = X
%%   ...
%% end
\begin{Fortress}
\(\VAR{unequalRows}\llbracket\KWD{nat} m, \KWD{nat} n\rrbracket(X\COLONOP{}R^{2m \times 4n}) = \;\KWD{do}\)\\
{\tt~~}\pushtabs\=\+\(  [\,\null\)\pushtabs\=\+\([\,\mathrm{A}_{m \times n}  \mathrm{B}_{m \times 3n}]\)\\
\(    [\,\mathrm{C}_{m \times 3n} \mathrm{D}_{m \times n}\,] \,] = X\)\-\\\poptabs
\(  \ldots\)\-\\\poptabs
\(\KWD{end}\)
\end{Fortress}
