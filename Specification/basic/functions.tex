%%%%%%%%%%%%%%%%%%%%%%%%%%%%%%%%%%%%%%%%%%%%%%%%%%%%%%%%%%%%%%%%%%%%%%%%%%%%%%%%
%   Copyright 2009, Oracle and/or its affiliates.
%   All rights reserved.
%
%
%   Use is subject to license terms.
%
%   This distribution may include materials developed by third parties.
%
%%%%%%%%%%%%%%%%%%%%%%%%%%%%%%%%%%%%%%%%%%%%%%%%%%%%%%%%%%%%%%%%%%%%%%%%%%%%%%%%

\chapter{Functions}
\chaplabel{functions}

\note{Abstract function declarations and their checking,
modifiers on functions such as \KWD{atomic} and \KWD{io},
keyword and varargs parameters, where clauses,
contract checking, and tail-call optimization
are not supported yet.

Varargs and keyword examples and abstract function declarations examples
are not run by the interpreter.}

\note{Guy's email titled ``Re: Function equality issues'' on 02/13/08

Yes with returning false for overloaded functions (with the implementation-dependent features).
(03/03/08)}

A \emph{function} is a value that has a function type
(described in \secref{function-types}).
Each function takes exactly one argument, which may be a tuple, and returns
exactly one result, which may be a tuple.
A function may be declared as top level or local as described
in \secref{decl-kinds}.
Fortress allows functions to be \emph{overloaded} (as described in
\chapref{multiple-dispatch}); there may be multiple function
declarations with the same function name in a single lexical scope.
Functions can be passed as arguments and returned as values.
Single variables may be bound to functions
including overloaded functions.

\section{Function Declarations}
\seclabel{function-decls}

\begin{Grammar}
\emph{FnDecl}
&::=& \option{\emph{FnMods}} \emph{FnHeaderFront} \emph{FnHeaderClause}
\EXP{=} \emph{Expr} \\
&$|$& \emph{FnSig} \\

\emph{FnSig} &::=& \emph{SimpleName} \EXP{\mathrel{\mathtt{:}}} \emph{Type}\\

\emph{FnMods} &::=& \emph{FnMod}$^+$\\

\emph{FnMod} &::=& \emph{AbsFnMod} $|$ \KWD{private}\\

\emph{AbsFnMod} &::=& \emph{LocalFnMod} $|$
\KWD{test}\\

\emph{LocalFnMod} &::=& \KWD{atomic} $|$ \KWD{io}\\

\emph{FnHeaderFront}
&::=& \emph{NamedFnHeaderFront}\\
&$|$& \emph{OpHeaderFront} \\

\emph{NamedFnHeaderFront}
&::=& \emph{Id} \option{\emph{StaticParams}} \emph{ValParam} \\

\emph{ValParam} &::=& \emph{BindId}\\
&$|$& \texttt(\option{\emph{Params}}\texttt)\\

\emph{Params}
&::=&
(\emph{Param}\EXP{,})$^*$ \options{\emph{Varargs}\EXP{,}} \emph{Keyword}(\EXP{,}\emph{Keyword})$^*$\\
&$|$&
(\emph{Param}\EXP{,})$^*$  \emph{Varargs}\\
&$|$& \emph{Param}(\EXP{,} \emph{Param})$^*$\\

\emph{VarargsParam} &::=& \emph{BindId}\EXP{\COLONOP}\emph{Type}\EXP{...} \\

\emph{Varargs} &::=& \emph{VarargsParam}\\

\emph{Keyword} &::=& \emph{Param}\EXP{=}\emph{Expr} \\

\emph{PlainParam} &::=& \emph{BindId} \option{\emph{IsType}} \\
&$|$& \emph{Type} \\

\emph{Param} &::=& \emph{PlainParam}\\

\emph{FnHeaderClause} &::=& \option{\emph{IsType}} \emph{FnClauses} \\

\emph{FnClauses} &::=& \option{\emph{Throws}} \option{\emph{Where}}
\emph{Contract} \\

\emph{Throws} &::=& \KWD{throws} \emph{MayTraitTypes}\\

\end{Grammar}

Syntactically, a function declaration consists of
an optional sequence of modifiers followed by
the name of the function, optional static parameters
(described in \chapref{trait-parameters}),
the value parameter with its (optionally) declared type,
an optional type of a return value,
an optional declaration of thrown checked exceptions
(discussed in \chapref{exceptions}),
an optional \KWD{where} clause (discussed in \secref{where-clauses}),
a contract for the function (discussed in \secref{contracts}),
and finally an optional body expression preceded by the token \EXP{=}.
A \KWD{throws} clause does not include naked type variables.
Every element in a \KWD{throws} clause is a subtype of
\TYP{CheckedException}.
When a function declaration includes a body expression, it is called a
\emph{function definition}.
Function declarations can be mutually recursive.

Function declarations can include the following special modifiers:
\paragraph{\KWD{atomic}:}
A function with the modifier \KWD{atomic} acts as if its entire
body were surrounded in an \KWD{atomic} expression discussed in
\secref{atomic}.

\paragraph{\KWD{io}:}
Functions that perform externally visible input/output actions
are said to be \KWD{io} functions.
An \KWD{io} function must not be invoked from a non-\KWD{io} function.

A function takes exactly one argument, which may be a tuple.
When a function takes a tuple argument, we abuse terminology by saying that
the function takes multiple arguments.
Value parameters cannot be mutated inside the function body.

A function's value parameter consists of a parenthesized,
comma-separated list of bindings where each binding is one of:
\begin{itemize}
\item A plain binding ``\VAR{identifier}''
or ``\EXP{\VAR{identifier}\COLONOP{}T}''
\item A varargs binding ``\EXP{\VAR{identifier}\COLONOP{}T\ldots}''
\item A keyword binding ``\EXP{\VAR{identifier}=\exp}''
or ``\EXP{\VAR{identifier}\COLONOP{}T=\exp}''
\end{itemize}
When the parameter is a single plain binding without a declared type,
enclosing parentheses may be elided.
The following restrictions apply:  No two bindings may have the same
    identifier.
No keyword binding may precede a plain binding.
No varargs binding may follow a keyword binding or precede a plain binding.
Note that it is permitted to have a single plain binding,
or to have no bindings.  The latter case, ``()'', is considered equivalent
to a single plain binding of the ignored identifier ``\_'' of type (), that
is, ``\EXP{(\_\COLONOP())}''.
Also, there can be at most one varargs binding.

A parameter declared by keyword binding is called
a \emph{keyword parameter};
a keyword parameter must be declared with
a \emph{default} expression,
which is used when no argument is bound to the parameter explicitly.
Syntactically, the default expression is specified
after an \EXP{=} sign.
The default expression of a parameter $x$ of function $f$
is evaluated each time the function is called
without a value provided for $x$ at the call site.
All parameters occurring to the left of $x$
are in scope of its default expression.
All parameters following $x$ must include default expressions as well;
$x$ is in scope of their default expressions and the body of the function.
When an argument is passed explicitly for a keyword parameter,
that argument must be passed as a \emph{keyword argument}.
(See \secref{function-app}.)
If no type is declared for a keyword parameter,
the type is inferred from the static type of its default expression.

A parameter declared by varargs binding is called
a \emph{varargs parameter};
it is used to pass a variable number of arguments to a function
as a single heap sequence.
The type of a varargs parameter
    is \EXP{\TYP{HeapSequence}\llbracket{}T\rrbracket} where \VAR{T} is
    the type mentioned in (or inferred for) that binding.
\note{See the library section for a discussion of
\TYP{HeapSequence}.}
Note that the type of a varargs parameter cannot be omitted.
If a function does not have a varargs parameter
then the number of arguments is fixed by the function's type.
Note that a varargs parameter is not allowed to have a default expression.

For example, here is a simple
polymorphic function for creating lists:
\input{\home/basic/examples/Fun.Decl.tex}

\section{Function Applications}
\seclabel{function-app}

\note{varargs arguments (10/30/07)

Is there any reason to use \TYP{HeapSequence} which is mutable to implement varargs tuples?
How about implementing them in terms of some covariant data type?

We agreed to implement varargs tuples using \TYP{ReadOnlyHeapSequence} which is immutable and
invariant. We'll add \TYP{MakeReadOnlyHeapSequence} into the core library.
}

Fortress allows functions to be overloaded; there may be multiple function
declarations with the same function name in a single lexical scope.
Thus, we need to determine which functional declarations are applicable to
a function application.

An overloaded function has multiple arrow types in its function type,
and it associates a declaration with each constituent arrow type.
When an overloaded function of type \VAR{T} is called with an argument of type
\VAR{A}, the call is ``dispatched'' to the declaration associated with
the most specific type of \VAR{T} applicable to \VAR{A}
(if no such type exists, then the function is not applicable to \VAR{A}).
This is well defined because the overloading rules guarantee that
the type of any function value is a well-formed function type.

If a function's argument type is \EXP{()},
then function declarations with the following forms of parameter lists are
considered to be applicable:
\begin{itemize}
\item   \EXP{()}
which means the same thing as \EXP{(\_\COLONOP())}
\item   \EXP{(x\COLONOP())}
which is something programmers don't ordinarily write
\item   \EXP{(x\COLONOP{}T\ldots)}
\end{itemize}
In the last case, \VAR{x} is bound to an empty
\EXP{\TYP{HeapSequence}\llbracket{}T\rrbracket}.

If a function's argument type \VAR{A} is
    neither \EXP{()} nor a tuple type, then function declarations with the following forms of parameter lists are considered
    to be applicable:
\begin{itemize}
\item
        \EXP{(x\COLONOP{}T)}       where \VAR{A} is a subtype of \VAR{T}
\item
        \EXP{(x\COLONOP{}T\ldots)} where \VAR{A} is a subtype of \VAR{T}
\end{itemize}
In the last case, \VAR{x} is bound to a
    \EXP{\TYP{HeapSequence}\llbracket{}T\rrbracket} of length \EXP{1},
    containing  the actual argument value.

If a function's argument type \VAR{A} is
a tuple type, then function declarations with the following forms of
parameter lists are considered to be applicable:
\begin{itemize}
\item
        \EXP{(x\COLONOP{}T)}       where \VAR{A} is a subtype of \VAR{T}
\item
        \EXP{(x\COLONOP{}T\ldots)} where \VAR{A} is a subtype of \VAR{T}
\item
a parameter list with no varargs binding, provided that
 \begin{itemize}
 \item type \VAR{A} has exactly as many plain types as the parameter
list has plain bindings, and
 \item for every keyword-type pair (described in
\secref{tuple-types}) in \VAR{A}, the parameter list has a
binding with the same keyword, and
 \item for every element type in \VAR{A}, the type in the element type
is a subtype of the type of the corresponding binding in the parameter list.
 \end{itemize}
\item
a parameter list with a varargs binding, provided that
 \begin{itemize}
 \item type \VAR{A} has at least as many plain types as the parameter
list has plain bindings, and
 \item for every keyword-type pair in
\VAR{A}, the parameter list has a binding with the same keyword, and
 \item for every element type in \VAR{A}, the type in the element type
is a subtype of the type of the corresponding binding in the parameter
list---but if there is no corresponding binding, then the type in the
element type must be a subtype of the type in the varargs binding.
 \end{itemize}
\end{itemize}
    In the latter case, the parameter named by the identifier in the varargs
    binding is bound to a \EXP{\TYP{HeapSequence}\llbracket{}T\rrbracket}
    that contains, in order, all the
 values of the tuple that did not correspond to plain
    bindings, followed by all the values in the varargs \TYP{HeapSequence}
    of the tuple, if any.

When an argument is passed explicitly for a keyword parameter, that
argument must be passed as a \emph{keyword argument}.
Syntactically, a keyword argument is a keyword-value pair
``\EXP{\VAR{identifier} = e}''.
Keyword parameters not explicitly bound are bound to their default values.
If a parameter that has no default value is not explicitly bound to an
argument, it is a static error.

When a function is called (See \secref{function-calls} for a discussion
of function call expressions), function arguments are evaluated in parallel,
keyword parameters not explicitly bound are bound to their
default values sequentially,
and the body of the function is evaluated in
a new environment, extending the environment in which
it is defined with all parameters bound to their arguments.

If the application of a function \VAR{f} ends by calling another
function \VAR{g}, tail-call
optimization must be applied.  Storage used by the new environments
constructed for the application of \VAR{f} must be reclaimed.

Here are some examples:
%sin(pi)
%arctan(y, x)
%makeColor(red=5, green=3, blue=43)
%processString(s, start=5, finish=43)
\input{\home/basic/examples/Fun.App.a.tex}
If the function's argument is not a tuple, then the argument need not be
parenthesized:
%sin x
%log log n
\input{\home/basic/examples/Fun.App.b.tex}

Here are a few varargs and keyword examples:
%f(x:ZZ, y:ZZ..., z:ZZ = 0) = <| x, <| q | q<-y |>, z |>
%
%f(1)  returns  <| 1, <| |>, 0 |>
%f(1, 2, 3)  returns  <| 1, <| 2, 3 |>, 0 |>
%f(1, [2 3]...)  returns  <| 1, <| 2, 3 |>, 0 |>
%f(1, 2, 3, [4 5]...)  returns  <| 1, <| 2, 3, 4, 5 |>, 0 |>
%f(1, 2, 3, 17#3...)  returns  <| 1, <| 2, 3, 17, 18, 19 |>, 0 |>
%f(1, 2, 3, z=8)  returns  <| 1, <| 2, 3 |>, 8 |>
%f([2 3]...)  declaration not applicable
\begin{Fortress}
\(f(x\COLONOP\mathbb{Z}, y\COLONOP\mathbb{Z}\ldots, z\COLONOP\mathbb{Z} = 0) =
\langle x, \langle q \mid q\leftarrow{}y \rangle, z \rangle\)\\
\end{Fortress}

\begin{tabular}{lcl}
\EXP{f(1)}  & \emph{returns} &
\EXP{\langle 1, \langle \rangle, 0 \rangle}
\\
\EXP{f(1, 2, 3)}  & \emph{returns} &
\EXP{\langle 1, \langle 2, 3 \rangle, 0 \rangle}
\\
\EXP{f(1, [2\;3]\ldots)}  & \emph{returns} &
\EXP{\langle 1, \langle 2, 3 \rangle, 0 \rangle}
\\
\EXP{f(1, 2, 3, [4\;5]\ldots)}  & \emph{returns} &
\EXP{\langle 1, \langle 2, 3, 4, 5 \rangle, 0 \rangle}
\\
\EXP{f(1, 2, 3, 17\mathinner{\hbox{\tt\char'43}}3\ldots)} & \emph{returns} &
\EXP{\langle 1, \langle 2, 3, 17, 18, 19 \rangle, 0 \rangle}
\\
\EXP{f(1, 2, 3, z=8)} & \emph{returns} &
\EXP{\langle 1, \langle 2, 3 \rangle, 8 \rangle}
\\
\EXP{f([2\;3]\ldots)}  & \emph{declaration not applicable}
\end{tabular}

\section{Abstract Function Declarations}
\seclabel{abstractFunctionDeclarations}

\begin{Grammar}
\emph{AbsFnDecl} &::=& \option{\emph{AbsFnMods}} \emph{FnHeaderFront}
\emph{FnHeaderClause}\\
&$|$& \emph{FnSig} \\

\emph{AbsFnMods} &::=& \emph{AbsFnMod}$^+$\\

\emph{AbsFnMod} &::=& \emph{LocalFnMod} $|$ \KWD{test}\\

\emph{LocalFnMod} &::=& \KWD{atomic} $|$ \KWD{io}\\

\emph{FnSig} &::=& \emph{Name} \EXP{\mathrel{\mathtt{:}}} \emph{ArrowType}\\
\end{Grammar}

In a component, overloaded functions may be declared separately from their
definitions using \emph{abstract function declarations}.
Abstract function declarations themselves may be overloaded.
For an abstract function declaration with name \VAR{f}, argument type
\VAR{T}, and return type \VAR{U} (\VAR{T} and \VAR{U} may be tuple types),
any concrete declaration for \VAR{f} must be for a function whose argument
type is a subtype of \VAR{T} and whose return type is a subtype of \VAR{U}.
Furthermore, the union of the argument types of the concrete declarations for
\VAR{f} must be equal to \VAR{T}.
Concrete function declarations must satisfy the overloading rules.
Unless \VAR{T} has a \KWD{comprises} clause (or is a tuple type, at least
one of whose entries has a \KWD{comprises} clause), this implies that some
concrete declaration for \VAR{f} must have argument type \VAR{T}.

Syntactically, an abstract function declaration is a function declaration
without a body.  Parameter names may be elided but parameter types cannot
be omitted.  Additionally, when a function's type is not parameterized,
Fortress provides an alternative mathematical notation for
an abstract function declaration: function name followed by the token
`\EXP{\COLONOP}', followed by an arrow type.

For example, after the following abstract function declaration:
%printMolecule(Molecule): ()
\begin{Fortress}
\(\VAR{printMolecule}(\TYP{Molecule})\COLON ()\)
\end{Fortress}
where trait \TYP{Molecule} is defined as follows:
%trait Molecule comprises {OrganicMolecule, InorganicMolecule} end
\begin{Fortress}
\(\KWD{trait} \TYP{Molecule} \KWD{comprises} \{\TYP{OrganicMolecule}, \TYP{InorganicMolecule}\} \KWD{end}\)
\end{Fortress}
the programmer could write:
%printMolecule(molecule: Molecule) = ...
\begin{Fortress}
\(\VAR{printMolecule}(\VAR{molecule}\COLON \TYP{Molecule}) = \ldots\)
\end{Fortress}
or could write:
%printMolecule(molecule: OrganicMolecule) = ...
%printMolecule(molecule: InorganicMolecule) = ...
\begin{Fortress}
\(\VAR{printMolecule}(\VAR{molecule}\COLON \TYP{OrganicMolecule}) = \ldots\)\\
\(\VAR{printMolecule}(\VAR{molecule}\COLON \TYP{InorganicMolecule}) = \ldots\)
\end{Fortress}
For the latter, the programmer must provide a
definition for every immediate subtype of \TYP{Molecule}, or
it is a static error.

\section{Function Contracts}
\seclabel{contracts}

\note{For nested contracts, the inner \KWD{outcome} shadows the outer \KWD{outcome}.}

\begin{Grammar}
\emph{Contract} &::=& \option{\emph{Requires}} \option{\emph{Ensures}} \option{\emph{Invariant}}\\

\emph{Requires} &::=&
\KWD{requires} \{ \option{\emph{ExprList}} \} \\

\emph{Ensures}  &::=&
\KWD{ensures} \{ \option{\emph{EnsuresClauseList}} \} \\

\emph{EnsuresClauseList} &::=&
\emph{EnsuresClause}(\EXP{,} \emph{EnsuresClause})$^*$ \\

\emph{EnsuresClause} &::=& \emph{Expr} \options{\KWD{provided} \emph{Expr}} \\

\emph{Invariant}&::=&
\KWD{invariant} \{ \option{\emph{ExprList}} \} \\

\end{Grammar}

Function contracts consist of three optional clauses: a \KWD{requires}
clause, an \KWD{ensures} clause, and an \KWD{invariant} clause.
All three clauses are evaluated in the scope of the function body.
\note{
extended with a special variable \emph{arg}, bound to an immutable
array of all function arguments. (This array is useful for describing
contracts in the presence of higher-order functions;
see \secref{arrow-types}).
}

\note{
Jan: Which of the following must be pure?  Presumably all of them.
We should say so.

Eric: Requiring declared purity on all called functions in a contract is
too restrictive.
}

The  \KWD{requires} clause consists of a sequence of
expressions of type \TYP{Boolean} separated by commas and
enclosed in curly braces.
The \KWD{requires} clause is evaluated during a function call before
the body of the function. If any expression in a \KWD{requires} clause
does not evaluate to \VAR{true}, a \TYP{CallerViolation} exception is
thrown.

The \KWD{ensures} clause consists of a sequence of
\KWD{ensures} subclauses. Each such subclause consists of an
expression of type \TYP{Boolean},
optionally followed by
a \KWD{provided} subclause. A \KWD{provided} subclause begins with
\KWD{provided} followed by an expression of type
\TYP{Boolean}.  For each subclause in the \KWD{ensures} clause of a
contract, the \KWD{provided}
subclause is evaluated immediately after the \KWD{requires} clause during a
function call (before the function body is evaluated).
If a \KWD{provided} subclause evaluates to \VAR{true},
then the expression preceding this \KWD{provided} subclause is evaluated
after the function body is evaluated.
If the expression evaluated after function evaluation does
not evaluate to \VAR{true}, a \TYP{CalleeViolation} exception is thrown.
The expression preceding the \KWD{provided} subclause can refer to
the return value of the function.
A \KWD{outcome} variable is implicitly bound to a return value of the
function and is in scope of
the expression preceding the \KWD{provided} subclause.
The implicitly declared \KWD{outcome} shadows any other declaration with the
same name in scope.

The \KWD{invariant} clause consists of a sequence of expressions of
\emph{any type} enclosed by curly braces.
These expressions are evaluated before and after a function call.  For
each expression $e$ in this sequence, if the value of $e$ when evaluated
before the function call is not equal to the value of $e$ after
the function call,  a \TYP{CalleeViolation} exception is thrown.

Here are some examples:
\input{\home/basic/examples/Fun.Contract.tex}

Overloaded function contracts are handled similarly with method contracts
described in \secref{method-contracts}.  In particular, substitutability is
preserved: the statically most applicable function to a call should be
substitutable with the dynamically most applicable function to the call.
For a call of function $f$,
we use the term \emph{static contract} of $f$ to refer to
a contract declared in the statically most applicable function declaration
and the term \emph{dynamic contract} of $f$ to refer to a contract declared
in the dynamically most applicable function declaration.
Three exceptions may be thrown due to an overloaded function contract
violation: \TYP{CallerViolation} is thrown when the \KWD{requires} clause
of the static contract fails, \TYP{CalleeViolation} is thrown when the
\KWD{ensures} or \KWD{invariant} clause of the dynamic contract fails,
and \TYP{ContractOverloadingViolation} is thrown
when the \KWD{requires} clause of the dynamic contract
or the \KWD{ensures} or \KWD{invariant} clause of the static contract fails.


Evaluation of a call of function $f$ proceeds as follows.
Let $C$ and $C'$ be the static and dynamic contracts of $f$, respectively.
If the \KWD{requires} clause of $C$ fails,
a \TYP{CallerViolation} exception is thrown.
%
Otherwise, if the \KWD{requires} clause of $C'$ fails,
a \TYP{ContractOverloadingViolation} exception is thrown.
%
Otherwise, the \KWD{provided} subclauses of $C$ and $C'$ are evaluated. For
every \KWD{provided} subclause that evaluates to \VAR{true}, the
corresponding \KWD{ensures} subclause is recorded in a table
$E$ for later comparison.  Similarly, the \KWD{invariant} clauses of $C$
and $C'$ are evaluated and the results are stored in $E$ for later
comparison.
%
Then the body of the dynamically most applicable function declaration of $f$
is evaluated.  After evaluation of the body,
%
all \KWD{ensures} subclauses of the dynamic contract recorded in $E$
are checked to ensure that they evaluate to \VAR{true}, and
all \KWD{invariant} clauses of the dynamic contract recorded in $E$
are checked to ensure that they evaluate to values equal to the
values they evaluated to before evaluation of the body.
%
If any such check fails, a \TYP{CalleeViolation}
exception is thrown. Otherwise, all \KWD{ensures} subclauses and
\KWD{invariant} clauses of the static contract in $E$ are checked.
If any of these checks fails, a \TYP{ContractOverloadingViolation} exception
is thrown.

\section{Local Function Declarations}
\seclabel{local-fn-decls}
\begin{Grammar}
\emph{LocalFnDecl} &::=&
\option{\emph{LocalFnMods}}
\emph{NamedFnHeaderFront} \emph{FnHeaderClause} \EXP{=}
\emph{Expr} \\

\emph{LocalFnMods} &::=& \emph{LocalFnMod}$^+$\\

\emph{LocalFnMod} &::=& \KWD{atomic} $|$ \KWD{io}\\
\end{Grammar}


Functions can be declared within expression blocks (described in
\secref{block-expr}) via the same syntax as is used for top-level
function declarations (described in \chapref{functions})
except that locally declared functions must not be operators
and must not include the modifiers \KWD{private} and \KWD{test}.
As with top-level function declarations, locally declared functions in a
single scope are allowed to be overloaded and mutually recursive.
