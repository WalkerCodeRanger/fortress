%%%%%%%%%%%%%%%%%%%%%%%%%%%%%%%%%%%%%%%%%%%%%%%%%%%%%%%%%%%%%%%%%%%%%%%%%%%%%%%%
%   Copyright 2009, Oracle and/or its affiliates.
%   All rights reserved.
%
%
%   Use is subject to license terms.
%
%   This distribution may include materials developed by third parties.
%
%%%%%%%%%%%%%%%%%%%%%%%%%%%%%%%%%%%%%%%%%%%%%%%%%%%%%%%%%%%%%%%%%%%%%%%%%%%%%%%%

\section{Throw Expressions}
\seclabel{throw-expr}
\begin{Grammar}
\emph{FlowExpr} &::=& \KWD{throw} \emph{Expr}
\end{Grammar}

A \KWD{throw} expression consists of \KWD{throw}
followed by a subexpression.
The type of the subexpression must be a subtype of the type
\TYP{Exception} (see \chapref{exceptions}).
A \KWD{throw} expression evaluates its subexpression to an exception value
and throws the exception value;
the expression completes abruptly and has \TYP{BottomType}.


The type \TYP{Exception} has exactly two direct mutually exclusive subtypes,
\TYP{CheckedException} and \TYP{UncheckedException}.
Every \TYP{CheckedException} that is thrown
must be caught or forbidden by an enclosing \KWD{try} expression
(see \secref{try-expr}),
or it must be declared in the \KWD{throws} clause
of an enclosing functional declaration (see \secref{function-decls}).
Similarly, every \TYP{CheckedException} declared to be thrown
in the static type of a functional called
must be either caught or forbidden by an enclosing \KWD{try} expression,
or declared in the \KWD{throws} clause
of an enclosing functional declaration.
