%%%%%%%%%%%%%%%%%%%%%%%%%%%%%%%%%%%%%%%%%%%%%%%%%%%%%%%%%%%%%%%%%%%%%%%%%%%%%%%%
%   Copyright 2009, Oracle and/or its affiliates.
%   All rights reserved.
%
%
%   Use is subject to license terms.
%
%   This distribution may include materials developed by third parties.
%
%%%%%%%%%%%%%%%%%%%%%%%%%%%%%%%%%%%%%%%%%%%%%%%%%%%%%%%%%%%%%%%%%%%%%%%%%%%%%%%%

\section{If Expressions}
\seclabel{if-expr}

\begin{Grammar}
\emph{DelimitedExpr}
&::=& \KWD{if} \emph{GeneratorClause} \KWD{then} \emph{BlockElems} \option{\emph{Elifs}}
\option{\emph{Else}} \KWD{end}\\
&$|$& \texttt{(}\KWD{if} \emph{GeneratorClause} \KWD{then} \emph{BlockElems}
\option{\emph{Elifs}} \emph{Else} \option{\KWD{end}}\texttt{)}\\


\emph{Elifs} &::=& \emph{Elif}$^+$\\

\emph{Elif} &$::=$& \KWD{elif} \emph{GeneratorClause} \KWD{then} \emph{BlockElems}\\

\emph{Else} &$::=$& \KWD{else} \emph{BlockElems}\\

\emph{GeneratorClause}
&::=& \emph{GeneratorBinding}\\
&$|$& \emph{Expr} \\

\emph{GeneratorBinding}
&::=& \emph{BindIdOrBindIdTuple}\EXP{\leftarrow}\emph{Expr} \\
\end{Grammar}


An \KWD{if} expression consists of \KWD{if} followed by a generator clause
(discussed in \secref{generators}), followed by \KWD{then}, an expression block,
an optional sequence of \KWD{elif} clauses (each consisting of \KWD{elif}
followed by a generator clause, \KWD{then}, and an expression block),
an optional \KWD{else} clause (consisting of \KWD{else} followed by
an expression block), and finally \KWD{end}.
Each clause forms an expression block and has the various
properties of expression blocks (described in \secref{block-expr}).
%
An \KWD{if} expression first evaluates the expression of its generator clause
(\emph{condition expression}).
If the generator clause is a generator binding,
the type of the condition expression must be a subtype of
\EXP{\TYP{Condition}\llbracket{}E\rrbracket} for some \VAR{E}
and the type of the identifiers bound in the clause must be a subtype of \VAR{E}.
A \TYP{Condition} is a generator that generates 0 or 1 element.
Otherwise, the generator clause must be an expression of type \TYP{Boolean}.
Note in particular that \VAR{true} is a \TYP{Boolean} value yielding \EXP{()}
exactly once, while \VAR{false} is a \TYP{Boolean} value that yields no elements.
If the evaluation of the condition expression completes normally and generates
exactly one element, the \KWD{then} clause is evaluated.
If the condition expression generates no elements,
the next clause (either \KWD{elif} or \KWD{else}), if any, is evaluated.
An \KWD{elif} clause works just as the original \KWD{if},
evaluating its condition expression and continuing with its \KWD{then} clause
if the condition generates exactly one element.
%
An \KWD{else} clause simply evaluates its expression block.
The type of an \KWD{if} expression is the union of the types of the
expression blocks of each clause.  If there is no \KWD{else} clause in an
\KWD{if} expression, then every clause must have type \EXP{()}.
The result of the \KWD{if} expression is the result of the expression block
which is evaluated; if no expression block is evaluated it is \EXP{()}.
The reserved word \KWD{end} may be elided if the \KWD{if} expression
is immediately enclosed by parentheses.  In such a case, an \KWD{else}
clause is required.

For example,
\input{\home/basic/examples/Expr.If.tex}
