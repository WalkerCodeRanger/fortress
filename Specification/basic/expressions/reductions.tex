%%%%%%%%%%%%%%%%%%%%%%%%%%%%%%%%%%%%%%%%%%%%%%%%%%%%%%%%%%%%%%%%%%%%%%%%%%%%%%%%
%   Copyright 2009, Oracle and/or its affiliates.
%   All rights reserved.
%
%
%   Use is subject to license terms.
%
%   This distribution may include materials developed by third parties.
%
%%%%%%%%%%%%%%%%%%%%%%%%%%%%%%%%%%%%%%%%%%%%%%%%%%%%%%%%%%%%%%%%%%%%%%%%%%%%%%%%

\section{Summations and Other Reduction Expressions}
\seclabel{reduction-expr}

\begin{Grammar}
\emph{FlowExpr} &::=&
\emph{Accumulator} \option{\emph{StaticArgs}}
\options{\texttt[ \emph{GeneratorClauseList} \texttt]} \emph{Expr} \\

\emph{Accumulator} &::=& $\sum$ $|$ $\prod$ $|$ \KWD{BIG} (\emph{Encloser} $|$ \emph{Op})
\end{Grammar}

A \emph{reduction expression} begins with a big operator such as
$\sum$ or $\prod$ followed by an optional static arguments and
an optional generator clause list (described in
\secref{generators}), followed by a body expression.
There is no explicit relationship between \EXP{\OPR{BIG}\:\emph{Op}} and
\emph{Op}.  Instead, a reduction expression corresponds to a call to the
\EXP{\OPR{BIG}\:\emph{Op}} operator, which has the following header:
%% opr BIG Op[\T\](g:(Reduction[\R0\],T->R0)->R0):R
\begin{Fortress}
\(\KWD{opr} \mathord{\OPR{BIG}}\: \emph{Op}\llbracket{}T\rrbracket(g\COLONOP(\TYP{Reduction}\llbracket{}R_{0}\rrbracket,T\rightarrow{}R_{0})\rightarrow{}R_{0})\COLONOP{}R\)
\end{Fortress}
Here \VAR{R} is the result type of the operator, and \VAR{g} corresponds to
the method %generate[\R0\]
\EXP{\VAR{generate}\llbracket{}R_{0}\rrbracket}
of the trait %Generator[\T\]
\EXP{\TYP{Generator}\llbracket{}T\rrbracket}:
it is a function that takes a reduction (of type %Reduction[\R0\]
\EXP{\TYP{Reduction}\llbracket{}R_{0}\rrbracket})
and a body (of type \EXP{T\rightarrow{}R_{0}})
and returns a value of type \EXP{R_{0}} by running body on each
generated element and combining them using the reduction.
When a generator clause
list is provided, generator clauses produce values and bind the values to the
identifiers that are used in the subexpression.  Each iteration of the
body expression is assumed to be evaluated in a separate implicit
thread as described in \secref{generators}.  The resulting values are
combined together using \emph{Op}
\label{redexpr-redvar}
as with reduction variables (see \secref{reduction-vars}).
The value of a reduction expression is this combined value.
When no generator clause list is provided, the body is taken to be an
object of type \TYP{Generator} and the elements it generates are
combined.  Thus, the reduction expression \EXP{\sum a} is equivalent
to \EXP{\sum\limits_{x\leftarrow{}a}\mskip 4mu plus 4mu{x}}.

A reduction expression with a generator list:
%% \sum [v_1 \gets g_1, v_2 \gets g_2, \ldots] e
\begin{Fortress}
\(\sum [v_{\mathrm{1}} \gets g_{\mathrm{1}}, v_{\mathrm{2}} \gets g_{\mathrm{2}}, \ldots] e\)
\end{Fortress}
is equivalent to the following code:
%do
%  var result: ZZ32 = 0
%  for v1 <- g1, v2 <- g2, \ldots do
%    result += e
%  end
%  result
%end
\begin{Fortress}
\(\KWD{do}\)\\
{\tt~~}\pushtabs\=\+\(  \KWD{var}\:\VAR{result}\COLON \mathbb{Z}32 = 0\)\\
\(  \KWD{for}\:v_{1} \leftarrow g_{1}, v_{2} \leftarrow g_{2}, \ldots \KWD{do}\)\\
{\tt~~}\pushtabs\=\+\(    \VAR{result} \mathrel{+}= e\)\-\\\poptabs
\(  \KWD{end}\)\\
\(  \VAR{result}\)\-\\\poptabs
\(\KWD{end}\)
\end{Fortress}
where \VAR{result} is a fresh variable.
%
A reduction expression without a generator clause list:
%% \sum g
\begin{Fortress}
\(\sum g\)
\end{Fortress}
is equivalent to the following:
%\sum [x \gets g] x
\begin{Fortress}
\(\sum [x \gets g] x\)
\end{Fortress}
Note that reduction expressions without generator clause lists can be used to
conveniently sum any aggregate expression (described in
\secref{aggregate-expr}), since every aggregate expression
is a generator.

\section{Parallel Prefix and Parallel Suffix}
\note{This section was not in F1.0$\beta$ nor in F1.0.  We don't quite have a story on this yet.}
