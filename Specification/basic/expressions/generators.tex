%%%%%%%%%%%%%%%%%%%%%%%%%%%%%%%%%%%%%%%%%%%%%%%%%%%%%%%%%%%%%%%%%%%%%%%%%%%%%%%%
%   Copyright 2009, Oracle and/or its affiliates.
%   All rights reserved.
%
%
%   Use is subject to license terms.
%
%   This distribution may include materials developed by third parties.
%
%%%%%%%%%%%%%%%%%%%%%%%%%%%%%%%%%%%%%%%%%%%%%%%%%%%%%%%%%%%%%%%%%%%%%%%%%%%%%%%%

\section{Generators}
\seclabel{generators}

\note{Distributions are not yet supported.}

\begin{Grammar}
\emph{GeneratorClauseList} &::=& \emph{GeneratorBinding}(\EXP{,} \emph{GeneratorClause})$^*$ \\

\emph{GeneratorBinding}
&::=& \emph{BindIdOrBindIdTuple}\EXP{\leftarrow}\emph{Expr} \\

\emph{GeneratorClause}
&::=& \emph{GeneratorBinding} \\
&$|$& \emph{Expr} \\

\emph{BindIdOrBindIdTuple}
&::=& \emph{BindId}\\
&$|$& \texttt{(} \emph{BindId}\EXP{,} \emph{BindIdList} \texttt{)}\\

\emph{BindIdList} &::=& \emph{BindId}(\EXP{,} \emph{BindId})$^*$\\

\emph{BindId} &::=& \emph{Id}\\
&$|$& \KWD{\_}\\
\end{Grammar}

Fortress makes extensive use of comma-separated \emph{generator clause lists}
to express parallel iteration.  Generator clause lists occur in
generated expressions (described in \secref{generated}) and
\KWD{for} loops (described in \secref{for-expr}),
sums and big operators (described in \secref{reduction-expr}),
and comprehensions (described in \secref{comprehensions}).
We refer to these
collectively as \emph{expressions with generator clauses}.  Every expression with
generator clauses contains a \emph{body expression} which is evaluated for each
combination of values bound in the generator clause list
(each such combination yields an \emph{iteration} of the body).

A generator clause is either a \emph{generator binding} or
an expression of type \EXP{\TYP{Generator}\llbracket()\rrbracket}
(this includes the type \TYP{Boolean}).
A generator clause list must begin with a generator binding.
A generator binding consists of one or more
comma-separated identifiers followed by the token \EXP{\leftarrow},
followed by a subexpression (called the \emph{generator expression}).
A generator expression evaluates to an object whose type is
\TYP{Generator}.  A generator encapsulates zero or more
\emph{generator iterations}.  By default, the programmer must assume
that generator iterations are run in parallel in separate implicit
threads unless the generators are instances of \TYP{SequentialGenerator}; the
actual behavior of generators is dictated by library code, as
described in \secref{defining-generators}.
No generator iterations are run until the generator expression completes.
For each generator iteration, a generator object produces a value or a tuple
of values.  These values are bound to the identifiers to the left of the arrow,
which are in scope of the subsequent generator clause list and of the
body of the construct containing the generator clause list.

An expression of type \EXP{\TYP{Generator}\llbracket()\rrbracket} in a
generator clause list is interpreted as a \emph{filter}.  A generator
iteration is performed only if the filter yields \EXP{()}.
If the filter yields no value, subsequent expressions in
the generator clause list will not be evaluated.  Note in particular that \VAR{true} is a \TYP{Boolean} value yielding \EXP{()} exactly once, while \VAR{false} is a \TYP{Boolean} value that yields no elements.

The order of nesting of generators need not imply anything about the
relative order of nesting of iterations.  In most cases, multiple
generators can be considered equivalent to multiple nested loops.
However, the compiler will make an effort to choose the best possible
iteration order it can for a multiple-generator loop, and may even
combine generators together; there may be no such guarantee for nested
loops.  Thus loops with multiple generators are preferable to distinct nested loops in general.
Note that the early termination behavior of nested looping is subtly
different from a single multi-generator loop, since nested loops give
rise to nested thread groups; see \secref{early-termination}.

Each generator iteration of the innermost generator clause
corresponds to a \emph{body iteration}, or simply an \emph{iteration}
of the generator clause list.  Each iteration is run in its own implicit
thread.  Each expression in the generator clause list can be considered
to evaluate in a separate implicit thread.  Together these implicit
threads form a thread group.  Evaluation of an expression with generators
completes only when this thread group has completed.

Some common \TYP{Generator}s include:
\begin{quote}
\begin{tabular}{ll}
\EXP{l\COLONOP{}u}              & Any range expression \\
\EXP{a}                         & Array \VAR{a} generates its elements \\
\EXP{a.\VAR{indices}}           & The index set of array \VAR{a} \\
\EXP{\{0,1,2,3\}}               & The elements of an aggregate expression \\
\EXP{\VAR{sequential}(g)}       & A sequential version of generator \VAR{g}
\end{tabular}
\end{quote}
The generator \EXP{\VAR{sequential}(g)} forces the iterations using
values from \VAR{g} to be performed in order.  Every generator has an
associated \emph{natural order} which is the order obtained by
\EXP{sequential}.  For example, a sequential \KWD{for} loop starting
at $1$ and going to $n$ can be written as follows:
%for i <- sequential(1:n) do
%  \cdots
%end
\begin{Fortress}
\(\KWD{for} i \leftarrow \VAR{sequential}(1\COLONOP{}n) \KWD{do}\)\\
{\tt~~}\pushtabs\=\+\(  \cdots\)\-\\\poptabs
\(\KWD{end}\)
\end{Fortress}
The \VAR{sequential} generator respects generator clause list ordering; it
will always nest strictly inside preceding generator clauses and
outside succeeding ones.

Given a multidimensional array, the \VAR{indices} generator returns
a tuple of values, which can be bound by a tuple of variables to the
left of the arrow:
%(i,j) <- my2DArray.indices
\begin{Fortress}
\((i,j) \leftarrow \VAR{my2DArray}.\VAR{indices}\)
\end{Fortress}
The parallelism of a loop on this generator follows the spatial
distribution (discussed in \secref{distributions})
of \VAR{my2DArray} as closely as possible.
