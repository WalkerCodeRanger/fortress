%%%%%%%%%%%%%%%%%%%%%%%%%%%%%%%%%%%%%%%%%%%%%%%%%%%%%%%%%%%%%%%%%%%%%%%%%%%%%%%%
%   Copyright 2009, Oracle and/or its affiliates.
%   All rights reserved.
%
%
%   Use is subject to license terms.
%
%   This distribution may include materials developed by third parties.
%
%%%%%%%%%%%%%%%%%%%%%%%%%%%%%%%%%%%%%%%%%%%%%%%%%%%%%%%%%%%%%%%%%%%%%%%%%%%%%%%%

\section{Do Expressions}
\seclabel{block-expr}

\begin{Grammar}
\emph{Do} &::=& (\emph{DoFront} \KWD{also})$^*$ \emph{DoFront} \KWD{end}\\

\emph{DoFront} &::=& \options{\KWD{at} \emph{Expr}} \option{\KWD{atomic}} \KWD{do} \option{\emph{BlockElems}}\\

\emph{BlockElems} &::=& \emph{BlockElem}$^+$ \\

\emph{BlockElem}
&::=& \emph{LocalVarFnDecl}\\
&$|$& \emph{Expr}\options{\EXP{,} \emph{GeneratorClauseList}}\\

\emph{LocalVarFnDecl}
&::=& \emph{LocalFnDecl}$^+$\\
&$|$& \emph{LocalVarDecl}\\

\end{Grammar}

A \KWD{do} expression consists of a series of \emph{expression blocks}
separated by \KWD{also} and
terminated by \KWD{end}.  Each expression block is preceded by
an optional \KWD{at} expression (described in \secref{parallelism-fundamentals}),
an optional \KWD{atomic}, and
\KWD{do}.  When prefixed by \KWD{at} or \KWD{atomic}, it is as though
that expression block were evaluated as the body expression of an \KWD{at} or
\KWD{atomic} expression (described in \secref{atomic}), respectively.
An expression block
consists of a (possibly empty) series of \emph{elements}--expressions,
generated expressions (described in \secref{generated}),
local variable declarations, or local function declarations--separated
by newlines or semicolons.

A single expression block evaluates its elements in order: each
element must complete before evaluation of the next can begin, and the
expression block as a whole does not complete until the final element completes.
Each expression except the last element of the expression block
must have type \TYP{()}.
There are two ways to make an expression \VAR{e} of type non-\TYP{()}
have type \TYP{()} in an expression block:
1) ``\EXP{\VAR{ignore}(e)}'' and 2) ``\EXP{{\tt\_} = e}''.
If the last element of the expression block is an
expression, the value and type of this expression are the value and
type of the expression block as a whole.  Otherwise, the value and
type of the expression block is \EXP{()}.  Each expression block
introduces a new scope.  Some compound expressions have clauses that
are implicitly expression blocks.

Because a local declaration shares a syntax with an
\emph{equality testing expression}, we require that any equality testing
expression within an expression block be parenthesized.


Here are examples of function
declarations whose bodies are \KWD{do} expressions:
\input{\home/basic/examples/Expr.Do.f.tex}

\input{\home/basic/examples/Expr.Do.foo.tex}

\input{\home/basic/examples/Expr.Do.mySum.tex}

%%%%%%%%%%%%%%%%%%%%%%%%%%%%%%%%%%%%%%%%%%%%%%%%%%%%%%%%%%%%%%%%%%%%%%%%%%%%%%%%
%   Copyright 2009, Oracle and/or its affiliates.
%   All rights reserved.
%
%
%   Use is subject to license terms.
%
%   This distribution may include materials developed by third parties.
%
%%%%%%%%%%%%%%%%%%%%%%%%%%%%%%%%%%%%%%%%%%%%%%%%%%%%%%%%%%%%%%%%%%%%%%%%%%%%%%%%

\subsection{Generated Expressions}
\seclabel{generated}

If a subexpression of a \KWD{do} expression has type \TYP{()}, the
expression may be followed by a `\EXP{,}' and a generator clause list
(described in \secref{generators}).  Writing ``\EXP{\VAR{expr},
\VAR{gens}}'' is equivalent to writing ``\EXP{\KWD{for}\mskip 4mu plus
4mu\VAR{gens} \KWD{do}\mskip 4mu plus 4mu\VAR{expr} \KWD{end}}''.  See
\secref{for-expr} for the semantics of the \KWD{for} expression.  Note
in particular that \VAR{expr} can be a reducing assignment of the form
``\EXP{\VAR{variable}\; \mathrel{\OPR{OP}}= \VAR{expr}}''.


%%%%%%%%%%%%%%%%%%%%%%%%%%%%%%%%%%%%%%%%%%%%%%%%%%%%%%%%%%%%%%%%%%%%%%%%%%%%%%%%
%   Copyright 2009 Sun Microsystems, Inc.,
%   4150 Network Circle, Santa Clara, California 95054, U.S.A.
%   All rights reserved.
%
%   U.S. Government Rights - Commercial software.
%   Government users are subject to the Sun Microsystems, Inc. standard
%   license agreement and applicable provisions of the FAR and its supplements.
%
%   Use is subject to license terms.
%
%   This distribution may include materials developed by third parties.
%
%   Sun, Sun Microsystems, the Sun logo and Java are trademarks or registered
%   trademarks of Sun Microsystems, Inc. in the U.S. and other countries.
%%%%%%%%%%%%%%%%%%%%%%%%%%%%%%%%%%%%%%%%%%%%%%%%%%%%%%%%%%%%%%%%%%%%%%%%%%%%%%%%

\subsection{Parallel Do Expressions}
\seclabel{also-block}

\note{Reduction variables are not yet supported.}

A series of blocks may be run in parallel using the \KWD{also}
construct.  Any number of contiguous blocks may be joined together by
\KWD{also}.  Each block is run in a separate
implicit thread; these threads together form a group.  The expression
as a whole completes when the group is complete.
A thread can be placed in a particular region by using an \KWD{at} expression
as described in \secref{parallelism-fundamentals}.
When multiple
expression blocks are separated by \KWD{also}, each expression block must
have type \EXP{()}; the result and type of the parallel \KWD{do}
expression is also \EXP{()}.


For example:
\input{\home/basic/examples/Expr.Do.treeSum.tex}
Note the use of the reduction variable \VAR{accum}
(\secref{reduction-vars}) within the threads in the group.

