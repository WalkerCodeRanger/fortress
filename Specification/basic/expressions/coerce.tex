%%%%%%%%%%%%%%%%%%%%%%%%%%%%%%%%%%%%%%%%%%%%%%%%%%%%%%%%%%%%%%%%%%%%%%%%%%%%%%%%
%   Copyright 2009, Oracle and/or its affiliates.
%   All rights reserved.
%
%
%   Use is subject to license terms.
%
%   This distribution may include materials developed by third parties.
%
%%%%%%%%%%%%%%%%%%%%%%%%%%%%%%%%%%%%%%%%%%%%%%%%%%%%%%%%%%%%%%%%%%%%%%%%%%%%%%%%

\subsection{Coercion}
\seclabel{coercion-expr}

An identity function \EXP{\mathrm{coerce}} is defined in \library\
to convert the type of its argument to its type argument:
%% coerce_[\T\](x: T) = x
\begin{Fortress}
\(\mathrm{coerce}\llbracket{}T\rrbracket(x\COLON T) = x\)
\end{Fortress}
The function returns its argument as the given type.
Unlike coercions described in \chapref{conversions-coercions},
the \EXP{\mathrm{coerce}} function can apply to an argument whose type is
a subtype of the type being coerced to.
