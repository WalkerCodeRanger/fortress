%%%%%%%%%%%%%%%%%%%%%%%%%%%%%%%%%%%%%%%%%%%%%%%%%%%%%%%%%%%%%%%%%%%%%%%%%%%%%%%%
%   Copyright 2009, Oracle and/or its affiliates.
%   All rights reserved.
%
%
%   Use is subject to license terms.
%
%   This distribution may include materials developed by third parties.
%
%%%%%%%%%%%%%%%%%%%%%%%%%%%%%%%%%%%%%%%%%%%%%%%%%%%%%%%%%%%%%%%%%%%%%%%%%%%%%%%%

\section{Function Expressions}
\seclabel{func-expr}

\note{Syntax is not up to date.

Allow \KWD{io} function expressions?

Types in \KWD{throws} clause can't be naked type parameter?

Check that \emph{Expr} type is subtype of the declared return type.
}

\note{We want to allow function expressions to carry contracts; we need to work
out how these contracts are evaluated, and what the implications are on
variable scoping. -- Eric}

\begin{Grammar}
\emph{Expr} &::=&
\KWD{fn} \emph{ValParam} \option{\emph{IsType}} \option{\emph{Throws}} \EXP{\Rightarrow} \emph{Expr}\\
\end{Grammar}

Function expressions denote function values; they do not
require evaluation.  Syntactically,
they start with \KWD{fn} followed by a parameter,
optional return type,
optional \KWD{throws} clause,
$\Rightarrow$, and finally an expression.
The type of a function expression is an arrow type consisting of the
function's parameter type followed by the token \EXP{\rightarrow},
followed by the function's return type,
and the function's  optional \KWD{throws} clause.
Unlike declared functions (described in
\chapref{functions}), function expressions are not
allowed to include static parameters
nor \KWD{where} clauses (described in \chapref{trait-parameters}).


Here is a simple example:
\input{\home/basic/examples/Expr.FnExpr.tex}
