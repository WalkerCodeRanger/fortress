%%%%%%%%%%%%%%%%%%%%%%%%%%%%%%%%%%%%%%%%%%%%%%%%%%%%%%%%%%%%%%%%%%%%%%%%%%%%%%%%
%   Copyright 2009, Oracle and/or its affiliates.
%   All rights reserved.
%
%
%   Use is subject to license terms.
%
%   This distribution may include materials developed by third parties.
%
%%%%%%%%%%%%%%%%%%%%%%%%%%%%%%%%%%%%%%%%%%%%%%%%%%%%%%%%%%%%%%%%%%%%%%%%%%%%%%%%

\section{Ranges}
\seclabel{ranges}

\note{Static expressions, static range types, and some range operations are not yet supported.}

\begin{Grammar}
\emph{Range} &::=& \option{\emph{Expr}}\EXP{:}\option{\emph{Expr}}\options{\EXP{:}\option{\emph{Expr}}}\\
&$|$& \emph{Expr}\EXP{\#}\emph{Expr}
\end{Grammar}

%% :
%% ::
%% ::c
%% :e
%% :e:
%% :e:e
%%
%% e:
%% e::
%% e::e
%% e:e
%% e:e:
%% e:e:e


A \emph{range expression} is used to create a special kind of \TYP{Generator} for
a set of integers, called a \TYP{Range}, useful
for indexing an array or controlling a \KWD{for} loop.  Generators in
general are discussed further in \secref{generators}.

An \emph{explicit range}
is self-contained and completely describes a set of integers.  Assume that
$a$, $b$, and $c$ are expressions that produce integer values.

\begin{itemize}
\item The range \EXP{a\COLONOP{}b} is the set of $n=\max(0,b-a+1)$ integers $\{a, a+1, a+2, \ldots, b-2, b-1, b\}$.
This is a \emph{nonstrided} range.
If \VAR{a} and \VAR{b} are both static expressions
(described in \secref{static-expr}),
then it is a \emph{static range} of type \EXP{\TYP{StaticRange}\llbracket{}a,n,1\rrbracket}
and therefore also a \emph{range of static size} of type \EXP{\TYP{RangeOfStaticSize}\llbracket{}n\rrbracket}.

\item The range \EXP{a\COLONOP{}b\COLONOP{}c} is
the set of $n=\max\left(0, \left\lfloor {b-a+c \over c} \right\rfloor \right)$ integers
$\{a, a+c, a+2c, \ldots, a + \left\lfloor {b-a \over c} \right\rfloor c\}$,
unless $c$ is zero, in which case it throws an exception.
(If $c$ is a static expression, then it is a static error if $c$ is zero.)
This is a \emph{strided} range.
If \VAR{a}, \VAR{b}, and \VAR{c} are all static expressions,
then it is a \emph{static range} of type \EXP{\TYP{StaticRange}\llbracket{}a,n,c\rrbracket}
and therefore also a \emph{range of static size} of type \EXP{\TYP{RangeOfStaticSize}\llbracket{}n\rrbracket}.

\item The range \EXP{a\mathinner{\hbox{\tt\char'43}}n} is the set of
\EXP{\max(0,n)} integers $\{a, a+1, a+2, \ldots, a+n-3, a+n-2, a+n-1\}$.
This is a \emph{nonstrided} range.
If \VAR{a} and \VAR{n} are both static expressions,
then it is a \emph{static range} of type \EXP{\TYP{StaticRange}\llbracket{}a,n,1\rrbracket}
If \VAR{n} is a static expression, then it is a \emph{range of static size}
of type \EXP{\TYP{RangeOfStaticSize}\llbracket{}n\rrbracket},
even if \VAR{a} is not a static expression.
\end{itemize}

Non-static components of a range expression are computed in separate
implicit threads.  The range is constructed when all components have
completed normally.

An \emph{implicit range} may be used only in certain contexts, such as array subscripts,
that can supply implicit information.  Suppose an implicit range is used as a subscript
for an axis of an array for which the lower bound is $l$ and the upper bound is $u$.
\begin{itemize}
\item The implicit range \EXP{\COLONOP} is treated as \EXP{l\COLONOP{}u}.
\item The implicit range \EXP{\COLONOP\mathrel{\mathtt{:}}c} is
  treated as \EXP{l\COLONOP{}u\COLONOP{}c}.
\item The implicit range \EXP{\COLONOP{}b} is treated as \EXP{l\COLONOP{}b}.
\item The implicit range \EXP{\COLONOP{}b\COLONOP{}c} is treated as
\EXP{l\COLONOP{}b\COLONOP{}c}.
\item The implicit range \EXP{a\COLONOP} and \EXP{a\mathinner{\hbox{\tt\char'43}}} are treated as \EXP{a\COLONOP{}u}.
\item The implicit range \EXP{a\COLONOP\mathrel{\mathtt{:}}c} is treated as
\EXP{a\COLONOP{}u\COLONOP{}c}.
\item The implicit range \EXP{\mathinner{\hbox{\tt\char'43}}s} is treated as \EXP{l\mathinner{\hbox{\tt\char'43}}s}.
\end{itemize}

One may test whether an integer is in a range by using the operator $\in$:
\input{\home/basic/examples/Expr.Range.tex}

Ranges may be compared as if they were sets of integers by using
\EXP{\subset} (\STR{SUBSET}) and \EXP{\subseteq} (\STR{SUBSETEQ})
and \EXP{=} and \EXP{\supseteq} (\STR{SUPSETEQ}) and \EXP{\supset} (\STR{SUPSET}).

Ranges may be intersected using the operator \EXP{\cap} (\STR{INTERSECTION}).

The size of a range (the number of integers in the set)
may be found by using the set-cardinality operator
\EXP{\left|\ldots\right|}.  For example, the value of \EXP{\left|3\COLONOP{}7\right|} is \EXP{5}
and the value of \EXP{\left|1\COLONOP{}100\COLONOP{}2\right|} is \EXP{50}.

Note that a range is very different from an interval with integer endpoints.
The range \EXP{3\COLONOP{}5} contains only the values 3, 4, and 5,
whereas the interval \EXP{[3,5]} contains all real numbers \VAR{x} such that \EXP{3\leq{}x\leq{}5}.
