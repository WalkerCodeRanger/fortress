%%%%%%%%%%%%%%%%%%%%%%%%%%%%%%%%%%%%%%%%%%%%%%%%%%%%%%%%%%%%%%%%%%%%%%%%%%%%%%%%
%   Copyright 2009, Oracle and/or its affiliates.
%   All rights reserved.
%
%
%   Use is subject to license terms.
%
%   This distribution may include materials developed by third parties.
%
%%%%%%%%%%%%%%%%%%%%%%%%%%%%%%%%%%%%%%%%%%%%%%%%%%%%%%%%%%%%%%%%%%%%%%%%%%%%%%%%

\section{While Loops}
\seclabel{while-expr}

\begin{Grammar}
\emph{DelimitedExpr} &::=&
\KWD{while} \emph{GeneratorClause} \emph{Do} \\

\end{Grammar}

A \KWD{while} loop consists of a generator clause
(discussed in \secref{generators}) followed by a simple \KWD{do} expression
(see \secref{block-expr}).
An iteration of a \KWD{while} loop evaluates the expression of its generator
clause (\emph{condition expression}).  If the generator clause is a generator
binding, the type of the condition expression must be a subtype of
\EXP{\TYP{Condition}\llbracket{}E\rrbracket} for some \VAR{E}
and the type of the identifiers bound in the clause must be a subtype of \VAR{E}.
A \TYP{Condition} is a generator that generates 0 or 1 element.
Otherwise, the generator clause must be an expression of type \TYP{Boolean}.
Note in particular that \VAR{true} is a \TYP{Boolean} value yielding \EXP{()}
exactly once, while \VAR{false} is a \TYP{Boolean} value that yields no elements.
If the evaluation of the condition expression completes normally and generates
exactly one element, the body expression is evaluated.
When one iteration completes a new one is run until either an iteration
completes abruptly (in which case the evaluation of the \KWD{while} expression
completes abruptly), or the condition expression generates no elements
(in which case the \KWD{while} loop completes normally with value \EXP{()}).
