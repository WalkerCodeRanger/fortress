%%%%%%%%%%%%%%%%%%%%%%%%%%%%%%%%%%%%%%%%%%%%%%%%%%%%%%%%%%%%%%%%%%%%%%%%%%%%%%%%
%   Copyright 2009, Oracle and/or its affiliates.
%   All rights reserved.
%
%
%   Use is subject to license terms.
%
%   This distribution may include materials developed by third parties.
%
%%%%%%%%%%%%%%%%%%%%%%%%%%%%%%%%%%%%%%%%%%%%%%%%%%%%%%%%%%%%%%%%%%%%%%%%%%%%%%%%

\section{Comprehensions}
\seclabel{comprehensions}

\note{Array comprehensions are not yet supported.

Array comprehensions do not have to be comprehensive --
may not define a value for every cell.
When an unitialized array cell is accessed, an uncaught exception is raised.}

\begin{Grammar}
\emph{Comprehension}
&::=&
\option{\KWD{BIG}} \texttt{[} \option{\emph{StaticArgs}}
\emph{ArrayComprehensionClause}$^+$ \texttt{]}\\
&$|$&
\option{\KWD{BIG}} \{ \option{\emph{StaticArgs}}
\emph{Entry} \texttt{|} \emph{GeneratorClauseList} \} \\
&$|$& \option{\KWD{BIG}} \emph{LeftEncloser} \option{\emph{StaticArgs}}
\emph{Expr} \texttt{|} \emph{GeneratorClauseList} \emph{RightEncloser} \\

\emph{Entry} &::=& \emph{Expr} \ensuremath{\mapsto} \emph{Expr} \\

\emph{ArrayComprehensionClause} &::=&
\emph{ArrayComprehensionLeft} \texttt{|} \emph{GeneratorClauseList}\\

\emph{ArrayComprehensionLeft} &::=&
\emph{IdOrInt} \ensuremath{\mapsto} \emph{Expr}\\
&$|$& \texttt( \emph{IdOrInt}\EXP{,} \emph{IdOrIntList} \texttt) \ensuremath{\mapsto} \emph{Expr}\\

\emph{IdOrInt} &::=& \emph{Id}\\
&$|$& \emph{IntLiteralExpr}\\

\emph{IdOrIntList} &::=& \emph{IdOrInt}(\EXP{,} \emph{IdOrInt})$^*$ \\
\end{Grammar}

Fortress provides \emph{comprehension} syntax, in which a generator clause list
binds values used in the body expression on the left-hand side of the
token \EXP{|}.  As described in \secref{generators}, each iteration of
the body expression must be assumed to execute in its own implicit
thread.  Comprehensions evaluate to aggregate values and have
corresponding aggregate types.  The rules for evaluation of a
comprehension are similar to those for a reduction expression (see
\secref{reduction-expr}).

The relationship between a comprehension and an aggregate expression (\secref{aggregate-expr}) is similar to the relationship between a reduction expression (\secref{reduction-expr}) and the corresponding infix operator application (\secref{operator-app-expr}).  The language does not enforce an explicit connection between comprehension syntax and the corresponding aggregate syntax, but in practice libraries that provide definitions for aggregate expressions are expected to define a corresponding comprehension and \emph{vice versa}.  As with reduction expressions, a comprehension using a particular set of enclosers corresponds to a call to a \emph{big bracketing operator}.  Thus the definition of set comprehensions is given by a function with the following signature:
%% opr BIG {[\T\] g:(Reduction[\R0\],T->R0)->R0}: Set[\T\]
\begin{Fortress}
\(\KWD{opr} \mathord{\OPR{BIG}} \{\llbracket{}T\rrbracket g\COLONOP(\TYP{Reduction}\llbracket{}R_{0}\rrbracket,T\rightarrow{}R_{0})\rightarrow{}R_{0}\}\COLON \TYP{Set}\llbracket{}T\rrbracket\)
\end{Fortress}
This is almost identical to the signature required to define a reduction
expression \EXP{\OPR{BIG}\:\emph{Op}}, shown in \secref{reduction-expr}.
Further information on defining comprehensions are described
in \secref{defining-generators}.

A set comprehension is enclosed in braces, with a left-hand body
separated by the token \EXP{|} from a generator clause list.
For example, the comprehension:
\input{\home/basic/examples/Expr.SetComp.tex}
evaluates to the set
\begin{Fortress}
\(\{0, 4, 16\}\)
\end{Fortress}

Map comprehensions are like set comprehensions,
except that the left-hand body must be of the form
\EXP{e_{1} \mapsto e_{2}}.
If \EXP{e_{1}} produces the same value but \EXP{e_{2}} a different value on more
than one iteration of the generator list,
a \TYP{KeyOverlap} exception is thrown.
For example:
\input{\home/basic/examples/Expr.MapComp.tex}
evaluates to the map
\begin{Fortress}
\(\{0 \mapsto 0, 4 \mapsto 8, 16 \mapsto 64\}\)
\end{Fortress}


List comprehensions are like set comprehensions,
except that they are syntactically enclosed in angle brackets.
For example:
\input{\home/basic/examples/Expr.ListComp.tex}
evaluates to the list
\begin{Fortress}
\(\langle 0, 4, 16 \rangle\)
\end{Fortress}
Note that the order of elements in the resulting list corresponds to
the \emph{natural order} of the generators in the generator clause
list (see \secref{defining-generators}).

Array comprehensions are like set comprehensions, except that they are
syntactically enclosed in brackets, and the left-hand body must be of
the form \txt{($\mathit{index}_1$, $\mathit{index}_2$, \ldots,
  $\mathit{index}_n$) $\mapsto$ $\mathit{e}$}.  Moreover an array
comprehension may have multiple clauses separated by semicolons or
line breaks.  Each clause conceptually corresponds to an independent loop.
Clauses are run in order.  The result is an $n$-dimensional array.
For example:
\begin{Fortress}
\(a = [\null\)\pushtabs\=\+\((x,y,1) \mapsto 0.0   \mid x \gets 1\COLONOP\VAR{xSize}, y \gets 1\COLONOP\VAR{ySize}\)\\
\(      (1,y,z) \mapsto 0.0   \mid y \gets 1\COLONOP\VAR{ySize}, z \gets 2\COLONOP\VAR{zSize}\)\\
\(      (x,1,z) \mapsto 0.0   \mid x \gets 2\COLONOP\VAR{xSize}, z \gets 2\COLONOP\VAR{zSize}\)\\
\(      (x,y,z) \mapsto x+y\cdot z \mid x \gets 2\COLONOP\VAR{xSize}, y \gets 2\COLONOP\VAR{ySize}, z \gets 2\COLONOP\VAR{zSize}\,]\)\-\\\poptabs
\end{Fortress}
