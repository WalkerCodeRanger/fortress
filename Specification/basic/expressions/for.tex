%%%%%%%%%%%%%%%%%%%%%%%%%%%%%%%%%%%%%%%%%%%%%%%%%%%%%%%%%%%%%%%%%%%%%%%%%%%%%%%%
%   Copyright 2009, Oracle and/or its affiliates.
%   All rights reserved.
%
%
%   Use is subject to license terms.
%
%   This distribution may include materials developed by third parties.
%
%%%%%%%%%%%%%%%%%%%%%%%%%%%%%%%%%%%%%%%%%%%%%%%%%%%%%%%%%%%%%%%%%%%%%%%%%%%%%%%%

\section{For Loops}
\seclabel{for-expr}

\note{Reduction variables are not yet supported.}

\begin{Grammar}
\emph{DelimitedExpr} &::=&
\KWD{for} \emph{GeneratorClauseList} \emph{DoFront} \KWD{end}\\

\emph{DoFront} &::=& \options{\KWD{at} \emph{Expr}} \option{\KWD{atomic}} \KWD{do} \option{\emph{BlockElems}}\\

\end{Grammar}

A \KWD{for} loop consists of \KWD{for}
followed by a generator clause list (discussed in \secref{generators}),
followed by a non-parallel \KWD{do} expression (the loop \emph{body};
see \secref{block-expr}).  Parallelism in \KWD{for} loops is specified
by the generators used (see \secref{generators}); in general the
programmer must assume that each loop iteration will occur
independently in parallel unless every generator is explicitly
\VAR{sequential}.  For each iteration, the body expression is
evaluated in the scope of the values bound by the generators.
The body of a \KWD{for} expression can make use of reduction variables as
dscribed in \secref{reduction-vars}.
The value and type of a \KWD{for} loop is \EXP{()}.
