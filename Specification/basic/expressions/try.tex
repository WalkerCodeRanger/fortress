%%%%%%%%%%%%%%%%%%%%%%%%%%%%%%%%%%%%%%%%%%%%%%%%%%%%%%%%%%%%%%%%%%%%%%%%%%%%%%%%
%   Copyright 2009, Oracle and/or its affiliates.
%   All rights reserved.
%
%
%   Use is subject to license terms.
%
%   This distribution may include materials developed by third parties.
%
%%%%%%%%%%%%%%%%%%%%%%%%%%%%%%%%%%%%%%%%%%%%%%%%%%%%%%%%%%%%%%%%%%%%%%%%%%%%%%%%

\section{Try Expressions}
\seclabel{try-expr}

\note{Chained exceptions are not yet supported.}

\begin{Grammar}
\emph{DelimitedExpr}
&::=& \KWD{try} \emph{BlockElems} \option{\emph{Catch}}
\options{\KWD{forbid} \emph{TraitTypes}}
\options{\KWD{finally} \emph{BlockElems}} \KWD{end} \\
\emph{Catch} &::=& \KWD{catch} \emph{BindId} \emph{CatchClauses}\\

\emph{CatchClauses} &::=& \emph{CatchClause}$^+$\\

\emph{CatchClause} &::=& \emph{TraitType} \EXP{\Rightarrow} \emph{BlockElems} \\
\end{Grammar}


A \KWD{try} expression starts with \KWD{try}
followed by an expression block (the \emph{\KWD{try} block}),
followed by an optional \KWD{catch} clause, an optional \KWD{forbid}
clause, an optional \KWD{finally} clause, and finally
\KWD{end}.  A \KWD{catch} clause consists of
\KWD{catch} followed by an identifier, followed by a sequence
of subclauses (each consisting of an exception type followed by the token
\EXP{\Rightarrow} followed by an expression block).
A \KWD{forbid} clause consists of \KWD{forbid}
followed by a set of exception types.
A \KWD{finally} clause consists of \KWD{finally}
followed by an expression block.
Note that the \KWD{try} block and the clauses
form expression blocks and have the various properties of
expression blocks (described in \secref{block-expr}).


The expressions in the \KWD{try} block are first evaluated in order
until they have all completed normally, or until one of them completes
abruptly.  If the \KWD{try} block completes normally, the
\emph{provisional} value of the \KWD{try} expression is the value of the
last expression in the \KWD{try} block.  In this case, and in case of
exiting to an enclosing \KWD{label} expression, the \KWD{catch} and
\KWD{forbid} clauses are ignored.


If an expression in the \KWD{try} block completes abruptly by throwing
an exception, the exception value is bound to the identifier specified in
the \KWD{catch} clause, and the type of the exception is matched
against the subclauses of the \KWD{catch} clause in turn, exactly as in a
\KWD{typecase} expression (\secref{typecase}).
The right-hand-side expression block
of the first matching subclause is evaluated.  If it
completes normally, its value is the provisional value of the
\KWD{try} expression.  If the \KWD{catch} clause completes abruptly,
the \KWD{try} expression completes abruptly.  If a thrown exception is
not matched by the \KWD{catch} clause (or this clause is omitted), but
it is a subtype of the exception type listed in a
\KWD{forbid} clause, a new
\TYP{ForbiddenException} is created with the thrown exception as its
argument and thrown.
The exception thrown by the \KWD{try} block is \emph{chained} to the
\TYP{ForbiddenException} as described in
\secref{chained-exceptions}.


If an exception thrown from a \KWD{try} block is matched by both
\KWD{catch} and \KWD{forbid} clauses, the exception is caught by the
\KWD{catch} clause.
If an exception thrown from a \KWD{try} block is not matched by any
\KWD{catch} or \KWD{forbid} clause, the \KWD{try} expression completes
abruptly.


The expression block of the \KWD{finally} clause
is evaluated after completion of the \KWD{try} block
and any \KWD{catch} or \KWD{forbid} clause.
The type of this expression block must be ().
The expressions in the \KWD{finally} clause are evaluated in order until
they have all completed normally, or until one of them completes abruptly.
In the latter case, the \KWD{try} expression completes abruptly exactly as
the subexpression in the \KWD{finally} clause does.
If the \KWD{finally} clause completes abruptly by throwing an exception,
any thrown exception earlier in the \KWD{try} expression is chained to the
exception thrown by the \KWD{finally} clause as described in
\secref{chained-exceptions}.


If the \KWD{finally} clause completes normally,
and the \KWD{try} block or the \KWD{catch}
clause completes normally, then the \KWD{try} expression completes
normally with the provisional value of the \KWD{try} expression.
Otherwise, the \KWD{try} expression completes abruptly as specified above.
The type of a \KWD{try} expression is the union
of the type of the \KWD{try} block
and the types of all the right-hand sides of the \KWD{catch} clauses.

For example, the following \KWD{try} expression:
\input{\home/basic/examples/Expr.Try.a.tex}
is equivalent to:
\input{\home/basic/examples/Expr.Try.b.tex}
The following example ensures that \VAR{file} is closed properly even if an
IO error occurs:
\input{\home/basic/examples/Expr.Try.c.tex}
