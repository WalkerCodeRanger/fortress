%%%%%%%%%%%%%%%%%%%%%%%%%%%%%%%%%%%%%%%%%%%%%%%%%%%%%%%%%%%%%%%%%%%%%%%%%%%%%%%%
%   Copyright 2009, Oracle and/or its affiliates.
%   All rights reserved.
%
%
%   Use is subject to license terms.
%
%   This distribution may include materials developed by third parties.
%
%%%%%%%%%%%%%%%%%%%%%%%%%%%%%%%%%%%%%%%%%%%%%%%%%%%%%%%%%%%%%%%%%%%%%%%%%%%%%%%%

\section{Case Expressions}
\seclabel{case-expr}

\begin{Grammar}
\emph{DelimitedExpr}&::=&
\KWD{case} \emph{Expr} \option{(\emph{Encloser} $|$ \emph{Op})} \KWD{of} \emph{CaseClauses}
\option{\emph{CaseElse}} \KWD{end} \\

\emph{CaseClauses} &::=& \emph{CaseClause}$^+$\\

\emph{CaseClause} &::=& \emph{Expr} \EXP{\Rightarrow} \emph{BlockElems}\\

\emph{CaseElse} &::=& \KWD{else} \EXP{\Rightarrow} \emph{BlockElems}\\

\end{Grammar}

A \KWD{case} expression begins with \KWD{case}
followed by a condition expression, followed by an optional operator,
\KWD{of}, a sequence of case clauses (each consisting
of a \emph{guarding expression} followed by the token $\Rightarrow$,
followed by an expression block), an optional
\KWD{else} clause (consisting of
\KWD{else} followed by the token $\Rightarrow$,
 followed by an expression block), and finally \KWD{end}.

A \KWD{case} expression evaluates its condition expression and checks
each case clause to determine which case clause matches.  To find a
matched case clause, the guarding expression of each case clause is
evaluated in order and compared to the value of the condition
expression.  For the first clause, the condition expression and
guarding expression are evaluated in separate implicit threads; for
subsequent clauses the value of the condition expression is retained
and only the guarding expression is evaluated.  Once both guard and
condition expressions have completed normally, the two values are
compared according to an optional operator specified.  If the
operator is omitted, it defaults to \EXP{=} or $\in$.
If the type of the guarding expression is a subtype of type \TYP{Contains}
and the condition expression does not,
the default operator is $\in$;
otherwise, it is \EXP{=}.
It is a static error
if the specified operator is not defined for these types
or if the operator's return type is not \TYP{Boolean}.

If the operator application completes normally and returns \VAR{true}, the
corresponding expression block is evaluated (see \secref{block-expr})
and its value is returned.  If the operator application returns \VAR{false},
matching continues with the next clause.
If no matched clause is found, a \TYP{MatchFailure} exception is thrown.
The optional \KWD{else} clause always matches without requiring a comparison.
The value of a \KWD{case} expression is the value of the right-hand side
of the matched clause.  The type of a \KWD{case} expression is the
union of the types of all right-hand sides of the case clauses.

For example, the following \KWD{case} expression specifies the operator
$\in$:
\input{\home/basic/examples/Expr.Case.a.tex}
but the following does not:
\input{\home/basic/examples/Expr.Case.b.tex}



\section{Extremum Expressions}
\seclabel{extremum-expr}

\note{Reduction variables are not yet supported.}

\begin{Grammar}
\emph{DelimitedExpr}&::=&
\KWD{case} \KWD{most} (\emph{Encloser} $|$ \emph{Op}) \KWD{of}
\emph{CaseClauses} \KWD{end}\\
\emph{CaseClauses} &::=& \emph{CaseClause}$^+$\\

\emph{CaseClause} &::=& \emph{Expr} \EXP{\Rightarrow} \emph{BlockElems}\\
\end{Grammar}

An extremum expression uses the same syntax as a \KWD{case} expression
(described in \secref{case-expr}) except that \KWD{most} is used
where a \KWD{case} expression would have a condition expression,
the specified operator is not optional,
and an extremum expression does not have an optional \KWD{else} clause.

All guarding expressions are evaluated in parallel in separate implicit threads
as part of the same group in which the guarding expressions themselves
are evaluated, in a manner analogous to reduction (see \secref{reduction-vars}).
The values of the guarding expressions are
compared in parallel according to the operator specified.
The specified operator must be a total order operator.
Which pairs of guarding expressions are
compared is unspecified, except that the pairwise comparisons
performed will be sufficient to determine that the chosen clause is
indeed the extremum (largest or smallest depending on the specified operator)
assuming a total order.  Any or all pairwise comparisons may be considered.


The expression block of the clause with the extremum
guarding expression (and only that clause) is evaluated.  If more than
one guarding expressions are tied for the extremum, the first
clause in textual order is evaluated to yield the result of the
extremum expression.  The type of an extremum expression is the union
of the types of all right-hand sides of the clauses.

For example, the following code:
\input{\home/basic/examples/Expr.Extremum.tex}
evaluates to \EXP{\hbox{\rm``\STR{miles~are~larger}''}}.
