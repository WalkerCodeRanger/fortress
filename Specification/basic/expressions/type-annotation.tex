%%%%%%%%%%%%%%%%%%%%%%%%%%%%%%%%%%%%%%%%%%%%%%%%%%%%%%%%%%%%%%%%%%%%%%%%%%%%%%%%
%   Copyright 2009 Sun Microsystems, Inc.,
%   4150 Network Circle, Santa Clara, California 95054, U.S.A.
%   All rights reserved.
%
%   U.S. Government Rights - Commercial software.
%   Government users are subject to the Sun Microsystems, Inc. standard
%   license agreement and applicable provisions of the FAR and its supplements.
%
%   Use is subject to license terms.
%
%   This distribution may include materials developed by third parties.
%
%   Sun, Sun Microsystems, the Sun logo and Java are trademarks or registered
%   trademarks of Sun Microsystems, Inc. in the U.S. and other countries.
%%%%%%%%%%%%%%%%%%%%%%%%%%%%%%%%%%%%%%%%%%%%%%%%%%%%%%%%%%%%%%%%%%%%%%%%%%%%%%%%

\section{Type Ascription}
\seclabel{type-ascription}

\begin{Grammar}
\emph{Expr} &::=& \emph{Expr} \KWD{as} \emph{Type}
\end{Grammar}

An expression consisting of a single subexpression, followed by
\KWD{as}, followed by a type, is a \emph{type ascription}.  The value
of the expression is the value of the subexpression.  The static type
of the expression is the ascribed type.
The type of the subexpression must be a subtype of the ascribed type.
A type ascription does not affect the dynamic type of the value the
expression evaluates to
(unlike a type assumption described in \secref{type-assumption}).
Type ascription can be used to provide
type information when type inference (see \chapref{type-inference})
cannot infer a type for an expression.


\section{Type Assumption}
\seclabel{type-assumption}

\begin{Grammar}
\emph{Expr} &::=& \emph{Expr} \KWD{asif} \emph{Type}
\end{Grammar}

An expression consisting of a single subexpression, followed by
\KWD{asif}, followed by a type, is a
\emph{type assumption}.
The value of the expression is the value of the subexpression.
The static type of the expression is the given type.
The type of the subexpression must be a subtype of the given type.
A type assumption considers both the static and the dynamic type of the value
of the expression to be the specified type for the purposes of the
immediately enclosing function, method, or operator invocation or
field access.  This is in contrast to type ascription, which only
gives a \emph{static} type to an expression.  Type assumption is used
to access a method provided by a supertrait when multiple supertraits
provide different methods with the same name.  Fortress thus provides
a richer version of type assumption operations such as \texttt{super}
in the Java Programming Language.
