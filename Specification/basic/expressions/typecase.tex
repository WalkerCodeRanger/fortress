%%%%%%%%%%%%%%%%%%%%%%%%%%%%%%%%%%%%%%%%%%%%%%%%%%%%%%%%%%%%%%%%%%%%%%%%%%%%%%%%
%   Copyright 2009, Oracle and/or its affiliates.
%   All rights reserved.
%
%
%   Use is subject to license terms.
%
%   This distribution may include materials developed by third parties.
%
%%%%%%%%%%%%%%%%%%%%%%%%%%%%%%%%%%%%%%%%%%%%%%%%%%%%%%%%%%%%%%%%%%%%%%%%%%%%%%%%

\section{Typecase Expressions}
\seclabel{typecase}

\note{This section should be revised according to the changes in the pattern matching proposal.}

\note{Do we want to support \EXP{\KWD{typecase} \mathord{\KWD{self}} \KWD{of} \ldots}?

How about \EXP{\KWD{typecase} (x, \KWD{self}, y) \KWD{of} \ldots}?

[07/23/2007] \EXP{\mathord{\KWD{self}}} is not allowed in a shorthand form of
\KWD{typecase} expresiosn.
% foo(x) = 8
% bar(x) = 3
% trait Parent
%    foo(x) = 10
%    baz(x) =
%      typecase self of
%        Child => bar(x)
%        Parent => 23
%      end
% end
% trait Child extends Parent
%    bar(x) = foo(x)
% end
\begin{Fortress}
{\tt~}\pushtabs\=\+\( \VAR{foo}(x) = 8\)\\
\( \VAR{bar}(x) = 3\)\\
\( \KWD{trait}\:\TYP{Parent}\)\\
{\tt~~~}\pushtabs\=\+\(    \VAR{foo}(x) = 10\)\\
\(    \VAR{baz}(x) =\)\\
{\tt~~}\pushtabs\=\+\(      \KWD{typecase} \mathord{\KWD{self}} \KWD{of}\)\\
{\tt~~}\pushtabs\=\+\(        \TYP{Child} \Rightarrow \VAR{bar}(x)\)\\
\(        \TYP{Parent} \Rightarrow 23\)\-\\\poptabs
\(      \KWD{end}\)\-\-\\\poptabs\poptabs
\( \KWD{end}\)\\
\( \KWD{trait}\:\TYP{Child} \KWD{extends}\:\TYP{Parent}\)\\
{\tt~~~}\pushtabs\=\+\(    \VAR{bar}(x) = \VAR{foo}(x)\)\-\\\poptabs
\( \KWD{end}\)\-\\\poptabs
\end{Fortress}
}

\begin{Grammar}
\emph{DelimitedExpr}
&::=& \KWD{typecase} \emph{TypecaseBindings} \KWD{of}
\emph{TypecaseClauses} \option{\emph{CaseElse}} \KWD{end}\\

\emph{TypecaseBindings} &::=& \emph{TypecaseVars}
\options{\EXP{=} \emph{Expr}}\\

\emph{TypecaseVars} &::=& \emph{BindId} \\
&$|$& \texttt{(} \emph{BindId}(\EXP{,} \emph{BindId})$^+$ \texttt{)} \\

\emph{TypecaseClauses} &::=& \emph{TypecaseClause}$^+$\\

\emph{TypecaseClause} &::=&
\emph{TypecaseTypes} \EXP{\Rightarrow} \emph{BlockElems} \\

\emph{TypecaseTypes}
&::=& \texttt{(} \emph{TypeList} \texttt{)}\\
&$|$& \emph{Type} \\

\emph{CaseElse} &::=& \KWD{else} \EXP{\Rightarrow} \emph{BlockElems}\\

\end{Grammar}

A \KWD{typecase} expression begins with \KWD{typecase} followed by
an identifier or a sequence of parenthesized identifiers,
optionally followed by the token \EXP{=} and a \emph{binding} expression,
followed by \KWD{of}, a sequence of typecase clauses
(each consisting of a sequence of \emph{guarding types}
followed by the token \EXP{\Rightarrow}, followed by an expression block),
an optional \KWD{else} clause
(consisting of \KWD{else} followed by the token \EXP{\Rightarrow},
followed by an expression block),
and finally \KWD{end}.

A \KWD{typecase} expression with a binding expression
evaluates the expression first;
if it completes normally, the value of the expression is bound
to the identifiers and the first matching typecase clause is chosen.
If there are multiple identifiers, the binding expression must be evaluated
to a tuple value of the same number of elements.
A typecase clause matches if the type of the value bound to each identifier
is a subtype of the corresponding type in the clause.
The expression block of the first matched clause
(and only that clause) is evaluated (see \secref{block-expr}) to yield
the value of the \KWD{typecase} expression.
If no matched clause is found, a \TYP{MatchFailure} exception is thrown
(\TYP{MatchFailure} is an unchecked exception).
Unlike bindings in other contexts,
the static type of such an identifier
is \emph{not} determined by the static type of
the expression it is bound to.
If the static type of a subexpression
in the bindings for a \KWD{typecase} expression has type \VAR{T},
when typechecking each typecase clause,
the static type of the corresponding identifier is the intersection of \VAR{T}
and the corresponding guarding type for that clause.
The type of a \KWD{typecase} expression is the union
of types of all right-hand sides of the typecase clauses.


\note{Q. [Victor] In a typecase expression, suppose a more specific clause
appears after a less specific one?  Do we want that to be a static error
(as it never makes sense to write that later clause since it will never
be evaluated)?

A. [Sukyoung \& Eric] No.  We may want to use the feature for reasoning
and debugging in a devlopment environment.  Some IDE may give a warning.}

For example:
\input{\home/basic/examples/Expr.Typecase.tex}
Note that \VAR{x} has a different type in each clause.

For a \KWD{typecase} expression without a binding expression,
the identifiers must be immutable variables in scope at that point.
In that case, the \KWD{typecase} expression is equivalent to one
that rebinds the variables to their values,
with the new bindings shadowing the old ones:
the first matching typecase clause is evaluated,
and within that clause, the static type of the variable
is the intersection of its original type and the guarding type.
Otherwise, it is a static error.

\note{
 \begin{itemize}
 \item \KWD{typecase} shorthand form with a mutable variable:
%% typecase x of     (* suppose x has static type Object *)
%%     String => x
%%     else => x.toString()
%%   end
\begin{Fortress}
\(\KWD{typecase}\:x \KWD{of}     \mathtt{(*}\;\hbox{\rm  suppose x has static type Object \unskip}\;\mathtt{*)}\)\\
{\tt~~~~}\pushtabs\=\+\(    \TYP{String} \Rightarrow x\)\\
\(    \KWD{else} \Rightarrow x.\VAR{toString}()\)\-\\\poptabs
{\tt~~}\pushtabs\=\+\(  \KWD{end}\)\-\\\poptabs
\end{Fortress}
Forbid this form if \VAR{x} is a mutable variable.
We need to give an explanation with a concrete example why we disallow this. In general, we have a rule against rebinding an expression in a local scope.
 \item \EXP{\mathord{\KWD{self}}} is not allowed in a shorthand form of
 \KWD{typecase} expresiosn:
%% foo(x) = 8
%% bar(x) = 3
%% trait Parent
%%    foo(x) = 10
%%    baz(x) =
%%      typecase self of
%%        Child => bar(x)
%%        Parent => 23
%%      end
%% end
%% trait Child extends Parent
%%    bar(x) = foo(x)
%% end
\begin{Fortress}
\(\VAR{foo}(x) = 8\)\\
\(\VAR{bar}(x) = 3\)\\
\(\KWD{trait}\:\TYP{Parent}\)\\
{\tt~~~}\pushtabs\=\+\(   \VAR{foo}(x) = 10\)\\
\(   \VAR{baz}(x) =\)\\
{\tt~~}\pushtabs\=\+\(     \KWD{typecase} \mathord{\KWD{self}} \KWD{of}\)\\
{\tt~~}\pushtabs\=\+\(       \TYP{Child} \Rightarrow \VAR{bar}(x)\)\\
\(       \TYP{Parent} \Rightarrow 23\)\-\\\poptabs
\(     \KWD{end}\)\-\-\\\poptabs\poptabs
\(\KWD{end}\)\\
\(\KWD{trait}\:\TYP{Child} \KWD{extends}\:\TYP{Parent}\)\\
{\tt~~~}\pushtabs\=\+\(   \VAR{bar}(x) = \VAR{foo}(x)\)\-\\\poptabs
\(\KWD{end}\)
\end{Fortress}
\end{itemize}
}
