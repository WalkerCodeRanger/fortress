%%%%%%%%%%%%%%%%%%%%%%%%%%%%%%%%%%%%%%%%%%%%%%%%%%%%%%%%%%%%%%%%%%%%%%%%%%%%%%%%
%   Copyright 2009, Oracle and/or its affiliates.
%   All rights reserved.
%
%
%   Use is subject to license terms.
%
%   This distribution may include materials developed by third parties.
%
%%%%%%%%%%%%%%%%%%%%%%%%%%%%%%%%%%%%%%%%%%%%%%%%%%%%%%%%%%%%%%%%%%%%%%%%%%%%%%%%

\section{Literals}
\seclabel{literals}

\note{We need to check whether the various types and values defined in this section are still correct.}

\begin{Grammar}
\emph{LiteralExpr} &::=& \texttt{(} \texttt{)}\\
&$|$& \emph{NumericLiteralExpr}\\
&$|$& \emph{CharLiteralExpr}\\
&$|$& \emph{StringLiteralExpr}\\
\end{Grammar}

Fortress provides boolean literals, \TYP{()} literal, character literals,
string literals, and numeric literals.  Literals are values; they do not
require evaluation.

\note{From here, copied from F1.0$\beta$.}

The literal \VAR{false} has type
\EXP{\TYP{BooleanLiteral}\llbracket\VAR{false}\rrbracket}.
The literal \VAR{true} has type
\EXP{\TYP{BooleanLiteral}\llbracket\VAR{true}\rrbracket}.
\note{And these are the only values of those types.}

The literal \EXP{()} is the only value with type \TYP{()}.
Whether any given occurrence of \TYP{()} refers to the value \EXP{()} or
to the type \TYP{()} is determined by context.

A character literal has type \TYP{Character}.
Each character literal consists of an abstract character in
\unicode~\cite{Unicode},
enclosed in single quotation marks (for example, `\txt{a}', `\txt{A}',
`\txt{\$}', `\txt{$\alpha$}', `\txt{$\oplus$}').
\marginnote{This is not true: Character literals include ``\b'' and
``LATIN CAPITAL A'', which do not ``consist of'' an abstract character enclosed in
single quotation marks.  I think it's better to remove most of this discussion and
direct the reader to the relevant section of Lexical Structure.
This section should be about {\bf semantics} of literals:
their values and types.}
The single quotes may be either true typographical ``curly'' single quotation marks
or a pair of ordinary apostrophe characters (for example, \txt{'a'}, \txt{'A'}, \txt{'\$'}, \txt{'$\alpha$'}, \txt{'$\oplus$'}).
See \secref{preprocessing-character-literals} for a description of how names of
characters may be used rather than actual characters within character literals,
for example `\txt{APOSTROPHE}' and `\txt{GREEK\_CAPITAL\_LETTER\_GAMMA}'.

A string literal has type \TYP{String}.
Each string literal is a sequence of \unicode\ characters enclosed in double
quotation marks (for example,
``\txt{Hello, world!}'' or ``\txt{$\pi\thinspace r^2$}'').
\marginnote{Again, I think this paragraph should mostly be eliminated and replaced with a pointer to the relevant section of Lexical Structure.

Here's a sentence I wrote as a replacement:
``A string literal (i.e., a sequence of characters enclosed in double quotation marks---see Section 5.10 for details) has type String.''}
The double quotes may be either true typographical ``curly'' double  quotation marks
or a pair of ``neutral'' double-quote characters (for example,
\txt{"Hello, world!"} or \txt{"$\pi\thinspace r^2$"}).
\secref{preprocessing-string-literals} also describes how names of
characters may be used rather than actual characters within string literals.
One may also use the escape sequences \txt{{\char'134}b} and \txt{{\char'134}t}
and \txt{{\char'134}n} and \txt{{\char'134}f} and \txt{{\char'134}r} as described in \cite{JLS}.


\note{We may want to eliminate the distinction between simple and compound numerals,
and just distinguish them by whether they have a radix point or not, when necessary
(i.e., ``Numerals may have a radix point.'').  Again,
that distinction is primarily a syntactic one, and can be discussed there if appropriate.}
Numeric literals in Fortress are referred to as \emph{numerals},
corresponding to various expressible numbers.
Numerals may be either \emph{simple} or \emph{compound}
(as described in \secref{numerals}).

A numeral containing only digits (let $n$ be the number of digits)
has type \EXP{\TYP{NaturalNumeral}\llbracket{}n,10,v\rrbracket} where $v$ is the value of the numeral
interpreted in radix ten.  If the numeral has no leading zeros, or is
the literal \EXP{0}, then it also has type \EXP{\TYP{Literal}\llbracket{}v\rrbracket}.
\note{Can the type \EXP{\TYP{NaturalNumeral}\llbracket{}n, 10, v\rrbracket}
have any value other than what you get from evaluating the corresponding literal?

What is the relationship between \EXP{\TYP{Literal}\llbracket{}v\rrbracket}
and \EXP{\TYP{NaturalNumeral}\llbracket{}n, 10, v\rrbracket}?

We need to describe the \TYP{Numeral} type hierarchy.
All the numeric/numeral types should be defined and their relations to each other given.
And a reference to where that is done should be appear in this section (probably at the beginning of the section).
}

\note{Ought we discuss digit-group separators here?
Or is that just part of lexical structure?
(I think we intend that a numeral consisting of digits and digit-group separators has
type \EXP{\TYP{NaturalNumeral}\llbracket{}n,10,v\rrbracket}
for appropriate values of \VAR{n} and \VAR{v}.)}

A numeral containing only digits (let $n$ be the number of
digits) then an underscore, then a radix indicator (let $r$ be the
radix) has type \EXP{\TYP{NaturalNumeralWithExplicitRadix}\llbracket{}n,r,v\rrbracket}
where $v$ is
the value of the $n$-digit numeral interpreted in radix $r$.

A numeral containing only digits (let $n$ be the total number
of digits) and a radix point (let $m$ be the number of digits after
the radix point) has type
\EXP{\TYP{RadixPointNumeral}\llbracket{}n,m,10,v\rrbracket} where $v$ is
the value of the numeral, with the radix point deleted,
interpreted in radix ten.

A numeral containing only digits (let $n$ be the total number
of digits) and a radix point (let $m$ be the number of digits after
the radix point), then an underscore, then a radix indicator (let $r$
be the radix) has the following type:
\begin{Fortress}
\EXP{\TYP{RadixPointNumeralWithExplicitRadix}\llbracket{}n,m,r,v\rrbracket}
\end{Fortress}
where $v$ is the value of the $n$-digit numeral, with the radix point
deleted, interpreted in radix $r$.

Every numeral also has type
\EXP{\TYP{Numeral}\llbracket{}n,m,r,v\rrbracket}
for appropriate values of \VAR{n}, \VAR{m}, \VAR{r}, and \VAR{v}.


Numerals are not directly converted to any of the number types
because, as in common mathematical usage,
we expect them to be polymorphic.
For example, consider the numeral \EXP{3.1415926535897932384};
converting it immediately to a floating-point number
may lose precision.
If that numeral is used in an expression
involving floating-point intervals,
it would be better to convert it directly to an interval.
Therefore, numerals have their own types as described above.
This approach allows library designers to decide how numerals should
interact with other types of objects by defining coercion operations
(see \secref{coercion} for an explanation of coercion
in Fortress).
\Library\ define coercions
from numerals to integers (for simple numerals)
and rational numbers (for compound numerals).


\marginnote{What is an ``integer'' in this context?}
In Fortress, dividing two integers using the $/$ operator produces
a \emph{rational number};\marginnote{Not if the second integer is 0.}
this is true regardless of whether
the integers are of type $\mathbb{Z}$ (or \txt{ZZ}),
$\mathbb{Z}64$ (or \txt{ZZ64}), $\mathbb{N}32$ (or \txt{NN32}), or
whatever.  Addition, subtraction, multiplication, and division of rationals
are always exact; thus values such as \EXP{1/3} are represented exactly in
Fortress.

\note{We should give an example with a radix specifier.}
Numerals containing a radix point are actually rational
literals; thus \EXP{3.125} has the rational value \EXP{3125/1000}.
The quotient of two integer literals is a static expression
(described in \secref{static-expr}) whose value is rational.
Similarly, a sequence of digits with a radix point followed by the symbol
$\times$ and an integer literal raised to an integer literal,
such as \EXP{6.0221415 \times 10^{23}}
is a static expression whose value is rational.
If such static expressions are mentioned as part of a floating-point
computation, the compiler performs the rational arithmetic exactly and then
converts the result to a floating-point value, thus incurring
at most one floating-point rounding error.
But in general rational computations may also be performed at run time,
not just at compile time.
\note{This specifies too much about the implementation:
I don't think we should mention a compiler at all, except perhaps as informal discussion.
In particular, wouldn't it be correct if this arithmetic were actually deferred and
performed exactly at run time?}


A rational number can be thought of as a pair of integers $p$ and $q$
that have been reduced to ``standard form in lowest terms'';
that is, $q > 0$ and there is no nonzero integer $k$
such that $p \over k$ and $q \over k$ are integers
and $\left| {p \over k} \right| + \left| {q \over k} \right|
	< \left| p \right| + \left| q \right|$.
The type $\mathbb{Q}$ includes all such rational numbers.

The type $\mathbb{Q}^*$ relaxes the requirement $q>0$ to $q \geq 0$ and
includes two extra values, \EXP{1/0} and \EXP{-1/0} (sometimes called ``the
infinite rational'' and ``the indefinite rational'').
\marginnote{Is there any subtyping or coercion relationship between \EXP{\mathbb{Q}} and
\EXP{\mathbb{Q}*}?
If \EXP{\mathbb{Q}} is a subtype of \EXP{\mathbb{Q}*}
then static checking would normally disallow the assignment,
so is this a relaxation to that rule?  If so, the static checker needs to know about it.}
If a value of type $\mathbb{Q}^*$ is assigned to a
variable of type $\mathbb{Q}$, a \TYP{DivisionByZero} is thrown
at run time if the value is \EXP{1/0} or \EXP{-1/0}.
The type \EXP{\mathbb{Q}^{\#}} includes all of \EXP{1/0}, \EXP{-1/0}, and
\EXP{0/0}.
The advantage of $\mathbb{Q}^{\#}$ is that it is closed under the rational
operations $+$, $-$, $\times$, and $/$.
In ASCII, \EXP{\mathbb{Q}}, \EXP{\mathbb{Q}^*}, and
\EXP{\mathbb{Q}^\#} are written as \txt{QQ}, \txt{QQ\_star}, and
\txt{QQ\_splat}, respectively.
See \secref{advanced-rationals} for definitions of \EXP{\mathbb{Q}}, \EXP{\mathbb{Q}^*},
and \EXP{\mathbb{Q}^\#}.



\subsection{Pi}
\seclabel{pi}

The object named \EXP{\pi} (or \txt{pi}) may be used to represent the ratio
of the circumference of a circle to its diameter rather than a specific
floating-point value or interval value.
In Fortress, \EXP{\pi} has type
\EXP{\TYP{RationalValueTimesPi}\llbracket\VAR{false},1,1\rrbracket}.
When used in a floating-point computation, it becomes
a floating-point value of the appropriate precision; when used in an interval
computation, it becomes an interval of the appropriate precision.

\subsection{Infinity and Zero}

The object named \EXP{\infty} has type \EXP{\TYP{ExtendedIntegerValue}\llbracket\VAR{true},0,\VAR{true}\rrbracket}.
One can negate \EXP{\infty} to get a negative infinity.

\note{What about spaces?  That is, is ``- 0'' treated the same as ``-0'', for example?}
Negating the literal \EXP{0} produces a special negative-zero object,
which refuses to participate in static arithmetic
(discussed in \secref{static-expr}).  It
has type \TYP{NegativeZero}.  The main thing it is good for is coercion to
a floating-point number (discussed in
\chapref{conversions-coercions}).
(Negating any other zero-valued expression simply produces zero.)
